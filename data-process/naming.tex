\section{数据重命名}
用rdseed软件从SEED格式中解压得到的SAC数据,一般都具有固定格式的文件名。
示例如下:
\begin{minted}{bash}
2012.055.12.34.56.7777.YW.MAIO.01.BHE.Q.SAC
2012.055.12.34.50.6666.YW.MAIO.01.BHN.Q.SAC
2012.055.12.34.54.5555.YW.MAIO.01.BHZ.Q.SAC
\end{minted}
这三个文件是YW台网MAIO台站的宽频地震仪记录的宽频带三分量(BHE、BHN、BHZ)波形数据。
文件名中每一项的具体含义在``\nameref{chap:naming}''中有介绍,这里不再重复。

默认的文件名比较长,在数据处理时可能会显得比较麻烦,一般都会根据实际需求进行适当的简化。
数据重命名这一步骤可以单独操作,如果数据有执行合并操作的话,可以在把合并后的文件
写入磁盘的同时进行重命名操作。

\subsection{重命名格式}
是否要对数据做重命名,以及按照什么格式重命名,都是没有一定的标准的。通常需要
用户根据自己所做研究的实际情况来决定。

在某些情况下,我们会将同一事件在所有台站的波形数据放在同一个文件夹下,并将文件名以事件的
发生日期/时间来命名。那么,SAC文件名中的时间等信息就可以被省略掉:
\begin{minted}{bash}
YW.MAIO.BHE
YW.MAIO.BHN
YW.MAIO.BHZ
\end{minted}

有时候,我们会将不同事件在同一个台站的波形数据放在同一个文件夹下,并将文件名以台站
名来命名,此时数据文件名中可能需要保留事件的日期信息:
\begin{minted}{bash}
YW.MAIO.20120224.BHE
YW.MAIO.20120224.BHN
YW.MAIO.20120224.BHZ
\end{minted}

\subsection{重命名脚本}
\subsubsection*{Bash}
Bash下可以借助于awk来实现文件重命名。下面的例子中,首先用点号对文件名做分割,
\verb+$0+表示原始文件名,\verb+$7+表示用逗号分割后的第7段字符,即台网名,
其他同理。最后将awk的输出传给sh去执行。
\begin{minted}{console}
$ ls *.SAC | awk -F "." '{printf "mv %s %s.%s.%s", $0, $7, $8, $10}' | sh
\end{minted}

\subsubsection*{Perl}
Perl下先用split函数将文件名用点号分割
\footnote{由于split函数的第一个参数实际上是正则表达式,故而这里的点号需要转义。}。
点号分割之后的字符串数组赋值给一系列变量,然后使用rename函数实现重命名。
\begin{minted}{perl}
#!/usr/bin/env perl
use strict;
use warnings;

foreach my $file (glob "*.SAC") {
    my ($year, $jday, $hour, $min, $sec, $msec,
        $net, $sta, $loc, $chn, $q, $suffix) = split /\./, $file;
    rename $file, "$net.$sta.$chn";
}
\end{minted}

上面的方法中定义了一堆变量但是却没有用到。可以将split函数的结果赋值给数组,然后
直接取数组中需要的元素即可。需要注意的是,Perl的数组下标从零开始。
\begin{minted}{perl}
#!/usr/bin/env perl
use strict;
use warnings;

foreach my $file (glob "*.SAC") {
    my @item = split /\./, $file;
    rename $file, "$item[6].$item[7].$item[9]";
}
\end{minted}

\subsubsection*{Python}
与上面的Perl解法类似,先用split方法将文件名分割成列表,然后直接从列表中取
需要的元素。
\begin{minted}{python}
#!/usr/bin/env python
# -*- coding: utf8 -*-
import os
import glob

for file in glob.glob("*.SAC"):
    item = file.split('.')
    os.rename(file, "%s.%s.%s" % (item[6], item[7], item[9]))
\end{minted}
