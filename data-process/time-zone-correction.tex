\section{时区校正}
\label{sec:time-zone-correction}
假设有一个数据文件,数据中的时间都是中国时间,即东八区时间,现想将数据修改至
国际标准时间,即要对数据做时区校正,将数据的绝对时间整体减少8个小时。
前面说过,时区校正可以通过修改头段b值来实现。

\begin{SACCode}
SAC> r nykl.z                           // 读入数据
SAC> lh b e kzdate kztime               // 查看头段信息

          b = 1.999622e+02              // b=200
          e = 1.600968e+03              // e=1600
     kzdate = SEP 10 (254), 1984
     kztime = 03:14:07.000

SAC> ch b (&1,b& - 8*3600)              // 取b值,减去8小时,再赋值给b
SAC> lh

          b = -2.860004e+04             // 此时b=200-8*3600=28600
          e = -2.719903e+04
     kzdate = SEP 10 (254), 1984        // 参考时间不变
     kztime = 03:14:07.000

SAC> ch allt (0 - &1,b&) iztype IB      // 将参考时间设置为文件起始时间
SAC> lh b e kzdate kztime

          b = 0.000000e+00
          e = 1.401006e+03
     kzdate = SEP 09 (253), 1984        // 中国时间的9月10日3时
     kztime = 19:17:26.963              // => 国际标准时间的9月9日19时
\end{SACCode}

从上面的例子中可以看出,头段变量b的值被减去了8个小时,而数据的参考时间并没有
改变,因而数据整体向前移动了8个小时,即完成了时区校正。

需要注意的是,若数据中头段o、a、f或tn这些相对时间是有定义的,则这些相对时间
都会由于b值的修改而出错,因而时区校正要尽早做。
