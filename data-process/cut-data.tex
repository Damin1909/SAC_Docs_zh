\section{数据截窗}
相关命令:\nameref{cmd:cut}

数据申请时一般会选择尽可能长的时间窗,而实际进行数据处理和分析时可能只需要其中
的一小段时间窗,这就需要对数据时间窗进行截取。

SAC中的~\nameref{cmd:cut}~命令可以实现数据截取,需要注意的是,该命令时
``参数设定类'' 命令,即需要先执行cut命令再执行read命令。

\subsection{pdw}
\label{subsec:pdw}
使用cut命令对数据进行截取时需要定义数据时间窗。除了截取数据之外,其他一些命令
也会需要定义时间窗,比如~\nameref{cmd:rms}、\nameref{cmd:mtw}、\nameref{cmd:xlim}~等,
这些命令都使用同样的方式定义时间窗,在SAC中称为pdw,即partial data window。

pdw定义了一个开始时间和一个结束时间,其格式为~\verb+ref offset ref offset+。
其中ref为参考时刻,可以取Z、B、E、O、A、F、N和Tn(n=0-9),而offset为相对于参考时刻的时间偏移量。

参考时刻ref可以取如下值:
\begin{itemize}
\item B: 磁盘文件起始值
\item E: 磁盘文件结束值
\item O: 事件开始时间
\item A: 初动到时
\item F: 信号结束时间
\item Tn: 用户自定义时间标记,n = 0,1...9
\item Z: 参考时刻
\item N: 将offset解释为数据点数而非时间偏移量,其仅可以用于结束值
\end{itemize}

若开始或结束的offset省略则认为其值为0。若开始ref省略则认为其为Z;若结束ref省略则
认为其值与开始ref相同。

下面的例子中展示了一些常见的pdw及其含义:
\begin{SACCode}
 B E            // 文件开始到文件结束,即与cut off相同
 B 0 30         // 文件开始的30秒
 A -10 30       // 初动前10秒到初动后30秒
 B N 2048       // 文件最初的2048个点
 T0 -10 N 1000  // 从T0前10秒起的1000个点
 30.2 48        // 相对磁盘文件0点的30.2到48秒
\end{SACCode}

\subsection{cut}
\nameref{cmd:cut}命令是``参数设定类''命令,因而需要先cut再read:
\begin{SACCode}
SAC> cut t0 -5 5        // 截取t0前后各5秒,共计10秒的数据
SAC> r *.SAC            // 先cut再read
\end{SACCode}
