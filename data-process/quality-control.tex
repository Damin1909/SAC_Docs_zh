\section{质量控制}

质量控制就是标记/删除信噪比低或不合适的波形。

用户可以自己在程序中判断数据的好坏以进行质量控制,但这样做很困难,因为实际情况中,
会遇到各种奇形怪状的``坏''波形,很难用统一的程序将这些``坏''波形挑选出来,所以
更多时候需要人工的参与。


我个人的做法如下:
\begin{SACCode}
SAC> r *.SAC        // 读入全部的SAC数据
SAC> ppk p 5        // plotpk,每次绘制5个波形
// 若波形质量很差,则用t9标记
SAC> wh
SAC> q
\end{SACCode}

解释一下以上做法,首先读入所有的SAC数据,然后利用plotpk,每次绘制n(n取5-10比较合适)
个波形。若波形质量很好,则不理会;若波形质量很差,则在波形的任意时刻标记t9的值
(具体如何标记可以参考下一节的内容),然后使用wh将标记的t9保存到头段中,再退出SAC。

完成上面的步骤之后,所有``坏''波形的t9都被标记,一般来说都是一个正值,而所有``好''波形的
t9则都处于未定义状态,其值为~\verb+t9=-12345.0+。

鉴于此,可以通过如下命令删除``坏''波形:
\begin{minted}{console}
$ saclst t9 f seis | awk '$2>0{print "rm",$1}' | sh
\end{minted}

注意:一定要在理解该命令的含义的前提下才可使用!
