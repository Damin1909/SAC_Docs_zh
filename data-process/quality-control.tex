\section{质量控制}

质量控制就是标记/删除信噪比低或不合适的波形。

用户可以自己在程序中判断数据的好坏以进行质量控制,但这样做很困难,
因为实际情况中,会遇到各种奇形怪状的``坏''波形,很难用统一的程序将
这些``坏''波形挑选出来,所以更多时候需要人工的参与。

一种常规的做法如下:
\begin{SACCode}
SAC> r *.SAC        // 读入全部的SAC数据
SAC> ppk p 5        // plotpk,每次绘制5个波形
// 若波形质量很差,则用t9标记
SAC> wh
SAC> q
\end{SACCode}

解释一下以上做法,首先读入所有的SAC数据,然后利用 \nameref{cmd:plotpk},
每次绘制n个波形,如果是3分量数据,n一般取3。若波形质量很好,则不理会;
若波形质量很差,则在波形的任意时刻标记 !t9! 的值(具体如何标记
可以参考下一节的内容),然后使用 !wh! 将标记的 !t9! 保存
到头段中,再退出SAC。

完成上面的步骤之后,所有``坏''波形的 !t9! 都被标记,一般来说都
是一个正值,而所有``好''波形的 !t9! 则都处于未定义状态,其值为
-12345.0。

鉴于此,可以通过如下命令删除``坏''波形:
\begin{minted}{console}
$ saclst t9 f *.SAC | awk '$2>0 {print "rm",$1}' | sh
\end{minted}
\begin{note}
一定要在理解该命令的含义的前提下才可使用,否则可能会造成数据的丢失!
\end{note}

当然,也可以用如下命令将``坏''数据移动到专门的目录中:
\begin{minted}{console}
$ mkdir BAD
$ saclst t9 f *.SAC | awk '$2>0 {print "mv", $1, "BAD/"}' | sh
\end{minted}
\begin{note}
awk命令中,目录名BAD后最好加上斜杠。若不加斜杠,且忘记新建目录BAD,
则所有应该放在目录BAD中的文件都会被重命名为BAD,进而导致文件丢失。
\end{note}
