\section{数据格式转换}
\subsection{数据格式}
通常,地震数据以SEED格式进行保存和传输。SEED\footnote{SEED格式的详细
说明参考官方文档:\url{www.fdsn.org/seed_manual/SEEDManual_V2.4.pdf}。}即
Standard for the Exchange of Earthquake Data,其可以存储多台站多分量
的波形数据以及台站元数据\footnote{台站元数据(Metadata)中包含了台站相关
的全部信息,比如台站位置、分量信息、仪器响应等。}。SEED格式本质上是一个
压缩格式,因而可以大大减少网络传输的数据量以及硬盘空间,同时又可以通过
特定的软件将其中的波形数据解压成常见的地震数据格式,也可以将从台站元数据
中提取出仪器响应信息。

除了SEED格式,还有miniSEED格式和dataless SEED格式。miniSEED格式中仅包含
波形数据,dataless SEED格式中仅包含台站元数据。之所以要将SEED格式拆分成
miniSEED和dataless SEED,是因为若每个SEED文件中都包含台站元数据,会造成
台站元数据的冗余,浪费网络资源及硬盘容量。

除了SEED格式之外,还有其他数据格式,比如为数据库设计的CSS 3.0格式,以及
众多数据处理软件自定义的格式,如SAC、AH、evt等等。不同国家的台网也可能会
自定义自己的数据格式,比如日本Hi-net台网的数据使用自己定义的win32格式。

\subsection{格式转换}
IRIS提供了 \href{http://ds.iris.edu/ds/nodes/dmc/forms/rdseed/}{rdseed}
软件,用于提取SEED数据中的连续波形数据以及台站元数据,并可将连续波形数据
保存为多种地震数据格式。

下面的命令可以从SEED数据中提取SAC格式的波形数据,以及台站的RESP仪器响应文件:
\begin{minted}{console}
$ rdseed -Rdf file.seed
\end{minted}

下面的命令可以从SEED数据中提取SAC格式的波形数据,以及台站的PZ仪器响应文件:
\begin{minted}{console}
$ rdseed -pdf file.seed
\end{minted}
