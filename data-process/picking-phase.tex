\section{震相拾取}
\label{sec:phase-picking}
相关命令:\nameref{cmd:plotpk}

震相拾取,或者说标定到时,是SAC的一种常用功能。

\subsection{ppk模式的进入与退出}
要进行震相拾取,首先要进入``ppk模式''。

读取波形数据后,在终端中键入 \texttt{plotpk}(简写为 \texttt{ppk}),
就会出现一个绘图窗口。若之前未曾打开过绘图窗口,则此时焦点位于ppk
打开的绘图窗口中;若之前曾经打开过绘图窗口,则需要鼠标点击一下绘图
窗口以使得焦点位于绘图窗口而不是终端中。此时,SAC就进入了``ppk模式'',
终端中光标所在行没有SAC提示符``\texttt{SAC>}''。

\begin{SACCode}
SAC> fg seis
SAC> ppk    // 焦点位于绘图窗口中,进入ppk模式
            // 光标所在行没有提示符"SAC> "
\end{SACCode}

学会如何进入ppk模式后,还要学会退出ppk模式。首先,确保焦点位于绘图窗口
而不是终端,然后将光标移动到绘图窗口中,按下``\texttt{q}''键即可退出
ppk模式。此时,终端中光标所在行会重新出现SAC提示符``\texttt{SAC>}''。

\begin{note}
只有当使用了ppk命令,焦点位于当前绘图窗口,且鼠标位于当前绘图窗口内才
称为ppk模式。在ppk模式下,所有的键盘输入都会被解释为"ppk命令",但不
会在终端中显示出来。若使用ppk命令后,不慎使焦点位于终端内,即脱离了
ppk模式,此时所有的键盘输入都会出现在终端中,但不会被SAC解释,当退出
ppk模式时,SAC才会依次解释终端中的命令。
\end{note}

\subsection{ppk模式下拾取震相}
下面介绍如何在ppk模式中拾取震相。先进入ppk模式,此时焦点位于绘图窗口,
并保证鼠标位于绘图区(即四个边框)的内部,移动鼠标到要标记到时的地方,
依次按下 \texttt{t}、\texttt{0},在要标记的到时处会出现一条竖线,
旁边有标识 \texttt{T0} ,此时已经将要标记的到时(即竖线对应的X轴位置)
保存到头段变量 \texttt{T0} 中。再按下 \texttt{q} 以退出ppk模式,
最后在终端键入 \texttt{wh} 将内存中的头段变量写回到磁盘文件中。

除了可以键入 \texttt{t} 和 \texttt{0} 之外,0还可以用1到9的任意数字替换,
分别表示将要标记的到时保存到 \texttt{T0} 到 \texttt{T9} 中。

\begin{SACCode}
SAC> fg seis
SAC> ppk
// 键入"t"和"0"标记到时,然后按"q"退出ppk模式
SAC> lh t0
     t0 = 1.255385e+01
SAC> wh         // 保存头段
\end{SACCode}

\begin{note}
在键入"t"时,鼠标不仅要在绘图窗口内,还要在绘图区(即四个边框)的内部,
否则会得到"Bad cursor position. Please retry."的错误提示。
\end{note}

\begin{note}
SAC全局变量 \verb|SAC_PPK_USE_CROSSHAIRS| 可以控制ppk模式下鼠标在绘图
窗口内的形态。若其值为 \texttt{0},则鼠标会以十字线的形式出现,即
"{\Large$+$}";当其值为 \texttt{1} 时,会在十字线的基础上加上水平线
和垂直线。通常建议设置其值为 \texttt{1},使得拾取到时时更精确。该全局
变量的设置方式参考 \nameref{sec:sac-install-for-linux} 一节。
\end{note}

\subsection{qdp off}
SAC在默认情况下会打开快速绘图选项,即 \texttt{qdp on}。关于 \texttt{qdp},
可以参考``\nameref{sec:plot-appearance}''一节以及命令 \nameref{cmd:qdp}
的说明。

在拾取震相时,若打开了快速绘图选项,则由于数据没有完全绘制而导致震相的可
识别度降低,也导致波形拾取精度降低。为了提高拾取精度,通常会在进入ppk
模式前关闭快速绘图选项,即使用 \texttt{qdp off} 命令:
\begin{SACCode}
SAC> dg sub reg elk.z
SAC> qdp on     // 打开快速绘图选项(默认值)
SAC> ppk
SAC> qdp off    // 关闭快速绘图选项
SAC> ppk        // 注意观察与之前的区别
\end{SACCode}
每次启动SAC后进入ppk模式前,都要手动执行 \texttt{qdp off} 以关闭快速
绘图选项,这样相对比较麻烦,可以使用``\nameref{sec:init-macro}''一节中
介绍的方法使得每次SAC启动时自动关闭快速绘图选项。

\subsection{放大与缩小}
有时数据时间较长,难以精确标定到时,此时需要将图幅放大,以显示整个波形的
一小部分。

首先需要将光标移动到绘图区域中的某位置,键入``\texttt{x}'',
再移动至另一位置,再次键入``\texttt{x}''。这样,两次键入确定了一个时间窗。
这时,绘图窗口中将只显示该时间窗内的波形,也就实现了图幅的放大。
可不断重复此步骤,进行多次放大。

SAC 101.5之后的版本有更方便的方式:在绘图窗口中某位置按下鼠标左键,
并拖动至另一位置再松开鼠标左键,则两个位置之间的时间窗内的波形会被放大。

图幅的缩小通过键入``\texttt{o}''来实现,``\texttt{o}''最多可以回退5次
绘图历史。

\subsection{同时标记三分量}
通常,震相在同一台站的三分量数据上具有相同的到时,因而将同一台站的
三分量数据画在一张图上,一方面可以综合三分量的波形信息以更准确地识别
震相,另一方面,一次标定三分量的震相到时可以减少工作量并保证震相在
三分量上的到时相同。使用命令``\texttt{ppk p 3 a m}''进入ppk模式即可
每次只显示并同时标记三个波形数据。

通常在拾取震相时会一次性读入多个台站的波形数据,而``\texttt{ppk p 3 a m}''
一次只能显示三个波形数据,可以在ppk模式下不断键入``\texttt{n}''以依次显示
接下来的三个波形,也可以键入``\texttt{b}''以显示前三个波形。当不断键入
``\texttt{n}''直到所有波形数据都显示完毕的时候,会自动退出ppk模式。

\begin{SACCode}
SAC> dg sub teleseis ntkl.[nez] nykl.[nez] onkl.[nez] sdkl.[nez]
SAC> ppk p 3 a m
// 键入"t0"标记ntkl台站的三分量到时
// 键入"n"以绘制接下来的三个数据
// 键入"t0"标记nykl台站的三分量到时
// 键入"n"以绘制接下来的三个数据
// 键入"b"以绘制之前的三个数据
// 键入"t0"重新标记nykl台站的三分量到时
// 键入"n"以绘制接下来的三个数据
// 键入"t0"标记onkl台站的三分量到时
// 键入"n"以绘制接下来的三个数据
// 键入"t0"标记sdkl台站的三分量到时
// 键入"n"自动退出ppk模式
SAC> wh
SAC> q
\end{SACCode}

在使用``\texttt{ppk p 3 a m}''选项同时标记三分量时需要注意:
\begin{itemize}
\item 三分量数据的参考时刻必须相同;若参考时刻不相同,则标记的结果是错误的
\item 该命令每次会按照顺序显示内存中的三个波形数据,当且仅当每次显示的
    三个波形数据恰好是同一台站的三分量数据时,该命令才能用作同时标记
    同一台站的三个分量
\end{itemize}

要使得每次显示的恰好是同一台站的三分量波形数据,则要求同一台站的三个分量
在内存中分别位于第$n$、$n+1$和$n+2$位,其中n为正整数。通常情况下,一次性读入
全部数据的时候,都可以满足这一要求。但也有一些例外:
\begin{itemize}
\item 数据文件名比较奇葩,导致读入时同一台站的三分量数据不是紧挨着读入的,
    可以使用``\texttt{ls *.SAC}''命令检查文件的读入顺序;
\item 某个台站丢失了一个分量的数据,导致后面的所有台站都出现问题;
\end{itemize}

\subsection{ppk命令}
除了上面介绍的若干ppk命令之外,还有很多其他ppk命令。
表 \ref{table:plotpk-commands} 列出了ppk模式下的所有命令,
其中常用的命令包括``\texttt{b}''、``\texttt{l}''、``\texttt{n}''、
``\texttt{o}''、``\texttt{q}''、``\texttt{t}''和``\texttt{x}''。

\begin{center}
\small\ttfamily
\begin{longtable}{cll}
\multicolumn{3}{r}{接上页} \\
\toprule
命令    &   含义    &   说明    \\
\midrule
\endhead
\caption{ppk模式命令一览表} \label{table:plotpk-commands}   \\
\toprule
命令    &   含义    &   说明    \\
\midrule
\endfirsthead
\bottomrule
\multicolumn{3}{r}{接下页\dots} \\
\endfoot
\bottomrule
\endlastfoot

a       &   定义事件初至a                           &   1,7     \\
b       &   如果有,则显示上一张绘图                &           \\
c       &   计算事件的初至和结束                    &   1,4,7   \\
d       &   设置震相方向为DOWN                      &           \\
e       &   设置震相起始为EMERGENT(急始)          &           \\
f       &   定义事件结束f                           &  1,2,3,7  \\
g       &   以HYPO格式将拾取显示到终端              &   4       \\
h       &   将拾取写成HYPO格式                      &   3,4     \\
i       &   设置震相起始为IMPULSIVE                 &           \\
j       &   设置噪声水平                            &   2,6,8   \\
k       &   即kill,退出ppk模式                     &           \\
l       &   显示光标当前位置                        &   2,4     \\
m       &   计算最大振幅波形                        &   2,3,5   \\
n       &   显示下一绘图                            &           \\
o       &   显示前一个绘图窗,最多可以保存5个绘图窗 &           \\
p       &   定义P波到时                             &   1,2,3,7 \\
q       &   即quit,退出ppk模式                             &           \\
s       &   定义S波到时                             &   1,2,3,7 \\
t       &   用户自定义到时tn,输入t之后需要输入0到9中的任一数   &   1,2,7\\
u       &   设置震相方向为UP                        &           \\
v       &   定义一个Wood-Anderson波形               &   2,5     \\
w       &   定义一个通用波形                        &   2,5     \\
x       &   使用一个新的x轴时间窗,简单说就是放大。 &           \\
z       &   设置参考水平                            &   2,6,8   \\
\textbackslash      &   删除当前全部拾取的定义。当一个文件中包含多个事件时有用。&   \\
+       &   设置震相方向为PLUS                      &           \\
-       &   设置震相方向为MINUS                     &           \\
\lstinline[showspaces]! !   &   设置震相方向为NEUTRAL                   &           \\
n       &   设置震相质量为n,n取0-4                 &           \\
\end{longtable}
\end{center}

注意:ppk模式的命令几乎都是由单个字符组成的,比如退出``\texttt{q}'',
唯一的例外是命令``\texttt{t}'',由字符``\texttt{t}''和0--9的整数构成。

不同的命令效果可能不同,有些会在绘图窗口显示信息,有些会将信息写入头段i
变量,下面对表 \ref{table:plotpk-commands} 中的说明进行一个说明:
\begin{description}
    \item [1] 会将信息写入头段变量
    \item [2] 写入字符型震相拾取文件(若已打开)
    \item [3] 写入HYPO格式震相拾取文件(若已打开)
    \item [4] 在绘图窗口中显示信息
    \item [5] 窗口显示包含波形的矩形
    \item [6] 在指定的水平处放置水平光标
    \item [7] 绘图窗口显示含有到时标识的垂直线
    \item [8] 绘图窗口显示含有标识的水平线
\end{description}
