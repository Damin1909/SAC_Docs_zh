\section{数据格式转换}
\subsection{数据格式}
通常,地震波形数据以SEED格式进行保存和传输。
SEED即Standard for the Exchange of Seismic Data,其可以存储多台站多分量
的连续波形数据以及台站相关元信息。SEED格式本质上
是一个压缩格式,因而可以大大减少网络传输的数据量以及硬盘空间。

除了SEED格式之外,还有miniSEED格式和dataless SEED格式。其中miniSEED格式中仅包含
连续波形数据,dataless SEED格式中仅包含台站元信息。之所以要将SEED格式拆分成miniSEED
和dataless SEED,是因为若每个SEED文件中都包含台站元信息,会造成台站元信息的冗余,
浪费网络资源及硬盘容量。

除了SEED格式之外,还有其他数据格式,比如为数据库设计的CSS 3.0格式,以及众多数据处理
软件自定义的格式,如SAC、AH、evt等等。不同国家的台网也可能会自定义自己的数据格式,
比如国内的SEED格式是在国际标准SEED上修改得到的(二者不完全兼容),日本Hi-net台网的
数据则使用自己定义的win32格式。

\subsection{格式转换}
IRIS提供了rdseed\footnote{\url{http://ds.iris.edu/ds/nodes/dmc/forms/rdseed/}}
软件,用于提取SEED数据中的连续波形数据以及台站元信息,并可将连续波形数据保存为
多种地震数据格式。

下面的命令可以从SEED数据中提取SAC格式的波形数据,以及台站的RESP仪器响应文件:
\begin{minted}{console}
$ rdseed -Rdf file.seed
\end{minted}

下面的命令可以从SEED数据中提取SAC格式的波形数据,以及台站的PZ仪器响应文件:
\begin{minted}{console}
$ rdseed -pdf file.seed
\end{minted}
