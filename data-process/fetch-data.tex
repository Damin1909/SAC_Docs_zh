\section{数据获取与转换}
地震波形数据的获取途径很多,列举如下:
\begin{enumerate}
\item IRIS:\url{www.iris.edu}
\item Hi-net:\url{www.hinet.bosao.go.jp}
\item GEOFON:\url{geofon.gfz-potsdam.de}
\end{enumerate}

数据的传输介质主要是网络传输为主,也有以移动硬盘、光盘或U盘为传输介质的。

最常见的数据交换格式是SEED格式,用于储存多台站多分量的连续波形数据
以及台站相关元信息。SEED格式本质上是一个压缩文件,因而可以大大减少网络
传输数据量以及硬盘空间。除了SEED格式之外,miniSEED格式仅包含连续波形数据,
而dataless SEED格式仅包含台站元信息。

IRIS提供了rdseed软件,用于读取SEED格式,并将其中的波形数据解压成为多种
地震数据格式。Hi-net提供了win32\_tools,用于将win32格式的数据文件解压成为SAC格式。