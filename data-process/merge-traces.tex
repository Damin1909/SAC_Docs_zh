\section{合并数据}
相关命令: \nameref{cmd:merge}

很多时候,从SEED数据中解压出来的同一个台站的SAC波形数据会被切割成多个等长或不等长的数据段。
可能是因为仪器在某些时刻存在问题导致连续数据出现间断,也可能是出于其它考虑将数据进行切割。
用户需要将这些数据段合并成单个包含连续波形数据的文件。

假定解压出来的台网NET、台站STA、位置ID为LOC的BHZ分量的连续波形被分割成了多个文件,
需要将多个文件合并成单个文件
\footnote{注意!由于SAC在101.6版重写了merge函数,本例仅在101.6之后的版本有效!具体
细节请参考本文档的~\nameref{cmd:merge}~命令以及本文档的之前版本。}:

\begin{SACCode}
SAC> r *.NET.STA.LOC.BHZ        // 读入所有需要合并的文件
SAC> merge                      // 内存中的所有文件被合并为一个文件
SAC> w NET.STA.LOC.BHZ          // 写出到磁盘中
\end{SACCode}

对于所有要合并的数据文件,SAC会检测其knetwk、kstnm、kcmpnm和delta是否完全匹配,并
智能判断每个文件的合并顺序。

实际合并的过程中,可能会出现数据间断或数据重叠的情况。若数据存在间断,可对其直接
补零或线性插值;若数据存在重叠,则可以比较重叠部分数据是否相同或对重叠的波形进行平均。
