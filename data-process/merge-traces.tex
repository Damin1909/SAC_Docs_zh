\section{合并数据}
相关命令:\nameref{cmd:merge}

有些时候,从SEED数据中解压出来的同一台站同一分量的连续波形数据,会被
分割成多个等长或不等长的文件,数据断开的可能原因是仪器在某些时刻存在
问题导致连续数据出现间断,也可能是出于其它考虑将数据进行切割。此时需要
首先对数据进行合并。

假定解压出来的台网 \texttt{NET}、台站 \texttt{STA}、位置为 \texttt{00}的
\texttt{BHZ} 分量的连续波形被分割成了多个文件,需要将多个文件合并成单个
文件。

对于v101.6之前的版本,只能用如下命令进行合并:
\begin{SACCode}
SAC> r 2012.055.12.00.00.0000.NET.STA.00.BHZ.Q.SAC
SAC> merge 2012.055.12.25.00.0000.NET.STA.00.BHZ.Q.SAC
SAC> merge 2012.055.12.40.00.0000.NET.STA.00.BHZ.Q.SAC
SAC> ...
SAC> merge 2012.055.13.20.00.0000.NET.STA.00.BHZ.Q.SAC
SAC> w NET.STA.00.BHZ
\end{SACCode}
即先读取第一段数据,然后合并第二段数据,再合并第三段数据。对于多个
数据段的合并,需要执行多次合并命令,且合并时文件的顺序必须按照绝对时间
的先后顺序。

SAC从v101.6开始重写了 \nameref{cmd:merge} 命令,可以使用如下更简洁的形式:
\begin{SACCode}
SAC> r *.NET.STA.00.BHZ        // 读入所有需要合并的文件
SAC> merge                     // 内存中的所有文件被合并为一个文件
SAC> w NET.STA.00.BHZ          // 写回到磁盘中
\end{SACCode}

对于所有要合并的数据文件,SAC会检测 \texttt{knetwk}、\texttt{kstnm}、
\texttt{kcmpnm} 和 \texttt{delta} 是否完全匹配,并智能判断每个文件的
合并顺序。

实际合并的过程中,可能会出现数据间断或数据重叠的情况。若数据存在间断,
可对其直接补零或线性插值;若数据存在重叠,则可以比较重叠部分数据是否
相同或对重叠的波形进行平均。
