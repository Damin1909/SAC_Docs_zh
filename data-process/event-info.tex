\section{事件信息}
\label{sec:event-info}
相关头段:\texttt{evla}、\texttt{evlo}、\texttt{evdp}、\texttt{mag}、
    \texttt{o}、\texttt{nzyear}、\texttt{nzjday}、\texttt{nzhour}、
    \texttt{nzmin}、\texttt{nzsec}、\texttt{nzmsec}

一般来说,从SEED连续波形中解压得到的SAC数据中是没有事件信息的。这就需要
用户从地震目录中获取事件的发震时刻、经度、纬度、深度和震级信息,并将这些
信息写入到SAC文件的头段中。SAC提供了用于可以修改头段变量的命令
\nameref{cmd:chnhdr},以及将修改后的头段变量写到磁盘文件的命令
\nameref{cmd:writehdr}\footnote{也可以使用 \texttt{w over} 将修改写回
磁盘文件。关于 \texttt{wh} 和 \texttt{w over} 的区别,参考
\nameref{sec:wh-and-wover} 一节。}。

\subsection{经纬度、深度与震级}
想要修改事件的经纬度、深度和震级,操作如下:
\begin{SACCode}
SAC> r cdv.?
SAC> ch evla 37.52 evlo -121.68 evdp 5.95   // 修改三个头段变量
SAC> ch mag 5.0                             // 修改一个头段变量
SAC> wh                                     // 将修改后的头段写入文件
\end{SACCode}

\subsection{发震时刻}
通常,需要将发震时刻信息写入SAC头段,并设置SAC文件的参考时刻为发震时刻。
这样设置的好处在于,可以直观地从X轴坐标上读取震相走时。要实现这一操作,
需要用到 \nameref{cmd:chnhdr} 的两个特殊用法。

假设发震时刻为1987年06月22日11时10分10.363秒:
\label{code:origin-time}
\begin{SACCode}
SAC> r ./cdv.?
SAC> ch o gmt 1987 173 11 10 10 363   // 06月22日是第173天
SAC> lh kzdate kztime o

     kzdate = JUN 22 (173), 1987
     kztime = 11:09:56.363
          o = 1.400000e+01       // 发震时刻相对于参考时刻的时间为14秒
SAC> ch allt -14 iztype IO       // 参考时间加14秒,其他时间减14秒
SAC> lh kzdate kztime o

     kzdate = JUN 22 (173), 1987
     kztime = 11:10:10.363
          o = 0.000000e+00
SAC> wh                          // 写回磁盘
\end{SACCode}

在上面的例子中,首先从地震目录中获取了地震的发震时刻,然后计算发震日期
对应一年中的第几天,本例中为第173天,再利用``\texttt{ch o gmt yyyy ddd hh mm sss xxx}''
的语法将发震时刻赋值给头段变量 \texttt{o},SAC会自动将发震时刻转换为
相对于参考时刻的相对时间。由于此时SAC文件的参考时刻为``1987-06-22T11:09:56.363'',
而 \texttt{o} 值对应的时刻为发震时刻``1987-06-22T11:10:10.363'',所以
头段变量 \texttt{o} 的值为发震时刻相对于参考时刻的时间差,即 \SI{14}{\s}。

将发震时刻写入头段之后,还需要将参考时刻修改为发震时刻,与此同时还要
修改所有的相对时间。\texttt{ch allt xx.xx} 的功能是将所有已定义的
相对时间加上 \texttt{xx.xx} 秒,同时从参考时刻中减去 \texttt{xx.xx} 秒,
此时参考时刻即为发震时刻,而 \texttt{o} 值为0。

实际数据处理的时候,不可能像上面的示例那样使用 \texttt{lh o} 查看
\texttt{o} 的值再使用 \texttt{ch allt} 选项,此时就需要使用SAC提供的
引用头段变量的值的功能。具体的语法以及用法会在``\nameref{chap:sac-programming}''
一章中介绍。

\begin{SACCode}
SAC> ch o gmt 1987 173 11 10 10 363
SAC> ch allt (0 - &1,o&) iztype IO
\end{SACCode}

需要注意,上例中的``\verb|(0 - &1,o&)|''必须原样输入!由于101.6版本
重写了SAC宏的语法分析器,所以很多用法都发生了改变。你可能会发现,在
101.6及其以后的版本中,\verb|(0-&1,o&)|、\verb|(-&1,o&)|、\verb|(-&1,o)| 
都可以正常使用,但是上例中的版本是唯一一个在所有SAC版本中都正确的,
为了命令的通用性,只要记住这个就可以了。

进行实际数据处理时,可以参考第 \ref{subsec:ch-origin} 节中提供的Perl脚本。
