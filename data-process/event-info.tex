\section{事件信息}
\label{sec:event-info}
相关头段:\texttt{evla}、\texttt{evlo}、\texttt{evdp}、\texttt{mag}、
    \texttt{o}、\texttt{nzyear}、\texttt{nzjday}、\texttt{nzhour}、
    \texttt{nzmin}、\texttt{nzsec}、\texttt{nzmsec}

相关脚本:\hyperref[subsec:event-info-perl]{Perl脚本}、
          \hyperref[subsec:event-info-python]{Python脚本}

一般来说,从SEED连续波形中解压得到的SAC数据中是没有事件信息的。这就需要
用户从地震目录中获取事件的发震时刻、经度、纬度、深度和震级信息,并将这些
信息写入到SAC文件的头段中。SAC提供了用于可以修改头段变量的命令
\nameref{cmd:chnhdr},以及将修改后的头段变量写到磁盘文件的命令
\nameref{cmd:writehdr}\footnote{也可以使用 \texttt{w over} 将修改写回
磁盘文件。关于 \texttt{wh} 和 \texttt{w over} 的区别,参考
\nameref{sec:wh-and-wover} 一节。}。

\subsection{经纬度、深度与震级}
想要修改事件的经纬度、深度和震级,操作如下:
\begin{SACCode}
SAC> r cdv.?
SAC> ch evla 37.52 evlo -121.68 evdp 5.95   // 修改三个头段变量
SAC> ch mag 5.0                             // 修改一个头段变量
SAC> wh                                     // 将修改后的头段写入文件
\end{SACCode}

\subsection{发震时刻}
通常,需要将发震时刻信息写入SAC头段,并设置SAC文件的参考时刻为发震时刻。
这样设置的好处在于,可以直观地从X轴坐标上读取震相走时。要实现这一操作,
需要用到 \nameref{cmd:chnhdr} 的两个特殊用法。

先看看如何修改\textbf{一个}SAC文件的发震时刻,假设发震时刻为1987年06月22日
11时10分10.363秒:
\label{code:origin-time}
\begin{SACCode}
SAC> r ./cdv.z
SAC> ch o gmt 1987 173 11 10 10 363   // 06月22日是第173天
SAC> lh kzdate kztime o

     kzdate = JUN 22 (173), 1987
     kztime = 11:09:56.363
          o = 1.400000e+01       // 发震时刻相对于参考时刻的时间为14秒
SAC> ch allt -14 iztype IO       // 参考时间加14秒,其他时间减14秒
SAC> lh kzdate kztime o

     kzdate = JUN 22 (173), 1987
     kztime = 11:10:10.363
          o = 0.000000e+00
SAC> wh                          // 写回磁盘
\end{SACCode}
上面的例子中,首先从地震目录中获取了地震的发震时刻,然后计算发震日期
是一年中的第几天,本例中为第173天,再利用``\texttt{ch o gmt yyyy ddd hh mm sss xxx}''
语法将发震时刻赋值给头段变量 \texttt{o},SAC会自动将发震时刻转换为
相对于参考时刻的相对时间。此时SAC文件的参考时刻为``\texttt{1987-06-22T11:09:56.363}'',
而 \texttt{o} 值对应的时刻为发震时刻``\texttt{1987-06-22T11:10:10.363}'',所以
头段变量 \texttt{o} 的值为发震时刻相对于参考时刻的时间差,即 \SI{14}{\s}。

将发震时刻写入头段之后,还需要将参考时刻修改为发震时刻,与此同时还要
修改所有的相对时间。\texttt{ch allt xx.xx} 的功能是将所有已定义的
相对时间加上 \texttt{xx.xx} 秒,同时从参考时刻中减去 \texttt{xx.xx} 秒,
此时参考时刻即为发震时刻,而 \texttt{o} 值为0。

上面的做法需要执行5个命令才能实现,而且需要人工查看 \texttt{o} 的值,
因而无法用于处理大量数据。下面就对这一例子做进一步简化,这其中需要使用
SAC提供的``在命令中引用头段变量的值''的功能。具体的语法以及用法在
``\nameref{chap:sac-programming}''一章中会介绍。
\begin{SACCode}
SAC> r cdv.z
SAC> ch o gmt 1987 173 11 10 10 363
SAC> ch allt (0 - &1,o&) iztype IO
SAC> wh
\end{SACCode}
在这个例子中,\verb|(0 - &1,o&)| 代替了上个例子中的 \texttt{-14}。
简单介绍一下 \verb|(0 - &1,o&)|的含义,\verb|&1,o&| 表示引用内存中
第一个SAC文件的头段变量 \texttt{o} 的值(即14),然后 \verb|(0 - 14)|
得到的结果即为 \texttt{-14}。此处简化的优点在于,不需要使用 \texttt{lh o}
查看头段变量的值,完全可以实现自动化。

\begin{note}
在SAC v101.6及之后的版本中,上例中的\verb|(0 - &1,o&)|还可以写成
\verb|(0-&1,o&)|、\verb|(-&1,o&)|或\verb|(-&1,o)|。
而在SAC v101.5c及之前的版本中,只能使用\verb|(0 - &1,o&)|,注意减号
两边的空格。考虑到命令的通用性,建议使用上面示例中的写法。
\end{note}

上面的示例只适用于为一个SAC数据添加发震时刻的情况。如果要一次性为多个
SAC数据添加同样的发震时刻,最直观的想法是:
\begin{SACCode}
SAC> r *.SAC
SAC> ch o gmt 1987 173 11 10 10 363
SAC> ch allt (0 - &1,o&) iztype IO
SAC> wh
\end{SACCode}
这样的做法是有很大风险的。因为内存中一次性读入了很多SAC数据,而在使用
\verb|ch allt| 命令时,\verb|&1,o&| 引用的是第一个SAC数据的 \texttt{o}
头段。第二个命令已经保证了内存中所有的数据的 \texttt{o} 都有相同的绝对
时刻(即发震时刻),只要所有数据的参考时刻是一致的,那么所有数据的头段
变量 \texttt{o} 的值也必然是一样的。所以当且仅当内存中的所有数据的参考
时刻完全一致时,上面的例子才是安全的。实际处理数据时会遇到很复杂的情况,
``所有数据的参考时刻完全一致''这一假设不一定成立。

在上面的例子的基础上再加一个命令:
\begin{SACCode}
SAC> r *.SAC
SAC> synchronize            // 同步所有数据的参考时间
SAC> ch o gmt 1987 173 11 10 10 363
SAC> ch allt (0 - &1,o&) iztype IO
SAC> wh
\end{SACCode}
\nameref{cmd:synchronize} 的作用是使内存中所有的数据拥有相同的参考时刻,
在此命令的基础上,所有数据的头段变量 \texttt{o} 将拥有相同的值,所以
直接引用第一个头段变量的 \texttt{o} 值就不再是一件危险的事情了。
