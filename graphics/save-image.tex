\section{图像保存}
\label{sec:save-image}

\subsection{xwindows}
xwindows是SAC中最常用绘图设备,对于震相拾取等交互式操作更是必不可缺。
\begin{SACCode}
SAC> fg seis
SAC> bd x       // begindevice xwinows,可省略
SAC> p          // 绘图
SAC> ed x       // enddevice xwindows,可省略
SAC> q
\end{SACCode}

对于xwindows,最简单的保存图像的方式是截图,常用的工具包括gnome下的screenshot或者
ImageMagick的import命令。

\subsection{sgf}
SGF图形设备会将图像信息保存到SGF文件中。其使用方式为:``启用sgf图形设备''$\rightarrow$
``绘图到sgf''$\rightarrow$``关闭sgf设备,退出SAC''$\rightarrow$``将sgf文件转换为其它格式''。
\begin{SACCode}
SAC> fg seis
SAC> bd sgf         // 启动sgf设备,不可省略
SAC> p
SAC> ed sgf         // 关闭sgf设备,可省略
SAC> q
$ ls
f001.sgf            // 生成sgf文件
\end{SACCode}

生成的sgf文件可以通过sgftops等命令转换为其它图像格式,在``\nameref{sec:sgftops}''中会介绍,
也可以使用sgftox直接将sgf文件显示在绘图窗口中。

\subsection{PS和PDF}
自101.5之后,SAC加入了saveimg命令,可以将当前xwindow或sgf图形设备中的图像保存到
其它ps或pdf格式的图像文件中,以获得更高质量的绘图效果。
\footnote{该命令也支持输出为png和xpm格式,但png和xpm为位图图像格式,精度不够,
且依赖于其它函数库,因而不推荐使用。}

\begin{SACCode}
SAC> fg seis
SAC> p                      // 首先在xwindows上绘图
SAC> saveimg foo.ps         // 将xwindows上的图像保存到foo.ps中
save file foo.ps [PS]
SAC> q
\end{SACCode}

\subsection{pssac}
pssac是Prof. Lupei Zhu写的用于绘制SAC文件的C程序。该程序利用了GMT的PS绘图库,
直接读取SAC文件并绘制到PS文件中。得益于GMT的PS库的灵活性,利用pssac可以绘制出
超高质量的复杂图像。具体参见``\nameref{sec:pssac}''一节。

\subsection{几种图像保存方法的比较}
\begin{itemize}
\item 在xwindows上绘图简单省事,直接截图的效果较差,仅可用于非正式的演示;
\item sgf转换为其它图像格式稍显麻烦,但适合在脚本中批量做图;
\item saveimg生成图像文件质量相对较高,可以满足大多数需求;
\item pssac功能强大,在一般绘图以及复杂图像时非常有用,适合在发表文章时使用;
\end{itemize}
