一个基本的编程语言需要包含哪些特性呢?变量、参数、函数、条件判断、循环
控制等等。

这一章介绍SAC所设计的一个基本的编程语言,在官方文档中直接称其为SAC宏。
SAC与SAC宏的关系在某种程度上更像是matlab与matlab脚本之间的关系。
最简单的,将一系列要一起执行的SAC命令放在一个文件中即构成了SAC宏文件。

本文档中使用了稍有不同的说法,并将这一章命令为``\nameref{chap:sac-programming}''。
其包含了三个主要的部分:
\begin{itemize}
\item 变量:由于SAC的特殊性,又分为头段变量和一般变量;
\item 内置函数:基本的数学和字符串函数;
\item SAC宏:参数、条件判断、循环控制等;
\end{itemize}

在官方文档中,变量和内置函数都是SAC宏的一部分,本文档将其从SAC宏中提取出来是出于
如下几个方面的考虑:
\begin{enumerate}
\item 变量和内置函数既可以在SAC命令中使用,也可以在SAC宏中使用;而其它特性如参数、
    条件判断、循环控制等几乎只能在SAC宏中使用;
\item SAC设计的编程语言功能简单,不够友好;非常建议使用Bash、Perl或Python这些更
    成熟的脚本语言来替代SAC的编程功能。宏参数、条件判断、循环控制等特性都可以被
    脚本的相应功能完全取代,而由于SAC设计的特殊性,诸如变量和内置函数等特性在
    某些情况下不能完全取代。
\end{enumerate}

所以,建议的做法是读完本章的内容,掌握如何引用头段变量、如何使用黑板变量以及内联
函数,简单了解SAC宏的特性,选择Perl或者Python脚本语言
\footnote{Bash在很多地方还是不如Perl或Python方便,不推荐。}
进行数据批量处理,尽量避免
使用黑板变量和内联函数。就目前的个人经验而言,在脚本语言中,SAC的引用头段变量
功能必不可少,黑板变量和内联函数这两个功能或多或少都可以被取代。

另外,尤其需要注意的是,SAC自101.6起彻底重写了SAC宏的语法分析器,因而导致101.6以后
的SAC宏与之前的SAC宏有很大不同,本文档只讨论101.6重写的SAC宏语法。
