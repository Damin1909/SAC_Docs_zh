\section{文档约定}
约定这个事情,说起来容易做起来难,遇到不符合约定的地方只能靠读者自己领悟了。

\subsection*{语法约定}
\begin{enumerate}
\item 命令和选项使用大写字母,参数使用小写字母
\item 命令和选项均使用全称,简写形式可省略的部分用灰色表示
\item ``![XXX]!''表示中括号内的 !XXX! 为可选项
\item ``!A|B|C!''表示可以在A、B、C中任选一项
\end{enumerate}

示例如下:
\begin{SACSTX}
B!AND!P!ASS! [BU!TTER!|BE!SSEL!|C1|C2] [C!ORNERS! v1 v2] [N!POLES! n] [P!ASSES! n]
    [T!RANBW! v] [A!TTEN! v]
\end{SACSTX}

需要特别说明的是,命令语法中选项的简写形式是在保证不产生歧义下的前提下
所允许的\textbf{最简形式}。本例中,!CORNERS! 的最简形式为首字符
!C! ,用户也可以使用 !CO!、!COR! 等来表示 !CORNERS!。

\subsection*{示例约定}
\begin{enumerate}
\item 命令、选项、参数均使用小写字母
\item 常见的命令和选项均使用简写表示
\item 含有提示符``!SAC>!''的行是用户键入的命令,无提示符的行是SAC输出行;
\item SAC输出行中可能会删除一些不重要的信息;
\item 示例中加入注释以帮助用户理解,注释使用了C语言的行注释符号``!//!'';
\item 命令长度过长时会被拆分成多行,每一行的行尾会加上续行符``!\!'',
    但需要注意,SAC中不能使用续行符;
\item 示例中若出现``!...!'',表示省略了一堆对数据的处理流程;
\item 除非上下文说明,否则每个例子都运行在单独的SAC会话中,即每个命令都
    省略了启动sac和退出sac的命令;
\item 除特别情况外,均省略 \nameref{cmd:plot} 命令,用户应该学会随时
    !plot! 以查看当前内存中的波形结果;
\end{enumerate}

示例如下:
\begin{SACCode}
$ sac                           // 该行省略
SAC> fg seis                    // 这是注释
SAC> p                          // 该行省略
SAC> lh o
     o = -4.143000e+01          // SAC输出行
SAC> q                          // 该行省略
\end{SACCode}
