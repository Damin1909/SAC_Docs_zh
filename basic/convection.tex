\section{文档约定}
约定这个事情,说起来容易做起来难,遇到不符合约定的地方只能靠读者自己领悟了。

\subsection*{语法约定}
\begin{enumerate}
\item 命令和选项使用大写字母,参数使用小写字母;
\item 命令和选项均使用全称,简写形式可省略的部分用灰色表示;
\item ``\texttt{[ ]} ''表示该项为可选项;
\item ``\texttt{A|B|C} ''表示A、B、C中任选一项;
\end{enumerate}

示例如下:
\begin{SACSTX}
B!AND!P!ASS! [BU!TTER!|BE!SSEL!|C1|C2] [C!ORNERS! v1 v2] [N!POLES! n] [P!ASSES! n]
    [T!RANBW! v] [A!TTEN! v]
\end{SACSTX}

需要特别说明的是,命令语法中选项的简写形式是在保证不产生歧义下的前提下所允许的\textbf{最简形式}。
本例中,\texttt{CORNERS} 的最简形式为首字符 \texttt{C} ,用户也可以使用
\texttt{CO}、\texttt{COR} 等来表示 \texttt{CORNERS}。

\subsection*{示例约定}
\begin{enumerate}
\item 命令、选项、参数均使用小写字母;
\item 常见的命令和选项均使用简写表示;
\item 含有提示符``\texttt{SAC>}''的行是用户键入的命令,无提示符的行是SAC输出行;
\item SAC输出行中可能会删除一些不重要的信息;
\item 示例中加入注释以帮助用户理解,注释使用了C语言的行注释符号``\texttt{//}'';
\item 命令长度过长时会被拆分成多行,每一行的行尾会加上续行符``\verb|\|'',但需要
    注意,SAC中不能使用续行符;
\item 示例中若出现``\texttt{...}'',表示省略了一堆对数据的处理流程;
\item 除非上下文说明,否则每个例子都运行在单独的SAC会话中,即每个命令都
    省略了启动sac和退出sac的命令;
\item 除特别情况外,均省略 \nameref{cmd:plot} 命令,用户应该学会随时 \nameref{cmd:plot} 以查看当前内存中的波形结果;
\end{enumerate}

示例如下:
\begin{SACCode}
$ sac                           // 该行省略
SAC> fg seis                    // 这是注释
SAC> p                          // 该行省略
SAC> lh o
     o = -4.143000e+01          // SAC输出行
SAC> q                          // 该行省略
\end{SACCode}
