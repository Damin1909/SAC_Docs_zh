\section{绘图}
\label{sec:display}

SAC中有四个常用的绘图命令,分别是 \nameref{cmd:plot}、\nameref{cmd:plot1}、
\nameref{cmd:plot2}、\nameref{cmd:plotpk}。这一节只介绍最基础的
\texttt{plot} 命令,其他的命令及更多的绘图功能将在
\nameref{chap:sac-graphics} 中说明。

\texttt{plot} 命令会在单个图形窗口中显示单个波形:
\begin{SACCode}
SAC> r cdv.[nez]
SAC> p
Waiting
Waiting
SAC>
\end{SACCode}

上面的示例中,首先将三个波形数据读入内存,然后使用 \texttt{plot} 命令
绘图,此时焦点位于绘图窗口,且绘图窗口中只显示第一个波形,终端中出现
``Waiting''字样;将焦点切换\footnote{Linux下的快捷键是Alt+Tab。}回终端,
敲击回车键,绘图窗口中显示第二个波形,终端中出现第二个``Waiting''字样,
焦点位于终端中;再次敲击回车键,窗口中显示第三个波形,焦点位于终端,
由于已经没有更多的波形需要显示,此时终端中显示SAC提示符。

如果内存中还有波形在``Waiting'',而你想要退出plot,不想要再继续查看
后面的波形,可以在终端中键入 \texttt{kill}(简写为 \texttt{k}),即可
直接退出plot,如下例:
\begin{SACCode}
SAC> r cdv.[nez]
SAC> p
Waitingk
SAC>
\end{SACCode}
