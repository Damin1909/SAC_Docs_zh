\section{样本数据}
想要学习SAC,手头必须有SAC格式的数据,SAC提供了两个命令可以用于生成SAC
格式数据,分别是 \nameref{cmd:funcgen} 和 \nameref{cmd:datagen}。

\subsection{funcgen}
\nameref{cmd:funcgen}(简写为 \texttt{fg})表示``function generator'',
即该命令可以生成一些特定的函数,比如脉冲、阶跃、正弦等等,还可以生成
一个地震波形样本:
\begin{SACCode}
SAC> fg impulse         // 生成脉冲函数
\end{SACCode}
上面的命令生成了一个脉冲函数并存储在SAC的内存中,可以用命令
\nameref{cmd:plot}(写为 \texttt{p})在图形界面上查看这个函数的样子:
\begin{SACCode}
SAC> p
\end{SACCode}

在学习SAC的过程中,\texttt{funcgen} 可以生成地震波形样本:
\begin{SACCode}
SAC> fg seismogram      // 生成地震波形样本,简写为fg seis
\end{SACCode}
这个命令在SAC内存中产生了一个地震波形样本,同时删除了内存中刚才生成的
脉冲信号,可以使用 \texttt{plot} 命令查看地震波形。这个地震波形样本在
以后的教程中经常用到。

\subsection{datagen}
\nameref{cmd:datagen}(简写为 \texttt{dg})表示``data generator''。
顾名思义,就是用来生成数据的。

下面的示例在内存中生成了CDV台站记录到的一个近震的三分量波形数据
\footnote{101.4的软件包中没有自带波形数据,因而无法使用该命令。},
并用 \nameref{cmd:plot1}(简写 \texttt{p1})将三个波形画在一张图上:
\begin{SACCode}
SAC> dg sub local cdv.?
SAC> p1
\end{SACCode}
更多示例参考 \nameref{cmd:datagen} 命令的语法说明。
