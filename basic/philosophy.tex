\section{SAC设计思想}
SAC的设计思想大概可以总结如下:
\begin{enumerate}
\item 每个信号\footnote{信号,或称之为trace,即\textbf{单个}台站
    \textbf{单个}仪器\textbf{单个}分量记录到的连续时间序列。}
    被保存到单独的SAC格式数据文件中;
\item SAC格式包含了描述数据特征的头段区和存储信号的数据区,
    参见``\nameref{chap:sac-file-format}''一章;
\item 将单个或多个SAC文件从磁盘读入内存;
\item 通过各种命令对内存中的数据进行操作;
\item 操作完毕,将内存中的数据写入到磁盘,可以覆盖原SAC文件或写入新文件中。
\end{enumerate}

读取SAC文件时的若干限制:
\begin{itemize}
\item SAC一次性最多处理1000个SAC文件;想要修改这个上限,参考
    \nameref{sec:mdfl} 一节;
\item 单个文件名所允许的最大长度为128字符;
\end{itemize}
