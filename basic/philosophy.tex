\section{SAC设计思想}
SAC的设计思想大概可以总结如下:
\begin{enumerate}
    \item 每个信号\footnote{信号,或称之为trace,即\textbf{单个}台站\textbf{单个}仪器\textbf{单个}分量记录到的连续时间序列。}
被保存到单独的SAC格式数据文件中;
\item SAC格式包含了描述数据特征的头段区和存储信号的数据区,参见``~\nameref{chap:sac-file-format}~''一章;
\item 将单个或多个\footnote{一次性最多处理\textbf{1000}个任意大小的文件,记住1000这个值!}
    SAC文件\footnote{单个文件名所允许的最大长度为128字符。}从磁盘读入内存;
\item 通过各种命令对内存中的数据进行操作;
\item 操作完毕,将内存中的数据写入到磁盘,可以覆盖原SAC文件或写入新文件中。
\end{enumerate}
