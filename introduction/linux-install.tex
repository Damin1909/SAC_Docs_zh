\section{在Linux下安装SAC}
\label{sec:sac-install-for-linux}

Linux下安装SAC,可以直接使用官方提供的二进制包,也可以手动编译源码包。
对于大多数用户而言,建议安装二进制包。下面会分别介绍两种安装方法,要求
读者了解Linux的一些基本概念和操作。

\subsection{安装二进制包}
\subsubsection{安装依赖包}
官方提供的二进制包中的可执行文件可以直接使用,在运行时需要用到几个动态
链接库。大部分Linux发行版下,都默认安装了这几个动态链接库。若不幸没有
安装或不确定有没有安装,可以通过如下命令安装所需的软件包。

对于Ubuntu/Debian:
\begin{minted}{console}
$ sudo apt-get update
$ sudo apt-get install libc6 libsm6 libice6 libxpm4 libx11-6
$ sudo apt-get install zlib1g libncurses5
\end{minted}

对于CentOS/Fedora/RHEL:
\begin{minted}{console}
$ sudo yum install glibc libSM libICE libXpm libX11
$ sudo yum install zlib ncurses
\end{minted}

\subsubsection{安装二进制包}
直接将官方提供的二进制包解压并移动到安装目录即可:
\begin{minted}{console}
$ tar -xvf sac-101.6a-linux_x86_64.tar.gz   # 解压
$ sudo mv sac /usr/local                    # 安装
\end{minted}

\subsection{编译源码}
\subsubsection{安装依赖包}
编译源码时需要安装若干软件包。

对于Ubuntu/Debian系:
\begin{minted}{console}
$ sudo apt-get update
$ sudo apt-get install build-essential
$ sudo apt-get install libncurses5-dev libsm-dev libice-dev
$ sudo apt-get install libxpm-dev libx11-dev zlib1g-dev
\end{minted}

对于CentOS/Fedora/RHEL系:
\begin{minted}{console}
$ sudo yum install gcc gcc-c++ make
$ sudo yum install glibc ncurses-devel libSM-devel libICE-devel
$ sudo yum install libXpm-devel libX11-devel zlib-devel
\end{minted}

\subsubsection{编译源码}
将源码按如下命令解压、配置、编译、安装:
\begin{minted}{console}
$ tar -xvf sac-101.6a_source.tar.gz
$ cd sac-101.6a
$ mkdir build
$ cd build
$ ../configure --prefix=/usr/local/sac
$ make
$ sudo make install
\end{minted}

\subsection{配置变量}
向 \verb|~/.bashrc|\footnote{某些发行版需要修改 \verb|~/.bash_profile|}
中加入如下语句以配置环境变量和SAC全局变量:
\begin{minted}{bash}
export SACHOME=/usr/local/sac
export SACAUX=${SACHOME}/aux
export PATH=${SACHOME}/bin:${PATH}

export SAC_DISPLAY_COPYRIGHT=1
export SAC_PPK_LARGE_CROSSHAIRS=1
export SAC_USE_DATABASE=0
\end{minted}

其中,
\begin{itemize}
\item \texttt{SACHOME} 为SAC的安装目录
\item \texttt{SACAUX} 目录中包含了SAC运行所需的辅助文件
\item \texttt{PATH} 为Linux系统环境变量
\item \verb|SAC_DISPLAY_COPYRIGHT| 用于控制是否在启动SAC时显示版本和版权
    信息,一般设置为1。在脚本中多次调用SAC时会重复显示版本和版权信息,
    干扰脚本的正常输出,因而在脚本中一般将其值设置为0。具体的设置方法
    可以参考``\nameref{chap:sac-script}''中的相关内容
\item \verb|SAC_PPK_LARGE_CROSSHAIRS| 用于控制震相拾取过程中光标的大小,
    在 \nameref{sec:phase-picking} 时会用到
\item \verb|SAC_USE_DATABASE| 用于控制是否允许将SAC格式转换为GSE2.0格式,
    一般用不到该特性,故而设置其值为0
\end{itemize}

修改完 \verb|~/.bashrc| 后,执行以下命令使配置的环境变量生效:
\begin{minted}{console}
$ source ~/.bashrc
\end{minted}

\subsection{启动SAC}
终端键入小写的sac\footnote{Ubuntu的源里有一个名叫sac的软件,是用来显示
登录账户的一些信息;CentOS的源里也有一个名叫sac的软件,是CSS语法分析器的
Java接口。所以一定不要试图用发行版自带的软件包管理器安装sac!!!},显示
如下则表示SAC安装成功:
\begin{minted}{console}
$ sac
 SEISMIC ANALYSIS CODE [11/11/2013 (Version 101.6a)]
 Copyright 1995 Regents of the University of California

SAC>
\end{minted}
