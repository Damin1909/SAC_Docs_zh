\section{\texttt{rdseed}}
\label{sec:rdseed}

!rdseed! 用于读取SEED格式,从中提取出波形信息,并将波形数据保存
为SAC、AH、CSS、SEGY或ASCII等多种数据格式。

\subsection{语法说明}
终端键入 !rdseed -h! 即可查看 !rdseed! 的选项及语法说明。
!rdseed! 命令的选项众多,下面按照选项的重要性从高到低排序。

比较重要且常用的选项:
\begin{itemize}
\item !-f file!:输入的SEED文件名。!rdseed!一次只能处理一个SEED文件。
\item !-d!:从SEED数据中提取波形数据
\item !-o n!:输出波形数据的格式,默认为SAC格式。!n! 可以取1--9,
    分别表示SAC(1)、AH(2)、CSS(3)、miniSEED(4)、SEED(5)、
    SAC ASCII(6)、SEGY(7)、Simple ASCII(SLIST)(8)和
    Simple ASCII(TSPAIR)(9)。
\item !-R!:输出RESP格式的仪器响应文件
\item !-p!:输出SAC PZ格式的仪器响应文件
\item !-E!:输出的波形数据的文件名中包含结束时间
\item !-q!:指定输出目录,该目录必须已经存在。默认输出到当前目录。
\item !-Q!:选择波形数据的质量,可以取 !E!、!D!、
    !M!、 !R!、!Q!,其中 !E! 代表输出全部
    质量的波形数据,其他值的含义参考``\nameref{sec:quality-control}''一节。
\item !-b n!:输出的波形数据的最大数据点数,默认值为20000000,
    能所取到的最大值是4字节整型的上限,即2147483647。若波形数据的的数据
    点数超过该值,则会给出警告并把数据分割成多段。
\item !-g file!:为SEED或miniSEED数据单独指定响应文件。响应文件
    可以是SEED格式也可以是dateless SEED格式,也可以通过设置环境变量
    !ALT_RESPONSE_FILE! 指定响应文件,这样做的好处在于可以多个SEED
    文件共用一个响应文件
\item !-h! 或 !-u!:显示命令的用法
\item !-z n!:检查并校正数据极性,参考接下来的``\nameref{subsec:polarity-correction}''一节
\end{itemize}

不常用的选项:
\begin{itemize}
\item !-a!:提取缩略词词典
\item !-c!:提取文件内容的目录信息
\item !-C STN|CHN!: 提取台站或分量的注释信息
\item !-l!:列出每个block的内容
\item !-s!:输出全部台站的RESP格式仪器响应文件到终端
\item !-S!:提取台站的汇总信息到文件 !rdseed.stations!,
    内容包括台站名、台网名、经纬度、海拔、分量、台站开始时间和结束时间
\item !-t!:输出波形相关信息到终端,包括台站名、分量名、台网名、
    位置码、质量控制符、波形开始时间和结束时间、采样率、数据点数
\item !-v n!:选择卷号,默认值为1。对于SEED文件 !n! 只能取1
\item !-k! 跳过数据点数为0的记录
\item !-e!:提取事件/台站数据到文件 !rdseed.events!
\item !-i!:忽略位置码
\item !-x file!:使用 !JWEED! 生成的summary文件,根据summary
    文件提取指定台站、分量和时间窗内的波形数据
\end{itemize}

\subsection{正负极性及其校正}
\label{subsec:polarity-correction}

地震仪的每个分量都有一个传感器,每个传感器都有一个敏感轴,仪器记录的就是
敏感轴方向的运动物理量。每个敏感轴都有一个正方向,若地面运动与敏感轴的正
方向一致,则输出为正值,若地面运动与敏感轴的正方向相反,则输出为负值。

SAC头段中的 !cmpaz! 和 !cmpinc! 是用于描述仪器敏感轴正方向
的最通用也是最准确的方法。几个比较特殊的方向是:垂直方向、正东西向、
正南北向,在SAC中方位码分别为 !Z!、!E! 和 !N!。下表
列出了这六个方向所对应的 !cmpaz! 和 !cmpinc!。
\begin{table}[H]
\caption{6个标准方向的 \texttt{cmpaz} 和 \texttt{cmpinc}}
\label{table:six-cmpaz-cmpinc}
\centering
\begin{tabular}{ccccc}
\toprule
方向     &   !cmpaz! & !cmpinc!  & 方位码 & 极性  \\
\midrule
垂直向上 &   0              &  0               & Z      & 正    \\
垂直向下 &   0              &  180             & Z      & 负    \\
正北     &   0              &  90              & N      & 正    \\
正南     &   180            &  90              & N      & 负    \\
正东     &   90             &  90              & E      & 正    \\
正西     &   270            &  90              & E      & 负    \\
\bottomrule
\end{tabular}
\end{table}

对于一个方位码为 !Z! 的数据,其分量方向有两种可能性:垂直向上和
垂直向下。根据SAC中NEU坐标系的定义(图 \ref{fig:cmpaz-cmpinc}),垂直
向上方向为正极性,垂直向下方位为负极性。同理,正东和正北是正极性,正西
和正南为负极性。

由上表可知,通过检查分量的 !cmpaz! 和 !cmpinc! 即可判断是
是正极性还是负极性。某些情况下,分量角度是正常的,但仪器响应中的总增益
是负值,也可用于表示负极性,但这种情况很少见到,目前缺乏数据做测试,因而
暂且先不考虑增益为负的这种情况。

!rdseed! 中 !-z n! 选项可以用于检测并校正负极性。
\begin{itemize}
\item !n=0! 表示不做极性检测;
\item !n=1! 表示只检查 !cmpaz! 或 !cmpinc!;若是负
    极性,则反转所有数据点的正负号并修改 !cmpaz! 或 !cmpinc!
    的值;
\item !n=2! 表示只检查总增益的正负值;若为负值即表示负极性,则反转
    所有数据的正负号但不修改 !cmpaz! 或 !cmpinc!;
\item !n=3! 表示同时检查 !cmpaz! 或 !cmpinc! 以及
    总增益的正负值,仅当其中之一符合负极性的要求时才做校正;
\end{itemize}

需要注意,正负极性的概念仅适用于6个标准分量方向。对于垂向分量而言,通常
需要校正极性,否则在查看Z分量的波形数据时,可能会出现某个台站的波形极性
不对的状况;对于水平向分量而言,由于通常会旋转到大圆路径方向,所以不做
极性校正,也不会有问题。总之,建议使用 !-z 1! 选项做极性校正。

\subsection{用法示例}
从SEED文件中提取波形数据和RESP仪器响应文件:
\begin{minted}{console}
rdseed -R -d -f infile.seed
\end{minted}
其中,选项 !-R -d -f! 可以合写成 !-Rdf!。

从SEED文件中提取波形数据和SAC PZ仪器响应文件:
\begin{minted}{console}
rdseed -pdf infile.seed
\end{minted}

从miniSEED文件中提取波形数据,并指定dataless SEED文件作为仪器响应文件:
\begin{minted}{console}
rdseed -Rdf infile.miniseed -g infile.dataless
\end{minted}

\subsection{警告与错误}
使用 !rdseed! 的过程中可能会遇到一些警告和错误。这些警告和错误
会显示在终端,也会记录到日志文件 !rdseed.err_log! 中。

\subsubsection{警告1}
\begin{verbatim}
Warning... Azimuth and Dip out of Range on AAK,BH1
Defaulting to subchannel identifier (for multiplexed data only)
\end{verbatim}
若分量的 !cmpaz! 和 !cmpinc! 所指定的传感轴方向与垂直方向
的偏差很小,比如偏差在两度以内,则将该分量的方位码设置为 !Z!。
对于近正东西和近南北方向,设置分量方位码为 !E! 和 !N!。

若分量的敏感轴方向不与垂直向、正东西向、正南北向相近,则会出现此警告,
此时可能会设置分量的范围码为 !1! 或其他的字符。因而该警告可忽略。

\subsubsection{警告2}
\begin{verbatim}
Warning... Azimuth/Dip Reversal found FURI.BHZ, Data inversion was not selected
\end{verbatim}
该警告表示,根据分量的 !cmpaz! 和 !cmpinc! 检测到当前分量
是负极性,但不对数据作极性校正。这种情况下使用 !-z 1! 选项,会
修改数据的正负号,并将台站角度修改为正极性方向。
