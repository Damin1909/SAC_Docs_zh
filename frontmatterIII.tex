{\ctexset{section/format+=\centering}\section*{写在第三版前的一些废话}}

\begin{shadequote*}
\Large\emph{工欲善其事,必先利其器。}
\par\hfill\emph{\normalsize---《论语 $\cdot$ 卫灵公》}
\end{shadequote*}

2010年10月,大三,开始接触并学习SAC;2011年的暑假
开始着手SAC文档的翻译工作;2012年01月,文档的v1.0版本发布;2013年03月,
文档的v2.0版发布。3年多的时间过去了,文档更新到了v3.0版。

v2.0大体上算是官方英文文档的译本,整体结构上完全遵循了官方文档的风格。
整个文档的条理不够清晰,教程部分稍显单薄,命令部分也不够完善。

2013年11月,George Helffrich著的
``\emph{The Seismic Analysic Code: A Primer and User's Guide}''一书
出版了。该书基于MacSAC,与本文档所关注的SAC有一些区别,但核心部分是
一致的。v3.0版借鉴了该书的整体结构和部分内容,重新设计了文档结构并重写
了教程的大部分内容,希望能够有一个结构更清晰、内容更丰富的版本。

整个文档分为教程部分和命令部分。教程部分又分为如下几章:
\begin{description}
\item[SAC简介] SAC软件的相关信息
\item[SAC基础] 继续阅读手册所需的基础知识
\item[SAC文件格式] SAC文件格式及SAC头段变量
\item[SAC数据处理] 利用SAC命令进行地震数据处理和分析
\item[SAC图像] 用SAC绘制地震波形,并控制绘图的细节
\item[SAC编程] 利用SAC的宏编程功能实现数据批处理
\item[SAC与脚本语言] 脚本(Bash、Perl和Python)中调用SAC
\item[SAC函数库] C、Fortran程序中调用SAC提供的库函数
\item[SAC I/O] C、Fortran、Matlab和Python中实现SAC I/O
\item[SAC相关工具] SAC相关工具
\item[SAC技巧与陷阱] 使用SAC时的若干技巧与陷阱
\end{description}

本文档的源码开源托管在GitHub上,欢迎更多的SAC用户参与到该项目,共同
维护并完善本文档。

本文档仅供个人学习使用,希望不涉及版权问题。

\begin{flushleft}
个人博客:\url{http://seisman.info}                         \\
项目主页:\url{https://github.com/seisman/SAC_Docs_zh}      \\
联系方式:\url{seisman.info@gmail.com}                      \\
文档发布及更新:\url{http://seisman.info/sac-manual.html}   \\
捐赠页面:\url{http://seisman.info/donations.html}          \\
\end{flushleft}

\begin{flushright}
作者:SeisMan \\
2014年04月14日
\end{flushright}
