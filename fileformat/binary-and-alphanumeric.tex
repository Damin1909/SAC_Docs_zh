\section{两种数据形式}
SAC文件格式有两种形式:二进制型和字符型\footnote{原为alphanumeric,译为文数字。}。
字符型与二进制型是完全等价的,只是字符型是给人看的,二进制型是给机器读写的。
从C程序的角度来看,两者的区别在于,写文件时前者使用~\verb+fprintf+~后者使用~\verb+fwrite+~。

二进制型的SAC数据,占用更小的磁盘空间,读写速度更快,因而是最常用的SAC格式形式。
当文件出现问题时,字符型数据便于查看文件内容。

\subsection{两种形式的互相转换}
你是否想要一个字符型的SAC文件,用编辑器打开好好看看SAC数据究竟长什么样。SAC自带
的命令可以实现两种形式的转换
\footnote{也可以使用SAC的~\verb+convert+~命令进行转换,不过此命令即将被淘汰,
因而这里不会介绍。}。
\begin{SACCode}
SAC> fg seis
SAC> w seis             // 先生成一个二进制型SAC数据,以做测试
\end{SACCode}

将二进制型转换成字符型:
\begin{SACCode}
SAC> r seis             // 读二进制型文件
SAC> w alpha seis.a     // 以字符型写入
\end{SACCode}

将字符型转换成二进制型:
\begin{SACCode}
SAC> r alpha seis.a     // 读字符型文件
SAC> w sac seis.b       // 以二进制型写入,可以省略sac,写成w seis.b
\end{SACCode}

试试用你最喜欢的文本编辑器打开字符型的~\verb+seis.a+~吧,其内容如下:
\begin{minted}{bash}
     0.01000000      -1.569280       1.520640      -12345.00      -12345.00
       9.459999       19.45000      -41.43000       10.46400      -12345.00
      -12345.00      -12345.00      -12345.00      -12345.00      -12345.00
      -12345.00      -12345.00      -12345.00      -12345.00      -12345.00
      -12345.00      -12345.00      -12345.00      -12345.00      -12345.00
      -12345.00      -12345.00      -12345.00      -12345.00      -12345.00
      -12345.00       48.00000      -120.0000      -12345.00      -12345.00
       48.00000      -125.0000      -12345.00       15.00000      -12345.00
      -12345.00      -12345.00      -12345.00      -12345.00      -12345.00
      -12345.00      -12345.00      -12345.00      -12345.00      -12345.00
       373.0627       88.14721       271.8528       3.357465      -12345.00
      -12345.00    -0.09854718       0.000000       0.000000      -12345.00
      -12345.00      -12345.00      -12345.00      -12345.00      -12345.00
      -12345.00      -12345.00      -12345.00      -12345.00      -12345.00
      1981        88        10        38        14
         0         6         0         0      1000
    -12345    -12345    -12345    -12345    -12345
         1        50         9    -12345    -12345
    -12345    -12345        42    -12345    -12345
    -12345    -12345    -12345    -12345    -12345
    -12345    -12345    -12345    -12345    -12345
         1         1         1         1         0
CDV      K8108838
-12345  -12345  -12345
-12345  -12345  -12345
-12345  -12345  -12345
-12345  -12345  -12345
-12345  -12345  -12345
-12345  -12345  -12345
-12345  -12345  -12345
    -0.09728001    -0.09728001    -0.09856002    -0.09856002    -0.09728001
    -0.09600000    -0.09472002    -0.09344001    -0.09344001    -0.09344001
    -0.09344001    -0.09344001    -0.09472002    -0.09472002    -0.09344001
    ......
\end{minted}

第1--30行是头段区,31之后的行是数据区。
目前你可能还看不懂头段区的这些数字或者字符代表什么。没关系,在下一节会详细介绍
SAC头段区,记得一定要一边看下一节的内容,一边对照着这个例子,好好琢磨SAC的头段。
