\section{SAC格式简介}
一个地震波形数据包含了时间上连续的一系列数据点,数据点可以是等间隔或
不等间隔采样。SAC的数据格式要求一个文件中只包含一个地震波形数据,这样
的定义更适合单个地震波形的处理。

每个SAC文件包含两个部分:一个头段区和一个数据区。

头段区位于每个文件的起始处,其大小是固定的,用于描述数据的相关信息,
比如数据点数、采样周期等等。

数据区紧跟在头段区之后,数据区又包含了一个或多个子数据区:
\begin{itemize}
\item 如果数据是时间序列,且是等间隔采样的,则只有一个子数据区,包含
    因变量(Y,也就是数据)的值,因变量(X)的信息可以直接从头段区中获得;
\item 如果数据是时间序列,但是不等间隔采样的,则有两个子数据区,
    分别包含因变量(Y)和自变量(X)的值;
\item 如果数据是谱数据而非时间序列,则有两个子数据区,分别包含振幅和
    相位或者实部和虚部;
\item 如果数据是三维数据(XYZ),则包含 \texttt{nxsize*nysize} 个子数据区。
\end{itemize}
最经常使用的数据是等间隔采样的数据,即文件中只有一个头段区和一个数据区。
