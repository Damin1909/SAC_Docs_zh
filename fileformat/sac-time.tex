\section{SAC中的时间概念}
\label{sec:sac-time}

\subsection{基本思路}
SAC的头段区有很多与时间相关的头段变量,包括nzyear、nzjday、nzhour、nzmin、nzsec、
nzmsec、b、e、o、a、f、tn(n=0-9),正确使用它们的前提是理解SAC中的时间概念。
这一节将试着说清楚这个问题。

首先,SAC处理的是地震波形数据,SAC格式里保存的是时间序列数据。先不管其它的一些台站
经纬度、事件经纬度信息,就数据而言,至少需要一系列数据值以及每个数据值所对应的时刻。

在本节接下来的内容中,将严格区分两个高中物理学过的概念:时刻和时间。简单地说,
在时间轴上,时刻是一个点,时间是一个线段。

一个简单的例子如下:
\begin{minted}{bash}
2014-02-26T20:45:00.000     0.10
2014-02-26T20:45:01.000     0.25
2014-02-26T20:45:02.000     0.33
2014-02-26T20:45:03.000     0.21
2014-02-26T20:45:04.000     0.35
2014-02-26T20:45:05.000     0.55
2014-02-26T20:45:06.000     0.78
2014-02-26T20:45:07.000     0.66
2014-02-26T20:45:08.000     0.42
2014-02-26T20:45:09.000     0.34
2014-02-26T20:45:10.000     0.25
\end{minted}
其中第二列是数据点,每个数据点所对应的时刻放在第一列,格式为``yyyy-mm-ddThh:mm:ss.xxx''。
数据点是以1s的等间隔进行采样的。

若把这堆时刻以及数据点直接写入文件中,将占据大量的磁盘空间,读写也很不方便。
考虑将某一个时刻定义为参考时刻,并把其它所有的时刻都用相对于该参考时刻的时间来表示。
这样可以简化不少。

比如取``\verb+2014-02-26T20:45:00.000+''为参考时刻,即
\begin{minted}{bash}
nzyear = 2014
nzjday = 57
nzhour = 20
nzmin  = 45
nzsec  = 00
nzmsec = 000
\end{minted}
则上面的数据可以简化为
\begin{minted}{bash}
00.000     0.10
01.000     0.25
02.000     0.33
03.000     0.21
04.000     0.35
05.000     0.55
06.000     0.78
07.000     0.66
08.000     0.42
09.000     0.34
10.000     0.25
\end{minted}
其中第二列是数据点,第一列是每个数据点对应的时刻相对于参考时刻的相对时间,下面
简称其为相对时间。

显然参考时刻的选取是任意的,若取``\verb+2014-02-26T20:45:05.000+''为参考时刻,
则上面的数据简化为
\begin{minted}{bash}
-05.000     0.10
-04.000     0.25
-03.000     0.33
-02.000     0.21
-01.000     0.35
 00.000     0.55
 01.000     0.78
 02.000     0.66
 03.000     0.42
 04.000     0.34
 05.000     0.25
\end{minted}

一般来说,会选取一个比较特殊的时刻作为参考时刻,比如第一个数据点对应的时刻,或者
地震波形数据中的发震时刻。

下面还是回到以``\verb+2014-02-26T20:45:00.000+''为参考时刻简化得到的结果。
因为数据是等间距的,相对时间这一列完全可以进一步简化,比如用``起始相对时间+采样间隔
+数据点数''或者``起始相对时间+采样间隔+结束相对时间''就完全可以表征第一列的相对
时间。如果只能从二者之中选一个的话,我会选择第一种,毕竟``数据点数''太有用了。

SAC选择了另外一种简化模式,``起始相对时间+采样间隔+数据点数+结束相对时间'',即
头段变量中的``\verb|b+delta+npts+e|'',这其实是存在信息冗余的,这就
造就了头段变量~\verb+e+~的一些特殊性,后面会提到。

按照SAC的模式在对相对时间进行简化之后,整个数据可以表示为
\begin{minted}{bash}
nzyear = 2014
nzjday = 57
nzhour = 20
nzmin  = 45
nzsec  = 00
nzmsec = 000
b      = 0.0
e      = 10.0
delta  = 1.0
npts   = 11

0.10
0.25
0.33
0.21
0.35
0.55
0.78
0.66
0.42
0.34
0.25
\end{minted}

似乎到这里就结束了。

地震学里的一个重要问题是拾取震相到时(时刻),所以还需要几个额外的头段变量来保存
这些震相到时(时刻),不过显然我们不会真的把时刻保存到这些头段变量中,不然上面的
一大堆就真是废话了。SAC将震相到时(时刻)相对于参考时刻的时间差(即相对时间)保存
到头段变量o、a、f、tn中。

综上,SAC中跟时间有关的概念有三个:
\begin{description}
    \item [参考时刻] 由头段变量nzyear、nzjday、nzhour、nzmin、nzsec、nzmsec决定;
    \item [相对时间] 即某个时刻相对于参考时刻的时间差(单位为秒),保存到头段变量b、e
    o、a、f、tn;
    \item [绝对时刻] =参考时刻+相对时间;
\end{description}

\subsection{一些测试}
下面以一个具体的数据为例,通过修改各种与时间相关的头段来试着去进一步理解SAC的时间概念。

\subsubsection{生成样例数据}
\begin{SACCode}
SAC> fg seis
SAC> lh iztype

  FILE: SEISMOGRAM - 1
   ------------

    iztype = BEGIN TIME
SAC> ch iztype IUNKN
SAC> w seis
\end{SACCode}
lh是命令~\nameref{cmd:listhdr}~的简写,用于列出头段变量的值。ch是~\nameref{cmd:chnhdr}~
的简写,用于修改头段变量的值。

这里额外多做了一个操作修改iztype的操作,这是由于这个数据稍稍有一点bug。

iztype指定了参考时刻的类型,其显示为BEGIN TIME,实际上其枚举值是IB,也就是说这个数据
选取文件第一个数据点的时刻作为参考时刻,那么b的值应该为0。而实际上这个数据的b值
并不为0,这其实是这个数据的一点小bug。这也从另一个侧面说明
SAC只有在修改与时间相关的头段变量时才可能会检查到这个错误/警告,所以这里
先将其修正为IUNKN。

\subsubsection{修改文件起始时间b}
\begin{SACCode}
SAC> r seis
SAC> lh kzdate kztime b delta npts e o a f

  FILE: seis - 1
 --------------

     kzdate = MAR 29 (088), 1981
     kztime = 10:38:14.000
          b = 9.459999e+00
      delta = 1.000000e-02
       npts = 1000
          e = 1.945000e+01
          o = -4.143000e+01
          a = 1.046400e+01
SAC> ch b 10
SAC> lh

  FILE: seis - 1
   ----------

     kzdate = MAR 29 (088), 1981
     kztime = 10:38:14.000
          b = 1.000000e+01
      delta = 1.000000e-02
       npts = 1000
          e = 1.999000e+01
          o = -4.143000e+01
          a = 1.046400e+01
\end{SACCode}

修改b前后的变化仅在于b和e值的变化,而参考时刻以及其它相对时间并没有发生变化。

这意味着整段SAC数据中的任意一个数据点所对应的时刻\footnote{好长的修饰语}都向后
延迟了0.54秒!这样做很危险,因为b和e的绝对时刻被修改了,而其它头段如o、a、f、tn的
绝对时刻却没有变。

使用的时候必须非常小心:
\begin{itemize}
\item 如果o、a、f、tn都没有定义,那么修改b值可以用于校正仪器的时间零飘\footnote{零飘,
    即仪器中的时刻与标准时刻不同。}以及时区差异\footnote{时区差异可以理解成另一种零飘。}。
\item 如果o、a、f、tn已经被定义,则修改b值会导致与震相相关的头段变量出现错误!
    \footnote{如果只定义了o值,或者a、f、tn为理论震相到时而非计算机拾取或人工拾取的
    到时,修改b也是没有问题的。有些乱,不多说了。}
\end{itemize}

\subsubsection{修改文件结束时间e}
\begin{SACCode}
SAC> r ./seis
SAC> lh kzdate kztime b delta npts e o a f

  FILE: ./seis - 1
 ------------

     kzdate = MAR 29 (088), 1981
     kztime = 10:38:14.000
          b = 9.459999e+00
      delta = 1.000000e-02
       npts = 1000
          e = 1.945000e+01
          o = -4.143000e+01
          a = 1.046400e+01
SAC> ch e 0
SAC> lh

  FILE: ./seis - 1
 ------------

     kzdate = MAR 29 (088), 1981
     kztime = 10:38:14.000
          b = 9.459999e+00
      delta = 1.000000e-02
       npts = 1000
          e = 1.945000e+01
          o = -4.143000e+01
          a = 1.046400e+01
\end{SACCode}

可以看到,修改前后所有变量均没有发生变化,即e的值是不可以随意改变的,根据上面的
结果可知,e的值是通过b、delta、npts的值动态计算的。这也与上一节说到的头段变量
冗余问题相符合。不要试图修改delta、npts,这不科学!

\subsubsection{修改o、a、f、tn}
这几个头段变量完全是由用户自定义的,因而任何的定义、修改、取消定义都不会对数据
的正确性产生影响,因而这里不再测试。

\subsubsection{修改参考时间}
\begin{SACCode}
SAC> r ./seis
SAC> lh kzdate kztime b delta npts e o a f

  FILE: ./seis - 1
 ------------

     kzdate = MAR 29 (088), 1981
     kztime = 10:38:14.000
          b = 9.459999e+00
      delta = 1.000000e-02
       npts = 1000
          e = 1.945000e+01
          o = -4.143000e+01
          a = 1.046400e+01
SAC> ch nzsec 15
SAC> lh

  FILE: ./seis - 1
 ------------

     kzdate = MAR 29 (088), 1981
     kztime = 10:38:15.000
          b = 9.459999e+00
      delta = 1.000000e-02
       npts = 1000
          e = 1.945000e+01
          o = -4.143000e+01
          a = 1.046400e+01

\end{SACCode}

试图修改参考时刻,整个SAC头段,除了参考时刻外其它时间变量都没有发生变化。根据
``绝对时刻=参考时刻+相对时间''可知,这导致所有SAC数据点的绝对时刻发生了平移,
这一点理论上可以用于校正零飘或者时区,但是由于SAC不支持智能判断时间(比如不知道
1时80分实际上是2时20分),所以修改时区时需要获取参考时刻6个头段变量,加上时区的
校正值,再写入到参考时刻6个变量中,相对较为繁琐。

\subsubsection{修改发震时刻}
数据处理中一个常见的需求是修改发震时刻,这可以通过修改头段变量o来实现,但是经常
需要将参考时刻设置为发震时刻。
上面的测试表明,直接修改参考时刻是很危险的,所以SAC的ch命令提供了~\verb+allt+~选项来实现
这一功能,在~``\nameref{sec:event-info}''~一节中会具体解释。

\subsection{总结}
将SAC中的时间变量分为三类:
\begin{enumerate}
\item 参考时刻:即nzyear、nzjday、nzhour、nzmin、nzsec、nzmsec;
\item 相对时间:即o、a、f、tn;
\item 特殊的相对时间:即b\footnote{由于e不可独立修改,所以不再考虑};
\end{enumerate}

第二类时间变量可以随意修改,即震相拾取。

第一、三类时间变量的修改会导致数据绝对时刻发生改变。若小心使用,可用于校正时间零飘或时区
不一致问题。

而为了修改发震时刻(此时应保证数据的绝对时刻不发生改变),则需要使用ch提供的allt
选项来实现。
