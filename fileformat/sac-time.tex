\section{SAC中的时间概念}
\label{sec:sac-time}

\subsection{基本思路}
SAC的头段区有很多与时间相关的头段变量,包括 \texttt{nzyear}、
\texttt{nzjday}、\texttt{nzhour}、\texttt{nzmin}、\texttt{nzsec}、
\texttt{nzmsec}、\texttt{b}、\texttt{e}、\texttt{o}、\texttt{a}、
\texttt{f}、\texttt{tn}(n=0--9)。
正确使用它们的前提是理解SAC中的时间概念。这一节将试着说清楚这个问题。

首先,SAC处理的是地震波形数据,SAC格式里保存的是时间序列数据。先不管
其它的一些台站经纬度、事件经纬度信息,就数据而言,至少需要一系列数据值
以及每个数据值所对应的时刻。

在本节接下来的内容中,将严格区分两个高中物理学过的概念:时刻和时间。
简单地说,在时间轴上,时刻是一个点,时间是一个线段。

一个简单的例子如下:
\begin{verbatim}
        2014-02-26T20:45:00.000     0.10
        2014-02-26T20:45:01.000     0.25
        2014-02-26T20:45:02.000     0.33
        2014-02-26T20:45:03.000     0.21
        2014-02-26T20:45:04.000     0.35
        2014-02-26T20:45:05.000     0.55
        2014-02-26T20:45:06.000     0.78
        2014-02-26T20:45:07.000     0.66
        2014-02-26T20:45:08.000     0.42
        2014-02-26T20:45:09.000     0.34
        2014-02-26T20:45:10.000     0.25
\end{verbatim}
其中第二列是数据点,每个数据点所对应的时刻放在第一列,格式为
``\texttt{yyyy-mm-ddThh:mm:ss.xxx}''。数据点是以 \SI{1}{\s} 的等间隔
进行采样的。

若把这堆时刻以及数据点直接写入文件中,将占据大量的磁盘空间,读写也很
不方便。考虑将某一个时刻定义为参考时刻,并把其它所有的时刻都用相对于
该参考时刻的秒数来表示,这样可以简化不少。

比如取``\texttt{2014-02-26T20:45:00.000}''为参考时刻,即
\begin{verbatim}
        nzyear = 2014
        nzjday = 57
        nzhour = 20
        nzmin  = 45
        nzsec  = 00
        nzmsec = 000
\end{verbatim}
则上面的数据可以简化为
\begin{verbatim}
        00.000     0.10
        01.000     0.25
        02.000     0.33
        03.000     0.21
        04.000     0.35
        05.000     0.55
        06.000     0.78
        07.000     0.66
        08.000     0.42
        09.000     0.34
        10.000     0.25
\end{verbatim}
其中第二列是数据点,第一列是每个数据点对应的时刻相对于参考时刻的相对时间,
下面简称其为相对时间。

显然参考时刻的选取是任意的,若取``\texttt{2014-02-26T20:45:05.000}''为
参考时刻,则上面的数据简化为
\begin{verbatim}
        -05.000     0.10
        -04.000     0.25
        -03.000     0.33
        -02.000     0.21
        -01.000     0.35
         00.000     0.55
         01.000     0.78
         02.000     0.66
         03.000     0.42
         04.000     0.34
         05.000     0.25
\end{verbatim}

一般来说,会选取一个比较特殊的时刻作为参考时刻,比如第一个数据点对应的
时刻,或者地震波形数据中的发震时刻。

下面还是回到以``\texttt{2014-02-26T20:45:00.000}''为参考时刻简化得到的结
果。因为数据是等间距的,相对时间这一列完全可以进一步简化,比如用``起始相
对时间+采样间隔+数据点数''或者``起始相对时间+采样间隔+结束相对时间''就完
全可以表征第一列的相对时间。

SAC选择了另外一种简化模式,``起始相对时间+采样间隔+数据点数+结束相对时间'',
即头段变量中的``\texttt{b+delta+npts+e}'',这其实是存在信息冗余的,这就
造就了头段变量 \texttt{e} 的一些特殊性,后面会提到。

按照SAC的模式在对相对时间进行简化之后,整个数据可以表示为
\begin{verbatim}
        nzyear = 2014
        nzjday = 57
        nzhour = 20
        nzmin  = 45
        nzsec  = 00
        nzmsec = 000
        b      = 0.0
        e      = 10.0
        delta  = 1.0
        npts   = 11

        0.10
        0.25
        0.33
        0.21
        0.35
        0.55
        0.78
        0.66
        0.42
        0.34
        0.25
\end{verbatim}
似乎到这里就结束了。

地震学里的一个重要问题是拾取震相到时(时刻),所以还需要几个额外的
头段变量来保存这些震相到时(时刻),不过显然不需要真的把``时刻''保存
到这些头段变量中,不然上面的一大堆就真是废话了。SAC将震相到时(时刻)
相对于参考时刻的时间差(即相对时间)保存到头段变量 \texttt{o}、
\texttt{a}、\texttt{f}、\texttt{tn} 中。

综上,SAC中跟时间有关的概念有三个:
\begin{description}
\item [参考时刻] 由头段变量 \texttt{nzyear}、\texttt{nzjday}、
    \texttt{nzhour}、\texttt{nzmin}、\texttt{nzsec}、\texttt{nzmsec} 决定
\item [相对时间] 即某个时刻相对于参考时刻的时间差(单位为秒),保存到
    头段变量 \texttt{b}、\texttt{e}、\texttt{o}、\texttt{a}、\texttt{f}、
    \texttt{tn}(n=0--9)
\item [绝对时刻] =参考时刻+相对时间
\end{description}

\subsection{一些测试}
下面以一个具体的数据为例,通过修改各种与时间相关的头段来试着去进一步
理解SAC的时间概念。

\subsubsection{生成样例数据}
\begin{SACCode}
SAC> fg seis
SAC> lh iztype
    iztype = BEGIN TIME
SAC> ch iztype IUNKN
SAC> w seis
\end{SACCode}
\texttt{lh} 是命令 \nameref{cmd:listhdr} 的简写,用于列出头段变量的值。
\texttt{ch} 是 \nameref{cmd:chnhdr} 的简写,用于修改头段变量的值。这里
额外多做了一个操作修改 \texttt{iztype} 的操作,这是由于这个数据稍稍有
一点bug。

\texttt{iztype} 指定了参考时刻的类型,其显示为 \texttt{BEGIN TIME},
实际上其枚举值是 \texttt{IB},也就是说这个数据选取文件第一个数据点的
时刻作为参考时刻,那么 \texttt{b} 的值应该为0。而实际上这个数据的
\texttt{b} 值并不为0,这其实是这个数据的一点小bug。这也从另一个侧面说明
SAC只有在修改与时间相关的头段变量时才可能会检查到这个错误/警告,所以这里
先将其修正为 \texttt{IUNKN}。

\subsubsection{修改文件起始时间b}
\begin{SACCode}
SAC> r seis
SAC> lh kzdate kztime b delta npts e o a f

     kzdate = MAR 29 (088), 1981
     kztime = 10:38:14.000
          b = 9.459999e+00
      delta = 1.000000e-02
       npts = 1000
          e = 1.945000e+01
          o = -4.143000e+01
          a = 1.046400e+01
SAC> ch b 10
SAC> lh

     kzdate = MAR 29 (088), 1981
     kztime = 10:38:14.000
          b = 1.000000e+01
      delta = 1.000000e-02
       npts = 1000
          e = 1.999000e+01
          o = -4.143000e+01
          a = 1.046400e+01
\end{SACCode}

修改 \texttt{b} 前后的变化仅在于 \texttt{b} 和 \texttt{e} 值的变化,而
参考时刻以及其它相对时间并没有发生变化。

这意味着整段SAC数据中的任意一个数据点所对应的时刻\footnote{好长的修饰语}
都向后延迟了0.54秒!这样做很危险,因为 \texttt{b} 和 \texttt{e} 的绝对
时刻被修改了,而其它头段如 \texttt{o}、\texttt{a}、\texttt{f}、
\texttt{tn} 的绝对时刻却没有变。

使用的时候必须非常小心:
\begin{itemize}
\item 如果 \texttt{o}、\texttt{a}、\texttt{f}、\texttt{tn} 都没有定义,
    那么修改 \texttt{b} 值可以用于校正仪器的时间零飘\footnote{零飘,即
    仪器中的时刻与标准时刻不同。}以及时区差异\footnote{时区差异可以理
    解成另一种零飘。}。关于时区的校正,参考``\nameref{sec:time-zone-correction}''
    一节。
\item 如果 \texttt{o}、\texttt{a}、\texttt{f}、\texttt{tn} 已经被定义,
    则修改 \texttt{b} 值会导致与震相相关的头段变量出现错误!\footnote{
    如果只定义了 \texttt{o} 值,或者 \texttt{a}、\texttt{f}、\texttt{tn}
    为理论震相到时而非计算机拾取或人工拾取的到时,修改 \texttt{b} 也是
    没有问题的。有些乱,不多说了。总之不要随便修改 \texttt{b} 的值。}
\end{itemize}

\subsubsection{修改文件结束时间e}
\begin{SACCode}
SAC> r ./seis
SAC> lh kzdate kztime b delta npts e o a f

     kzdate = MAR 29 (088), 1981
     kztime = 10:38:14.000
          b = 9.459999e+00
      delta = 1.000000e-02
       npts = 1000
          e = 1.945000e+01
          o = -4.143000e+01
          a = 1.046400e+01
SAC> ch e 0
SAC> lh

     kzdate = MAR 29 (088), 1981
     kztime = 10:38:14.000
          b = 9.459999e+00
      delta = 1.000000e-02
       npts = 1000
          e = 1.945000e+01
          o = -4.143000e+01
          a = 1.046400e+01
\end{SACCode}

可以看到,修改前后所有变量均没有发生变化,即 \texttt{e} 的值是不可以
随意改变的,根据上面的结果可知,\texttt{e} 的值是通过 \texttt{b}、
\texttt{delta}、\texttt{npts}的值动态计算的。这也与上一节说到的头段变量
冗余问题相符合。不要试图修改 \texttt{delta}、\texttt{npts},这不科学!

\subsubsection{修改o、a、f、tn}
这几个头段变量完全是由用户自定义的,因而任何的定义、修改、取消定义都
不会对数据的正确性产生影响,因而这里不再测试。

\subsubsection{修改参考时间}
\begin{SACCode}
SAC> r ./seis
SAC> lh kzdate kztime b delta npts e o a f

     kzdate = MAR 29 (088), 1981
     kztime = 10:38:14.000
          b = 9.459999e+00
      delta = 1.000000e-02
       npts = 1000
          e = 1.945000e+01
          o = -4.143000e+01
          a = 1.046400e+01
SAC> ch nzsec 15
SAC> lh

     kzdate = MAR 29 (088), 1981
     kztime = 10:38:15.000
          b = 9.459999e+00
      delta = 1.000000e-02
       npts = 1000
          e = 1.945000e+01
          o = -4.143000e+01
          a = 1.046400e+01
\end{SACCode}

试图修改参考时刻,整个SAC头段,除了参考时刻外其它时间变量都没有发生变化。
根据``绝对时刻=参考时刻+相对时间''可知,这导致所有SAC数据点的绝对时刻
发生了平移,这一点理论上可以用于校正零飘或者时区,但是由于SAC不支持智能
判断时间(比如不知道1时80分实际上是2时20分),所以修改时区时需要获取
参考时刻6个头段变量,加上时区的校正值,再写入到参考时刻6个变量中,相对
较为繁琐,因而若要校正时区,建议直接修改头段变量中的 \texttt{b} 值。

\subsubsection{修改发震时刻}
数据处理中一个常见的需求是修改发震时刻,这可以通过修改头段变量 \texttt{o}
来实现,但是经常需要将参考时刻设置为发震时刻。上面的测试表明,直接修改
参考时刻是很危险的,所以SAC的 \texttt{ch} 命令提供了 \texttt{allt} 选项来
实现这一功能,在``\nameref{sec:event-info}''一节中会具体解释。

\subsection{总结}
将SAC中的时间变量分为三类:
\begin{enumerate}
\item 参考时刻:即 \texttt{nzyear}、\texttt{nzjday}、\texttt{nzhour}、
    \texttt{nzmin}、\texttt{nzsec}、\texttt{nzmsec};
\item 相对时间:即\texttt{o}、\texttt{a}、\texttt{f}、\texttt{tn};
\item 特殊的相对时间:即\texttt{b}\footnote{由于e不可独立修改,所以不再考虑};
\end{enumerate}

第二类时间变量可以随意修改,即震相拾取。

第一、三类时间变量的修改会导致数据绝对时刻发生改变。一般通过修改第三类
时间变量来校正时间零漂和时区差异。在设置了发震时刻后,应使用
\nameref{cmd:chnhdr} 命令的 \texttt{allt} 选项修改第一、三类时间变量。
