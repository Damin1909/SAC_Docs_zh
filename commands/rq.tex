\SACCMD{rq}
\label{cmd:rq}

\SACTitle{概要}
从谱文件中去除Q因子

\SACTitle{语法}
\begin{SACSTX}
RQ [Q v] [R v] [C v]
\end{SACSTX}

\SACTitle{输入}
\begin{description}
\item [Q v] 设置质量因子为v
\item [R v] 设置距离为v,单位为 \si{\km}
\item [C v] 设置群速度为v,单位 \si{\km\per\s}
\end{description}

\SACTitle{缺省值}
\begin{SACDFT}
rq q 1. r 0. c 1.
\end{SACDFT}

\SACTitle{说明}
该命令用于从波形数据中去除Q衰减效应,用于校正振幅谱的方程如下:
\[ AMP_{corrected}(f) = AMP_{uncorrected}(f) * e^{\frac{\pi R f}{Q C}} \]
其中$f$为频率,单位为 \si{\Hz},$R$为距离,单位 \si{\km},
$C$是群速度,单位为 \si{\km\per\s}。Q是一个无量纲衰减因子。

\SACTitle{头段变量}
depmin、depmax、depmen

\SACTitle{限制}
实际上各参数应是频率的函数,目前限制为常数值。
