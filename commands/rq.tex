\SACCMD{rq}
\label{cmd:rq}

\SACTitle{概要}
从谱文件中去除Q因子

\SACTitle{语法}
\begin{SACSTX}
RQ [Q v] [R v] [C v]
\end{SACSTX}

\SACTitle{输入}
\begin{description}
\item [Q v] 设置质量因子为v 
\item [R v] 设置距离为v,单位为km 
\item [C v] 设置群速度为v,单位km/s 
\end{description}

\SACTitle{缺省值}
\begin{SACDFT}
rq q 1. r 0. c 1.
\end{SACDFT}

\SACTitle{说明}
该命令用于从波形数据中去除Q衰减效应,用于校正振幅谱的方程如下:
\[ AMP_{corrected}(f) = AMP_{uncorrected}(f) * e^{\frac{\pi R f}{Q C}} \]
其中$f$为频率,单位为$Hz$,$R$为距离,单位$km$,$C$是群速度,单位为$km/s$。
Q是一个无量纲衰减因子。

\SACTitle{头段变量}
depmin、depmax、depmen

\SACTitle{错误消息}
\begin{itemize}
\item[-]1301: 未读入数据文件
\item[-]1305: 对时间序列的非法操作
\end{itemize}

\SACTitle{警告消息}
\begin{itemize}
\item[-]1604: 对于实部-虚部格式的谱文件,rq会自动将其转换给实部-虚部格式,并给出警告消息。
\end{itemize}

\SACTitle{限制}
实际上各参数应是频率的函数,目前限制为常数值。
