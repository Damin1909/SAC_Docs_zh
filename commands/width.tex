\SACCMD{width}
\label{cmd:width}

\SACTitle{概要}
控制线宽

\SACTitle{语法}
\begin{SACSTX}
WIDTH [ON|OFF|width] [SK!ELETON! width] [I!NCREMENT! ON|OFF]
    [L!IST! S!TANDARD!|widthlist]
\end{SACSTX}
其中 \texttt{width} 只能取整数

\SACTitle{输入}
\begin{description}
\item [WIDTH width] 设置数据的线宽为 \texttt{width}
\item [WIDTH ON] 打开 \texttt{WIDTH} 选项但是不改变当前线宽值
\item [WIDTH OFF] 关闭 \texttt{WIDTH} 选项
\item [SKELETON width] 设置图形边框宽度为 \texttt{width}
\item [LIST STANDARD] 设置为标准线宽列表,设置数据宽度为标准列表中的
    第一个宽度,并打开 \texttt{WIDTH} 选项
\item [LIST widthlist] 改变宽度列表的内容。输入宽度列表。设置数据宽度
    为列表中的第一个宽度,并打开 \texttt{WIDTH} 选项
\item [INCREMENT ON] 按照 \texttt{widthlist} 表中的次序,依次改变一个宽度值
\item [INCREMENT OFF] 关闭线宽递增功能
\end{description}

\SACTitle{缺省值}
\begin{SACDFT}
width off skeleton 1 increment off list standard
\end{SACDFT}

\SACTitle{说明}
\texttt{width} 指定了绘制数据时的线条宽度。\texttt{SKELETON}指定了
坐标轴的宽度,其就仅修改坐标轴的宽度,网格、文本、标签和框架号总是
用1号细线表示。

若将 \texttt{WIDTH} 设置为递增,则每次绘图之后,宽度都会按照宽度表中的
顺序自动修改。

如果在同一张绘图中同时绘制几个数据文件,也许需要对每个文件使用不同的宽度。
此时可使用 \texttt{INCREMENT} 选项。在这个选项打开时,每次绘制一个数据
文件后,都按照宽度表中的次序自动地变成另一个宽度。宽度值和次序在标准宽度
表中为:
\begin{SACCode}
1, 2, 3, 4, 5, 6, 7, 8, 9, 10
\end{SACCode}
你可以使用 \texttt{LIST} 选项改变这个表的次序或内容。这个命令常用于
重叠绘图(参见 \nameref{cmd:plot2}),此时你可能需要每张图上的数据宽度
都按相同的顺序排列。

\SACTitle{示例}
选择自动变换的数据宽度起始值为1:
\begin{SACCode}
SAC> width 1 increment
\end{SACCode}

边框宽度起始值为2,并按1、3、5的增量变化:
\begin{SACCode}
SAC> width skeleton 2 increment list 1 3 5
\end{SACCode}
