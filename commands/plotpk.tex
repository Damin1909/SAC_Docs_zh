\SACCMD{plotpk}
\label{cmd:plotpk}

\SACTitle{概要}
产生一个用于拾取到时的图

\SACTitle{语法}
\begin{SACSTX}
P!LOT!PK [P!ERPLOT! ON|OFF|n] [BELL ON|OFF] [A!BSOLUTE!|R!ELATIVE!]
    [R!EFERENCE! ON|OFF|v] [MARKALL ON|OFF] [S!AVELOC! ON|OFF]
\end{SACSTX}

\SACTitle{输入}
\begin{description}
\item [PERPLOT ON] 一张图绘制n个文件,使用n的旧值
\item [PERPLOT OFF] 在一张图上绘制所有文件
\item [PERPLOT n] 一张图上绘制n个文件,比如拾取震相时常使用n=3(三分量)
\item [BELL ON|OFF]  击键时是否响铃
\item [ABSOLUTE] 显示绝对GMT格式时间
\item [RELATIVE] 显示相对于每个文件的参考时间
\item [REFERENCE ON] 打开参考线显示
\item [REFERENCE OFF] 关闭参考线选项
\item [REFERENCE v] 打开参考线选项并设置参考值为v,设置为0会很方便
\item [MARKALL ON] 同时储存一张图上的所有文件的时间标记,用于拾取震相
\item [MARKALL OFF] 只保存光标位置所在的那个文件的震相拾取标记
\item [SAVELOCS ON]  保存拾取的位置(由命令L得到,稍后介绍)到暂存块变量中
\item [SAVELOCS OFF]  不保存拾取位置到暂存块变量
\end{description}

\SACTitle{缺省值}
\begin{SACDFT}
plotpk perplot off absolute reference off markall off savelocs off
\end{SACDFT}

\SACTitle{说明}
这个命令的格式类似于PLOT1,这个绘图需要一个带有光标的终端。在该命令后将在屏幕上出现光标,输入单个字符即可执行不同的操作。有些但不是全部字符将在屏幕上产生图形输出。错误以及输出信息将在屏幕上部打印,当前在头段中的震相拾取将自动	显示在屏幕上。不同的光标响应的输出可以定向到SAC头段中,也可以定向到一个字符数字型震相拾取文件(OAPF),或者到一个HYPO震相拾取文件,或者到终端。如果你使用SAVELOCS选项保存由L光标选项得到的光标位置到暂存块变量,那么下面的黑板变量将被定义:
\begin{itemize}
\item NLOCS: 在命令执行期间拾取的光标位置数,每次执行PPK时初始化为0,每次拾取光标位置则加1
\item XLOCn: 拾取光标位置的x值,如果头段中的参考时间被定义其是GMT时间,否则其为偏移时间
\item YLOCn: 第n个拾取光标位置的y值
\end{itemize}

\SACTitle{相关命令}
\nameref{cmd:plot1}、\nameref{cmd:ohpf}、\nameref{cmd:oapf}、\nameref{cmd:apk}
