\SACCMD{rotate}
\label{cmd:rotate}

\SACTitle{概要}
将成对的正交分量旋转一个角度

\SACTitle{语法}
\begin{SACSTX}
ROT!ATE! [TO G!CP!|TO v|TH!ROUGH! v] [N!ORMAL!|R!EVERSED!]
\end{SACSTX}

\SACTitle{输入}
\begin{description}
\item [TO GCP] 旋转到大圆弧路径(``great circle path'')。两个分量必须都是水平分量且
    头段中必须定义台站和事件的经纬度
\item [TO v] 旋转一定角度使得第一个分量的方位角为v度。两个分量必须都是水平分量
\item [THROUGH v] 顺时针旋转v度。其中一个分量可以是垂直分量
\item [NORMAL|REVERSED] 输出分量为正/负极性
\end{description}

\SACTitle{缺省值}
\begin{SACDFT}
rotate to gcp normal
\end{SACDFT}

\SACTitle{说明}
此命令可以对多对分量旋转一定的角度,内存中的每两个文件为一对分量。
每对分量必须拥有相同的台站名、事件名
、采样率和数据点数,且头段变量cmpaz和cmpinc必须定义,程序会检查两个分量
是否正交(允许0.02度的偏差)。

``THROUGH''选项表示将一对正交分量旋转一定的角度。这对正交分量可以均为
水平分量(cmpinc=90)或包含一个垂直分量(cmpinc=0)。其中,水平面内的旋转
是相对于北向顺时针的角度;垂直面内的旋转是相对于垂直向上方向的角度。

``TO''选项表示将一对正交分量旋转\textbf{到}一定的角度(方位角),这对
正交分量必须都是水平分量(cmpinc=90)。
对于``TO v''而言,表示将一对分量旋转一定角度,
使得旋转后的第一个分量沿着方位角v的方向;对``TO GCP''而言,首先根据
台站和事件经纬度计算出反方位角,并将分量旋转一定角度,使得旋转后的
第一个分量沿着反方位角加/减180度的方向,此时第一个水平分量由事件位置指向
台站位置,即地震学中的径向(Radial)分量,第二个水平分量垂直于R分量,
即地震学的切向(Tangential)分量。

NORMAL和REVERSED用于指定输出分量的极性,仅用于一对水平分量的旋转中。
在rotate命令中,就一对水平分量而言,若第二个分量比第一个分量超前90度
(可以理解为方位角大90度)则称为正极性;若第二个分量比第一个分量落后90度
则称为负极性。对于一对输入分量而言,无论是N分量在前还是E分量在前均可,
该命令会自动判断一对输入分量是正极性还是负极性,并对旋转公式进行调整,
NORMAL和REVERSED仅用于控制一对输出分量的极性。

\SACTitle{示例}
将一对水平分量旋转30度:
\begin{SACCode}
SAC> dg sub tele ntkl.[ne]          // 内存中的顺序是E分量先于N分量
SAC> lh cmpinc cmpaz

  FILE: /opt/sac/aux/datagen/teleseis/ntkl.e - 1
 ------------------------------------------

     cmpinc = 9.000000e+01
      cmpaz = 9.000000e+01

  FILE: /opt/sac/aux/datagen/teleseis/ntkl.n - 2
 ------------------------------------------

     cmpinc = 9.000000e+01
      cmpaz = 0.000000e+00
SAC> rot through 30                 // 顺时针旋转30度
SAC> lh

  FILE: /opt/sac/aux/datagen/teleseis/ntkl.e - 1
 ------------------------------------------

     cmpinc = 9.000000e+01
      cmpaz = 1.200000e+02

  FILE: /opt/sac/aux/datagen/teleseis/ntkl.n - 2
 ------------------------------------------

     cmpinc = 9.000000e+01
      cmpaz = 3.000000e+01
\end{SACCode}

旋转两对水平分量到大圆弧路径:
\begin{SACCode}
SAC> read abc.n abc.e def.n def.e
SAC> rotate to gcp
SAC> w abc.r abc.t def.r def.t
\end{SACCode}
上面的例子中若头段变量baz为33度,则径向分量指向213度,切向分量指向303度,如果设置反极性,切向分量指向123度。

\SACTitle{头段变量}
cmpaz、cmpinc
