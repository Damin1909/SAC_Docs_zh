\SACCMD{xvport}
\label{cmd:xvport}

\SACTitle{概要}
定义X轴的视口

\SACTitle{语法}
\begin{SACSTX}
XVP!ORT! xvmin xvmax
\end{SACSTX}

\SACTitle{输入}
\begin{description}
\item [xvmin] X轴视口的最小值,范围为0.0到xvmax
\item [xvmax] X轴视口的最大值,范围为xmin到1.0
\end{description}

\SACTitle{缺省值}
\begin{SACDFT}
xvport 0.1 0.9
\end{SACDFT}

\SACTitle{说明}
视口(viewspace)是实际绘制的视窗的一部分。用于定义视口和视窗的坐标系称为虚拟坐标系。
虚拟坐标系不依赖于特定物理设备显示面的尺寸、形状或分辨率。SAC的坐标系在x和y方向都是0到1的范围。
视窗左下角的坐标为 \texttt{(0.0, 0.0)},右上角的坐标为 \texttt{(1.0, 1.0)}。
这个坐标系的使用便于你指定一个图形的位置,而不必考虑特定的输出设备。

XVPORT和YVPORT命令控制在视窗中哪个位置上绘制指定的图形。默认值使用了视窗的部分,
在图形的每边留下一些空间绘制坐标轴、标签和标题。你可以使用这个命令把一个给定的图形安
排在任何位置。当与BEGINFRAME和ENDFRAME命令一起使用,这些命令让你创建特殊的构图以在相同的
框架中放置若干不同的图形。

\SACTitle{示例}
参见 \nameref{cmd:beginframe}
