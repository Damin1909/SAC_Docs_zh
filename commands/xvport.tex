\SACCMD{xvport}
\label{cmd:xvport}

\SACTitle{概要}
定义X轴的视口

\SACTitle{语法}
\begin{SACSTX}
XVP!ORT! xvmin xvmax
\end{SACSTX}

\SACTitle{输入}
\begin{description}
\item [xvmin] X轴视口的最小值,范围为 \texttt{0.0} 到 \texttt{xvmax}
\item [xvmax] X轴视口的最大值,范围为 \texttt{xmin} 到 \texttt{1.0}
\end{description}

\SACTitle{缺省值}
\begin{SACDFT}
xvport 0.1 0.9
\end{SACDFT}

\SACTitle{说明}
视口(viewport)是视窗(viewspace)的一部分。在SAC中,定义视口时使用的是
相对坐标系,即坐标系在X和Y方向都是0到1的范围。视窗左下角的坐标为
\texttt{(0.0, 0.0)},右上角的坐标为 \texttt{(1.0, 1.0)}。该坐标系是与
输出设备无关的,因而可以很方便地指定一个图形的位置。

\texttt{xvport} 和 \texttt{yvport} 定义了视口相对于视窗的位置,后续的
命令将在定义的视口中绘图。默认值 \texttt{xvport 0.1 0.9} 在X方向上使用了
视窗的80\%,在图形的左右两边留下一些空间绘制坐标轴、标签和标题。

当与 \nameref{cmd:beginframe} 和 \nameref{cmd:endframe} 命令一起使用时,
可以让你能够在一个视窗内绘制若干不同的图形,构成复杂的组合图。

\SACTitle{示例}
参见 \nameref{sec:composite-plots} 一节。
