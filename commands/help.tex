\SACCMD{help}
\label{cmd:help}

\SACTitle{概要}
在终端显示SAC命令的语法和功能信息

\SACTitle{语法}
\begin{SACSTX}
H!ELP! [item ... ]
\end{SACSTX}

\SACTitle{输入}
\begin{description}
\item [item] 命令(全称或简写)、模块、子程序等等。若item为空,则显示SAC的帮助文档的介绍
\end{description}

\SACTitle{说明}
SAC的帮助文档位于 \verb|$SACHOME/aux/help| 中,该命令的实质是读取帮助文档并显示
到终端中。item列表中每一项会按照顺序依次显示在终端中,若输出超过一屏,可以使用PgUp、
PgDn、Enter、空格、方向键等实现翻页。直接输入``q''则退出当前项文档并显示下一项文档。

\SACTitle{示例}
\begin{SACCode}
SAC> h                  //获得帮助文档包的介绍
SAC> h r cut bd p       //一次获取多个命令信息
\end{SACCode}

\SACTitle{错误信息}
\begin{itemize}
\item[-]1103: 没有可用的帮助文档包(检测 \verb|$SACHOME/aux/help| 文件夹)
\end{itemize}
