\SACCMD{beginframe}
\label{cmd:beginframe}

\SACTitle{概要}
打开frame,用于绘制组合图

\SACTitle{语法}
\begin{SACSTX}
B!EGIN!F!RAME! [PRINT [pname]]
\end{SACSTX}

\SACTitle{输入}
\begin{description}
\item [PRINT pname] 当在BEGINFRAME中使用PRINT时,SAC将把图形打印到名为
    pname的打印机,如果pname没有指定则打印到默认的打印机
\end{description}

\SACTitle{说明}
一般情况下,在每次使用绘图命令时,SAC会对绘图设备执行刷新操作,以清除
上一个绘图命令绘制的图像,然后再显示本次命令绘制的图像,这样可以保证
每次绘图命令绘制的图像不会重叠在一起。

beginframe命令会关闭绘图设备的自动刷新功能,直到 \nameref{cmd:endframe}
命令恢复自动刷新功能为止。在这两个命令中间执行的所有绘图命令所产生的图像
将会叠加在一起,形成组合图。

通过这两个命令,并结合xvport和yvport定义每次绘图的viewport,可以很容易地绘制出
复杂的组合图。

关于如何绘制组合图以及这几个命令的使用,可以参考 ``\nameref{sec:composite-plots}''一节。
