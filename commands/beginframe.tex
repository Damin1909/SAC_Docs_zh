\SACCMD{beginframe}
\label{cmd:beginframe}

\SACTitle{概要}
打开frame,用于绘制组合图

\SACTitle{语法}
\begin{SACSTX}
B!EGIN!F!RAME! [PRINT [pname]]
\end{SACSTX}

\SACTitle{输入}
\begin{description}
\item [PRINT pname] 当在BEGINFRAME中使用PRINT时,SAC将把图形打印到名为pname的打印机,
    如果pname没有指定则打印到默认的打印机
\end{description}

\SACTitle{说明}
一般情况下,在每次使用绘图命令时,SAC会对绘图设备执行刷新操作,以清除上一个绘图
命令绘制的图像,然后再显示本次命令绘制的图像,这样可以保证每次绘图命令绘制的图像不会重叠在一起。

beginframe命令会关闭绘图设备的自动刷新功能,直到endframe命令恢复自动刷新功能为止。
在这两个命令中间执行的所有绘图命令所产生的图像将会叠加在一起,形成组合图。

通过这两个命令,并结合xvport和yvport定义每次绘图的viewport,可以很容易地绘制出
复杂的组合图。

关于如何绘制组合图以及这几个命令的使用,可以参考 ``\nameref{sec:composite-plots}''一节。

\SACTitle{相关命令}
\nameref{cmd:endframe}、\nameref{cmd:xvport}、\nameref{cmd:yvport}
