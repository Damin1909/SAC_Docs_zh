\SACCMD{grayscale}
\label{cmd:grayscale}

\SACTitle{概要}
产生内存中数据的灰度图像\footnote{这个命令使用了未在SAC中发布的命令,
    要使用这个命令你必须安装Utah Raster Toolkit。}

\SACTitle{语法}
\begin{SACSTX}
G!RAY!S!CALE! [VIDEOTYPE NORMAL|REVERSED] [SCALE v] [ZOOM n]
    [XCROP n1 n2|ON|OFF] [YCROP n1 n2|ON|OFF]
\end{SACSTX}

\SACTitle{输入}
\begin{description}
\item [VIDEO NORMAL] 设置video类型为 !normal!。在Normal模式中,
    最小值附近的数据位黑色,最大值附近的数据为白色
\item [VIDEO REVERSED] 设置video类型为 !reversed!。在Reversed模式
    中,最小值附近的数据位白色,最大值附近的数据为黑色
\item [SCALE v] 改变数据的比例因子为v,The data is scaled by raising it
    to the vth power。小于1的值将平滑图像、降低峰和谷,大于1的值将伸展
    整个数据
\item [ZOOM n] Image is increased to n times its normal size by pixel replication.
\item [XCROP n1 n2] Turn x cropping option on and change cropping limits
    to n1 and n2. The limits are in terms of the image size.
\item [XCROP ON] Turn x cropping option on and use previously specified cropping limits.
\item [XCROP OFF] Turn x cropping option off.  All of the data in the x direction is displayed.
\item [YCROP n1 n2] Turn y cropping option on and change cropping limits to n1 and n2. The limits are in terms of the image size.
\item [YCROP ON] Turn y cropping option on and use previous specified cropping limits.
\item [YCROP OFF] Turn y cropping option off.  All of the data in the y direction is displayed.
\end{description}

\SACTitle{缺省值}
\begin{SACDFT}
grayscale videotype normal scale 1.0 zoom 1 xcrop off ycrop off
\end{SACDFT}

\SACTitle{说明}
这个命令可以用于绘制 \nameref{cmd:spectrogram} 命令输出的灰度图,用这个
命令显示的SAC数据须是``xyz''文件。

注意:SAC启动了一个脚本来运行图像操作和显示程序,然后再显示SAC的提示符。
对于大型图像或较慢的机器,这中间会有个明显的延迟。

\SACTitle{限制}
最大只能显示512*1000

\SACTitle{头段变量改变}
需要:iftype、nxsize、nysize

\SACTitle{错误消息}
SAC> getsun: Command not found.  (需要Utah Raster Toolkit提供一些工具程序)
