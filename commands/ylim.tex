\SACCMD{ylim}
\label{cmd:ylim}

\SACTitle{概要}
设定图形Y轴的范围

\SACTitle{语法}
\begin{SACSTX}
YLIM [ON|OFF|ALL|min max|PM v]
\end{SACSTX}

\SACTitle{输入}
\begin{description}
\item [ALL] 根据内存中所有文件的最大和最小值限定Y轴的范围
\item [min max] 设定Y轴的范围为 !min! 到 !max! 之间
\item [PM v] 设置Y轴的范围为 !-v! 到 !+v! 之间
\item [ON] 打开Y轴范围选项,但不改变范围值
\item [OFF] 关闭Y轴范围选项
\end{description}

\SACTitle{缺省值}
\begin{SACDFT}
ylim off
\end{SACDFT}

\SACTitle{说明}
当内存中有多个数据文件需要绘制时,若关闭此选项,每个数据文件在绘图时都会
根据自己的数据因变量的范围决定绘图时Y轴的范围。当此选项打开时,将限制
所有绘图的Y轴的范围。!ALL! 选项会找到内存中所有数据文件的最大值
和最小值作为所有绘图的Y轴的范围。

可以为内存中不同的文件设定不同的Y轴范围,只要在命令中多次使用这些选项
即可。

\SACTitle{示例}
file1的Y轴范围为0.0到30.0,file2的y轴范围为内存中所有文件的最大、最小值,
file3的y轴范围将限定为文件自身的最大、最小值。如果文件多于三个,则其余的
所有文件都限定为文件自身的最大、最小值。
\begin{SACCode}
SAC> ylim 0.0 30.0 all off
SAC> r file1 file2 file3
SAC> p
\end{SACCode}
