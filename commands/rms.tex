\SACCMD{rms}
\label{cmd:rms}

\SACTitle{概要}
计算测量时间窗的信号的均方根

\SACTitle{语法}
\begin{SACSTX}
RMS [NOISE ON|OFF|pdw] [TO USERn]
\end{SACSTX}

\SACTitle{输入}
\begin{description}
\item [NOISE ON] 打开噪声归一化选项
\item [NOISE OFF] 关闭噪声归一化选项
\item [NOISE pdw] 打开噪声归一化选项并设置噪声的测量时间窗 \nameref{subsec:pdw}
\item [TO USERn] 将测量结果存储到头段变量 \texttt{USERn} 中
\end{description}

\SACTitle{缺省值}
\begin{SACDFT}
rms noise off to user0
\end{SACDFT}

\SACTitle{说明}
该命令用于计算当前测量时间窗(由 \nameref{cmd:mtw} 定义)内的数据的
均方根,并将测量结果写入头段变量 \texttt{USERn} 中。

均方根的定义为:
\[
 RMS = \sqrt{\frac{1}{N} \sum_{i=1}^n y_i^2}
\]

\texttt{NOISE} 选项定义了噪声的测量时间窗,用于计算噪声的均方根,以对
信号的均方根做修正。若使用该选项,则会首先计算 \texttt{mtw} 定义的时间
窗内的数据点的平方和的均值,然后再计算 \texttt{NOISE pdw} 定义的时间窗
内数据点的平方和的均值,最后将信号的平方和均值减去噪声的平方和均值,
并取其平方根作为测量结果。

\SACTitle{示例}
为了计算两个头段 \texttt{T1} 和 \texttt{T2} 间的数据的未修正的均方根,
并将结果保存在头段 \texttt{USER4} 中:
\begin{SACCode}
SAC> mtw t1 t2
SAC> rms to user4
\end{SACCode}

将 \texttt{T3} 前5秒作为噪声窗,计算修正后的均方根:
\begin{SACCode}
SAC> mtw t1 t2
SAC> rms noise t3 -5.0 0.0
\end{SACCode}

\SACTitle{头段变量}
USERn
