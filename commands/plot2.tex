\SACCMD{plot2}
\label{cmd:plot2}

\SACTitle{概要}
产生一个多波形单窗口绘图

\SACTitle{语法}
\begin{SACSTX}
P!LOT!2 [A!BSOLUTE!|R!ELATIVE!]
\end{SACSTX}

\SACTitle{输入}
\begin{description}
\item [ABSOLUTE] 所有文件时间为绝对时间,具有不同的文件开始时间的文件将
    会相对于第一个文件移动
\item [RELATIVE] 每个文件相对于自己的开始时间绘图
\end{description}

\SACTitle{缺省值}
\begin{SACDFT}
p2 absolute
\end{SACDFT}

\SACTitle{说明}
所有文件列表中的文件都将绘制在同一个绘图窗口中。可以选择绘制一个包含绘图
符号以及文件名的图例以区别不同的文件,在使用这个命令之前X和Y的范围可以被
定义。参见 \nameref{cmd:xlim} 和 \nameref{cmd:ylim} 命令。如果没有给出轴
的范围那么将根据文件列表中的极值确定轴范围。图例的位置可以通过 \nameref{cmd:fileid}
命令控制。不像 \nameref{cmd:plot} 和 \nameref{cmd:plot1},\nameref{cmd:plot2}
可以绘制谱文件。实部-虚部数据被绘制为实部-频率。振幅/相位数据被绘制为
振幅-频率。虚部和相位信息忽略。谱文件总是用相对模式绘图。注意到在频率域
\texttt{b}、\texttt{e} 和 \texttt{delta} 分别被设置为0,Nqquist频率以及
频率间隔 \texttt{df}。头段值 \texttt{depmin} 和 \texttt{depmax} 未改变。
如同 \nameref{cmd:plotsp},如果 \nameref{cmd:xlim} 关闭,绘图将从
\texttt{df=delta} 处开始,而非0。如果 \nameref{cmd:xlim} 或 \nameref{cmd:ylim}
在数据变换到频率域之前被改变了,最好在使用 \texttt{plot2}绘图之前输入
\texttt{xlim off} 和 \texttt{ylim off}。

注意:由于某些原因,可能在内存中同时存在时间序列文件和谱文件并且没有选择
相对绘图选项,则时间序列将以绝对模式绘图,谱文件将以相对模式绘图。相对
模式意味着相对于第一个文件。因而内存中文件的顺序将影响绘图之间的关系。

\SACTitle{示例}
\begin{SACCode}
SAC> read mnv.z.am knb.z.am elk.z.am
SAC> xlim 0.04 0.16
SAC> ylim 0.0001 0.006
SAC> linlog
SAC> symbol 2 increment
SAC> title 'rayleigh wave amplitude spectra for nessel'
SAC> xlabel 'frequency (hz)'
SAC> plot2
SAC> fft
SAC> xlim off ylim off
SAC> line increment list 1 3
SAC> plot2 print
\end{SACCode}
