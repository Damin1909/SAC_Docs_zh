\SACCMD{plotdy}
\label{cmd:plotdy}

\SACTitle{概要}
绘制一个带有误差棒的图

\SACTitle{语法}
\begin{SACSTX}
PLOTDY [ASPECT ON|OFF] [PRINT pname] name|number [name|number]
\end{SACSTX}

\SACTitle{输入}
\begin{description}
\item [ASPECT ON|OFF] ON表示保持3/4的纵横比。OFF允许纵横比随着窗口维度的变换而变换
\item [name] 数据文件列表中数据名
\item [number] 数据文件列表中的数据号
\end{description}

\SACTitle{说明}
这个命令允许你绘制一个带有误差棒的数据集。你选择的第一个数据文件(通过名字或文件号指定)将沿着y轴绘制第二个数据文件是dy值,如果选择了第三个数据文件则其为正的dy值

\SACTitle{示例}
假定你有一个等间距的ASCII文件,其包含了两列数据。第一列是y值,第二列是
dy值,你可以像下面那样读入SAC并用数据绘制误差棒:
\begin{SACCode}
SAC> readtable content yy myfile
SAC> plotdy 1 2
\end{SACCode}
