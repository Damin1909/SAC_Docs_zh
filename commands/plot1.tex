\SACCMD{plot1}
\label{cmd:plot1}

\SACTitle{概要}
绘制多波形多窗口图形

\SACTitle{语法}
\begin{SACSTX}
P!LOT!1 [A!BSOLUTE!|R!ELATIVE!] [P!ERPLOT! ON|OFF|n]
\end{SACSTX}

\SACTitle{输入}
\begin{description}
\item [ABSOLUTE] 图形文件的时间为绝对时间。具有不同开始时间的文件将做相对移动。
\item [RELATIVE] 所有的文件绘制时相对于自己的开始时间
\item [PERPLOT ON] 每张图绘制几个文件,使用n的旧值
\item [PERPLOT OFF] 所有文件绘制在一张图上
\item [PERPLOT n] 每张图上绘制n个文件
\end{description}

\SACTitle{缺省值}
\begin{SACDFT}
plot1 absolute perplot off
\end{SACDFT}

\SACTitle{说明}
每个数据文件共享一个共同的X轴,但是各自拥有一个单独的Y轴。绘图的总尺寸
由由当前视口决定(参见 \nameref{cmd:xvport} 和 \nameref{cmd:yvport})。
每一个子图的大小这个视口的大小以及当绘图的文件数目决定。每个子图的Y轴
范围由文件极值决定或者可以通过 \nameref{cmd:ylim} 命令自己设置。X轴的
范围可以是固定的(参见 \nameref{cmd:xlim})也可以是与数据成比例的。对于
这种类型的绘图X轴标度有两种类型:relative和absolute。在绝对标度中X轴的
范围为绘图文件的自变量的最小值和最大值。在不同子图上两点之间的时间差
也按照起始时间的不同而得到校正。在相对标度中,X轴将从0到文件中的最大
时间差(即取所有文件中开始时间和结束时间差最大的那个时间差以保证每个文件
都可以被完整地绘制,尽管这样会导致某些图有很多的无数据区域),每个文件
都将从图形的左边界即X轴的零点开始绘制,对每个文件这个零点所对应的真实
值是不同的,其将在文件名下方给出。这种标度类型在你相对某个时间标记(比如
初动到时)裁剪文件时很有用。这样使得很容易看到每个文件波形之间的相似或差别。

\SACTitle{示例}
下面的例子是由LLNL DSS的4个台站Elko、Kanab、Landers和Mina记录到的美国西部的一个地震。
参考时间被设置为事件发生时刻kzdate和kztime:
\begin{SACCode}
SAC> cut -5 200
SAC> read *v
 elk.v knb.v lac.v mnv.v
SAC> fileid location ul type list kstcmp
SAC> title 'regional earthquake:  :1,kztime:  :1,kzdate:'
SAC> qdp 2000
SAC> plot1
\end{SACCode}
