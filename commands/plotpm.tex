\SACCMD{plotpm}
\label{cmd:plotpm}

\SACTitle{概要}
针对一对数据文件产生一个``质点运动''图

\SACTitle{语法}
\begin{SACSTX}
P!LOT!PM
\end{SACSTX}

\SACTitle{说明}
在一个质点运动图中,一个等间距文件相对于另一个等间距文件绘图。对于没有
自变量的值,第一个文件的因变量的值将沿着y轴绘制,第二个文件的因变量值
沿着x轴绘制。对于一对地震图来说,这种图展示了一个质点在两个地震图所确定
的平面内随时间的运动。它将产生一个方形图,每个轴的范围为因变量的极大
极小值。注释轴沿着底部和左边绘制。轴标签以及标题可以通过 \nameref{cmd:xlabel}、
\nameref{cmd:ylabel} 和 \nameref{cmd:title} 命令设置。如果未设置x、y标签,
则台站名和方位角将用作轴标签。\nameref{cmd:xlim} 可以用于控制文件的
哪些部分需要被绘制。

\SACTitle{示例}
创建一个两个地震图的质点运动图:
\begin{SACCode}
SAC> read xyz.t xyz.r
SAC> xlabel 'radial component'
SAC> ylabel 'transverse component'
SAC> title 'particle-motion plot for station xyz'
SAC> plotpm
\end{SACCode}

如果你想要值绘制每个文件在初动附近的一部分,你可以使用xlim命令:
\begin{SACCode}
SAC> xlim a -0.2 2.0
SAC> PLOTPM
\end{SACCode}

你也可以使用plotpk在之前的绘图窗口中设置新的区域,然后绘制命令如下:
\begin{SACCode}
SAC> beginwindow 2
SAC> plotpk
 ... mark the portion you want using X and S
 ... terminate PLOTPK with a Q
SAC> beginwindow 1
SAC> plotpm
\end{SACCode}
