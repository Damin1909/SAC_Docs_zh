\SACCMD{markvalue}
\label{cmd:markvalue}

\SACTitle{概要}
在数据文件中搜索并标记某个值

\SACTitle{语法}
\begin{SACSTX}
MARKV!ALUE! [GE|LE v] [TO marker]
\end{SACSTX}

\SACTitle{输入}
\begin{description}
\item [GE v] 搜索并标记第一个大于或等于 !v! 的数据点
\item [LE v] 搜索并标记第一个小于或等于 !v! 的数据点
\item [TO marker] 用于保存数据点的时刻的时间标记头段,!marker!
    可以取 !Tn!(n=0--9)
\end{description}

\SACTitle{缺省值}
\begin{SACDFT}
markvalue ge 1 to t0
\end{SACDFT}

\SACTitle{说明}
该命令会在信号的测量时间窗内搜索第一个满足条件(大于或等于/小于或等于)的
数据点,并将该数据点所对应的时刻记录下来。默认情况下,测量时间窗为整个信号,
可以使用 \nameref{cmd:mtw} 命令设置新的测量时间窗。

\SACTitle{示例}
搜索文件中第一个值大于3.4的点并将结果保存在头段 !T7! 中:
\begin{SACCode}
SAC> markvalue ge 3.4 to t7
\end{SACCode}

设定测量时间窗为 !T4! 后的10秒,并搜索第一个小于-3的值:
\begin{SACCode}
SAC> mtw t4 0 10
SAC> markvalue le -3 to t5
\end{SACCode}

\SACTitle{头段变量改变}
Tn、KTn
