\SACCMD{plotalpha}
\label{cmd:plotalpha}

\SACTitle{概要}
从磁盘读入字符数据型文件到内存并将数据绘制出来

\SACTitle{语法}
\begin{SACSTX}
P!LOT!A!LPHA! [MORE] [DIR CURRENT|name] [FREE|FORMAT text] [CONTENT text] [filelist]
\end{SACSTX}

\SACTitle{输入}
\begin{description}
\item [MORE] 将新读入的文件加到内存中老文件之后。如果没有这个选项,新文件
    将代替内存中的老文件。参见 \nameref{cmd:read} 命令
\item [DIR CURRENT] 从当前目录读取并绘制所有文件
\item [DIR name] 从文件夹name中读取并绘制所有文件,其可以是相对或绝对路径
\item [FREE] 以自由格式(以空格分隔数据各字段)读取并绘制filelist中的数据文件
\item [FORMAT text] 以固定格式读取并绘制filelist中的数据文件,格式声明
    位于 !text! 中
\item [CONTENT text] 定义filelist中数据每个字段的含义。!text! 的
    含义参见 \nameref{cmd:readtable} 命令中的
\item [filelist] 字符数字型文件列表,其可以包含简单文件名、绝对/相对路径、
    通配符。
\end{description}

\SACTitle{缺省值}
\begin{SACDFT}
plotalpha free content y. dir current
\end{SACDFT}

\SACTitle{说明}
参考 \nameref{cmd:readtable} 命令的相关说明。该命令与 !readtable!
之后再 \nameref{cmd:plot} 不同,因为它允许你在每个数据点上绘制标签。

\SACTitle{示例}
读取并绘制一个自由格式的X-Y数据,且其第一个字段是标签:
\begin{SACCode}
SAC> plotalpha content lp filea
\end{SACCode}
