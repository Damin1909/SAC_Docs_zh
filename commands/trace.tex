\SACCMD{trace}
\label{cmd:trace}

\SACTitle{概要}
追踪黑板变量和头段变量

\SACTitle{语法}
\begin{SACSTX}
TRACE [ON|OFF] name [name ...]
\end{SACSTX}

\SACTitle{输入}
\begin{description}
\item [ON] 打开变量追踪选项
\item [OFF] 关闭变量追踪选项
\item [name] 要追踪的黑板变量或/和头段变量名。对于头段变量,其格式为\verb+filename,hdrname+,
    其中filename是要追踪的SAC文件名或文件号,hdrname是SAC头段变量名。
\end{description}

\SACTitle{缺省值}
\begin{SACDFT}
trace on
\end{SACDFT}

\SACTitle{说明}
该命令用于在SAC执行过程中追踪SAC黑板变量或/和头段变量的值,主要用于调试大型或复杂的宏文件。
当变量追踪选项被打开时,将显示变量的当前值。若变量追踪选项处于打开状态,则每次执行命令时将
对变量值进行检查,若变量的值发生改变则将其新值打印到终端。当变量追踪选项被关闭时,也会显示
变量的当前值。

\SACTitle{示例}
追踪黑板变量TEMP1和文件MYFILE的头段变量T0:
\begin{SACCode}
SAC> trace on temp1 myfile,t0
  TRACE  (on) TEMP1 = 1.45623
  TRACE  (on) MYFILE,T0 = UNDEFINED
\end{SACCode}

在执行命令时,SAC会检查变量值是否发生改变。若发生改变则将相关信息显示出来。假设在完成一些计算之后
改变了TEMP1,并定义了T0的值,则SAC将显示如下信息:
\begin{SACCode}
  TRACE (mod) TEMP1 = 2.34293
  TRACE (mod) MYFILE,T0 = 10.3451
\end{SACCode}

稍后的处理中TEMP1可能再次改变:
\begin{SACCode}
  TRACE (mod) TEMP1 = 1.93242
\end{SACCode}

当跟踪选项被关闭时,SAC最后一次显示变量当前值:
\begin{SACCode}
SAC> trace off temp1 myfile,t0
  TRACE (off) TEMP1 = 1.93242
  TRACE (off) MYFILE,T0 = 10.3451
\end{SACCode}
