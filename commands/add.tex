\SACCMD{add}
\label{cmd:add}

\SACTitle{概要}
为每个数据点加上同一个常数

\SACTitle{语法}
\begin{SACSTX}
add [v1 [v2 ... vn]]
\end{SACSTX}

\SACTitle{输入}
\begin{description}
\item [v1]  加到第一个文件的常数
\item [v2]  加到第二个文件的常数
\item [vn]  加到第n个文件的常数
\end{description}

\SACTitle{缺省值}
\begin{SACDFT}
add 0.0
\end{SACDFT}

\SACTitle{说明}
此命令给内存块中数据文件的每个数据点都加上同一个常数。

对于每个数据文件,这个常数可以是相同的也可以是不同的。
若内存块中数据文件的个数比命令中常数的个数要多,则余下的所有数据文件
将都加上命令的最后一个常数值;若内存块中数据文件的个数比给出的常数个数少,
则多余的常数被忽略。

\SACTitle{示例}
为了给文件f1的每个数据点加上常数5.1,f2和f3的每个数据点加上常数6.2:
\begin{SACCode}
SAC> r f1 f2 f3         // 三个文件
SAC> add 5.1 6.2        // 两个常数
\end{SACCode}

\SACTitle{头段变量}
depmin、depmax、depmen
