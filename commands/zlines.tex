\SACCMD{zlines}
\label{cmd:zlines}

\SACTitle{概要}
控制后续等值线绘图上的等值线线型

\SACTitle{语法}
\begin{SACSTX}
ZLINES  [ON|OFF] [LIST n1 n2 ... nn] [REGIONS v1 v2 ... vn]
\end{SACSTX}

\SACTitle{输入}
\begin{description}
\item [ON|OFF] 打开/关闭等值线显示选项
\item [LIST n1 n2 .. nn] 设置要使用的线型表,这个表上的每个输入用于
    相应的等值线。如果等值线的数目大于这个表中给出的线型的数目,则
    使用整个线型表
\item [REGIONS v1 v2 .. vn] 设置等值线范围表。这个表的长度应小于线型表的
    长度,小于范围值的等值线使用线型表中相应的线型。超过最后一个范围值的
    等值线采用线型表中最后一个线型的值
\end{description}

\SACTitle{缺省值}
\begin{SACDFT}
zlines on list 1
\end{SACDFT}

\SACTitle{示例}
循环四种不同线型,建立等值线:
\begin{SACCode}
SAC> zlines list 1 2 3 4
\end{SACCode}

设置虚线表示低于0.0等值线,实线表示高于0.0的等值线:
\begin{SACCode}
SAC> zlines list 2 1 regions 0.0
\end{SACCode}
