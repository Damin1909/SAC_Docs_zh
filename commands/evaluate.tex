\SACCMD{evaluate}
\label{cmd:evaluate}

\SACTitle{概要}
对简单算术表达式求值

\SACTitle{语法}
\begin{SACSTX}
EVAL!UATE! [TO TERM|name] [v] op v [op v ...]
\end{SACSTX}
其中 !op! 可以取下面中的一个:
\begin{SACSTX}
ADD|SUBTRACT|MULTIPLY|DIVIDE|POWER|SQRT|EXP|ALOG|ALOG10|
SIN|ASIN|COS|ACOS|TAN|ATAN|EQ|NE|LE|GE|LT|GT
\end{SACSTX}
其中, !ADD|SUBTRACT|MULTIPLY|DIVIDE|POWER! 可以分别用
!+!、!-!、!*!、!/!、!**! 替代。

\SACTitle{输入}
\begin{description}
\item [TO TERM] 结果写入终端
\item [TO name] 结果写入黑板变量name
\item [v] 浮点数或整数。SAC中所有的运算都是浮点运算,整数会首先转换为浮点型
\item [op] 算术或逻辑操作符
\end{description}

\SACTitle{缺省值}
\begin{SACDFT}
evaluate to term 1. * 1.
\end{SACDFT}

\SACTitle{说明}
这个命令允许你对算术或逻辑表达式求值。算术表达式可以是包含多个操作符的
复合表达式,在这种情况下表达式由左向右计算,不支持嵌套功能。逻辑表达式
只能包含一个操作符。计算结果可以写入用户终端或者指定的黑板变量。你可以
通过 \nameref{cmd:getbb} 命令使用该黑板变量的值。

\SACTitle{示例}
一个简单的例子:
\begin{SACCode}
SAC> eval 2*3
 6
SAC> eval tan 45
1.61978
\end{SACCode}

下面将一个以度为单位的角度转换为弧度并计算其正切值:
\begin{SACCode}
SAC> eval 45*pi/180
 0.785398
SAC> eval tan 0.785398
 1
\end{SACCode}

下面将计算的结果保存到黑板变量:
\begin{SACCode}
SAC> evaluate to temp1 45*pi/180
SAC> evaluate tan %temp1%
 1
\end{SACCode}
