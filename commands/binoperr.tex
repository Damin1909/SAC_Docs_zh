\SACCMD{binoperr}
\label{cmd:binoperr}

\SACTitle{概要}
控制二元操作addf、subf、mulf、divf中的错误

\SACTitle{语法}
\begin{SACSTX}
BINOPERR [N!PTS! F!ATAL!|W!ARNING!|I!GNORE!] [D!ELTA! F!ATAL!|W!ARNING!|I!GNORE!]
\end{SACSTX}

\SACTitle{输入}
\begin{description}
\item [NPTS]  改变数据点数不相等的错误条件
\item [DELTA] 改变采样周期不相等的错误条件
\item [FATAL] 设置错误条件为``致命''
\item [WARNING] 设置错误条件为``警告''
\item [IGNORE]  设置错误条件为``忽略''
\end{description}

\SACTitle{缺省值}
\begin{SACDFT}
binoperr npts fatal delta fatal
\end{SACDFT}

\SACTitle{说明}
对文件执行二元操作(addf、divf等)时,SAC会检测两个文件的数据点数和采样周期是否匹配。

使用这个命令,你可以控制SAC在遇到不匹配时如何处理。
\begin{itemize}
\item 如果设置错误条件为致命,
那么SAC在遇到错误时将停止执行当前命令,忽略当前行的其它剩余命令,输出错误信息到终端
并将控制权交还给用户。
\item 如果设置错误条件为警告,则遇到错误时会发送一个警告消息,程序
内部尽可能纠正错误并继续执行。
\item 如果设置错误条件为忽略,则SAC会纠正错误并继续而不告诉你发生了什么。
\end{itemize}

若要操作的两个数据文件的数据点数不匹配,SAC会使用数据点最少的那个文件的数据点数
作为最终结果文件的数据点数,以保证正常操作。

若要操作的两个文件采样周期不匹配,则SAC会使用第一个数据文件的采样周期作为结果文件的采样周期。

\SACTitle{示例}
假定file1有1000个数据点,file2有950个数据点:
\begin{SACCode}
SAC> binoperr npts fatal
SAC> read file1
SAC> addf file2
ERROR:  Header field mismatch: NPTS file1 file2
\end{SACCode}

上例中由于数据点数不匹配导致文件加法未执行,假设你输入:
\begin{SACCode}
SAC> binoperr npts warning
SAC> addf file2
WARNING:  Header field mismatch: NPTS file1 file2
\end{SACCode}
则仅对文件的前950个数据点执行加法操作。

\SACTitle{相关命令}
\nameref{cmd:addf}、\nameref{cmd:subf}、\nameref{cmd:mulf}、\nameref{cmd:divf}
