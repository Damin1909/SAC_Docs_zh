\SACCMD{fir}
\label{cmd:fir}

\SACTitle{概要}
应用一个有限脉冲响应滤波器

\SACTitle{语法}
\begin{SACSTX}
FIR [REC|FFT] file
\end{SACSTX}

\SACTitle{输入}
\begin{description}
\item [FFT] 通过FFT变换方法应用FIR滤波器
\item [REC] 递归应用FIR滤波器
\item [file] 包含FIR滤波器的文件名
\end{description}
 
\SACTitle{缺省值}
\begin{SACDFT}
fir fft fir
\end{SACDFT}

\SACTitle{说明}
这个命令中使用的滤波器必须首先用DFIR交互式滤波器设计。这个滤波器通过变换方法应用,除非你要求使用递归方法或者数据点数对于变换方法来说太大。这些滤波器都没有相位失真但在脉冲信号前会产生前驱波。

\SACTitle{头段变量}
depmin, depmax, depmen

\SACTitle{错误消息}
\begin{itemize}
\item[-]1301: 未读入文件
\item[-]1306: 对不等间隔文件非法操作
\item[-]1307: 对谱文件非法操作
\item[-]1601: 文件和滤波器采样频率不匹配
\item[-]1603: 内存不足以进行FIR滤波
\end{itemize}

\SACTitle{警告消息}
\begin{itemize}
\item[-]1602: 内存不足以使用FFT进行FIR滤波(自动使用递归方法)
\end{itemize}

\SACTitle{限制}
变换方法的最大数据点数是4096。
