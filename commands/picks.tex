\SACCMD{picks}
\label{cmd:picks}

\SACTitle{概要}
控制时间标记的显示

\SACTitle{语法}
\begin{SACSTX}
PICKS [ON|OFF] [pick V!ERTICAL!|H!ORIZONTAL!|C!ROSS!] [W!IDTH! v] [H!EIGHT! v]
\end{SACSTX}

\SACTitle{输入}
\begin{description}
\item [ON|OFF] 打开/关闭时间标记的显示
\item [pick] SAC中与时间标记有关的头段变量名,可以取 \texttt{O}、\texttt{A}、
    \texttt{F}、\texttt{Tn}(n=0--9)
\item [VERTICAL] 在时间标记处绘制垂直线,时间标记ID位于线的右下方
\item [HORIZONTAL] 在最接近时间标记的数据点处绘制水平线,时间标记ID位于
    线上或线下
\item [CROSS] 在时间标记处绘制垂直线,在最近的数据点处绘制水平线
\item [WIDTH v] 时间标记的宽度为 \texttt{v}
\item [HEIGHT v] 修改时间标记的高度为 \texttt{v}。高度和宽度仅对
    \texttt{HORIZONTAL} 和 \texttt{CROSS} 使用,其数值为占图形的比例
\end{description}

\SACTitle{缺省值}
\begin{SACDFT}
pick on width 0.1 height 0.1
\end{SACDFT}
所有的拾取标记类型都是 \texttt{VERTICAL}

\SACTitle{说明}
这个命令控制SAC绘图上时间标记的显示。这些时间标记标识了如震相到时、事件
发生时刻等。当打开显示选项时,每一个定义了的时间标记都会在绘图上相应时刻
处绘制一条线,并且在其旁有一个时间标记ID。时间标记ID是一个8字符长的头段
变量。头段变量中 \texttt{KA}、\texttt{KF}、\texttt{KO} 以及 \texttt{KTn}
分别是 \texttt{A}、\texttt{F}、\texttt{O} 和 \texttt{Tn} 的时间标记ID。
如果时间标记id未定义,那么标记的名字就是就头段变量本身。每一个时间标记
可以被显示为一条垂直线、一条水平线或一个交叉线。

\SACTitle{示例}
以交叉符号显示时间标记 \texttt{T4}、\texttt{T5} 和 \texttt{T6},并改变
交叉符号的高度和宽度:
\begin{SACCode}
SAC> picks t4 c t5 c t6 c w 0.3 h 0.1
\end{SACCode}
