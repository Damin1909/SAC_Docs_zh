\SACCMD{read}
\label{cmd:read}

\SACTitle{概要}
从磁盘读取SAC文件到内存

\SACTitle{语法}
\begin{SACSTX}
R!EAD! [MORE] [DIR CURRENT|name] [XDR|ALPHA|SEGY]
    [SCALE ON|OFF] [filelist]
\end{SACSTX}
所有的选项必须位于filelist之前。

\SACTitle{输入}
\begin{description}
\item [MORE] 将读入的新文件添加到内存中老文件之后。若选项此忽略,则读入
    的新数据将替代内存中的老数据
\item [DIR CURRENT] 从``当前目录''读取文件列表中的文件。``当前目录''为
    启动SAC的目录
\item [DIR name] 从目录name中读取文件列表中的文件,可以为绝对路径或相对路径
    \footnote{关于dir选项,有一个很大的陷阱,详见``\nameref{sec:read-dir}''。}
\item [XDR] 读取XDR格式的文件。此格式用于实现不同构架的二进制数据的转换
\item [ALPHA] 输入文件是SAC的字符数字型文件,该选项与XDR选项不兼容
\item [SEGY] 读取IRIS/PASSCAL定义的SEGY格式文件。该格式允许一个文件包含一个波形
\item [SCALE] 只能和SEGY选项搭配使用,该选项默认是关闭的。当 \texttt{SCALE}
    选项为OFF时,SAC直接从SEGY文件中读取数据值;当 \texttt{SCALE} 为ON时,
    SAC将每个数据值乘以以文件中给定的 \texttt{SCALE} 因子。若 \texttt{SCALE}
    为OFF,则这个文件中的 \texttt{SCALE} 值将储存在SAC头段 \texttt{SCALE}中;
    若 \texttt{SCALE} 为ON,SAC的 \texttt{SCALE} 头段将被设置为1.0。
\item [filelist] 文件列表。可以是简单的文件名,也可以包含相对或绝对路径,
    也可以使用通配符
\end{description}

\SACTitle{缺省值}
\begin{SACDFT}
read dir current
\end{SACDFT}

\SACTitle{说明}
该命令将SAC文件从磁盘读入到内存中,默认状态下会读取每个磁盘文件中的全部
数据点。

\nameref{cmd:cut} 命令可以用于指定读取文件的一部分数据。在2000年之后产生
的SAC文件会被假定年份为四位数字。年份为两个数字的文件被假定为20世纪,
会被加上1900。

在使用 \texttt{read} 命令时,正常情况下内存中的老数据会被新读取的数据
所替代。若使用 \texttt{more} 选项,则新数据将被读入内存并放在老数据的
后面。在如下三种情况下 \texttt{more} 选项可能会有用:
\begin{itemize}
\item 文件列表太长无法在一行中键入
\item 在长文件列表中某个文件名拼错而没有读入,可以使用 \texttt{more} 选项再次读入
\item 一个文件被读入,作了些处理,然后与原始数据比较
\end{itemize}

\SACTitle{示例}
\texttt{read} 命令的简单示例位于\nameref{sec:read-and-write} 一节。

如果你想要对一个数据进行高通滤波,并与原始数据进行对比:
\begin{SACCode}
SAC> r f01
SAC> hp c 1.3 n 6
SAC> r more f01
SAC> p1
\end{SACCode}

假设SAC的启动目录位于 \texttt{/me/data},你想要处理其子目录 \texttt{event1}
和 \texttt{event2} 下的文件。
\begin{SACCode}
SAC> read dir event1 f01 f02
\end{SACCode}
读取了目录 \texttt{/me/data/event1} 下的文件。

\begin{SACCode}
SAC> read f03 g03
\end{SACCode}
相同目录下的文件被读入。

\begin{SACCode}
SAC> read dir event2 *
\end{SACCode}
\texttt{/me/data/event2} 下的全部文件被读入。

\begin{SACCode}
SAC> read dir current f03 g03
\end{SACCode}
目录 \texttt{/me/data} 下的文件被读入。

\SACTitle{头段变量}
e、depmin、depmax、depmen、b
