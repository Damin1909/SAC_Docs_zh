\SACCMD{pickphase}
\label{cmd:pickphase}

\SACTitle{概要}
告诉SAC从一个用户定义的参考文件中读入震相列表(也可能是作者信息),或在
该命令行上输入震相列表

\SACTitle{语法}
\begin{SACSTX}
PICKPH!ASE! header phase author [header phase author ...]
PICKPH!ASE! FILE [filename]
PICKPH!ASE! AUTHOR [filename]
\end{SACSTX}

\SACTitle{输入}
\begin{description}
\item [header] 头段变量名:\texttt{t0-t9}
\item [phase] 对于给出的头段变量对应的拾取的震相名
\item [author] 告诉SAC作者列表,为某个头段所对应的作者,或者是 \texttt{-}
\item [FILE] 如果使用了 \texttt{FILE} 选项,SAC将从参考文件中读取震相列表。
    如果参考文件的文件名在命令行上给出,则SAC将读取这个指定的文件,否则
    SAC将根据上一次执行 \texttt{pickphase} 读取最近输入的文件名。如果
    未给出文件名,则SAC使用 \verb|$SACAUX/csspickprefs|
\item [AUTHOR] 与 \texttt{FILE} 选项相似,其另一个功能是允许SAC读取指定的
    头段变量信息
\end{description}

\SACTitle{缺省值}
\begin{SACDFT}
pickphase file
\end{SACDFT}

\SACTitle{说明}
\texttt{pickphase} 用于在命令行上覆盖参考文件。其可以用于在命令行提供指定
的头段/震相/作者信息,或者将SAC从一个参考文件重定向到另一个。更多关于参考
文件的信息,参见 \nameref{cmd:pickprefs} 以及 \nameref{cmd:readcss}。

注意:如果当数据在数据缓冲区内而用户修改了参考文件,那么在SAC缓冲区中的
震相拾取将可能被修改。(缓冲区的信息可以通过 \nameref{cmd:listhdr} 和
\nameref{cmd:chnhdr} 查看)。

例:如果当一些含有某些 \texttt{pP} 或 \texttt{PKiKP} 震相的SAC文件通过
\namere{cmd:read} 命令读入时,被允许的震相包括 \texttt{pP} 以及 \texttt{PKiKP},
那么这些拾取将出现在 \texttt{Tn} 时间标记中。如果 \texttt{pickphase}
在稍后再次使用将 \texttt{pP} 以及 \texttt{PKiKP} 从允许的震相中去除,那么
\texttt{pP} 以及 \texttt{PKiKP} 到时将不会从CSS文件中读取,已经在内存中的
数据的 \texttt{pP} 和 \texttt{PKiKP} 拾取将从 \texttt{Tn}  时间标记中去除。
