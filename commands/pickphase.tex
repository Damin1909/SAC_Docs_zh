\SACCMD{pickphase}
\label{cmd:pickphase}

\SACTitle{概要}
告诉SAC从一个用户定义的参考文件中读入震相列表(也可能是作者信息),或在该命令行上输入震相列表

\SACTitle{语法}
\begin{SACSTX}
PICKPH!ASE! header phase author [ header phase author ... ]
PICKPH!ASE! FILE [filename]
PICKPH!ASE! AUTHOR [filename]
\end{SACSTX}

\SACTitle{输入}
\begin{description}
\item [header] 头段变量名:t0--t9.
\item [phase] 对于给出的头段变量对应的拾取的震相名
\item [author] 告诉SAC作者列表,为某个头段所对应的作者,或者是"-"
\item [FILE] 如果使用了FILE选项,SAC将从参考文件中读取震相列表。如果参考文件的文件名在命令行上给出,则SAC将读取这个指定的文件,否则SAC将根据上一次执行PICKPHASE读取最近输入的文件名.如果未给出文件名,则SAC使用\texttt{sac/aux/csspickprefs}
\item [AUTHOR] 这个选项和FILE选项相似,其另一个功能是允许SAC读取指定的头段变量信息
\end{description}

\SACTitle{缺省值}
PICKPHASE FILE

\SACTitle{说明}
PICKPHASE用于在命令行上覆盖参考文件。其可以用于在命令行提供指定的头段/震相/作者信息,或者将SAC从一个参考文件重定向到另一个。更多关于参考文件的信息,参见PICKPREFS以及READCSS

注意:如果当数据在数据缓冲区内而用户修改了参考文件,那么在SAC缓冲区中的震相拾取将可能被修改。(缓冲区的信息可以通过LISTHDR和CHNHDR查看)。

例:如果当一些含有某些pP或PKiKP震相的SAC文件通过READ命令读入时,被允许的震相包括pP以及PKiKP,那么这些拾取将出现在Tn时间标记中。如果PICKPHASE在稍后再次使用将pP以及PKiKP从允许的震相中去除,那么pP以及PKiKP到时将不会从CSS文件中读取,已经在内存中的数据的pP和PKiKP拾取将从Tn时间标记中去除。
