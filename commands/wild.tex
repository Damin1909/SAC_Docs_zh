\SACCMD{wild}
\label{cmd:wild}

\SACTitle{概要}
设置读命令中用于扩展文件列表的通配符

\SACTitle{语法}
\begin{SACSTX}
WILD [ECHO ON|OFF] [SINGLE char] [MULTIPLE char] [CONCATENATION chars]
\end{SACSTX}

\SACTitle{输入}
\begin{description}
\item [ECHO ON] 打开扩展文件表回显选项;当该选项打开时,会回显被通配符展开的文件列表
\item [ECHO OFF] 关闭扩展文件表回显开关
\item [SINGLE char] 修改用于匹配单个字符的通配符
\item [MULTIPLE char] 修改用于匹配多个字符的通配符 
\item [CONCATENATION chars] 修改用于将联接字符串括起来的字符
\end{description}

\SACTitle{缺省值}
\begin{center}
\begin{tabular}{llll}
\toprule
选项	&	UNIX	&	VAX		&	PRIME	\\
\midrule
ECHO	&	ON 		&	ON 		&	ON		\\
SINGLE  &	?  		&	?  		&	+		\\
MULTIPLE&	* 		&	* 		&	'		\\
CONCATENATION& [,] & 	(,)  	&	[,]		\\
\bottomrule
\end{tabular}
\end{center}

\SACTitle{说明}
很多现代操作系统都提供了通配符特性,也可以称为文件扩展。它是一个可以让你使用
简短文件名以及简单的简写形式去指定一组文件的表示符号。SAC在READ、READTABLE以及READHDR
命令中使用通配符及一些扩展名,使用这些表示符号,你可以很容易地访问一组文件:
\begin{itemize}
\item 所有以字母``abc''开头的文件
\item 所有以``z''结尾的文件
\item 所有文件名中严格包含三个字母的文件
\end{itemize}

通配符代号有三个元素。对于不同的系统三个元素会有不同的缺省符。你可以使用这个命令改变通配符。多重匹配字符(``*'')用于匹配字符串中任意字符串,包括空字符串。单个匹配符(``?'')用于匹配任意单个字符。连接符号(``[''和``]'')用于包围由逗号分隔的要匹配的字符串。在这个字符串中,可以包含单通配符或多通配符。

SAC使用通配符完成文件名的扩展,通常有几个步骤:
\begin{enumerate}
\item 如果标识目录部分存在的话将将其去掉,否则使用当前目录
\item 做系统调用,以得到目录中所有文件的列表
\item 如果在标识中是一个连接表,就用其他字符形成连接表中每个字符的新的标识,然	  后匹配它们到文件表中。如果没有连接表标识,则可简单匹配标识到文件表
\item 去掉形成扩展文件表的所有重复的匹配
\item 如果需要,回显扩展文件表
\item 试着将扩展文件表读入内存
\end{enumerate}

每个操作系统都使用一些不同的步骤在一个目录中存取文件。上面第一步的系统调用反映了这些不同。例如,在UNIX中以字母顺序显示文件名,但在PRIME或VAX上就不是这样。在PRIME目录中文件次序是随意的。这些不同反映在扩展文件表的文件次序上。你可以用各种不同的通配符和连接表进行实验,以确定扩展文件表中的文件次序是否重要。

下例将帮助你理解怎样使用这些通配符元素,一个有用的特征是SAC保存包含在连接表上的字符串,当你输入一个空表,则前面的表将被重复使用,这可以节省许多输入的操作。

\SACTitle{示例}
假定当前目录中包含如下次序的文件:
\begin{SACCode}
ABC DEF STA01E STA01N STA01Z STA02E STA02N STA02Z STA03Z
\end{SACCode}

同样假定扩展文件设置回显,下面显示怎样使用各种通配符去将上面文件表的一部分读入内存:
\begin{SACCode}
SAC> READ S*
 STA01E STA01N STA01Z STA02E STA02N STA02Z STA03Z
SAC> READ *Z
 STA01Z STA02Z STA03Z
SAC> READ ???
 ABC DEF
SAC> READ STA01[Z,N,E]
 STA01Z STA01N STA01E
SAC> READ *[Z,N,E]
 STA01Z STA02Z STA03Z STA01N STA02N STA01E STA02E
SAC> READ *1[Z,N,E] *2[ ]
 STA01Z STA01N STA01E STA02Z STA02N STA02E
\end{SACCode}

\SACTitle{限制}
在一个标识中只可以有一个连接串

\SACTitle{相关命令}
\nameref{cmd:read}、\nameref{cmd:readtable}、\nameref{cmd:readhdr}
