\SACCMD{traveltime}
\label{cmd:traveltime}

\SACTitle{概要}
根据预定义的速度模型计算指定震相的走时

\SACTitle{语法}
\begin{SACSTX}
TRAVELTIME [M!ODEL! model] [PICKS n] [PHASE phaselist] [VERBOSE|QUIET] [M|KM]
\end{SACSTX}

\SACTitle{输入}
\begin{description}
\item [MODEL] 预定义的速度模型,可以取 !iasp91! 或 !ak135!,
    缺省值为 !为iasp91!
\item [PICKS] !number!的取值为0到9,表明将第一个震相的到时存储到对应的头段
    变量 !Tn! 中,其余震相依次写入到后面的头段变量 !Tn! 中;若未指定
    !PICKS!选项,则只计算震相到时但不写入头段中
\item [PHASE] 要计算/标记的震相列表,若使用了 !PICKS n! 选项,则震相到时
    信息将被写入头段 !Tn! 和 !KTn! 中
\item [VERBOSE|QUIET] 若使用 !VERBOSE!,则会在终端输出震相走时及其相对于
    文件参考时刻的秒数;若使用 !QUIET!,则不在屏幕上显示震相走时信息
\item [M|KM] 头段变量 !evdp! 的单位为 \si{\m} 或者 \si{\km}
\end{description}

\SACTitle{缺省值}
\begin{SACDFT}
traveltime MODEL iasp91 KM PHASE P S Pn Pg Sn Sg
\end{SACDFT}

\SACTitle{说明}
该命令使用 \href{https://seiscode.iris.washington.edu/projects/iaspei-tau}{iaspei-tau}
程序计算标准速度模型下的震相理论走时,要求内存中的SAC数据文件中必须包含
事件位置、台站位置以及发震时刻。震相名区分大小写,可使用的震相名参考
\href{https://seiscode.iris.washington.edu/projects/iaspei-tau}{iaspei-tau}
的相关文档。

若使用了 !PICKS n! 选项,则会将震相列表中第一个震相的到时存储在
头段变量 !Tn! 中,余下的震相到时依次存储到其余的 !Tn! 中。
对于每个震相,终端会输出两个时间,前者时震相到时相对于参考时刻的秒数,
后者是震相相对于发震时刻的走时。

由于历史原因,事件深度 !evdp! 的单位可以是 \si{\m} 或 \si{\km}。SAC 从 v101.5 开始,
!evdp! 的默认单位是 \si{\km},但由于某些程序输出的 SAC 波形依然以 \si{\m} 作为
!evdp! 的单位,为了能够兼容这些以 \si{\m} 为事件深度单位的波形数据,新版的SAC中
引入了 !km! 和 !m! 选项以指定深度单位。

\SACTitle{示例}
区域事件,使用默认震相列表:
\begin{SACCode}
SAC> fg seismo
SAC> traveltime
traveltime: depth: 15.000000
traveltime: error finding phase P
traveltime: error finding phase S
traveltime: setting phase Pn       at 10.464321 s [ t = 51.894321 s ]
traveltime: setting phase Pg       at 22.904724 s [ t = 64.334724 s ]
traveltime: setting phase Sn       at 50.047722 s [ t = 91.477722 s ]
traveltime: setting phase Sg       at 66.414337 s [ t = 107.844337 s ]
\end{SACCode}
由于震中距比较小,没有P和S震相,所以会出现 !error finding phase P!
的错误,该错误可忽略。以Pn震相为例,震相走时为 \SI{51.89}{\s},相对于参考
时刻的秒数为 \SI{10.46}{\s}。

上面的示例,只是计算了震相走时,不会写入到头段变量中。要将震相走时存储到
头段变量中,需要使用 !PICKS n!选项:
\begin{SACCode}
SAC> fg seismo
SAC> traveltime picks 0 phase Pn Pg Sn Sg
traveltime: depth: 15.000000
SAC> lh T0MARKER T1MARKER T2MARKER T3MARKER
T0MARKER = 10.464           (Pn)
T1MARKER = 22.905           (Pg)
T2MARKER = 50.048           (Sn)
T3MARKER = 66.414           (Sg)
SAC> write seismo-picks.z
\end{SACCode}
可以看到,!T0! 到 !T3! 分别保存了震相Pn、Pg、Sn、Sg震相的
到时相对于文件参考时刻的秒数。!KT0! 到 !KT3! 中则分别存储了
相应的震相名。此处,尽管没有使用 !VERBOSE! 选项,深度信息还是会被
打印出来,这是为了提醒用户注意 !evdp! 的单位,可以通过 !QUIET!
选项设置不显示深度信息。

!rdseed v5.0! 生成的波形数据,震源深度 !evdp! 的单位为 \si{\m}:
\begin{SACCode}
SAC> r 2008.052.14.16.03.0000.XC.OR075.00.LHZ.M.SAC
SAC> lh evdp
evdp = 6.700000e+03
SAC> traveltime M picks 0
traveltime: depth: 6.700000 km
SAC> lh t0marker t1marker t2marker t3marker
t0marker = 61.48            (Pn)
t1marker = 76.413           (Pg)
t2marker = 109.66           (Sn)
t3marker = 132.11           (Sg)
SAC> ch evdp (0.001 * &1,evdp&) // 将evdp的单位改成km
SAC> setbb station &1,KSTNM&
SAC> write %station%.z
\end{SACCode}
