\SACCMD{writebbf}
\label{cmd:writebbf}

\SACTitle{概要}
将黑板变量文件写入到磁盘

\SACTitle{语法}
\begin{SACSTX}
W!RITE!BBF [file]
\end{SACSTX}

\SACTitle{输入}
\begin{description}
\item [file] 黑板变量文件的文件名,其可以是简单文件名或包含相对路径或绝对路径
\end{description}

\SACTitle{缺省值}
\begin{SACDFT}
writebbf bbf
\end{SACDFT}

\SACTitle{说明}
这个命令让你能够将当前会话的所有黑板变量写入到磁盘文件中,稍后可以使用readbbf命令
将黑板变量文件重新读入SAC,该特性允许你保存某次SAC会话的信息,并用于另一次SAC会话中。
你也可以在自己的程序中调用SAC函数库以访问黑板变量文件中的信息。

\SACTitle{相关命令}
\nameref{cmd:readbbf}、\nameref{cmd:setbb}、\nameref{cmd:getbb}
