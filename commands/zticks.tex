\SACCMD{zticks}
\label{cmd:zticks}

\SACTitle{概要}
用方向标记标识等值线

\SACTitle{语法}
\begin{SACSTX}
ZTICKS [ON|OFF] [Spacing v] [LE!NGTH! v] [D!IRECTION! DOWN|UP]
    [!LIST! c1 c2 ... cn]
\end{SACSTX}

\SACTitle{输入}
\begin{description}
\item [ON|OFF] 打开/关闭等值线方向标记
\item [SPACING v] 在每条线段上设置项链标识之间的间隔为v(视口坐标系)
\item [LENGTH v] 设置每个标识的长度为v(视口坐标系)
\item [DIRECTION DOWN|UP] 标识在z值减小/增加的方向上
\item [LIST c1 c2 . cn] 设置要使用的等值线标识表。在这个表上的每个输入都用于相应的等值线。如果等值线数多于这个列表的长度,则重复使用整个标识表。ON意味着标识画在等值线上,OFF意味着标识不画在等值限上。
\end{description}

\SACTitle{缺省值}
\begin{SACDFT}
zticks off spacing 0.1 length 0.005 direction down list on
\end{SACDFT}

\SACTitle{示例}
参考~\nameref{cmd:contour}例子中zticks的使用

\SACTitle{相关命令}
\nameref{cmd:contour}
