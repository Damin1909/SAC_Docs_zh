\SACCMD{mathop}
\label{cmd:mathop}

\SACTitle{概要}
控制数学操作符的优先级

\SACTitle{语法}
\begin{SACSTX}
MATHOP NORMAL|MATH|FORTRAN|NONE|OLD
\end{SACSTX}

\SACTitle{输入}
\begin{description}
\item [NORMAL] 使用正常的数学操作符优先级
\item [MATH] 与 \texttt{NORMAL} 相同
\item [FORTRAN] 与 \texttt{NORMAL} 相同
\item [NONE] 不使用操作符优先级
\item [OLD] 与 \texttt{NONE} 相同
\end{description}

\SACTitle{缺省值}
\begin{SACDFT}
mathop NORMAL
\end{SACDFT}

\SACTitle{说明}
该命令控制数学操作符的优先级。正常情况下,乘法和除法的优先级要比加法和
减法高,指数运算拥有最高的优先级。

101.6之前的版本中,SAC在进行代数运算时没有考虑操作符的优先级,整个表达式
按照从左到右的顺序依次进行计算。

SAC在101.6之后的版本中,默认使用正常的操作符优先级。对于一些在老版本SAC
下写的脚本或宏来说,可能依赖于旧的优先级顺序。可以考虑修改脚本以适应新版
本的优先级或者直接设定适应 \texttt{OLD} 优先级。

\SACTitle{示例}
正常的操作符优先级:
\begin{SACCode}
SAC> mathop normal
SAC> eval 1+2*3
 7
SAC> eval 1+(2*3)
 7
\end{SACCode}

旧的操作符优先级:
\begin{SACCode}
SAC> mathop old
SAC> eval 1+2*3
 9
SAC> eval 1+(2*3)
 7
\end{SACCode}
