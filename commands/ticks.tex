\SACCMD{ticks}
\label{cmd:ticks}

\SACTitle{概要}
控制绘图上刻度轴的位置

\SACTitle{语法}
\begin{SACSTX}
TICKS [ON|OFF|ONLY] [A!LL!] [T!OP!] [B!OTTOM!] [R!IGHT!] [L!EFT!]
\end{SACSTX}

\SACTitle{输入}
\begin{description}
\item [ON] 在指定的边上显示刻度,其他不变
\item [OFF] 在指定的边上不显示刻度,其他不变
\item [ONLY] 仅在指定的边上显示刻度,其他关闭
\item [ALL] 所有四条边
\item [TOP] X轴的上边
\item [BOTTOM] X轴的下边
\item [RIGHT] Y轴的右边
\item [LEFT] Y轴的左边
\end{description}

\SACTitle{缺省值}
\begin{SACDFT}
ticks on all
\end{SACDFT}

\SACTitle{说明}
刻度轴可以画图形四边的一边或几边上,刻度间隔由 \nameref{cmd:xdiv} 命令控制。

\SACTitle{示例}
显示上部刻度轴,其他不变:
\begin{SACCode}
SAC> ticks on top
\end{SACCode}

关闭所有刻度轴:
\begin{SACCode}
SAC> ticks off all
\end{SACCode}

只显示底部刻度轴:
\begin{SACCode}
SAC> ticks only bottom
\end{SACCode}
