\SACCMD{readerr}
\label{cmd:readerr}

\SACTitle{概要}
控制在执行 \nameref{cmd:read} 命令过程中的错误的处理方式

\SACTitle{语法}
\begin{SACSTX}
R!EAD!ERR [B!ADFILE! F!ATAL!|W!ARNING!|I!GNORE!] [N!OFILES! F!ATAL!|W!ARNING!|I!GNORE!]
          [M!EMORY! S!AVE!|D!ELETE!]
\end{SACSTX}

\SACTitle{输入}
\begin{description}
\item [BADFILE] 当文件不可读或不存在时出现的错误
\item [NOFILES] 文件列表中没有文件可读时出现的错误
\item [FATAL] 设置错误条件为fatal,发送错误消息并停止执行命令
\item [WARNING] 发送警告消息,但继续执行命令
\item [IGNORE] 忽略错误,继续执行命令
\item [MEMORY] 如果无文件可读则对内存中原有的数据进行处理
\item [DELETE] 内存中的原数据将被删除
\item [SAVE] 内存中的原数据将保留在内存中
\end{description}

\SACTitle{缺省值}
\begin{SACDFT}
readerr badfile warning nofiles fatal memory delete
\end{SACDFT}

\SACTitle{说明}
当你试着使用 \nameref{cmd:read} 命令将数据文件读入内存时可能会发生错误。
文件可能不存在或虽然存在但不可读。当SAC遇到这些badfiles时,一般会发送
警告消息,然后试着读取文件列表中的其余文件。如果你想要SAC在遇到坏文件时
停止读取文件可以设置 \texttt{BADFILE} 为 \texttt{FATAL}。如果你不想看到
警告信息,可以设置 \texttt{BADFILE} 为 \texttt{INGORE}。如果文件列表中的
文件均不可读,SAC将发送错误信息并停止处理,如果你想要SAC发送警告信息或
完全忽略这个问题,设置 \texttt{NOFILES} 为 \texttt{INGORE}。当然,SAC内存
中先前的文件也可以从内存中删除或者保留在内存中。
