\SACCMD{whiten}
\label{cmd:whiten}

\SACTitle{概要}
对输入的时间序列的频谱进行平滑

\SACTitle{语法}
\begin{SACSTX}
W!H!IT!EN! n
\end{SACSTX}

\SACTitle{输入}
\begin{description}
\item [n]  阶数(极点数目)。阶数越大,结果数据就越平滑。高阶可以更好地
    清除一些数据,但是也可能会导致对数据处理过多而丢掉一些重要的数据。
    默认值为6。
\end{description}

\SACTitle{缺省值}
\begin{SACDFT}
whiten 6
\end{SACDFT}

\SACTitle{说明}
该命令对数据中加入白噪声,以平滑输入时间序列的频谱。在谱相关命令(比如
子程序SPE中的命令、\texttt{transfer}、\texttt{spectrogram})之前执行,
可以减少频谱值的动态范围,提高了对地震数据高频操作的精度。

\texttt{whiten} 可以在SPE子程序内部调用,也可以在SAC主程序中调用。
SPE中的 \texttt{whiten} 和主程序中的 \texttt{whiten} ,阶数是相互独立的,
即在主程序中修改 \texttt{whiten} 的阶数,不会影响SPE中 \texttt{whiten} 的
阶数。与此同时,主程序中的 \texttt{whiten} 命令与 \texttt{transfer} 命令的
\texttt{prewhiten} 选项共用一个阶数;SPE中 \texttt{whiten} 命令与SPE中的
\texttt{cor} 命令的 \texttt{prewhiten} 选项共用一个阶数。
