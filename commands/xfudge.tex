\SACCMD{xfudge}
\label{cmd:xfudge}

\SACTitle{概要}
设置X轴范围的附加因子

\SACTitle{语法}
\begin{SACSTX}
XFUDGE [ON|OFF|v]
\end{SACSTX}

\SACTitle{输入}
\begin{description}
\item [v] 设置附加因子为 !v!
\item [ON] 打开附加选项,但不改变附加因子
\item [OFF] 关闭附加选项
\end{description}

\SACTitle{缺省值}
\begin{SACDFT}
xfudge 0.03
\end{SACDFT}

\SACTitle{说明}
当坐标轴的范围设置为数据的时间极值时,!xmin=b!,!xmax=e!。实际绘图时
会将 !xmin! 调小一点,将 !xmax! 调大一点,使得绘图时波形的两端
与边框之间留有一些空隙。附加因子定义了这个空隙相对于时间极值的比例。

实际绘图时,会根据附加因子计算新的坐标轴范围:
\begin{verbatim}
            xdiff = xfudge * ( b - e )
            xmin = b - xdiff
            xmax = e + xdiff
\end{verbatim}
其中 !b! 和 !e! 是数据的时间范围,!xfudge! 是附加
因子,!xmin! 和 !xmax! 是计算得到的X坐标轴范围。
该算法对对数坐标也得类似的效果。
