\SACCMD{transcript}
\label{cmd:transcript}

\SACTitle{概要}
控制输出到副本文件

\SACTitle{语法}
\begin{SACSTX}
TRANSCRIPT [OPEN|CREATE|CLOSE|CHANGE|WRITE|HISTORY] [FILE filename]
    [CONTENTS ALL|ERRORS|WARNINGS|OUTPUT|COMMANDS|MACROS|PROCESSED]
    [MESSAGE text]
\end{SACSTX}

\SACTitle{输入}
\begin{description}
\item [OPEN] 打开副本文件,并在已存在的文件底部添加副本
\item [CREATE] 创建一个新的副本文件
\item [CLOSE] 关闭一个已经打开的副本文件
\item [CHANGE] 改变一个已经打开的副本文件的内容
\item [WRITE] 写信息到一个副本文件,不改变其状态或内容
\item [HISTORY] 将命令行历史保存到文件
\item [FILE filename] 定义副本文件的名字
\item [MESSAGE text] 向副本文件中写入文本。这个信息可以用于指定正在进行的
    进程或指定正在处理的不同事件,在两次执行这个命令的过程中这个信息不保存
\item [CONTENTS ALL] 定义副本文件的内容为全部输入输出的信息
\item [CONTENTS list] 定义副本文件的内容,即包含在文件中的输入输出的类型表
\end{description}
其中 \texttt{list} 可以取:
\begin{itemize}
\item \texttt{ERRORS}:执行命令期间产生的错误消息
\item \texttt{WARNINGS}:执行命令期间产生的警告消息
\item \texttt{OUTPUT}:执行命令期间的输出消息
\item \texttt{COMMANDS}:终端输出的原始命令
\item \texttt{MACROS}:宏文件中出现的原始命令
\item \texttt{PROCESSED}:经终端或宏处理之后的命令
\end{itemize}

\SACTitle{缺省值}
\begin{SACDFT}
transcript open file transcript contents all
\end{SACDFT}

\SACTitle{说明}
副本文件用于记录SAC执行的结果。其可以是一个完整或部分副本,可以包含一次
或多次执行的结果。你可以同时拥有5个活动的副本文件,每个文件用于追踪不同
的方面。其中的一个用途是记录终端输入的命令然后用于一个宏文件中,如下例所示。

\SACTitle{示例}
为了创建一个新的副本文件,文件名为mytran,包含了除已处理命令之外的其他
全部类型:
\begin{SACCode}
SAC> transcript create file mytran cont err warn out com macros
\end{SACCode}

如果之后不想把宏命令送入这个文件,你可以使用 \texttt{CHANGE} 选项:
\begin{SACCode}
SAC> transcript change file mytran contents e w o c
\end{SACCode}

为了定义一个名为myrecord的副本文件,其记录了终端输入的命令:
\begin{SACCode}
SAC> transcript create file myrecord contents commands
\end{SACCode}

以后,经过适当的编辑,这个文件可以用作宏命令文件,去自动执行相同的一组
命令。在最后的例子中假设你需要彻夜处理许多事件。你可以设置每个事件一个
副本文件(用不同的副本文件名)去记录处理的结果。另外你可以将处理所有事件
得到的任何错误消息保存到一个副本文件中:
\begin{SACCode}
SAC> transcript open file errortran contents errors
SAC> transcript write message 'processing event 1'
\end{SACCode}

这些命令将放在处理每个事件的宏文件中,它假设事件名作为第一个参数带入宏。
使用打开选项,运行信息和出错信息将添加到文件的后面,第二天检查一下这个
出错信息副本文件,就可以快速查阅在处理期间是否出现了错误。

为了将保存一个命令行副本,以记录SAC当前和将来要运行的命令:
\begin{SACCode}
SCA> transcript history file .sachist
\end{SACCode}
这就在当前目录创建并写入了一个副本文件``\verb|~/.sachist|''。任何储存在
那里的文件将被载入命令历史中。如果这个命令位于你的启动初始化宏文件中,
则每次你运行SAC时将在当前目录产生一个单独的命令行历史。在一个新执行的SAC中,
上下键将浏览完整的命令历史,你可以修改以前输入的命令并再次执行它,如果你
没有在SAC内或初始化宏文件中输入这个命令,则命令行历史将自动保存到
\verb|~/.sac_history|。
