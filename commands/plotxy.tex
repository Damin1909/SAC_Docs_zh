\SACCMD{plotxy}
\label{cmd:plotxy}

\SACTitle{概要}
以一个文件为自变量,一个或多个文件为因变量绘图

\SACTitle{语法}
\begin{SACSTX}
P!LOT!XY name|number name|number [name|number ...]
\end{SACSTX}

\SACTitle{输入}
\begin{description}
\item [name] 数据文件表中的一个文件名 
\item [number] 数据文件表中的一个文件号 
\end{description}

\SACTitle{说明}
你选择的第一个文件(通过文件名或文件号)为自变量,沿着x轴绘制。余下的数据文为因变量,沿着y轴绘制。所有的图形环境命令比如TITLE, LINE和SYMBOL都可以使用以控制绘图的属性。这个命令用于绘制由READTABLE命令读入的多列数据。在多情况下,其可以看作像用绘图命令一样作出的图表。

\SACTitle{示例}
假设你有一个ASCII文件,其包含4列数。你想要将其读入SAC并绘图。下面的命令读入这个文件,将其储存为SAC内部4个分开的文件,打开线型增量开关,然后以第二列为自变量,其他列为因变量绘图:
\begin{SACCode}
SAC> readalpha content ynnn myfile
SAC> line increment on
SAC> plotxy 2 1 3 4
\end{SACCode}
