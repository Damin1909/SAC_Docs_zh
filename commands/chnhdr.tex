\SACCMD{chnhdr}
\label{cmd:chnhdr}

\SACTitle{概要}
修改指定的头段变量的值

\SACTitle{语法}
\begin{SACSTX}
C!HN!H!DR! [FILE n1 n2 ...] field v [field v ...] [ALLT v]
\end{SACSTX}

\SACTitle{输入}
\begin{description}
\item [FILE n1 n2] 只修改内存中的指定文件的头段变量,n为内存中文件的文件号
\item [field v] SAC头段变量名及其值\footnote{为了保证数据内部一致性,以下头段变量的值
    不可用该命令修改:NVHDR、NPTS、NWFID、NORID和NEVID}
\item [ALLT v] 将所有已定义的时间相关头段变量的值加v秒,同时将参考时刻减去v秒
\end{description}

\SACTitle{说明}
关于值v的说明:
\begin{itemize}
\item 头段变量的类型和值的类型必须匹配;
\item 对于有内部空格的字符串要用单引号括起来;
\item 逻辑型头段变量的取值为TRUE或FALSE,YES或NO也可以接受;
\item 对于相对时间头段变量(B、E、O、A、F、Tn),v可以是相对参考时刻的时间偏移量(浮点型),
    也可以使用绝对时刻的形式~\verb+GMT v1 v2 v3 v4 v5 v6+~,其中v1、v2、v3、v4、
    v5、v6是GMT年、一年的第一天、时、分、秒、毫秒。如果v1是两位整数,SAC假定其为当前世纪,
    除非那个时间是未来时间,那种情况下SAC假定是上个世纪,最好还是用4位整数表示年。
\item 对于任意类型的头段变量,均可以设置其值为~\verb+undef+~,使头段变量未定义
\end{itemize}

该命令允许你修改指定的一个或多个文件的头段变量值。在未指定文件号的情况下,则对内
存中的所有文件进行操作。要将内存中修改后的头段覆盖磁盘文件的头段,需要使用write或
writehdr命令,SAC会对新值做有效性检查,不过你可以使用listhdr自己检查。

头段中用6个变量定义了参考时刻,这是SAC中唯一的绝对时刻,其它时刻都被转换成相对
于参考时刻的相对时间。可以使用``ALLT v''修改参考时刻以及相对时间。
参考时间被减去了v秒,相对时间被加上了v秒,
这保证了数据的绝对时刻不发生改变。为了方便,你可以通过输入绝对时刻
而非相对时间来改变时间偏移变量的值。绝对时刻首先被转换为相对时间,然后再存入头段中。

\SACTitle{示例}
为了定义内存中所有文件的事件经纬度、事件名::
\begin{SACCode}
SAC> ch evla 34.3 evlo -118.5
SAC> ch kevnm 'LA goes under'
\end{SACCode}

为了定义第二、四个文件的事件经纬度、事件名:
\begin{SACCode}
SAC> ch file 2 4 EVLA 34.3 EVLO -118.5
SAC> ch file 2 4 KEVNM 'LA goes under'
\end{SACCode}

设定初动到时为无定义状态:
\begin{SACCode}
SAC> ch a undef
\end{SACCode}

假设你知道事件的GMT起始时间,你想要快速改变头段中所有的时间变量,使得发震时刻是0
即参考时间为发震时刻,并且所有的相对时间根据这个时间去纠正相对值。

首先用GMT选项设置事件起始时间:
\begin{SACCode}
SAC> ch o GMT 1982 123 13 37 10 103
\end{SACCode}
现在使用LISTHDR检查发震时刻o相对于当前参考时间的描述:
\begin{SACCode}
SAC> lh o
 o = 123.103
\end{SACCode}
现在使用ALLT选项从所有的偏移时间中减去这个值,并加到参考时间上,你同时需要改变描述参考时间类型的字段:
\begin{SACCode}
SAC> ch allt -123.103 iztype iO
\end{SACCode}
注意这里的负号意味着从偏移时间中减去这个值。

更方便的做法是直接引用头段变量的值:
\begin{SACCode}
SAC> ch allt (0 - &1,o&) iztype IO
\end{SACCode}

\SACTitle{相关命令}
\nameref{cmd:listhdr}、\nameref{cmd:write}、\nameref{cmd:writehdr}
