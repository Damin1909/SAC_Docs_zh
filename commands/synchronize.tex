\SACCMD{synchronize}
\label{cmd:synchronize}

\SACTitle{概要}
同步内存中所有文件的参考时刻

\SACTitle{语法}
\begin{SACSTX}
SYNC!HRONIZE! [ROUND ON|OFF] [BEGIN ON|OFF]
\end{SACSTX}

\SACTitle{输入}
\begin{description}
\item [ROUND ON] 打开ROUND选项。若该打开该选项,则每个文件的开始时间都将以最接近的采样间隔的整数倍
\item [ROUND OFF] 关闭ROUND选项 
\item [BEGIN ON] 设置每个文件的开始时间为0 
\item [BEGIN OFF] 保持参考时间的绝对时刻不变
\end{description}

\SACTitle{缺省值}
\begin{SACDFT}
synchronize round off begin off
\end{SACDFT}

\SACTitle{说明}
该命令同步内存中所有文件的参考时刻。它通过检查所有文件的参考时刻和文件起始的偏移
时间来决定最晚的开始时间,并取最晚的开始时间作为内存中所有文件的参考时刻。
然后计算每个文件中的所有相对时间(B、E、A、Tn 等)相对于新参考时刻的值。

当数据文件具有不同的参考时刻,而你想要用CUT或XLIM命令分析或绘制这些文件中的一部分
数据时这个命令会很有用。一旦参考时间被同步了,则CUT操作将针对严格相同的GMT时间窗进行。

如果使用了BEGIN选项,参考时间的GMT值则不被保留,BEGIN选项将所有文件的kztime、kzdate设置为相同,
而且将所有文件的开始时间设置为0。其他的点保持与文件开始时间的关系。

\SACTitle{示例}
假定你读取两个不同参考时间的文件到内存:
\begin{SACCode}
SAC> read file1 file2
SAC> listhdr b kztime kzdate

  FILE: FILE1
  -
  B = 0.
  KZTIME = 10:38:14.000
  KZDATE = MAR 29 (088), 1981

  FILE: FILE2
  -
  B = 10.00
  KZTIME = 10:40:10.000
  KZDATE = MAR 29 (088), 1981
\end{SACCode}

这些文件有相同的参考日期,不同的参考时间以及不同的开始时间偏移量。你可以执行
SYNCHRONIZE然后使用LISTHDR,你会发现:
\begin{SACCode}
SAC> synchronize
SAC> listhdr

  FILE: FILE1
  -
  B = -126.00
  KZTIME = 10:40:20.000
  KZDATE = MAR 29 (088), 1981

  FILE: FILE2
  -
  B = 0.
  KZTIME = 10:40:20.000
  KZDATE = MAR 29 (088), 1981
\end{SACCode}
现在内存中的所有文件有相同的参考时间,如果头段中有任何已定义的时间标记,它们的值
也会调整以使得其GMT值是不变的。
