\SACCMD{synchronize}
\label{cmd:synchronize}

\SACTitle{概要}
同步内存中所有文件的参考时刻

\SACTitle{语法}
\begin{SACSTX}
SYNC!HRONIZE! [R!OUND! ON|OFF] [B!EGIN! ON|OFF]
\end{SACSTX}

\SACTitle{输入}
\begin{description}
\item [ROUND ON] 打开 \texttt{ROUND}选项。若打开该选项,则会对每个文件的
    开始时间做微调以使得开始时间是采样间隔的整数倍
\item [ROUND OFF] 关闭 \texttt{ROUND} 选项
\item [BEGIN ON] 设置每个文件的开始时间为0
\item [BEGIN OFF] 保持参考时间的绝对时刻不变
\end{description}

\SACTitle{缺省值}
\begin{SACDFT}
synchronize round off begin off
\end{SACDFT}

\SACTitle{说明}
该命令用于同步内存中所有文件的参考时刻。通过检查所有文件的参考时间和文件
起始时间(B),找到所有文件中最晚的文件起始时刻,并取该时刻作为内存中
所有文件共同的参考时刻,最后再计算每个文件中所有相对时间相对于新参考时刻
的值。

当数据文件具有不同的参考时刻时,对数据使用 \nameref{cmd:cut} 或 \nameref{cmd:xlim}
命令就会有些麻烦,因而可以使用该命令使得所有数据的参考时刻一致。

如果使用了 \texttt{BEGIN} 选项,则会将所有文件的 \texttt{kzdate} 和
\texttt{kztime} 设置为同样的值,并将所有文件的开始时间(B)设置为零,
这样做会使得数据丢失绝对时间信息。

\SACTitle{示例}
假定你读取两个不同参考时间的文件到内存:
\begin{SACCode}
SAC> read file1 file2
SAC> listhdr b kztime kzdate

  FILE: FILE1
  -
  B = 0.
  KZTIME = 10:38:14.000
  KZDATE = MAR 29 (088), 1981

  FILE: FILE2
  -
  B = 10.00
  KZTIME = 10:40:10.000
  KZDATE = MAR 29 (088), 1981
\end{SACCode}

这些文件有相同的参考日期,不同的参考时刻以及不同的开始时间偏移量。可以
执行 \texttt{synchronize} 同步参考时刻:
\begin{SACCode}
SAC> synchronize
SAC> listhdr

  FILE: FILE1
  -
  B = -126.00
  KZTIME = 10:40:20.000
  KZDATE = MAR 29 (088), 1981

  FILE: FILE2
  -
  B = 0.
  KZTIME = 10:40:20.000
  KZDATE = MAR 29 (088), 1981
\end{SACCode}
现在内存中的所有文件有相同的参考时间,如果头段中有任何已定义的时间标记,
它们的值也会调整以保证其绝对时刻是不变的。
