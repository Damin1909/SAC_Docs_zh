\SACCMD{writecss}
\label{cmd:writecss}

\SACTitle{概要}
将内存中的数据以 !CSS 3.0! 格式写入磁盘

\SACTitle{语法}
\begin{SACSTX}
W!RITE!CSS [B!INARY!|A!SCII!] [DIR ON|OFF|CURRENT|name] name
\end{SACSTX}

\SACTitle{输入}
\begin{description}
\item [ASCII] 以标准ASCII格式写入磁盘
\item [BINARY] 以 !CSS 3.0! 二进制文件输出
\item [DIR ON] 打开目录选项,但不改变目录名
\item [DIR OFF] 关闭目录选项,即写入当前目录
\item [DIR CURRENT] 打开目录选项并设置写目录为当前目录
\item [DIR name] 打开目录选项并设置写目录为 !name! 。将所有文件
    写入目录 !name! 中,其可以是相对路径或绝对路径
\item [name] 以文件名 !name! 写入磁盘。只能指定一个名字其不能包含
    通配符。对于 !ASCII! 型输出,以 !name! 为基础,在其后
    加上各个CSS表所对应的后缀(比如: !name.wfdisc!、!name.origin!)。
    对于 !BINARY! 输出,则 !name! 是输出文件名
\end{description}

\SACTitle{缺省值}
\begin{SACDFT}
wirtecss ascii dir off
\end{SACDFT}

\SACTitle{说明}
该命令允许你在数据处理过程中将数据以CSS 3.0格式保存到磁盘中。在ASCII模式
下(默认模式),会写入一个或多个ASCII文件到磁盘。输出的具体文件依赖于数据
来源,但输出可以是下面列出的CSS表中的任意一个或多个:
\begin{verbatim}
wfdisc, wftag, origin, arrival, assoc, sitechan, site, affiliation,
origerr, origin, event, sensor, instrument, gregion, stassoc, remark sacdata
\end{verbatim}

在BINARY模式下,所有的CSS表以及波形数据都会以二进制格式写入到同一个文件中。
