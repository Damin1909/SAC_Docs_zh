\SACCMD{line}
\label{cmd:line}

\SACTitle{概要}
控制绘图中的线型

\SACTitle{语法}
\begin{SACSTX}
LINE [ON|OFF|S!OLID!|D!OTTED!|n] [FILL ON|OFF|pos_color/neg_color]
    [I!NCREMENT! [ON|OFF]] [L!IST! STANDARD|nlist]
\end{SACSTX}

\SACTitle{输入}
\begin{description}
\item [ON] 打开线型选项,不改变线型
\item [OFF] 关闭线型选项
\item [SOLID] 改变线型为实线型,并打开线型开关
\item [DOTTED] 改变线型为虚线型,并打开线型开关
\item [n] 设置线型为n并绘制线条。若n取值为0则表示不绘制该线条
\item [FILL ON|OFF] 打开/关闭颜色填充
\item [FILL pos\_color/neg\_color] 对每个数据的正值和负值部分涂色
\item [INCREMENT ON] 每个数据被绘出之后,按照线型表中的次序改变为另一个线型
\item [INCREMENT OFF] 不改变线型
\item [LIST nlist] 改变线型列表
\item [LIST STANDARD] 设置为标准线型列表
\end{description}

\SACTitle{缺省值}
\begin{SACDFT}
line solid increment off list standard
\end{SACDFT}

\SACTitle{说明}
这个命令控制绘图时的线型,图形的框架(轴、标题等等)通常使用实线。网格的
线型用 \nameref{cmd:grid} 命令控制。并非所有的图形设备都有除实线型之外的
其他线型的,在那些设备上显然这个命令没有什么效果。而对于不同的设备线型号
n也可能是不同的。

还有其他命令可以控制数据显示的其他方面。\nameref{cmd:symbol} 命令用于将
每个数据点的值用一个符号显示在图上。\nameref{cmd:color} 命令控制彩色图形
设备的颜色选择。所有的这些属性都是独立的。如果你想的话你可以选择在每个数
据点上选择带符号的蓝色虚线。线型为0代表关闭画线选项。在 !LIST!
选项和 \nameref{cmd:symbol} 命令中可以利用线型为0的特性,在同一张图上将
某些数据用线显示某些数据用符号显示。

\SACTitle{示例}
选择依次变化的线型,从线型1开始:
\begin{SACCode}
SAC> line 1 increment
\end{SACCode}

改变线型表使之包含线型3、5和1:
\begin{SACCode}
SAC> line list 3 5 1
\end{SACCode}

使用 \nameref{cmd:plot2} 在同一个图形上绘制三个文件,第一个使用实线无符号,
第二个没有线条,用三角符号,第三个无线条,用十字符号:
\begin{SACCode}
SAC> read file1 file2 file3
SAC> line list 1 0 0 increment
SAC> symbol list 0 3 7 increment
SAC> plot2
\end{SACCode}

将地震图的正值部分涂上红色,负值部分涂上蓝色,如果线型为0,则涂色区域用
黑色描边:
\begin{SACCode}
SAC> fg seis
SAC> line 0 fill red/blue
SAC> p
\end{SACCode}
