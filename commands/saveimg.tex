\SACCMD{saveimg}
\label{cmd:saveimg}

\SACTitle{概要}
将绘图窗口中的图像保存到多种格式的图像文件中

\SACTitle{语法}
\begin{SACSTX}
SAVE!IMG! filename.format
\end{SACSTX}

\SACTitle{输入}
\begin{description}
\item [filename] 要保存的图像的文件名
\item [format] 图像文件格式,支持PS、PDF、PNG和XPM
\end{description}

\SACTitle{说明}
该命令将当前绘图窗口中的图像保存到图像文件中,可用的格式包PS、PDF、PNG
和XPM,命令会根据图像文件的扩展名自动识别文件格式。

\texttt{saveimg} 相对于SGF文件的好处在于,SGF文件中的字母和数字是由线段
组成的,而 \texttt{saveimg} 产生的ps或pdf图像采用Postscript的特性直接
产生字体。对大多数情况,低精度的png或xpm文件也能满足要求。出于可移植性的
考虑,SAC默认是不支持PNG格式的。

png和xpm将拥有当前窗口的横纵比,pdf或ps文件拥有固定的横纵比X/Y=11/8.5=1.2941,
对这些绘图,如果显示窗口设置为1.2941会看起来比较好。

与SGF文件类似,\texttt{saveimg} 生成的PS文件是没有BoundingBox的。对于
sgf文件,脚本 \texttt{sgftoeps.csh} 可以生成一个有BoundingBox的eps文件。
对于 \texttt{saveimg} 生成的PS文件,目前还没有相应的脚本。

为了使用 \texttt{saveimg} 保存一个绘图,图像必须是可见的,即通过
\nameref{cmd:plot}、\nameref{cmd:plot1} 等命令绘制出来。\texttt{saveimg}
在子程序SSS中无法工作,但如果输入 \nameref{cmd:quitsub} 退出子程序,
此时图像窗口未关闭,\texttt{saveimg} 此时可用于保存该图像。另外如果使用了
\nameref{cmd:beginframe} 命令在一个窗口中绘制多张图,必须等到执行
\nameref{cmd:endframe} 命令之后方可使用 \texttt{saveimg} 命令。

\SACTitle{示例}
将图像保存为PDF文件:
\begin{SACCode}
SAC> read PAS.CI.BHZ.sac
SAC> p1
SAC> saveimg pas.ci.pdf
\end{SACCode}

将谱图用多种格式保存:
\begin{SACCode}
SAC> fg seismo
SAC> spectrogram
SAC> save spectrogram.ps
SAC> save spectrogram.png
SAC> save spectrogram.pdf
\end{SACCode}
