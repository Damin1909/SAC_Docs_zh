\SACCMD{getbb}
\label{cmd:getbb}

\SACTitle{概要}
获取或打印黑板变量的值

\SACTitle{语法}
\begin{SACSTX}
GETBB [TO TERM!inal!|filename] [NAMES ON|OFF] [NEWLINE ON|OFF]
    ALL|variable [variable ...]
\end{SACSTX}

\SACTitle{输入}
\begin{description}
\item [TO TERMINAL] 打印值到终端
\item [TO filename] 将值追加到文件filename后
\item [NAMES ON] 输出格式为``黑板变量名=黑板变量值''
\item [NAMES OFF] 只打印黑板变量值
\item [NEWLINE ON] 打印每个黑板变量后换行
\item [NEWLINE OFF] 打印黑板变量后不换行
\item [ALL] 打印当前定义的全部黑板变量
\item [variable] 打印列表指定的黑板变量
\end{description}

\SACTitle{缺省值}
\begin{SACDFT}
getbb to terminal names on newline on all
\end{SACDFT}

\SACTitle{说明}
该命令用于获取或打印黑板变量的值。可以控制打印哪些黑板变量以及具体的打印格式。
你可以将黑板变量打印到终端或者文本文件中。你可以使用这些选项对一系列数据文件进行测量,
将结果保存到文本文件中,然后用 \nameref{cmd:readtable} 命令将这个文件读回SAC,绘图或者进行更多的分析。

\SACTitle{示例}
假设你已经设置了一些黑板变量:
\begin{SACCode}
SAC> setbb c1 2.45 c2 4.94
\end{SACCode}

稍后可以这样打印他们的值:
\begin{SACCode}
SAC> getbb c1 c2
 c1 = 2.45
 c2 = 4.94
\end{SACCode}

想要在一行内只打印其值:
\begin{SACCode}
SAC> getbb names off newline off c1 c2
 2.45 4.94
\end{SACCode}

假设你有一个宏文件叫GETXY,其可以对单个文件进行某些分析操作,并将结果储存在两个头段变量中X和Y中。
你想要对当前目录中所有垂直分量进行操作,保存每对X和Y的值,然后绘图。下面的宏文件的第一个参数是
用于储存这些结果的文本文件:
\begin{SACCode}
DO FILE WILD *Z
  READ FILE
  MACRO GETXY
  GETBB TO 1 NAMES OFF NEWLINE OFF X Y
ENDDO
GETBB TO TERMINAL
READALPHA CONTENT P 1
PLOT
\end{SACCode}
最终这个文本文件将包含成对的x-y数据点,每行一个,对应一个垂直分量的数据文件。为了关闭文本文件
并清空缓存区,最后将输出重定向到终端的getbb命令是必要的。
