\SACCMD{rglitches}
\label{cmd:rglitches}

\SACTitle{概要}
去掉信号中的坏点

\SACTitle{语法}
\begin{SACSTX}
RGL!ITCHES! [TH!RESHOLD! v] [TY!PE! L!INEAR!|Z!ERO!] [W!INDOW! ON|OF!F!|pdw]
    [METHOD A!BSOLUTE!|P!OWER!|R!UNAVG!]
\end{SACSTX}

\SACTitle{输入}
\begin{description}
\item [WINDOW ON|OFF|pdw] 指定需要做校正的数据段。缺省值为OFF,即校正
    整个数据文件;\nameref{subsec:pdw} 指定了时间窗,表示仅对该时间窗
    内的数据进行检测和校正;ON表示只校正上一次 \texttt{pdw} 定义的时间
    窗内的数据;
\item [THRESHOLD v] 设置阈值水平为 \texttt{v}。当某个特定的指标大于该
    阈值时即认为是坏点,并做校正
\item [METHOD ABSOLUTE] 若数据点的绝对值大于或等值阈值 \texttt{v},则做校正
\item [METHOD POWER] 用向后差分法计算信号的能量,若能量超过阈值v则做校正
\item [METHOD RUNAVG] 将一个长为 \texttt{SWINLEN} 的窗从整个数据的末尾以
    一个数据点的增量移动到数据的开头,并计算每个滑动窗内的平均值和标准差。
    若某个点与当前窗的均值的差的绝对值大于标准差\texttt{THRESH}的2倍,
    且大于 \texttt{MINAMP},则认为其是坏点,其值将用当前窗的均值代替。
    此方法不受 \texttt{WINDOW} 选项的影响,总是适用于整段波形数据。
\end{description}

对于RUNAVG方法,另有三个与之相关的选项:
\begin{description}
\item [SWINLEN v] 设置滑动平均窗的长度为v
\item [THRESH2 v] 设置坏点的阀值
\item [MINAMP v] 设置坏点的最小幅度
\end{description}

\SACTitle{缺省值}
\begin{SACDFT}
rglitches threshold 1.0e+10 type linear window off method absolute
    swinlen 0.5 thresh2 5.0 minamp 50
\end{SACDFT}

\SACTitle{说明}
此命令可以用于由于平滑数据采集系统故障或测试\footnote{有些台网会在每天的
指定时间生成若干个人工脉冲,以检测数据采集系统是否正常运行。}
而产生的坏点。该命令会检查每一个数据点是否超过指定的阈值,然后将这些坏点
置零或者线性插值。使用RUNAVG方法,甚至可以去掉那些比整个数据的最大值要小
的坏点。

\SACTitle{头段变量}
depmin、depmax、depmen
