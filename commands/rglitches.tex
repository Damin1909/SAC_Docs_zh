\SACCMD{rglitches}
\label{cmd:rglitches}

\SACTitle{概要}
去掉信号中的坏点

\SACTitle{语法}
\begin{SACSTX}
RGL!ITCHES! [TH!RESHOLD! v] [TY!PE! L!INEAR!|Z!ERO!] [W!INDOW! ON|OF!F!|pdw]
    [METHOD A!BSOLUTE!|P!OWER!|R!UNAVG!]
\end{SACSTX}

\SACTitle{输入}
\begin{description}
\item [WINDOW ON|OFF|pdw] 指定需要做校正的数据段。缺省值为OFF,即校正整个数据文件;
    pdw指定了时间窗,表示仅对该时间窗内的数据进行检测和校正;ON表示只校正上一次pdw定义
    的时间窗内的数据;
\item [THRESHOLD v] 设置阈值水平为v。当某个特定的指标大于该阈值时即认为
    是坏点,并做校正
\item [METHOD ABSOLUTE] 若数据点的绝对值大于或等值阈值v,则做校正
\item [METHOD POWER] 用向后差分法计算信号的能量,若能量超过阈值v则做校正
\item [METHOD RUNAVG] 将一个长为SWINLEN的窗从整个数据的末尾以一个数据点的增量移动到数据
    的开头,并计算每个滑动窗内的平均值和标准差。若某个点与当前窗的均值的差的绝对值大于标准差
    的THRESH2倍,且大于MINAMP,则认为其是坏点,其值将用当前窗的均值代替。此方法不受WINDOW选项
    的影响,总是适用于整段波形数据。
\end{description}

对于RUNAVG方法,另有三个与之相关的选项:
\begin{description}
\item [SWINLEN v] 设置滑动平均窗的长度为v
\item [THRESH2 v] 设置坏点的阀值
\item [MINAMP v] 设置坏点的最小幅度
\end{description}

\SACTitle{缺省值}
\begin{SACDFT}
rglitches threshold 1.0e+10 type linear window off method absolute
    swinlen 0.5 thresh2 5.0 minamp 50
\end{SACDFT}

\SACTitle{说明}
此命令可以用于由于平滑数据采集系统故障或测试
\footnote{有些台网会在每天的指定时间生成若干个人工脉冲,以检测数据采集系统是否正常运行。}
而产生的坏点。该命令会检查每一个数据点是否超过指定的阈值,然后将这些坏点置零或者线性插值。
使用RUNAVG方法,甚至可以去掉那些比整个数据的最大值要小的坏点。

\SACTitle{头段变量}
depmin、depmax、depmen
