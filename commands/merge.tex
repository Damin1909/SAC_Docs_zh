\SACCMD{merge}
\label{cmd:merge}

\SACTitle{概要}
将多个数据文件合并成一个文件

\SACTitle{语法}
\begin{SACSTX}
MERGE [V!ERBOSE!] [G!AP! Z!ERO!|I!NTERP!] [O!VERLAP! C!OMPARE!|A!VERAGE!] [filelist]
\end{SACSTX}

\SACTitle{输入}
\begin{description}
\item [VERBOSE] 输出数据合并的细节。
\item [GAP ZERO|INTERP] 如何处理数据间断。ZERO表示将数据间断处补零值;INTERP表示对
    数据间断处进行线性插值。
\item [OVERLAP COMPARE|AVERAGE] 如何处理数据重叠。COMPARE表示对重叠的时间段内的
    数据进行比较,若不匹配则退出;AVERAGE表示对重叠时间段内的数据进行平均。
\item [filelist] SAC二进制数据文件列表\footnote{暂时不支持通配符。}
\end{description}

\SACTitle{说明}
此命令在101.6版中完全重写,功能和用法上也与之前的版本完全不同。

在旧版本中,内存中的每个文件与filelist中的每个文件分别合并,且内存中的文件必须早于
filelist中的文件。因而在合并多段数据时,只能先读取第一段数据,再merge第二段数据,再
merge第三段数据,依次不断循环执行。

新版本的merge命令,会将内存中的全部数据文件以及filelist中的全部文件合并成单个文件。
若内存中无数据,则只合并filelist中的数据文件;若filelist为空,则只合并内存中的数据
文件。

新版的merge命令可以合并任意数目和任意顺序的数据文件。
merge会检查所有要合并的文件,以保证他们拥有相同的kstnm、kentwk、kcmpnm、delta。

若要合并的数据段之间存在间断,可以通过补零或线性插值的方式弥补间断;若数据段之间
存在重叠,可对重叠的部分进行比较判断或直接进行平均。

\SACTitle{示例}
下面看一个数据合并的例子
\footnote{这样的修改使得不同版本的语法不完全兼容,实际使用时新版的merge命令要更方便
也更符合人的正常思维。}:
\begin{SACCode}
SAC> read file1 file2
SAC> merge file3 file4
\end{SACCode}
在101.5c及其之前的版本中,这个例子的结果是,file3与file1合并成文件file1,file4与
file2合并成文件file2,此时内存中有两个文件file1和file2。而在101.6及其之后的版本中,
这个例子的结果是,文件file1、file2、file3、file4合并成文件file1,此时内存中只有
一个文件file1。

多个文件合并成单个文件的一种方法:
\begin{SACCode}
SAC> r file1                        // 读取一个文件
SAC> merge file2 file3 file4        // merge其余文件
SAC> w over
\end{SACCode}

另一种合并办法:
\begin{SACCode}
SAC> r file1 file2 file3 file4
SAC> merge                      // 合并内存中的所有文件
SAC> w over                     // 合并后的文件写入到file1中
\end{SACCode}

再一种合并方法:
\begin{SACCode}                     // 内存中无数据
SAC> merge file1 file2 file3 file4  // 合并filelist中的全部文件
SAC> w over                         // 保存到file1中
\end{SACCode}

\SACTitle{头段变量改变}
npts、depmin、depmax、depmen、e

\SACTitle{相关命令}
\nameref{cmd:read}
