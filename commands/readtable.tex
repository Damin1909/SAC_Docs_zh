\SACCMD{readtable}
\label{cmd:readtable}

\SACTitle{概要}
从磁盘读取列数据文件到内存

\SACTitle{语法}
\begin{SACSTX}
R!EAD!TAB!LE! [MORE] [DIR CURRENT|name] [FREE|FORMAT tex]
    [CONTENT text] [HEADER number] [filelist]
\end{SACSTX}
所有的选项必须位于filelist之前。最后两个选项可以放在每个文件的第一行。

\SACTitle{输入}
\begin{description}
\item [MORE] 将新文件追加到内存中老文件之后。若忽略该选项,则新数据将替代内存中的老数据
\item [DIR CURRENT] 从当前目录读取所有简单文件名。当前目录是你启动SAC的目录
\item [DIR name] 从目录name中读取全部简单文件,其可以为绝对/相对路径
\item [FREE] 用自由格式读取文件列表中的数据(以空格分隔)
\item [FORMAT text] 以固定格式读取文件列表中的数据。该选项目前不可用
\item [CONTENT text] 定义数据内容。text的具体格式见说明及示例
\item [HEADER] 文件中要跳过的几个头段行
\item [filelist] 列数据文件
\end{description}

\SACTitle{缺省值}
\begin{SACDFT}
readtable free content y. dir current
\end{SACDFT}

\SACTitle{说明}
该命令可以读取字符型列数据。最简单的用法就是读取一个Y数据,也可以通过修改content的
内容读入X-Y数据或更复杂的数据。因而该命令可以用于直接读取其他程序输出的复杂格式数据。
也可以用这个方法读入多个Y数据集,但只允许一个X数据集。

读入数据时会计算基本的头段变量,包括npts、b、e、delta、leven、depmin、depmax和depmin。
若只有一个Y数据集,内存中的数据文件名将和磁盘文件名相同;若有多个Y数据集,则在文件名之
加上一个两位数字。

字符数字型数据文件的每一行都将以自由格式或声明的格式读入,每行最多160个字符。
content选项用于决定对于数据每行的每个输入该如果处理。在content text中的每个字符
分别代表了不同的数据元素,这些字符的顺序与数据中每行的输入所代表的含义相对应。
content字段允许的字符如下:
\begin{itemize}
\item Y: 下一个输入属于Y(因变量)数据集
\item X: 下一个输入属于X(自变量)数据集
\item N: 下一个输入属于数据集
\item P: 下一对输入使用X-Y数据集
\item R: 下一对输入使用Y-X数据集
\item I: 忽略这个输入
\end{itemize}

还有一个重复计数器可以跟在上面的任何字符之后。这个重复计数器是一个1位或2位整数,其代表重复前面那个字符多少次,``.''是一个无穷次重复的计数器,其只能出现在content的text的最后,意味着最后一个字符可以表示接下来的所有输入列。

\SACTitle{示例}
为了读取一个或多个自由格式的X-Y数据对:
\begin{SACCode}
SAC> readtable content p. filea
\end{SACCode}

你不能在文件行之间打断一个X-Y数据对。假设你有一个包含了格式化数据的文件,在每行的中间有一个X-Y数据对。每行的其它数据都没有用。假设每行Y数据在X数据之前,一旦正确的格式声明给出了,就可以用下面的命令:
\begin{SACCode}
SAC> readtable content r format \(24x,f12.3,14x,f10.2\) fileb
\end{SACCode}
注意:在左括号和右括号两边的``\verb+\+''是SAC的转义字符,这很重要,因为SAC使用括号作为内联函数。由于没有重复计数器,因而只有一个Y-X数据对被从文件的每行读入。

假设你有一个文件FILEC,其每行包括一个X值和7个不同数据集的Y值,其为(8F10.2)格式。为了在内存中创建7个不同的数据集,可以使用下面的命令:
\begin{SACCode}
SAC> readtable content xn . format \(8f10.2\) filec
\end{SACCode}
这将在内存中产生7个不同的数据文件,其名称分别为FILEC01, FILEC02等等。

现在假设你不想读入第5个Y数据集,可以执行下面的命令:
\begin{SACCode}
SAC> readtable content xn6 format \(5f10.20x,2f10.2\) filec
\end{SACCode}
另一个可以少敲键盘但是稍微低效一点的命令如下:
\begin{SACCode}
SAC> READTABLE CONTENT XN4IN2 FORMAT \(8F10.2\) FILEC
\end{SACCode}

\SACTitle{头段变量改变}
b, e, delta, leven, depmin, depmax, depmen

\SACTitle{相关命令}
\nameref{cmd:read}、\nameref{cmd:write}
