\SACCMD{dif}
\label{cmd:dif}

\SACTitle{概要}
对数据进行微分

\SACTitle{语法}
\begin{SACSTX}
DIF [TW!O!|TH!REE!|F!IVE!]
\end{SACSTX}

\SACTitle{输入}
\begin{description}
\item [TWO]   使用两点差分算子
\item [THREE] 使用三点差分算子
\item [FIVE]  使用五点差分算子
\end{description}

\SACTitle{缺省值}
\begin{SACDFT}
dif two
\end{SACDFT}

\SACTitle{说明}
该命令通过对数据应用差分算法实现数据的微分操作,要求数据必须是等间隔采样的时间序列文件。
微分会使位移量变成速度量,速度量变成加速度量,因而微分的同时会修改头段变量 
!idep! 的值。

两点差分算法:
\[ Out(j) =\frac{Data(j+1) - Data(j)}{\Delta} \]
两点差分算子不是中心差分算法。此算法的最后一个输出值是未定义的,SAC的处理方式是:
令数据点数减1,文件起始时间 !B! 增加半个采样周期。

三点(中心两点)差分算法:
\[ Out(j) = \frac{1}{2} \frac{Data(j+1) - Data(j-1)}{\Delta} \]
此算法输出值的第一个和最后一个是未定义的,SAC将数据点数减去2,并将文件起始时间 !B!
增加一个采样周期。

五点(中心四点)差分算法:
\[ Out(j) = \frac{2}{3} \frac{Data(j+1) - Data(j-1)}{\Delta} - \frac{1}{12} \frac{Data(j+2) - Data(j-2)}{\Delta} \]
此算法输出值的首尾各两个点是未定义的,SAC使用三点差分算符计算第二个和倒数第二个点
的值,并将数据点数减2,将 !B! 增加一个采样周期。



\SACTitle{头段变量改变}
npts、b、e、depmin、depmax、depmen、idep
