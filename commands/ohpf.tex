\SACCMD{ohpf}
\label{cmd:ohpf}

\SACTitle{概要}
打开一个HYPO格式的震相文件

\SACTitle{语法}
\begin{SACSTX}
OHPF [file]
\end{SACSTX}

\SACTitle{输入}
\begin{description}
\item [file]  要打开的文件名。如果文件已经存在,则打开并将新的震相加到文件底部
\end{description}

\SACTitle{缺省值}
\begin{SACDFT}
OHPF HPF
\end{SACDFT}

\SACTitle{说明}
SAC产生的HYPO震相拾取文件可以用于程序HYPO71以及其他类似事件定位程序的输入。由APK和PPK得到的震相拾取信息被写入这个打开的文件中,这个文件可以使用CHPF关闭。打开一个新的HYPO文件会自动关闭前一个已经打开的文件。打开一个已经存在的HYPO文件的同时也会自动删除文件的最后一行,这一行原本有一个指令标记10作为HYPO文件的结束标志符,删除最后一行意味着可以在其后添加新的拾取。终止SAC也会自动关闭任何已经打开的拾取文件,事件分割符能够用WHPF命令写入震相拾取文件。

\SACTitle{错误消息}
\begin{itemize}
\item[-]1901: 无法打开HYPO震相拾取文件(可能是文件名只能中的非法字符)
\end{itemize}

\SACTitle{相关命令}
\nameref{cmd:apk}、\nameref{cmd:plotpk}、\nameref{cmd:whpf}、\nameref{cmd:chpf}
