\SACCMD{pickauthor}
\label{cmd:pickauthor}

\SACTitle{概要}
告诉SAC从一个用户定义的参考文件中读入作者列表(也可能是震相拾取信息),
或在该命令行上输入作者列表

\SACTitle{语法}
\begin{SACSTX}
PICKA!UTHOR! author1 [author2 author3 ...]
PICKA!UTHOR! FILE [filename]
PICKA!UTHOR! PHASE [filename]
\end{SACSTX}

\SACTitle{输入}
\begin{description}
\item [authorlist] SAC使用输入创建作者列表
\item [FILE] 如果使用了 \texttt{FILE} 选项,SAC将从参考文件中读取作者列表。
    如果参考文件的文件名在命令行上给出,则SAC将读取这个指定的文件,否则
    SAC将根据上一次执行 \nameref{cmd:pickauthor} 读取最近输入的文件名。
    如果未给出文件名,则SAC使用 \verb|$SACAUX/csspickprefs|
\item [PHASE] 与 \texttt{FILE} 选项相似,其另一个功能是允许SAC读取指定的
    头段变量信息
\end{itemize}

\SACTitle{缺省值}
\begin{SACDFT}
pickauthor file
\end{SACDFT}

\SACTitle{说明}
\texttt{pickauthor} 提供了一种在命令行上覆盖参考文件的方法。其可以用于
在命令行提供作者的优先列表信息,或者将SAC从一个参考文件重定向到另一个。
更多关于参考文件的信息,参考 \nameref{cmd:pickprefs} 以及 \nameref{cmd:readcss}。

注意:如果当数据在数据缓冲区内而用户修改了参考文件,那么在SAC缓冲区中的
震相拾取将可能被修改。(缓冲区的信息可以通过 \nameref{cmd:listhdr} 和
\nameref{cmd:chnhdr} 查看)。

例:如果作者列表是``john rachel michael'',并且一些文件被用 \nameref{cmd:readcss}
命令读入,一些到时可能以 \texttt{author=michael} 读入。(用户可能不会
注意到对于某个给出的震相拾取其作者是谁,因为在CSS中的作者字段在SAC格式中
并不会出现)。如果用户稍后使用了 \texttt{pickauthor} 命令并修改作者列表为
``peter doug rachel'',然后 \texttt{readcss more} 读入更多文件,没有
\texttt{author=michael} 的文件读入,已经在内存中的文件将丢失以michael作为
作者的拾取。不了解这个用户可能会莫名地发现似乎震相拾取会随机消失。更多的
信息参考 \texttt{pickprefs}。
