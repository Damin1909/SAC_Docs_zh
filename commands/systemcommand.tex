\SACCMD{systemcommand}
\label{cmd:systemcommand}

\SACTitle{概要}
在SAC内部执行系统命令

\SACTitle{语法}
\begin{SACSTX}
S!YSTEM!C!OMMAND! command [options]
\end{SACSTX}

\SACTitle{输入}
\begin{description}
\item [command] 系统命令名
\item [options] 命令需要的选项
\end{description}

\SACTitle{说明}
在SAC中是可以执行大部分系统命令的,比如常见的 \texttt{ls}、\texttt{cp}
等。但某些命令无法直接在SAC中执行,比如用于查看PS文件的 \texttt{gs} 命令
会首先被SAC解释为 \nameref{cmd:grayscale} 的简写,故而在SAC中无法直接
调用 \texttt{gs} 命令。

另一个经常使用但无法直接调用的命令是 \texttt{rm}。为了避免在读入数据时
将命令 \texttt{r *.SAC} 误敲成 \texttt{rm *.SAC} 而导致数据文件被误删除,
故而在SAC中禁止直接调用 \texttt{rm} 命令。

\texttt{systemcommand} 的作用就是为了能够在SAC内部间接调用这些无法
被SAC直接调用的系统命令。

\SACTitle{示例}
调用系统命令 \texttt{rm} 删除某些SAC文件:
\begin{SACCode}
SAC> rm junks           // 无法直接调用rm命令
 ERROR 1106: Not a valid SAC command.
SAC> sc rm junks        // 通过sc间接调用rm命令
\end{SACCode}
