\SACCMD{systemcommand}
\label{cmd:systemcommand}

\SACTitle{概要}
在SAC内部执行系统命令

\SACTitle{语法}
\begin{SACSTX}
S!YSTEM!C!OMMAND! command [ options ]
\end{SACSTX}

\SACTitle{输入}
\begin{description}
\item [command] 系统命令名
\item [options] 命令需要的选项
\end{description}

\SACTitle{说明}
在SAC中是可以执行大部分系统命令的,比如常见的 \texttt{ls}、\texttt{cp} 等。

但是某些命令无法直接在SAC中执行,比如用于查看PS文件的gs命令会首先被SAC解释为
\nameref{cmd:grayscale} 的简写,故而在SAC中无法直接调用gs命令。

另一个经常使用但无法直接调用的命令是 \texttt{rm}。由于 \nameref{cmd:read} 命令
可以被简写为 \texttt{r} ,在读入文件时键入 \texttt{r *.SAC} 很可能会一时手误
敲成 \texttt{rm *.SAC} ,为了避免类似的误操作,故而在SAC中禁止直接调用 \texttt{rm} 命令。

当需要在SAC内部执行删除命令时,则需要使用 \nameref{cmd:systemcommand} 调用系统命令。

\SACTitle{示例}
调用系统命令删除某些SAC文件:
\begin{SACCode}
SAC> rm junks
 ERROR 1106: Not a valid SAC command.
SAC> sc rm junks
\end{SACCode}