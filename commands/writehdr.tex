\SACCMD{writehdr}
\label{cmd:writehdr}

\SACTitle{概要}
用内存中文件的头段区覆盖磁盘文字中的头段区

\SACTitle{语法}
\begin{SACSTX}
W!RITE!H!DR!
\end{SACSTX}

\SACTitle{说明}
\nameref{cmd:write} 命令的 !over! 选项可以用内存中头段区和数据区
覆盖磁盘文件中的头段区和数据区。该命令用内存中头段区覆盖磁盘文件中的头段区,
数据区不会被覆盖。如果使用了 \nameref{cmd:cut} 命令,读取数据时将仅读入
部分数据,内存中的头段区将会做相应修改以反映 !cut! 命令的效果,
但是磁盘中的数据并没有被修改,因而此时不能使用 !writehdr! 命令。
对被 !cut! 的数据使用 !writehdr! 命令将可能导致磁盘中的数据
产生类似于平移或截断的效果。
