\SACCMD{correlate}
\label{cmd:correlate}

\SACTitle{概要}
计算自相关和互相关函数

\SACTitle{语法}
\begin{SACSTX}
    COR!RELATE! [M!ASTER! name|n] [N!UMBER! n] [L!ENGTH! ON|OFF|v] [NO!RMALIZED!]
    [T!YPE! R!ECTANGLE!|HAM!MING!|HAN!NING!|C!OSINE!|T!RIANGLE!]
\end{SACSTX}

\SACTitle{输入}
\begin{description}
\item [MASTER name|n] 通过文件名或文件号指定主文件,所有文件将与此文件做相关
\item [NUMBER n] 设置相关窗的个数
\item [LENGTH ON|OFF] 打开/关闭固定窗长选项开关
\item [LENGTH v] 打开固定窗长选项开关,并将窗长度设置为v秒
\item [NORMALIZED] 对相关结果进行归一化
\item [TYPE RECTANGLE] 给每个窗应用矩形函数,其等价于不对窗加函数
\item [TYPE HAMMING|HANNING] 对每个窗应用Hamming/Hanning函数
\item [TYPE COSINE] 对每个窗的前后10\%的数据点应用COSINE函数
\item [TYPE TRIANGLE] 对每个窗应用TRIANGLE函数
\end{description}

\SACTitle{缺省值}
\begin{SACDFT}
correlate master 1 number 1 length off type rectangle
\end{SACDFT}

\SACTitle{说明}
该命令在频率域计算信号间的相关函数。你可以指定内存中的某个信号为主信号,主信号将与内存
中的所有信号进行相关,即主信号与主信号计算自相关函数,与其它信号计算互相关函数。

该命令的窗特性允许你计算对多个窗计算平均相关函数,其中窗的数目以及窗函数均可以指定。
当窗特性被打开时,会将信号划分为n个固定长度的窗,计算每个窗的互相关函数,然后将所有
的互相关函数做平均、截取到与原信号相同的数据长度,并替换内存中的原始数据文件。

当窗长度(LENGTH选项)以及窗数目(NUMBER选项)超过文件中的数据点数(NPTS)时,
会自动计算窗之间的重叠。缺省情况下,此窗特性是关闭的。

\SACTitle{示例}
以内存中第三个文件为主文件计算互相关函数:
\begin{SACCode}
SAC> r file1 file2 file3
SAC> cor master 3
\end{SACCode}
也可以通过文件名来指定主文件。

假设有两个数据文件,每个包含1000个噪声数据。将数据划分为无重叠的10个窗,每个窗包含
100个数据点,且对窗应用HANNING函数,并计算10个窗的平均相关函数:
\begin{SACCode}
SAC> r file1 file2
SAC> cor type hanning number 10
\end{SACCode}

为了使窗之间有20\%的混叠,可以设置窗长度为120个数据点。假设数据采样周期为0.025(即每秒40个采样点),则窗长为3秒:
\begin{SACCode}
SAC> r file1 file2
SAC> cor type hanning number 10 length 3.0
\end{SACCode}

\SACTitle{头段变量}
depmin、depmax、depmen
