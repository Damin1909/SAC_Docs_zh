\SACCMD{addf}
\label{cmd:addf}

\SACTitle{概要}
将一组数据加到内存中的另一组数据中

\SACTitle{语法}
\begin{SACSTX}
ADDF [N!EWHDR! [ON|OFF]] filelist
\end{SACSTX}

\SACTitle{输入}
\begin{description}
\item [NEWHDR ON|OFF] 指定新生成的文件使用哪个文件的头段。!OFF!
    表示使用内存中原文件的头段区,!ON! 表示使用filelist中文件的
    头段区。缺省值为 !OFF!
\item [filelist] SAC二进制文件列表
\end{description}

\SACTitle{说明}
简单地说,该命令要做的就是C=A+B,其中A是已读入内存的文件,B是磁盘中要加到A的文件,C是
文件加法的生成物。该命令会将磁盘中的一组文件,与内存中的另一组文件分别相加。若内存中的
文件数多于filelsit中文件数,则filelist的最后一个文件将加到内存中余下的文件中;
若filelist中的文件数多于内存中的文件数,则filelist中多余的文件将被忽略。

要相加的两个文件必须满足如下条件才能执行加法操作:
\begin{itemize}
\item 为等间隔时间序列文件
\item 有相同的采样间隔 !delta!
\item 有相同的数据点数 !npts!
\item 文件开始时间有相同的绝对时刻
\end{itemize}

若两个待相加的文件的时刻不完全匹配,则会给出警告,但相加操作会继续执行。若采样间隔或数据
点数不匹配,默认情况下SAC会认为这是致命(!fatal!)错误,直接报错退出。可以通过 
\nameref{cmd:binoperr} 命令将采样间隔或数据点数的不匹配设置为忽略(!ignore!)、
警告(!warning!)或致命(!fatal!)。

在将不匹配设置为 !ignore! 或 !warning! 的前提下,SAC会忽略采样
周期的不匹配,直接对数据做加法,不论是否使用了 !newhdr on! 选项,都使用第一个
数据文件的采样周期作为生成数据的采样周期;对于数据点数不匹配的情况,则取最小的数据点数作为
最终结果文件的数据点数。

\SACTitle{示例}
将一个文件加到其他三个文件中:
\begin{SACCode}
SAC> r file1 file2 file3
SAC> addf file4
\end{SACCode}

将两个文件分别加到另两个文件中:
\begin{SACCode}
SAC> r file1 file2
SAC> addf file3 file4
\end{SACCode}

\SACTitle{头段变量}
depmin、depmax、depmen、npts、delta
