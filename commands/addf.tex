\SACCMD{addf}
\label{cmd:addf}

\SACTitle{概要}
使内存中的一组数据加上另一组数据

\SACTitle{语法}
\begin{SACSTX}
ADDF [N!EWHDR! [ON|OFF]] filelist
\end{SACSTX}

\SACTitle{输入}
\begin{description}
\item [NEWHDR ON|OFF] 若为OFF,则生成的数据文件使用内存中原文件的头段;
    若为ON,则生成的数据文件使用filelist中新文件的头段。缺省值为OFF
\item [filelist] SAC二进制文件列表
\end{description}

\SACTitle{说明}
这个命令用于将一组数据文件与另一组数据文件分别相加。若内存中的文件数
多于filelsit中文件数,则filelist的最后一个文件将加到剩余文件中;
若filelist中的文件数多余内存中的文件数,则filelist中多余的文件将被忽略。

该命令仅可对等间隔时间序列进行操作,且要求相加的两个数据文件有相同的
采样周期和数据点数,数据文件所对应的绝对时刻也应匹配。

对于采样周期和数据点数是否匹配的检查,可以通过 \nameref{cmd:binoperr}
命令进一步设置为ignore、warning或fatal。

在ignore或warning的前提下,若两个待相加的文件采样周期不等,SAC会忽略
采样周期的差异,直接进行数据点的加法;若文件的数据点数不等,则取最小
的数据点数作为最终结果文件的数据点数。

若两个待相加的文件所对应的时刻不完全匹配,则会给出警告,但相加操作会
继续执行。

\SACTitle{示例}
将一个文件加到其他三个文件中:
\begin{SACCode}
SAC> r file1 file2 file3
SAC> addf file4
\end{SACCode}

将两个文件分别加到另两个文件中:
\begin{SACCode}
SAC> r file1 file2
SAC> addf file3 file4
\end{SACCode}

\SACTitle{头段变量}
depmin、depmax、depmen、npts、delta
