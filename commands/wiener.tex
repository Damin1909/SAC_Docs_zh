\SACCMD{wiener}
\label{cmd:wiener}

\SACTitle{概要}
设计并应用一个自适应Wiener滤波器

\SACTitle{语法}
\begin{SACSTX}
W!IE!N!E!R [W!INDOW! pdw] [N!COEFF! n] [MU OFF|ON|v] [EPS!ILON! OFF|ON|e]
\end{SACSTX}

\SACTitle{输入}
\begin{description}
\item [WINDOW pdw] 设置滤波器设计窗口为pdw。关于pdw参见~\nameref{cmd:cut}~命令
\item [NCOEFF n] 设置滤波器系数为n个
\item [MU off|on|v] 设置自适应步长参数
\item [MU Off] 设置自适应步长参数为0
\item [MU ON] 设置自适应步长为1.95/Rho(0),其中Rho(0)是pdw中延迟为0时的自相关系数
\item [MU v] 设置自适应步长为v
\item [EPSILON e] 设置ridge回归参数为e。
\end{description}

\SACTitle{缺省值}
\begin{SACDFT}
wiener window b 0 10 ncoeff 30 mu off epsilon off
\end{SACDFT}

\SACTitle{说明}
预测误差滤波器使用Yule-Walker方法,从指定的部分数据窗中由自相关函数给出。
这个窗口可以是文件的任何部分,然后滤波器应用到整个个信号,即信号被残差序列替换。
这个滤波器可以用作预白化或用作瞬时信号的检波预处理器。若mu指定为非0值,则滤波器
为时域自适应的,大值mu可能会导致不稳定。

\SACTitle{示例}
下面的命令将应用一个非自适应滤波器,将第一个十秒指定为窗:
\begin{SACCode}
SAC> wiener window b 0 10 mu 0.
\end{SACCode}

下面命令将应用带40个系数的滤波器,指定设计窗为从文件开始到第一个到时前1秒:
\begin{SACCode}
SAC> wiener ncoeff 40 window b a -1
\end{SACCode}

\SACTitle{头段变量}
depmin、depmax、depmen

\SACTitle{相关命令}
\nameref{cmd:cut}
