\SACCMD{bandpass}
\label{cmd:bandpass}

\SACTitle{概要}
对数据文件使用无限脉冲带通滤波器

\SACTitle{语法}
\begin{SACSTX}
B!AND!P!ASS! [BU!TTER!|BE!SSEL!|C1|C2] [C!ORNERS! v1 v2] [N!POLES! n] [P!ASSES! n] 
    [T!RANBW! v] [A!TTEN! v]
\end{SACSTX}

\SACTitle{输入}
\begin{description}
\item [BUTTER] 应用一个Butterworth滤波器
\item [BESSEL] 应用一个Bessel滤波器
\item [C1] 应用一个Chebyshev I型滤波器
\item [C2] 应用一个Chebyshev II滤波器
\item [CORNERS v1 v2] 设定拐角频率分别为v1和v2,即频率通带为v1-v2
\item [NPOLES n] 设置极数为N,范围: 1-10
\item [PASSES n] 设通道数为N,范围: 1-2
\item [TRANBW v] 设置Chebyshev转换带宽为v
\item [ATTEN v] 设置Chebyshev衰减因子为v
\end{description}

\SACTitle{缺省值}
\begin{SACDFT}
bandpass butter corner 0.1 0.4 npoles 2 passes 1 tranbw 0.3 atten 30
\end{SACDFT}

\SACTitle{说明}
在SAC中有一系列无限脉冲滤波器(IIR)可以使用。这些递归的数字滤波器是基于
传统的模拟滤波器设计的:Butterworth, Bessel, Chebyshev I型以及Chebyshev II型
。这些模拟滤波器经过双线性变换(一种可以保持模拟滤波器稳定性的变换方式)转
换成数字滤波器。

一般来说,多数情况下Butterworth滤波器是个不错的选择。因为它有一个相当尖锐的
转换带以及平缓的群延迟响应。Butterworth滤波器是默认的滤波器类型,它的3 db点
指定在截止频率处。Bessel滤波器对于那些需要线性相位而没有双通滤波的应用来说是
最好的。它的振幅响应并不够好。SAC的Bessel滤波器经过归一化因此它的3 db点也在
指定的截止频率处。两个Chebyshev滤波器可以用于适应通带与阻带之间具有较为尖锐
的转变的要求。尽管他们有较好的振幅响应,但是它们的群延迟响应是SAC所有滤波器
中最差的。

在使用这些滤波器时需要小心。首先,所有的递归滤波器都有非线性相位响应,这将
导致滤波后波形的频散。对于滤波后波形的相位很重要的应用来说,SAC提供了一个递
归滤波的零相位工具。零相位滤波器可以通过正向和反向(而不仅仅只是正向滤波)两次
滤波来实现。这个双向操作的滤波器产生一个等效振幅响应,它等于原来振幅响应的平
方。它同时也产生一个非因果的滤波脉冲响应,使得信号在尖锐起始时间之前附
加一个虚假的前驱信号。因此双通滤波后数据无法正确的给出起跳到时。对于信号
前驱不可忽略的情况,例如读取震相起跳到时,使用双通滤波器不是一个很好的
选择。其次,当滤波器的通带宽度相比折叠频率很小时,滤波器可能会出现数值不稳定
。这个问题在增加极数时会更加严重。在要求使用一个非常窄的通带时,一个有效的办
法是首先对数据进行采样,对采样之后的数据用一个通带宽一些的滤波器进行滤波,最
后对数据进行插值回到原始采样率。当所需通带宽降到折叠频率的百分之几时很有必	
要使用这种策略。

一般来说,随着极数的增加滤波器将有一个从通带到阻带的尖锐的转变。然而,极数太
大是要付出代价的。滤波器群延迟一般随着极数的增加而变得更宽,结果导致波形混淆
。那需要3或4个极的应用应该重新考虑。

Butterworth和Bessel滤波器的设计特别简单。你只需要指定截止频率和极数即可。

Chebyshev滤波器设计起来更复杂一点,你还需要提供转换带宽以及阻带衰减因子。
转换带宽是滤波器通带和阻带之间的区域的宽度,它被指定为模拟滤波器通带宽度
的一部分由于双线性变换频率轴的非线性弯曲,递归数字滤波器的转换带宽可能会
比设计时指定的要小。在SAC里,模拟滤波器的截止频率在双线性变换之后要做补偿
以保证其满足设计要求。阻带边界同样也是不真实的,因此,如果明确的设置阻带
边界是重要的,当你的截止频率选定后你必须对其进行补偿。

阻带衰减是通带增益与阻带增益的比值。

\SACTitle{示例}
应用一个四极Butterworth滤波器,拐角频率为2 、5 Hz.:
\begin{SACCode}
SAC> bp n 4 c 2 5
\end{SACCode}

在此之后如果要应用一个二极双通具有相同频率的Bessel:
\begin{SACCode}
SAC> bp n 2 be p 2
\end{SACCode}

\SACTitle{错误消息}
\begin{itemize}
\item[-]1301: 未读入文件
\item[-]1306: 对不等间隔文件的非法操作
\item[-]1307: 对谱文件的非法操作
\item[-]1002: 由于拐角频率大于Nyquist频率而出现的坏值
\end{itemize}

\SACTitle{头段变量改变}
depmin, depmax, depmen
