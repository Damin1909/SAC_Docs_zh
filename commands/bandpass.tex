\SACCMD{bandpass}
\label{cmd:bandpass}

\SACTitle{概要}
对数据文件使用无限脉冲带通滤波器

\SACTitle{语法}
\begin{SACSTX}
B!AND!P!ASS! [BU!TTER!|BE!SSEL!|C1|C2] [C!ORNERS! v1 v2] [N!POLES! n] [P!ASSES! n]
    [T!RANBW! v] [A!TTEN! v]
\end{SACSTX}

\SACTitle{输入}
\begin{description}
\item [BUTTER] 应用Butterworth滤波器
\item [BESSEL] 应用Bessel滤波器
\item [C1] 应用Chebyshev I型滤波器
\item [C2] 应用Chebyshev II型滤波器
\item [CORNERS v1 v2] 设定拐角频率分别为 \texttt{v1} 和 \texttt{v2},
    即频率通带为 \texttt{v1-v2}
\item [NPOLES n] 设置极数为 \texttt{n},取值范围:1--10
\item [PASSES n] 设置通道数为 \texttt{n},取值范围:1--2
\item [TRANBW v] 设置Chebyshev转换带宽为v
\item [ATTEN v] 设置Chebyshev衰减因子为v
\end{description}

\SACTitle{缺省值}
\begin{SACDFT}
bandpass butter corner 0.1 0.4 npoles 2 passes 1 tranbw 0.3 atten 30
\end{SACDFT}

\SACTitle{说明}
SAC提供了一系列无限脉冲滤波器(IIR)可对数据进行滤波,包括Butterworth、
Bessel、Chebyshev I型和Chebyshev II型滤波器。这些递归数字滤波器都基于
传统的模拟滤波器,将模拟滤波器经过双线性变换(一种可以保留模拟滤波器
稳定性的数学变换)即可得到数字滤波器。

多数情况下Butterworth滤波器都是个不错的选择,因为Butterworth滤波器从通带
到阻带的转变相对比较尖锐,且群延迟响应相对平缓,因而Butterworth滤波是SAC
中默认的滤波器类型,其 \SI{3}{dB} 点位于指定的截止频率处。

Bessel滤波器适用于在不使用双通滤波的情况下要求线性相位的情形,其振幅响应
不够理想,SAC中的Bessel滤波器经过归一化以保证其 \SI{3}{\dB} 点也位于指定
的截止频率处。

两个Chebyshev滤波器用于需要通带与阻带间具有较为尖锐的转变的情况。
尽管他们有较好的振幅响应,但是它们的群延迟响应是SAC所有滤波器中最差的。

使用这些滤波器时需要小心。首先,所有的递归滤波器都具有非线性相位响应,
会导致滤波后波形的频散,即波形出现畸变。若要求滤波后的波形保留原始相位,
则要求滤波器具有零相位。对同一个数据做正向和反向两次滤波,即相当于对
数据做了一次零相位滤波,对应于命令中双通道的情况。双通道滤波器的振幅
响应,等于单通道滤波器的振幅响应的平方。与此同时,零相位滤波器具有非因果
的脉冲响应,会导致滤波后的信号在尖锐的跳起之前附加一个虚假的前驱信号。
因而双通滤波后的数据是不能用于拾取震相的跳起到时的,此时,双通滤波不是
一个好选择。

其次,当滤波器的通带宽度比数据的Nyquist频率小很多时,滤波器会出现数值
不稳定,当滤波器的级数增加时问题更加严重。对于需要极窄通带的情形,更可靠
的方式是先对数据进行减采样,然后对采样后的数据应用一个通带稍宽一些的
滤波器,再将滤波后的数据插值回原始数据的采样率。当滤波通带的带宽降到
Nyquist频率的百分之几以下时很有必要使用这种策略。

一般来说,随着极数的增加,滤波器从通带到阻带的过渡会更加尖锐。但是,使用
极数过大是要付出代价的。随着极数的增加,滤波器的群延迟也会变宽,进而导致
滤波后的波形频散更加严重。因而要仔细考虑是否真的需要超过3个或4个极点的
滤波器。

Butterworth和Bessel滤波器的设计特别简单,只需要指定截止频率和极数即可。
Chebyshev滤波器设计起来更复杂一点,除了截止频率和极点数目之外,还需要
提供转换带宽以及阻带衰减因子。转换带宽是滤波器通带和阻带之间的区域的宽度。
由于双线性变换频率轴的非线性弯曲,递归数字滤波器的转换带宽可能会比设计时
指定的要小。在SAC里,模拟滤波器的截止频率在双线性变换之后要做补偿
以保证其满足设计要求。阻带边界同样也是不准确的。因此,若要精确指定通带
边界,在选定截止频率后必须对其进行补偿。阻带衰减因子是通带增益与阻带增益
的比值。

\SACTitle{示例}
应用一个四极Butterworth滤波器,拐角频率为 \SI{2}{\Hz} 和 \SI{5}{\Hz}:
\begin{SACCode}
SAC> bp n 4 c 2 5
\end{SACCode}

在此之后如果要应用一个二极双通具有相同频率的Bessel:
\begin{SACCode}
SAC> bp n 2 be p 2
\end{SACCode}

\SACTitle{头段变量改变}
depmin、depmax、depmen
