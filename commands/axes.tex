\SACCMD{axes}
\label{cmd:axes}

\SACTitle{概要}
控制注释轴的位置

\SACTitle{语法}
\begin{SACSTX}
AXES [ON|OFF|ONLY] [A!LL!] [T!OP!] [B!OTTOM!] [R!IGHT!] [L!EFT!]
\end{SACSTX}

\SACTitle{输入}
\begin{description}
\item [ON] 显示列表中指定的注释轴,其他不变
\item [OFF] 不显示列表中指定的注释轴,其他不变
\item [ONLY] 只显示列表中指定的注释轴,其他的不显示
\item [ALL] 指定所有四个注释轴
\item [TOP] 绘图上部的X注释轴
\item [BOTTOM] 绘图下部的X注释轴
\item [RIGHT] 绘图右侧的Y注释轴
\item [LEFT] 绘图左侧的Y注释轴
\end{description}

\SACTitle{缺省值}
\begin{SACDFT}
axes only bottom left
\end{SACDFT}
即只有下边和左边使用注释轴

\SACTitle{说明}
坐标轴可以绘制在一张图四边的任意一或多个边,有很多命令可以控制坐标轴长什么样。
坐标轴的注释间隔用~\nameref{cmd:xdiv}命令设定(即隔多长显示一个数字),刻度标记的间距可以用~\nameref{cmd:ticks}命令单独控制。

only表示仅在后面列表中指定的边上使用注释轴,而on和off则表示仅对列表中的边打开
或关闭注释轴,对其他不在列表中的边不起作用。

要获得自己想要的效果,使用on或者off时你必须要知道当前已经显示的轴有哪些,
哪些是你想要打开或关闭的。这是一个有点容易弄错的问题,不如只使用only加上想要显示
的轴更加简单一点。

\SACTitle{示例}
\begin{SACCode}
SAC> fg seis
SAC> p           //看看SAC的默认设置,左边和底部有注释
SAC> axes on t   //打开顶部注释,左边和底部注释依然保留
SAC> p           //看到的结果是只有顶部注释,没有左边和底部注释,
                 //这里和说明中强调的不一样,应该是程序的bug,
                 //将on认为是only的简写了
SAC> axes on a   //打开所有注释轴
SAC> axes off b  //仅关闭底部注释轴(off选项和说明是一致的)
SAC> axes only b //仅显示底部注释轴
\end{SACCode}

\SACTitle{相关命令}
\nameref{cmd:xdiv}、\nameref{cmd:ticks}
