\SACCMD{deletechannel}
\label{cmd:deletechannel}

\SACTitle{概要}
从内存文件列表中删去一个或多个文件

\SACTitle{语法}
\begin{SACSTX}
D!ELETE!C!HANNEL! ALL
\end{SACSTX}
或
\begin{SACSTX}
D!ELETE!C!HANNEL! fname|filenumber|range [fname|filenumber|range ...]
\end{SACSTX}

\SACTitle{输入}
\begin{description}
\item [ALL] 删除内存中全部文件
\item [fname] 要删除的内存文件列表中的文件名
\item [filenumber] 文件列表中指定文件的文件号,第一个文件是1,第二个文件是2...
\item [range] 要删除的一系列文件号,用破折号隔开,如11-20
\end{description}

\SACTitle{示例}
\begin{SACCode}
  dc 3 5                         // 删除第三、五个文件
  dc SO01.sz SO02.sz             // 删除这些名字的文件
  dc 11-20                       // 删除11-20的全部文件
  dc 3 5 11-20 SO01.sz SO02.sz   // 删除上面的全部
\end{SACCode}

\SACTitle{错误消息}
\begin{itemize}
\item[-]5106: 文件名不在文件列表中
\item[-]5107: 文件号不在文件列表中
\end{itemize}

\SACTitle{相关命令}
\nameref{cmd:filenumber}
