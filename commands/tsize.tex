\SACCMD{tsize}
\label{cmd:tsize}

\SACTitle{概要}
控制文本尺寸属性

\SACTitle{语法}
\begin{SACSTX}
TSIZE [T!INY!|S!MALL!|M!EDIUM!|L!ARGE! v ] [RATIO v] [OLD|NEW]
\end{SACSTX}

\SACTitle{输入}
\begin{description}
\item [TINY|SMALL|MEDIUM|LARGE v] 设定标准文本尺寸的值为v
\item [RATIO v] 设定文本的宽高比为v
\item [OLD] 将所有文本尺寸值设置为旧值。旧值即SAC 9之前的版本中的文本尺寸值
\item [NEW] 设定所有文本尺寸值为SAC初始化时的缺省值
\end{description}

\SACTitle{缺省值}
\begin{SACDFT}
tsize ratio 1.0 new
\end{SACDFT}

\SACTitle{说明}
大多数的文本注释命令(\nameref{cmd:title}、\nameref{cmd:xlabel}、\nameref{cmd:fileid}等)
允许你改变要显示的文本的尺寸。
你可以从四个标准尺寸中选择(TINY、SMALL、MEDIUM、LARGE)。每一个标准尺寸都有一个初始值,
如下表所示,这个尺寸定义为一个字符的高度相对于整个视窗的百分比。有些时候你想要使用一些
不同于默认尺寸的文本。\nameref{cmd:tsize}~允许你重新定义这四个标准尺寸。你也可以使用这个命令改变字符的宽-高比。
\begin{table}[!ht]
\centering
\caption{SAC标准文本尺寸}
\begin{tabular}{lccccc}
\toprule
NAME	&	A	&	B	&	C	&	D	&	E	\\
\midrule
TINY 	& 0.015 &   66 	&  50  	&	68  &	110	\\
SMALL	& 0.020 &	50  &  37  	&	66  &	82	\\
MEDIUM  & 0.030 &	33  &  25  	&	44  &	55	\\
LARGE	& 0.040 &	25  &  18  	&	33  &	41	\\
\bottomrule
\end{tabular}
\end{table}

上面做各列的定义如下:
\begin{itemize}
\item A 自如相对整个视窗的高度
\item B 全视窗下文本的行数
\item C 正常视窗下文本的行数。正常视窗是指x为0.到1.,y为0.到0.75
\item D 正常视窗中,每行的最小字符数
\item E 正常视窗中每行字符的平均数
\end{itemize}

从SAC 9开始,系统的默认文本尺寸发生了变化,新的尺寸集覆盖了更宽的尺寸范围,在
多数设备上看上去更好。你可以使用OLD选项将文本尺寸修改为之前版本的文本尺寸。

\SACTitle{示例}
为了改变MEDIUM的定义,并使用它创建一个特别尺寸的标题:
\begin{SACCode}
SAC> tsize medium 0.35
SAC> title 'rayleigh wave spectra' size medium
SAC> plot2
\end{SACCode}

为了重置文本尺寸到其默认值:
\begin{SACCode}
SAC> tsize new
\end{SACCode}

\SACTitle{相关命令}
\nameref{cmd:title}、\nameref{cmd:xlabel}、\nameref{cmd:fileid}、\nameref{cmd:plotc}
