\SACCMD{marktimes}
\label{cmd:marktimes}

\SACTitle{概要}
根据一个速度集得到走时并对数据文件进行标记

\SACTitle{语法}
\begin{SACSTX}
MARKT!IMES! [T!O! marker] [D!ISTANCE! H!EADER!|v] [O!RIGIN! H!EADER!|v|GMT time]
    [V!ELOCITIES! v ... ]
\end{SACSTX}

\SACTitle{输入}
\begin{description}
\item [TO marker] 定义头段中用于储存结果的第一个时间标记。对下一个的速度
    使用下一个时间
\item [marker] \texttt{T0|T1|T2|T3|T4|T5|T6|T7|T8|T9}
\item [DISTANCE HEADER] 使用头段中的 \texttt{dist} 代表距离用于走时计算
\item [DISTANCE v] 使用 \texttt{v} 作为走时计算中的距离
\item [ORIGIN HEADER] 使用头段中的参考时间(\texttt{O})用于走时计算
\item [ORIGIN v] 使用相对参考时间偏移 \texttt{v} 秒用于走时计算
\item [ORIGIN GMT time] 使用GMT时间作为参考时间
\item [time] GMT时间包含六个整数:年、儒略日、时、分、秒、毫秒
\item [VELOCITIES v ...] 设置用于走时计算的速度集,最多可以输入10个速度
\end{description}

\SACTitle{缺省值}
\begin{SACDFT}
marktimes velocities 2. 3. 4. 5. 6. distance
    header origin header to t0
\end{SACDFT}

\SACTitle{说明}
这个命令在头段中标记震相的到时,给定事件的发生时间,震中距以及速度即可
计算到时。下面的方程用于简单估计走时:
 		\[ time(j) = origin + \frac{distance}{velocity(j)} \]
结果被写入指定的时间标记中

\SACTitle{示例}
使用默认的速度集,强制距离为 \SI{340}{\km},第一个时间标记为 \texttt{T4}:
\begin{SACCode}
SAC> marktimes distance 340. to t4
\end{SACCode}

选择一个不同的速度集:
\begin{SACCode}
SAC> markt v 3.5 4.0 4.5 5.0 5.5
\end{SACCode}

设置新的参考时间并将结果保存在 \texttt{T2} 中:
\begin{SACCode}
SAC> markt origin gmt 1984 231 12 43 17 237 to t2
\end{SACCode}

\SACTitle{头段变量改变}
Tn、KTn
