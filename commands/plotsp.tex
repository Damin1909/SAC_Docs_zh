\SACCMD{plotsp}
\label{cmd:plotsp}

\SACTitle{概要}
用多种格式绘制谱数据

\SACTitle{语法}
\begin{SACSTX}
P!LOT!SP [ASIS|RLIM|AMPH|RL|IM|AM|PH] [LINLIN|LINLOG|LOGLIN|LOGLOG]
\end{SACSTX}

\SACTitle{输入}
\begin{description}
\item [ASIS]  按照谱文件当前格式绘制分量
\item [RLIM]  绘制实部和虚部分量 
\item [AMPH]  绘制振幅和相位分量 
\item [RL] 只绘制实部分量 
\item [IM]  只绘制虚部分量 
\item [AM]  只绘制振幅分量 
\item [PH]  只绘制相位分量 
\item [LINLIN|LINLOG|LOGLIN|LOGLOG] 设置x-y轴为线型还是对数型,与单独的LINLIN等命令区分开
\end{description}

\SACTitle{缺省值}
\begin{SACDFT}
plotsp asis loglog
\end{SACDFT}

\SACTitle{说明}
SAC数据文件可能包含时间序列文件或谱文件,IFTYPE决定文件所有哪种类型。多数绘图命令只能对时间序列文件起作用,这个命令则可以绘制谱文件。

你可以使用这个命令绘制一或两个分量。每一个分量绘制在一张图上。你也可以设置横纵坐标为线型或对数型,其仅对该命令有效.

\SACTitle{示例}
获得一个谱文件振幅的对数-线性的绘图:
\begin{SACCode}
SAC> read file1
SAC> fft
SAC> plotsp am loglin
\end{SACCode}

\SACTitle{错误消息}
\begin{itemize}
\item[-]1301: 未读入数据文件
\item[-]1305: 对时间序列的非法操作
\end{itemize}
