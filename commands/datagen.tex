\SACCMD{datagen}
\label{cmd:datagen}

\SACTitle{概要}
产生样本波形数据并储存在内存中

\SACTitle{语法}
\begin{SACSTX}
D!ATA!G!EN! [MORE] [SUB L!OCAL!|R!EGIONAL!|T!ELESEIS!] [filelist]
\end{SACSTX}

\SACTitle{输入}
\begin{description}
\item [MORE] 将新生成的样本数据放在内存中旧文件后。若省略此项,则新数据
    将替代内存中的旧数据
\item [SUB LOCAL|REGIONAL|TELESEIS] 要生成的数据的子类型,每个子类型对应
    不同的样本数据
\item [filelist] 样本数据文件列表。每个子类型可选的文件列表在后面给出
\end{description}

\SACTitle{说明}
SAC提供了一些样本地震数据以供用户学习时使用,该命令将读取一个或多个样本
地震数据到内存中。事实上,该命令与 \nameref{cmd:read} 命令类似,只是该
命令是从特殊的数据目录(\verb|$SACAUX/datagen|)中读取文件。

该命令提供了三种子类型,分别是 \texttt{local}、\texttt{regional} 和
\texttt{teleseis},分别对应近震、区域地震和远震。不同的子类型,其所
包含的数据文件也不同。

\subsubsection*{LOCAL}
该local事件发生在加州的Livermore Valley,是一个很小的无感地震(ML=1.6),
其被Livermore Local Seismic Network (LLSN)所记录。

LLSN拥有一系列垂直分量和三分量台站。该数据集中包含了9个三分量台站的数据。
数据时长40秒,每秒100个采样点。台站信息、事件信息、p波及尾波到时都包含在
头段中,这些文件包括:
\begin{SACCode}
    cal.z, cal.n, cal.e
    cao.z, cao.n, cao.e
    cda.z, cda.n, cda.e
    cdv.z, cdv.n, cdv.e
    cmn.z, cmn.n, cmn.e
    cps.z, cps.n, cps.e
    cva.z, cva.n, cva.e
    cvl.z, cvl.n, cvl.e
    cvy.z, cvy.n, cvy.e
\end{SACCode}

\subsubsection*{REGIONAL}
该区域地震发生在Nevada,被Digital Seismic Network (DSS)所记录。DSS包含了
美国西部的四个宽频带三分量台站。数据包含了从发震前5秒开始的为300秒地震
数据,每秒含40个采样点,文件名为:
\begin{SACCode}
    elk.z, elk.n, elk.e
    lac.z, lac.n, lac.e
    knb.z, knb.n, knb.e
    mnv.z, mnv.n, mnv.e
\end{SACCode}

\subsubsection*{TELESEIS}
该远震事件于1984年9月10日发生在加州北海岸Eureka附近,其为中等偏大的地震
(ML 6.6、MB 6.1、MS 6.7),多地有感。该数据集中包含了Regional Seismic
Test Network (RSTN)的5个台站的中等周期和长周期数据(其中cpk台站的数据
无法获取,sdk台站的长周期数据被截断)。这个数据集的数据时长1600秒,长周
期数据每秒1个采样点,中等周期数据每秒4个采样点。文件包括:
\begin{SACCode}
    ntkl.z, ntkl.n, ntkl.e, ntkm.z, ntkm.n, ntkm.e
    nykl.z, nykl.n, nykl.e, nykm.z, nykm.n, nykm.e
    onkl.z, onkl.n, onkl.e, onkm.z, onkm.n, onkm.e
    sdkl.z, sdkl.n, sdkl.e, sdkm.z, sdkm.n, sdkm.e
\end{SACCode}

\SACTitle{示例}
下面的示例展示了一些基本的用法:
\begin{SACCode}
SAC> dg sub l cal.z    // 单个近震Z分量数据
SAC> dg sub r *.z      // 区域地震多台Z分量数据
SAC> dg sub t sdkl.?   // 远震的单台三分量数据
\end{SACCode}

生成一堆波形数据,并保存数据到磁盘中:
\begin{SACCode}
SAC> dg sub l cdv.e cdv.n cdv.z
SAC> w cdv.e cdv.n cdv.z
\end{SACCode}
在写文件时,需要手动指定文件名列表,当文件很多时,就会变得很麻烦。可以
利用 \nameref{cmd:write} 命令的语法简化这一命令:
\begin{SACCode}
SAC> dg sub local *
SAC> w delete /opt/sac/aux/datagen/local/
\end{SACCode}
\texttt{delete} 选项的作用是从原始的文件名中删去 \texttt{/opt/sac/aux/datagen/local/},
只留下文件名。
