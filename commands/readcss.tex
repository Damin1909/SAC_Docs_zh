\SACCMD{readcss}
\label{cmd:readcss}

\SACTitle{概要}
从磁盘读取CSS格式的文件到内存

\SACTitle{语法}
\begin{SACSTX}
R!EAD!CSS [BINARY|ASCII] [MAX!MEM! v] [MORE] [TRUST ON|OFF] [VER!BOSE! ON|OFF]
    [SHIFT ON|OFF] [SCALE ON|OFF] [MAG!NITUDE! MB|MS|ML|DEF] [DIR name] wfdisclist
    [filelist] [cssoptions]
\end{SACSTX}
其中 \texttt{cssoptions} 用于进一步从 \texttt{wfdisc} 文件中筛选满足
条件的数据文件,\texttt{cssoptions} 可以取:
\begin{SACSTX}
    [STA!TION! station] [CHAN!NEL! channel] [BAND!WIDTH! bandcode]
    [ORIENT!ATION! orientation-code]
\end{SACSTX}

\SACTitle{输入}
\begin{description}
\item [ASCII] 读取ASCII形式的CSS文件(默认值)
\item [BINARY] 读取二进制CSS文件,阅读writecss以了解更多信息
%\item [TRUST ON|OFF] 这个选项用于解决从SAC格式转为CSS格式时出现的冲突。
%    当转化这数据时,比较事件ID可能意味着这些文件有用于可识别的事件信息,
%    或者它们可能是两个非常不同格式的数据人为合并得来的。
%    当设置TRUST为ON时,相较于TRUSRT为OFF,
%    SAC更可能依赖于现在内存中的数据文件相关的READ命令的历史接受比较事件ID为事件识别信息。
\item [MAXMEM] 设定读取大量数据时所能使用的最大内存占物理内存的百分比。
    当使用的内存达到设定的上限时,即使已经读取了其他数据库表,也不会再
    读取更多的波形数据。\texttt{MAXMEN} 的默认值是0.3。
\item [MORE] 将读入的波形数据放在内存中的原有波形之后,若不使用该选项,
    则新读入的波形数据会覆盖内存中的原有波形数据,详情参考 \nameref{cmd:read}
    命令。
\item [VERBOSE ON|OFF] 如果 \texttt{VERBOSE} 是ON,SAC会显示正在读取的
    波形数据的扩展信息,并打印出CSS数据库表的概要信息以及数据格式转换的
    进度信息。
\item [SHIFT ON|OFF] 若 \texttt{SHIFT} 是ON,则发震时刻将被设置为0,其他
    相关时间头段变量也会做相应修改。与震中距相关的一些头段变量也会受影响。
    默认值为 \texttt{SHIFT ON}。
\item [SCALE ON|OFF] \texttt{SCALE} 选项的默认值是OFF。在 \texttt{wfdisc}
    文件中,有一个字段为校准因子 \texttt{CALIB}。当 \texttt{SCALE}选项是
    OFF时,SAC直接从 \texttt{.w} 文件读取数字信号数据,此时数据的单位是counts,
    并将 \texttt{CALIB} 的值保存到SAC头段变量 \texttt{SCALE} 中。当
    \texttt{SCALE} 选项是ON时,SAC会给读取的数据乘以 \texttt{CALIB} 值,
    并设置SAC的头段变量 \texttt{SCALE} 的值为1.0。设置 \texttt{SCALE ON},
    将数据乘以 \texttt{CALIB} 值,在某种程度上可以认为是对数据去除了仪器
    响应,但该方法很粗糙,完整地去除仪器响应应使用 \nameref{cmd:transfer}
    命令。仅当 \nameref{cmd:transfer} 命令所需的仪器响应信息无法获取时,
    才建议使用 \texttt{SCALE ON}。
\item [MAGNITUDE] 指定要将哪一种震级放在SAC的头段变量 \texttt{mag} 中。
    Mb是体波震级,Ms是面波震级,ML是地方震震级。
    默认值是 \texttt{DEF},其算法为:若Ms存在且大于或等于6.6,则最优先用Ms。
    否则,如果Mb存在,用Mb。如果Mb不存在,而Ms存在,用Ms。其他情况用ML。
%\item [COMMIT]如果MORE选项进行了设置,COMMIT选项会把头段和波形记录到SAC的内存中,
%   在读取更多文件前从RAM删除之前任何版本的头段和波形。
%   COMMIT是默认值。
%\item [ROLLBACK] 如果设置了MORE选项,ROLLBACK选项在读取更多文件前恢复到上次记录到的头段和波形的版本。
%\item [RECALLTRACE]如果设置了MORE选项,RECALLTRACE选项:
%   波形回滚到上一个版本,
%   些波形紧密相关的头段变量回滚到上一个版本,
%   记录那些和这些波形不紧密相关的头段变量(可用来获得哪些变量被记录了,哪些则是回滚到上一个版本)。
%   注意:如果MORE选项没有设置,COMMIT、ROLLBACK和RECALLTRACE选项就不发挥作用。
\item [DIR name] \texttt{wfdisc} 文件所在的路径
\item [wfdiscfiles] \texttt{wfdisc} 文件列表
\item [filelist] 若不指定filelist,则 \texttt{wfdisc} 文件所包含的所有
    波形数据都会被读入内存;若指定了filelist,则只有filelist中指定的波形
    数据才会被读取内存。需要注意,filelist所指定的波形文件名必须位于之前
    指定的 \texttt{wfdisc} 文件中。
\item[STATION station] \texttt{station} 是一个6个或更少字符构成的字符串。
    \texttt{wfdisc} 文件中台站名 \texttt{kstnm} 与 \texttt{station} 匹配
    的行会被选中并读取。\texttt{station} 中可以包含通配符 \texttt{*} 和
    \texttt{?} 。
\item[CHANNEL channel]  \texttt{channel} 是一个8个或更少字符构成的字符串。
    \texttt{wfdisc} 文件中通道名与 \texttt{channel} 匹配的行会被选中并读取。
    \texttt{channel} 中可以包含通配符 \texttt{*} 和 \texttt{?} 。
\item [BANDWIDTH bandcode] 单字符编码。常见的取值为E、S、H、B、M、L、V、U、R等。
    \texttt{bandcode} 的具体含义参考附录中表 \ref{tbl:bandcode}。
    \texttt{channel} 字段中第一个字符与 \texttt{bandcode} 匹配的行会被
    选择并读取。\texttt{bandcode} 中使用通配符 \texttt{*} 会匹配所有
    \texttt{bandcode}。
\item [ORIENTATION orientation-code] \texttt{orientation-code}通常可以取
    ``Z N E''(表示竖直、北和东)、``1 2 3''(表示正交但非标准的三个方向)。
    \texttt{channel} 字段中最后一个字符与 \texttt{orientatio-code} 相匹
    配的行会被选中并读取。\texttt{orientation-code} 使用通配符 \texttt{*}
    会匹配所有 \texttt{orientation-code}。
\end{description}

\SACTitle{默认值}
\begin{SACSTX}
readcss ascii maxmem 0.3 verbose off station * band * chan * orient
\end{SACSTX}

\SACTitle{说明}
CSS是一种数据库架构,该命令可以读取CSS 3.0或CSS 2.8中的文件。

每个CSS数据库包含了若干个数据库表表,每个数据库表包含若干个记录。对于
CSS 3.0而言,该命令支持读取如下数据库表:\texttt{wfdisc}、\texttt{wftag}、
\texttt{origin}、\texttt{arrival}、\texttt{assoc}、\texttt{sitechan}、
\texttt{site}、\texttt{affiliation}、\texttt{origerr}、\texttt{origin}、
\texttt{event}、\texttt{sensor}、\texttt{instrument}、\texttt{gregion}、
\texttt{stassoc} 和 \texttt{remark sacdata}。对于CSS 2.8而言,该命令只
支持表 \texttt{wfdisc}、\texttt{arrival} 和 \texttt{origin}。

关于CSS格式的详细介绍,请参考:
\begin{itemize}
\item \url{https://anf.ucsd.edu/pdf/css30.pdf}
\item \url{http://prod.sandia.gov/techlib/access-control.cgi/2002/023055.pdf}
\item \url{ftp://ftp.pmel.noaa.gov/newport/lau/tphase/data/css_wfdisc.pdf}
\end{itemize}

在CSS数据库的众多表中,最常用的是与波形相关的 \texttt{wfdisc} 表以及
波形数据 \texttt{.w} 文件。\texttt{wfdisc} 表中每行代表一个波形记录,
共19列,每列代表了波形记录的不同信息。详情参考上面列出的格式说明文档。

\texttt{readcss} 命令的 \texttt{BINARY} 选项,可以用于读取 \texttt{writecss}
命令生成的二进制CSS格式。在BINARY模式下,\texttt{cssoptions} 选项没有作用,
即 \texttt{wfdisc} 文件中包含的全部波形数据都会被读取。
