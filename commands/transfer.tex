\SACCMD{transfer}
\label{cmd:transfer}

\SACTitle{概要}
去仪器响应,并加上其他仪器响应

\SACTitle{语法}
\begin{SACSTX}
TRANS!FER! [FROM type [options]] [TO type [options]]
    [FREQ!LIMITS! f1 f2 f3 f4] [PREWH!ITENING! ON|OFF|n]
\end{SACSTX}

\SACTitle{输入}
\begin{itemize}
\item FROM type : 要去除的仪器响应类型;
\item TO type : 要加入的仪器响应类型;
\item FREQLIMITS f1 f2 f3 f4:压制大于f4以及低于f1的频段;
\item PREWHITENING ON|OFF|n:预白化处理。
\end{itemize}

\SACTitle{缺省值}
\begin{SACDFT}
trans from none to none
\end{SACDFT}

\SACTitle{说明}
关于仪器响应的详细说明,请参考附录中的``\nameref{chap:resp}''一章。

transfer命令的作用是从波形中去除仪器响应,如果需要,也可以再卷积上某类型的
仪器响应。\verb+from+给出了波形当前的仪器响应,也就是要去除的仪器响应。
\verb+to+给出了要卷积上的仪器响应。

freqlimits用于在去除仪器响应的时候对波形的低频和高频部分进行压制。通常这一选项
是很重要的。这是因为所有的地震仪在零频率处都具有零响应。当做反卷积但不卷积其他
响应的时候(即\verb+to none+),就有必要修改超低频率处的响应以防止频率域除0。
在高频区域,信噪比通常很低,因而有必要对其响应进行抑制。
freqlimites包含了低通和高通的尖灭函数(taper),可以实现高频和低频的有效压制。
四个频率满足f1<f2<f3<f4,尖灭函数在f2和f3间为1,在f1以下和f4以上为0。频率f1和
f2确定了高通滤波器的特性,频率f3和f4指定了低通滤波器的特性。
f1与f2之间以及f3与f4之间分别为余弦波的四分之一个周期。

如果你想要一个低通滤波器但在低频处不滤波,一种方法是设置f1=-2和f2=-1;
如果你想要一个高通滤波器但在高频处不滤波,对于Nyquist频率为0.5,设置f3=10和f4=20。

注意:由于滤波器还有零相位,因而其是非因果。如果数据点数npts不为2的指数幂次,
会导致在频段(f1,f4)之外振幅不为0。如果想要数据点数为2的幂次方个,可以参考SAC中的
\nameref{cmd:cut}命令。

\verb+prewhitening+用于控制数据的预白化。预白化可以将输入时间序列在变换到
频率域之前,进行谱的平化。这会减小谱值的动态范围,并提高数据在高频的计算精度。
参见\nameref{cmd:whiten}命令。打开预白化选项,会在谱操作之前在频率域进行谱白化,
并在谱操作后在时间域做谱白化的补偿,也可以设置预白化选项的阶数。默认情况下,
预白化选项是关闭的,阶数为n=6。

transfer中的from和to后,接具体的仪器类型。在SAC中,可以用5种方式来指定要使用的
仪器类型,即内置仪器类型、特殊内置仪器类型、RESP文件、PZ文件和FAP文件。

\subsubsection{内置仪器类型}
SAC内置了一堆标准仪器的响应函数,可以在命令中直接使用。但是因为日常中其实很少
使用这些内置仪器类型,或者只是用到很少的几个,故而把这些仪器类型放在附录的表
\ref{table:instrument-type}中。

为了从数据中去除LLL宽频带仪器响应,并卷积上SRO仪器响应,且对频带做尖灭及预白化:
\begin{SACCode}
SAC> read abc.z
SAC> rmean; rtr; taper
SAC> trans from lll to sro freq .02 .05 1. 2. prew 2
\end{SACCode}

某些内置仪器类型,还有子类型(subtype)。
当前仪器为RSTN类型的nykm.z子类型,为了从波形数据中去除该仪器响应,并卷积上DSS仪器响应:
\begin{SACCode}
SAC> read nykm.z
SAC> rmean; rtr; taper
SAC> trans from rstn subtype nykm.z to dss prew off
\end{SACCode}

\subsubsection{特殊仪器类型}
SAC中定义了三个特殊的仪器类型,即\verb+none+、\verb+vel+和\verb+acc+,分别表示
位移、速度和加速度。

\begin{itemize}
\item 若使用\verb+to none+,则表示在去除原有仪器响应后,再卷积上none对应的仪器响应。
    而none对应的响应函数在全频段内振幅为1,相位为零,即相当于没有卷积新的仪器响应。
    这样会得到位移波形,同时SAC头段变量idep将被设置为DISPLACEMENT。
\item 若使用\verb+to vel+或\verb+to acc+,则将得到速度场或加速度场。
\item 若使用\verb+from none+,则不会去除仪器响应,原始的地震波形数据认定为位移,
    这常用于向合成地震图中加入仪器响应。\verb+from vel+和\verb+from acc+同理。
\end{itemize}

从波形中去除WWSP的仪器响应,得到位移波形:
\begin{SACCode}
SAC> read xyz.z
SAC> rmean; rtr; taper
SAC> transfer from WWSP to none freq 0.05 0.01 5 10
\end{SACCode}

向合成的位移地震图中加入WWSP仪器响应:
\begin{SACCode}
SAC> r syn.z
SAC> transfer from none to WWSP
\end{SACCode}

\subsubsection{RESP文件}
可以直接用\verb+evalresp fname+后接RESP文件名来指定仪器响应:
\begin{SACCode}
SAC> r 2006.253.14.30.24.0000.TA.N11A..LHZ.Q.SAC
SAC> rmean; rtr; taper
SAC> trans from evalresp fname RESP.TA.N11A..LHZ to none freq 0.004 0.007 0.2 0.4
\end{SACCode}

若未使用fname显式指定RESP文件,evalresp选项会从SAC头段区中提取出台网名、台站名、
位置ID、通道名,并尝试在当前工作目录下去寻找文件名为
~``\verb+RESP.<NET>.<STA>.<LOCID>.<CHAN>+''~的RESP响应文件:
\begin{SACCode}
SAC> r 2006.253.14.30.24.0000.TA.N11A..LHZ.Q.SAC
SAC> rtr; taper
SAC> trans from evalresp to none freq 0.004 0.007 0.2 0.4
            // 在当前目录自动寻找文件RESP.TA.N11A..LHZ
\end{SACCode}

一个RESP文件可以包含多个响应函数,既可以是同一仪器在不同时期的响应函数,也可以是
多个台站的响应函数。transfer命令会从波形中的头段区提取出相关信息,并从RESP文件
中找出全部信息都匹配的响应函数。
\begin{SACCode}
SAC> r 2013.144.05.40.09.0195.IU.COLA.00.BHZ.M.SAC
SAC> rmean; rtr; taper
SAC> transfer from evalresp fname RESP.ALL to none freq 0.1 0.2 5 10
        // RESP.ALL中包含了多个台站的响应函数
\end{SACCode}

也可以通过给evalresp选项指定额外的选项和参数来覆盖SAC文件中对应的头段变量,
这些可能的选项包括:STATION、CHANNEL、NETWORK、DATE、TIME、LOCID。
每个选项都必须有一个合适的值,若DATE在SAC头段中为设定且在选项中未指定,
则使用当前系统日期,TIME同理。
若NETWORK未指定,则默认使用任意台站名;若LOCID未指定,则默认使用任意LOCID。

从文件中去除仪器响应,并卷积上COLB台站的仪器响应:
\begin{SACCode}
SAC> r 2013.144.05.40.09.0195.IU.COLA.00.BHZ.M.SAC
SAC> rtr; taper
SAC> trans from evalresp to evalresp station COLB
    // 当前目录下必须有COLA和COLB两个台站的仪器响应文件
    // 该命令会去除RESP.IU.COLA.00.BHZ的响应函数
    // 并卷积RESP.IU.COLB.00.BHZ的响应函数
    // PS: COLB台站是我瞎编的
\end{SACCode}

为了显示IU台网COL台站BHZ通道,1992年01月02日16:42:05的仪器响应:
\begin{SACCode}
SAC> fg impulse npts 16384 delta .05 begin 0.
SAC> trans to evalresp sta COL cha BHZ net IU date 1992/2 time 16:42:05
SAC> fft
SAC> psp am
\end{SACCode}

上面给了RESP文件的一堆例子,日常输出处理时最常用的还是直接用fname指定RESP响应
文件。

\subsubsection{PZ文件}
可以使用\verb+polezero subtype+后接PZ文件的方式来指定响应文件。
\begin{SACCode}
SAC> r XXX.Z
SAC> rmean; rtr; taper
SAC> trans from polezero subtype XXX.PZ to WWSP
\end{SACCode}

PZ文件中可以包括多个台站的PZ文件。假设用rdseed在当前目录产生了多个台站的PZ文件,
并将所有PZ文件合并得到一个新的PZ文件event.pz。下面的例子将读入全部BH*波形文件,
经仪器响应校正并覆盖原文件:
\begin{SACCode}
SAC> r *BH*SAC          // 读入全部数据
SAC> rtr
SAC> taper
SAC> trans from polezero s event.pz to none freq 0.05 0.1 10.0 15.0
SAC> mul 1.0e9
SAC> w over
\end{SACCode}

需要格外注意,在用PZ文件去仪器响应得到位移物理量时,得到的数据的单位是~\verb+m+,
而SAC中默认的单位是~\verb+nm+,因而需要将数据乘以10的9次方将数据的单位转换成~\verb+nm+。
对于转换得到速度或加速度,同理。

虽然没有进过验证,但根据经验做如下合理猜测:当把多个PZ文件合并成一个总PZ文件时,
PZ文件中的注释行就很重要了,transfer命令会从中提取相关信息与波形的头段信息进行
匹配,以找到合适的响应函数。因为对每个波形,都需要到总PZ文件中寻找匹配的响应
函数,所以上例所示的方法,虽然可以很简单地批量去仪器响应,但效率可能不高,尤其
是当台站数目很多、PZ文件较大的时候。

\subsubsection{FAP文件}
FAP文件不太常用,这里仅举一例。

假设有fapfile文件~\verb+fap.n11a.lhz_0.006-0.2+,其名字表示频率段为0.006 Hz到0.2 Hz,
要从波形~\verb+2006.253.14.30.24.0000.TA.N11A..LHZ.Q.SAC+中移除该仪器响应:
\begin{SACCode}
SAC> r 2006.253.14.30.24.0000.TA.N11A..LHZ.Q.SAC
SAC> rtr
SAC> taper
SAC> trans from fap s fap.n11a.lhz_0.006-0.2 to none freq 0.004 0.006 0.1 0.2
SAC> mul 1.0e9
\end{SACCode}
