\SACCMD{mlm}
\label{spe:mlm}

\SACTitle{概要}
利用最大似然方法计算谱估计

\SACTitle{语法}
\begin{SACSTX}
MLM [O!RDER! n] [N!UMBER! n]
\end{SACSTX}

\SACTitle{输入}
\begin{description}
\item [ORDER n] 设置预测误差滤波器的时滞阶数为n
\item [NUMBER n] 设置用于谱估计的点数
\end{description}

\SACTitle{缺省值}
\begin{SACDFT}
mlm order 25
\end{SACDFT}

\SACTitle{说明}
该命令实现了能量密度谱的最大似然法。该方法生成的谱估计能够代表一个窄带通滤波器的
能量输出,最终得到一个平滑的、参数化的能量密度谱,这些参数是有限脉冲响应窄带滤波器
的系数。用户可以指定这些参数。

该方法的特点在于其一般比传统方法有更高的分辨率。算法的阶数限制在100,因为它需要
一个维度与阶数相同的矩阵的逆。对于这个求逆运算,存在更快的方法,但对于大阶数估计
可能存在数值噪声。

\SACTitle{相关命令}
\nameref{spe:cor}、\nameref{spe:writespe}、\nameref{spe:plotspe}
