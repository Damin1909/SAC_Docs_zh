%
% LaTeX配置文件
%
\documentclass[UTF8, a4paper, 11pt, twoside, punct = CCT]{ctexbook}
% 页面设置
\usepackage[top=3.0cm, bottom=2.0cm, left=3.5cm, right=2.5cm]{geometry}
\ctexset{
    contentsname = {目 \quad 录},
    listfigurename = {图目录},
    listtablename = {表目录},
    chapter/number = \arabic{chapter},
    section/format+ = \raggedright,
    subsection/beforeskip = 0.25ex,
    subsection/afterskip = 0.25ex,
    subsubsection/beforeskip = -.1ex,
    subsubsection/afterskip = .1ex,
}
\addtolength{\parskip}{3pt}             % 段落间距
%设置SACTitle为subsection格式
\newcommand{\SACTitle}[1]{\subsection*{#1}}
\newcommand{\SACCMD}[1]{\section{\texttt{#1}}}

% 文档相关信息
\newcommand{\SACDOCTITLE}{SAC\textbf{参考手册}} % 文档标题
\newcommand{\SACDOCAUTHOR}{SeisMan}             % 文档作者
\newcommand{\SACDOCVERSION}{3.6-dev}            % 文档版本
\newcommand{\SACDOCDATE}{\today}                % 文档更新日期
\newcommand{\SACVERSION}{101.6a}                % SAC版本

\usepackage{datetime2}      % \today格式为YYYY-MM-DD

\usepackage{titletoc}           % 目录设置
\setcounter{tocdepth}{2}        % 目录深度
\titlecontents{chapter}[0em]    % 章
    {\vspace{0.2em}\bfseries\large}
    {\thecontentslabel\quad}
    {\hspace*{0em}}
    {\hfill \contentspage}
\titlecontents{section}[1em]    % 节
    {\normalsize}
    {\thecontentslabel\quad}
    {\hspace*{0em}}
    {\ \dotfill \ \contentspage}
    [\vspace{-0.3em}]
\titlecontents{subsection}[3em] % 小节
    {\small}
    {\thecontentslabel\quad}
    {\hspace*{0em}}
    {\ \dotfill \ \contentspage}
    [\vspace{-0.3em}]

% 双栏目录
\usepackage{multicol}
\makeatletter
\renewcommand{\tableofcontents}{%
\setlength{\columnsep}{2.5em}
\begin{multicols}{2}[\chapter*{\contentsname}]%
    \@starttoc{toc}%
\end{multicols}}
\makeatother

% 页眉页脚设置
\usepackage{titleps}
\newpagestyle{body}[\small]{
    \sethead
    [$\cdot$~\thepage~$\cdot$][][\S\,\thesection\quad\sectiontitle]
    {\CTEXthechapter\quad\chaptertitle}{}{$\cdot$~\thepage~$\cdot$}
    \setfoot{}{}{}\headrule
}

% 空白页
\makeatletter	% copy from lnotes
\def\cleardoublepage{
    \clearpage
    \if@twoside
        \ifodd
            \c@page
        \else
            \hbox{}
            \vspace*{\fill}
            \begin{center}
		    保护环境,从阅读电子文档开始!
            \end{center}
            \vspace{\fill}
            \thispagestyle{empty}
            \newpage
            \if@twocolumn
                \hbox{}
                \newpage
            \fi
        \fi
    \fi
}
\makeatother

% 超链接及书签
\usepackage[
    CJKbookmarks=true,
    colorlinks=true,
    linkcolor=blue,
    citecolor=blue,
    urlcolor=blue
]{hyperref}
\hypersetup{ % 文档元信息
    pdftitle={\SACDOCTITLE v\SACDOCVERSION},
    pdfauthor={\SACDOCAUTHOR},
}

% 代码宏包
\usepackage{listings}
\usepackage[usenames, dvipsnames, svgnames]{xcolor} % 用于code
\lstset{
    basicstyle=\footnotesize\ttfamily,
    xleftmargin=2pc,                    % 整体布局
    xrightmargin=2pc,
    backgroundcolor=\color{Lavender},   % 背景色
    rulecolor=\color{Silver},           % 边框颜色
    frame=single,                       % 边框
    columns=flexible,
}
\lstdefinelanguage{SAC} {
    keywords={SAC>},   % SAC提示符
    otherkeywords={SAC>, SAC/SSS>},   % SAC提示符
    sensitive=true,
    comment=[l][{\color[rgb]{0,0.4,0}}]{//},
}
\lstnewenvironment{SACCode}{
    \lstset{
        language={SAC},                     % 语言
        keywordstyle=\color{blue},          % 关键字颜色
    }
}{}
\lstnewenvironment{SACSTX}{
    \lstset{delim=[is][\textcolor{gray}]{!}{!}}
}{}
\lstnewenvironment{SACDFT}{}{}

\usepackage{minted}
\setminted{linenos, frame=leftline, xleftmargin=2pc, numbersep=6pt, breaklines=true}
\setminted[console]{linenos=false, frame=none, fontsize=\small}

% 解决代码复制问题
% http://tex.stackexchange.com/questions/83204/
\usepackage{accsupp}
\newcommand\emptyaccsupp[1]{\BeginAccSupp{ActualText={}}#1\EndAccSupp{}}
\let\theHFancyVerbLine\theFancyVerbLine
\def\theFancyVerbLine{\rmfamily\tiny\emptyaccsupp{\arabic{FancyVerbLine}}}

\usepackage{enumitem}	% 列表宏包
% itemsep 设置列表间距; topsep 设置列表前间距
\setenumerate{
    itemsep=-3pt, partopsep=0pt, parsep=\parskip, topsep=0pt
}
\setitemize{
    itemsep=-3pt, partopsep=0pt, parsep=\parskip, topsep=0pt
}
\setdescription{
    itemsep=-3pt, partopsep=0pt, parsep=\parskip,
    topsep=0pt, itemindent=0pt
}

\usepackage{float}      % 图表浮动体

% 图片
\usepackage{graphicx}
\graphicspath{{figures/}}
\usepackage{tikz}
\usepackage{tikz-3dplot}

\usepackage{longtable}	% 长表格
\usepackage{booktabs}   % 三线表

% 图表标题
\usepackage[labelfont={small,bf}, textfont=small]{caption}
\captionsetup[figure]{aboveskip=6pt, belowskip=-12pt}
\captionsetup[table]{aboveskip=6pt, belowskip=-6pt}

% 数学相关
\usepackage{siunitx}    % 数字与单位
\sisetup{per-mode = symbol}

% 符号
\usepackage{pifont}

\usepackage[perpage]{footmisc}	% 脚注在每一页单独编号
% 脚注中verb http://tex.stackexchange.com/questions/203
\usepackage{fancyvrb}
\VerbatimFootnotes

% 自定义quote环境
% http://tex.stackexchange.com/questions/16964/
\usepackage{framed}
\newcommand*\openquote{\makebox(25,-22){\scalebox{5}{``}}}
\newcommand*\closequote{\makebox(25,-22){\scalebox{5}{''}}}
\colorlet{shadecolor}{Azure}
\makeatletter
\newif\if@right
\def\shadequote{\@righttrue\shadequote@i}
\def\shadequote@i{\begin{snugshade}\begin{quote}\openquote}
    \def\endshadequote{%
\if@right\hfill\fi\closequote\end{quote}\end{snugshade}}
\@namedef{shadequote*}{\@rightfalse\shadequote@i}
\@namedef{endshadequote*}{\endshadequote}
\makeatother

% https://github.com/ElegantLaTeX/elegant/blob/master/elegantnote.cls
\usepackage{manfnt}
\newenvironment{note}{
    \par\ttfamily\itshape\noindent{
        \makebox[0pt][r]{
            \scriptsize\color{red!90}\textdbend\quad
        }\textbf{Note:}
    }
}{\par\vspace{.5\baselineskip}}
