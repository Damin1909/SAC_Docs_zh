SAC提供的命令可以帮助用户实现地震数据的预处理,但无法实现所有的数据分析功能。
日常科研中会需要自己写一套算法对数据进行处理,以得到想要的结果。这就需要
能够在自己的程序中读写SAC文件,即SAC I/O。

根据SAC格式的定义,SAC格式文件分为固定长度的头段区和非固定长度的数据区,通常
的做法是先读入头段区,然后从中提取出数据点数等信息,然后再据此读入数据区,
由此即可实现SAC数据的读写。

这一章将介绍如何在C、Fortran、Matlab、Python中读写SAC文件。
