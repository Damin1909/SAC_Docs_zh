\section{C程序中的SAC I/O}
SAC的官方库中提供了一系列用于读写SAC文件的子函数,可以直接在C程序中调用。
但这些子程序用起来不太方便,比如:
\begin{itemize}
\item 函数参数太多太复杂,有些参数基本不会用到,但还是需要定义;
\item 读文件时无法只读取部分文件,即没有截窗的功能;
\item 要获取某个头段变量的值,必须单独调用相应的子函数;
\end{itemize}

在了解了SAC文件的具体格式后,可以很容易的写一套函数来实现SAC文件的读写。

Prof. Lupei Zhu实现了一套相对比较易用的SAC I/O函数库,可以在C或Fortran程序
中直接调用,姑且称之为~\verb+sacio+。

\verb+sacio+函数库与Prof. Lupei Zhu的其他程序一起发布。你可以从Prof. Lupei Zhu
的主页下载~\verb+fk+\footnote{\url{http://www.eas.slu.edu/People/LZhu/downloads/fk3.2.tar}}软件包,
并从中提取出源文件~\verb+sac.h+和~\verb+sacio.c+。

\verb+sacio+简单易用,但也存在一些潜在的Bug及缺陷。我在~\verb+sacio+的基础上重写了
SAC I/O函数库以及SAC相关工具,项目地址为~\url{https://github.com/seisman/sac_tools}。
该项目目前属于玩具性质,仅供有兴趣的读者参考。

\subsection{sacio函数接口}
\verb+sac.h+中SAC格式的头段区被定义为~\verb+SACHEAD+类型的结构体,每一个头段
变量都是结构体的成员。\verb+sacio.c+定义了一系列用于读写SAC文件的函数,
表~\ref{table:sacio-function}~中列出了~\verb+sacio+提供的函数接口。

\begin{table}[H]
\centering
\caption{sacio函数列表}
\label{table:sacio-function}
\begin{tabular}{ll}
\toprule
函数名      &   说明        \\
\midrule
\verb+read_sachead+     &   仅读取SAC文件的头段部分 \\
\verb+read_sac+         &   读取整个SAC \\
\verb+read_sac2+        &   读取SAC文件中一部分 \\
\verb+write_sac+        &   将数据写到SAC文件中 \\
\verb+sachdr+           &   初始化SAC头段\\
\bottomrule
\end{tabular}
\end{table}

调用SAC I/O接口的程序,可以通过如下方式编译:
\begin{minted}{bash}
$ gcc -c sacio.c
$ gcc -c prog.c
$ gcc sacio.o prog.o -o prog -lm
\end{minted}

写成Makfile会更简单一些:
\begin{minted}{make}
all: readsachead clean

readsachead: sacio.o readsachead.o
    $(CC) -o $@ $^ -lm

clean:
    -rm *.o
\end{minted}

\subsection{读写SAC文件}
下面的例子展示了如何在C程序中读取一个SAC二进制文件,经过简单的数据处理后,
最后写回到原文件中:
\inputminted{C}{./sacio/readsac.c}
\verb+read_sac+函数将SAC文件的头段区保存到结构体~\verb+SACHEAD hd+中,
可以通过~\verb+hd.npts+、\verb+hd.delta+这样的方式引用头段变量的值。SAC文件的
数据区保存到指针~\verb+float *data+中,\verb+read_sac+会根据头段中的数据点数
为指针~\verb+data+分配内存空间并读入数据,在用完之后要记得用~\verb+free(data)+释放
已分配的内存,以避免内存泄露。

\subsection{仅读取SAC头段区}
有些时候只需要SAC文件的头段区的一些信息,此时若读取整个文件就有些浪费了,
\verb+read_sachead+仅读取SAC头段区而不读取数据区。
\inputminted{C}{./sacio/readsachead.c}

\subsection{读SAC数据的一段}
有些时候,数据可能很长,但用户只需要其中的一小段。为了读取一小段数据而把整个
文件都读入进去实在太浪费了。SAC中的~\nameref{cmd:cut}~命令可以实现数据的
截取,\verb+read_sac2+则是实现了类似的功能。下面的例子截取了数据中以~\verb+T0+
为参考的$-0.5$到$2.5$秒,即相当于~\verb+cut t0 -0.5 2.5+。
\inputminted{C}{./sacio/readsac2.c}
\verb+read_sac2+与~\verb+read_sac+相比,多了三个用于定义时间窗的参数,
其中\verb+tmark+表示参考时间标记,可以取值为:
\begin{description}
\item[-5] 以~\verb+B+作为时间标记
\item[-3] 以~\verb+O+作为时间标记
\item[-2] 以~\verb+A+作为时间标记
\item[0-9] 以~\verb+T0~T9+中的某个为时间标记
\end{description}
这里只读入了数据的一小部分数据,但子程序中对头段中的~\verb+B+、\verb+E+、\verb+NPTS+等
做了修改,因而返回的头段与数据是相互兼容的。

\begin{note}
tmark的有效取值为什么是-5、-3、-2、0~9?

在头段区中,若以T0作为参考位,T3是T0后的第3个变量,T7是T0后的第7个变量,
B是T0前的第5个变量,O是T0前的第3个变量。

\verb+sacio.c+的代码中使用了~\verb|tref = *((float *)hd + 10 + tmark)|来获取某个
时间头段变量的值。首先将结构体指针~\verb+hd+强制转换成~\verb+float+型,加上~\verb+10+使得
指针指向结构体的成员T0所在的位置,然后tmark取不同的值,则将指针相应地前移或
后移相应的浮点变量,最后取指针所在位置的浮点变量值。由此即可很简单地获取某个
时间标记头段中的值。这种用法在自己的程序中也会经常用到。
\end{note}

\subsection{从零创建SAC文件}
在做合成数据时,经常需要从无到有完全创建一个SAC文件。这相对于一般的“读->处理->写”
而言要更复杂一些,因为必须首先构建一个基本的头段区。下面的例子展示了如何用~\verb+sachdr+构建
一个最基本的头段区,并填充其他一些头段,最后将创建的头段及数据写入到文件中。
\inputminted{C}{./sacio/write_sac.c}
