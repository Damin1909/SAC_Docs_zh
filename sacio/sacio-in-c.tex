\section{C程序中的SAC I/O}
SAC自带的函数库中提供了一系列用于读写SAC文件的子函数,具体细节可以参考
``\nameref{sec:libsacio}''一节,这些子函数可以直接在C程序中调用。
但这些子函数用起来不太方便,比如:
\begin{itemize}
\item 函数参数太多太复杂,有些参数基本不会用到,但还是需要定义;
\item 读文件时无法只读取部分文件,即没有截窗的功能;
\item 要获取某个头段变量的值,必须单独调用相应的子函数;
\end{itemize}

在了解了SAC文件的具体格式后,可以很容易的写一套函数来实现SAC文件的读写。

Prof. Lupei Zhu实现了一套相对比较易用的SAC I/O函数库,可以在C或Fortran
程序中直接调用,姑且称之为 !sacio!。

!sacio! 函数库与Prof. Lupei Zhu的其他程序一起发布。你可以从
Prof. Lupei Zhu的主页下载 !fk! \footnote{
\url{http://www.eas.slu.edu/People/LZhu/downloads/fk3.2.tar}}软件包,
并从中提取出源文件 !sac.h! 和 !sacio.c!。

!sacio! 简单易用,但也存在一些潜在的Bug及缺陷。seisman在
!sacio! 的基础上重写了SAC I/O函数库以及SAC相关工具,项目地址为
\url{https://github.com/seisman/sac_tools}。该项目目前属于玩具性质,
仅供有兴趣的读者参考。

\subsection{sacio函数接口}
!sac.h! 中SAC格式的头段区被定义为 !SACHEAD! 类型的结构体,
每一个头段变量都是结构体的成员。!sacio.c! 定义了一系列用于读写
SAC文件的函数,表 \ref{table:sacio-function} 中列出了 !sacio!
提供的函数接口。

\begin{table}[H]
\centering
\caption{\texttt{sacio}函数列表}
\label{table:sacio-function}
\begin{tabular}{ll}
\toprule
函数名      &   功能        \\
\midrule
!read_sachead!      &   仅读取SAC文件的头段部分 \\
!read_sac!          &   读取整个SAC文件 \\
!read_sac2!         &   读取SAC文件中一部分 \\
!write_sac!         &   将数据写到SAC文件中 \\
!sachdr!            &   初始化SAC头段\\
\bottomrule
\end{tabular}
\end{table}

调用SAC I/O接口的程序,可以通过如下方式编译:
\begin{minted}{console}
$ gcc -c sacio.c
$ gcc -c prog.c
$ gcc sacio.o prog.o -o prog -lm
\end{minted}

写成Makfile会更简单一些:
\begin{minted}{make}
all: prog clean

prog: sacio.o prog.o
    $(CC) -o $@ $^ -lm

clean:
    -rm *.o
\end{minted}

\subsection{读写SAC文件}
下面的示例展示了如何在C程序中读取一个SAC二进制文件,经过简单的数据处理后,
最后写回到原文件中:
\inputminted{C}{./sacio/readsac.c}
!read_sac! 函数将SAC文件的头段区保存到结构体 !SACHEAD hd! 中,
可以通过 !hd.npts!、!hd.delta! 这样的方式引用SAC头段变量的值。
SAC文件的数据区保存到指针/数组 !float *data! 中,!read_sac!
会根据头段中的数据点数为指针 !data! 分配内存空间并读入数据,在用完
之后要记得用 !free(data)! 释放已分配的内存,以避免内存泄露。

\subsection{仅读取SAC头段区}
有时候只需要SAC文件头段区的一些信息,若读取整个文件就有些浪费了,
!read_sachead! 仅读取SAC头段区而不读取数据区。
\inputminted{C}{./sacio/readsachead.c}

\subsection{读SAC数据的一段}
有些时候,数据可能很长,但用户只需要其中的一小段。为了读取一小段数据
而把整个文件都读入进去实在太浪费了。SAC中的 \nameref{cmd:cut} 命令可以
实现数据的截取,!read_sac2! 则是实现了类似的功能。下面的例子截取了
数据中以 !T0!为参考的$-0.5$到$2.5$秒,即相当于 !cut t0 -0.5 2.5!。
\inputminted{C}{./sacio/readsac2.c}
!read_sac2! 与 !read_sac! 相比,多了三个用于定义时间窗的参数,
其中!tmark! 表示参考时间标记,可以取值为:
\begin{description}
\item[-5] 以 !B! 作为时间标记
\item[-3] 以 !O! 作为时间标记
\item[-2] 以 !A! 作为时间标记
\item[0--9] 以 !T0-T9! 中的某个为时间标记
\end{description}
这里只读入了数据的一小部分数据,但子程序中对头段中的 !b!、!e!、
!npts! 等做了修改,因而返回的头段与数据是相互兼容的。

\begin{note}
tmark的有效取值为什么是-5、-3、-2、0-9?

在头段区中,若以T0作为参考位,T3是T0后的第3个变量,T7是T0后的第7个变量,
B是T0前的第5个变量,O是T0前的第3个变量。

!sacio.c! 的代码中使用了 !tref = *((float *)hd + 10 + tmark)!
来获取某个时间头段变量的值。首先将结构体指针 !hd! 强制转换成
!float! 型,加上 !10! 使得指针指向结构体的成员 !T0!
所在的位置,然后 !tmark! 取不同的值,则将指针相应地前移或后移
相应的浮点变量,最后取指针所在位置的浮点变量值。由此即可很简单地获取某个
时间标记头段中的值。这种用法在自己的程序中也会经常用到。
\end{note}

\subsection{从零创建SAC文件}
在做合成数据时,经常需要从无到有完全创建一个SAC文件。这相对于一般的
``读$\rightarrow$处理$\rightarrow$写''而言要更复杂一些,因为必须首先
构建一个基本的头段区。下面的例子展示了如何用 !sachdr! 构建一个
最基本的头段区,并填充其他一些头段,最后将创建的头段及数据写入到文件中。
\inputminted{C}{./sacio/write_sac.c}
