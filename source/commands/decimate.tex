\SACCMD{decimate}
\label{cmd:decimate}

\SACTitle{概要}
对数据做减采样

\SACTitle{语法}
\begin{SACSTX}
DEC!IMATE! [n] [F!ILTER! ON|OFF]
\end{SACSTX}

\SACTitle{输入}
\begin{description}
\item [n] 设置减采样因子为n,即每n个点中取一个点,n取值范围为2到7
\item [FILTER ON|OFF] 打开/关闭抗混叠FIR滤波器
\end{description}

\SACTitle{缺省值}
\begin{SACDFT}
decimate 2 filter on
\end{SACDFT}

\SACTitle{说明}
此命令用于对内存中的数据进行减采样,减采样因子n表示从每n个数据点中取
一个点,因而经过减采样之后的数据点数近似为$npts/n$个。减采样因子的允许
取值为2到7,为了得到更大的减采样因子,可以多次执行该命令。

根据采样定理:
\begin{quote}
如果信号是带限的,并且采样频率大于信号带宽的2倍,那么,原来的连续信号
可以从采样样本中完全重建出来。
\end{quote}
若不满足此采样条件,采样后信号的频率就会重叠,即高于采样频率一半的频率
成分将被重建成低于采样频率一半的信号。这种频谱重叠导致的失真称为混淆
效应。

该命令提供了一个可选的FIR滤波器对数据进行低通滤波,以避免减采样过程中
可能出现的混淆效应。可用的的FIR滤波器的具体参数位于 !$SACHOME/aux/fir/decn!
中,这些滤波器是经过精心设计的,保留了相位信息。使用FIR滤波器有时会在
数据的两端产生瞬时跳变,因而减采样的结果需要在图形界面下人工审核。只有
当高频响应的准确度不重要的时候(比如绘图时),才可以关闭FIR滤波器。

\SACTitle{示例}
对数据减采样42倍:
\begin{SACCode}
SAC> r file1
SAC> decimate 7     // 减采样因子为7时FIR滤波器偶尔不稳定,慎用!
SAC> decimate 6
\end{SACCode}

\SACTitle{头段变量}
npts、delta、e、depmin、depmax、depmen
