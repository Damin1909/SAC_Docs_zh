\SACCMD{plot1}
\label{cmd:plot1}

\SACTitle{概要}
绘制多波形多窗口图形

\SACTitle{语法}
\begin{SACSTX}
P!LOT!1 [A!BSOLUTE!|R!ELATIVE!] [P!ERPLOT! ON|OFF|n]
\end{SACSTX}

\SACTitle{输入}
\begin{description}
\item [ABSOLUTE|RELATIVE] 绝对/相对绘图模式
\item [PERPLOT n] 每张图上绘制 !n! 个文件
\item [PERPLOT ON] 每张图绘制 !n! 个文件,使用 !n! 的旧值
\item [PERPLOT OFF] 所有文件绘制在一张图上
\end{description}

\SACTitle{缺省值}
\begin{SACDFT}
plot1 absolute perplot off
\end{SACDFT}

\SACTitle{说明}
!plot1! 用于一次性绘制多个波形,多个波形共用同一个X轴,但各自拥有
一个单独的Y轴。绘图的总尺寸由当前视口决定(\nameref{cmd:xvport} 和
\nameref{cmd:yvport})。每一个子图的大小由视口大小以及要绘制的波形数目
决定。绘图的X轴范围可以是固定的(\nameref{cmd:xlim})也可以是与数据长度
成比例的。每个子图的Y轴范围由文件极值决定或者可以通过 \nameref{cmd:ylim}
命令自己设置。

多个波形共用X轴时,有 !absolute! 和 !relative! 两种绘图
模式。在 !absolute! 模式下,所有波形将按照其绝对时刻对齐,即X轴
表示相对于某个固定参考时刻的秒数;在 !relative! 模式下,所有数据
将按照文件开始时间对齐,X轴的范围为0到最大时间差(所有文件中结束时间和
开始时间的最大时间差,即最大 !e-b!),每个波形从X轴的零点开始绘制,
该零点所对应的真实时刻,会在图中以 !OFFSET: xxx! 的形式标出。

\SACTitle{示例}
下面的例子是由LLNL DSS的4个台站Elko、Kanab、Landers和Mina记录到的美国
西部的一个地震。参考时间为事件发生时刻:
\begin{SACCode}
SAC> cut -5 200
SAC> read *v
 elk.v knb.v lac.v mnv.v
SAC> fileid location ul type list kstcmp
SAC> title 'regional earthquake:  &1,kztime&  &1,kzdate&'
SAC> qdp 2000
SAC> plot1
\end{SACCode}
