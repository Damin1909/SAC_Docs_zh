\SACCMD{arraymap}
\label{cmd:arraymap}

\SACTitle{概要}
利用SAC内存中的所有文件产生一个台阵或联合台阵的分布图

\SACTitle{语法}
\begin{SACSTX}
ARRAY!MAP! [A!RRAY!|C!OARRAY!]
\end{SACSTX}

\SACTitle{输入}
\begin{description}
\item [ARRAY] 根据头段变量中的偏移X、Y值绘制台站分布
\item [COARRAY] 根据各台站之间的相对坐标绘制台站分布图
\end{description}

\SACTitle{缺省值}
\begin{SACDFT}
arraymap array
\end{SACDFT}

\SACTitle{头段数据}
下面的两个头段变量必须使用SAC宏文件 !wrxyz! 或者与之功能相似的
其他函数提前设定,所有的偏移是相对于某个参考点的千米数。
\begin{itemize}
\item !USER7!:向东的偏移(X)
\item !USER8!:向北的偏移(Y)
\end{itemize}

\SACTitle{说明}
不是很清楚这个命令的作用是什么,对于每个数据来说,需要用宏文件 !wrxyz!
\footnote{位于 !$SACAUX/macros! 目录中}定义头段变量 !user7!
和 !user8!,然后才能利用该命令绘制出arraymap,从命令的名字来理解,
应该是绘制某个台站的台站分布图,理论上只需要台站的真实位置即可。不知这个
究竟在什么场合要使用。

\SACTitle{限制}
在 \nameref{cmd:bbfk} 中允许的最多台站数
