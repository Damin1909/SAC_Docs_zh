\SACCMD{plotpk}
\label{cmd:plotpk}

\SACTitle{概要}
绘图并拾取震相到时

\SACTitle{语法}
\begin{SACSTX}
P!LOT!PK [P!ERPLOT! ON|OFF|n] [B!ELL! ON|OFF] [A!BSOLUTE!|R!ELATIVE!]
    [REF!ERENCE! ON|OFF|v] [M!ARKALL! ON|OFF] [S!AVELOC! ON|OFF]
\end{SACSTX}

\SACTitle{输入}
\begin{description}
\item [PERPLOT n] 一张图绘制 !n! 个文件
\item [PERPLOT ON] 一张图绘制 !n! 个文件,使用 !n! 的旧值
\item [PERPLOT OFF] 一张图上绘制所有文件
\item [BELL ON|OFF] 绘图区内击键时是否响铃
\item [ABSOLUTE|RELATIVE] 绝对/相对绘图模式
\item [REFERENCE ON|OFF] 是否显示参考线
\item [REFERENCE v] 显示参考线,并设置参考值为 !v!
\item [MARKALL ON] 一次标记一张图上所有文件的到时
\item [MARKALL OFF] 只标记光标位置所对应的震相到时
\item [SAVELOCS ON|OFF] 是否将 !L! 命令(表 \ref{table:plotpk-commands})
    拾取的位置保存到黑板变量中
\end{description}

\SACTitle{缺省值}
\begin{SACDFT}
plotpk perplot off absolute reference off markall off savelocs off
\end{SACDFT}

\SACTitle{说明}
执行该命令后,即进入 !ppk! 模式,可用于手动拾取震相。在 !ppk!
模式下,键盘的输入将被解释为ppk命令(表 \ref{table:plotpk-commands})。
详情请参考 \nameref{sec:phase-picking} 一节。

!PERPLOT! 选项用于控制每张图上显示的文件数目,在标记三分量数据时
通常设置为``!p 3!'',使得一次显示一个台站的三分量数据。

在一张图上绘制多个波形时,所有波形共用X轴,此时存在 !absolute! 和
!relative! 两种绘图模式。在 !absolute! 模式下,所有波形将
按照其绝对时刻对齐,即X轴表示对于\textbf{某个固定参考时刻}的秒数;在 !relative!
模式下,所有波形将按照自己的参考时刻对齐,即X轴表示相对于\textbf{各自参考时刻}的
秒数。

在标记震相到时时,默认只会标记光标位置所对应的波形文件。!MARKALL!
选项可以一次标记当前绘图窗口内的所有波形,该选项会获取光标位置所对应的
横坐标值,并将其标记到当前绘图窗口内所有波形的头段变量中。该选项常用于
一次性拾取单个台站三分量的震相到时,需要注意的是,若三分量数据的参考
时刻不同,使用 !MARKALL! 标记的到时则是错的。

若使用 !SAVELOCS! 选项,则会将 !L! 命令所拾取的当前光标
位置会保存到黑板变量中:
\begin{itemize}
\item !NLOCS!:每次执行 !ppk! 命令时,该变量初始化为0,
    每使用 !L! 命令拾取一次光标位置,其值加1
\item !XLOCn!:保存第 !n! 次光标拾取的位置的X值;若参考
    时刻 !kzdate! 和 !kztime! 有定义,则 !XLOCn!
    保存绝对时刻的值,否则保存时间偏移的秒数
\item !YLOCn!:保存第 !n! 次光标拾取的位置的Y值
\end{itemize}

\SACTitle{BUGS}
\begin{itemize}
\item !BELL! 选项在某些平台上不可用(v101.6a)
\item !REFERENCE! 选项无效(v101.6a)
\item 按照说明文档,绘图时默认使用 !absolute! 模式,但似乎代码存在
    Bug。某些情况下默认使用 !relative! 模式,但在查看若干头段变量
    之后再绘图,则会使用 !absolute! 模式(v101.6a)

示例代码如下:
\begin{SACCode}
SAC> dg sub local cdv.z
SAC> w 1.SAC
SAC> ch nzsec 50
SAC> w 2.SAC
SAC> r *.SAC
SAC> ppk            // 第一次绘图为relative模式
SAC> lh kztime      // 查看头段变量的值
SAC> ppk            // 第二次绘图为absolute模式
\end{SACCode}
\end{itemize}
