\SACCMD{hilbert}
\label{cmd:hilbert}

\SACTitle{概要}
应用Hilbert变换

\SACTitle{语法}
\begin{SACSTX}
HILBERT
\end{SACSTX}

\SACTitle{说明}
一个实序列$f(t)$的Hilbert变换定义为
\[
    H(f(t)) = f(t) * (\frac{1}{\pi t}) =
    \frac{1}{\pi} \int_{-\infty}^{+\infty} \frac{f(t)}{t-\tau} d\tau
\]
其中星号表示卷积。

由于$\frac{1}{\pi t}$的Fourier变换为$-i sgn(\omega)=-e^{i\pi/2} sgn(\omega)$,
因而对一个信号做Hilbert变换可以理解为先将信号做Fourier变换,然后乘以
$-e^{i\pi/2} sgn(\omega)$,再做反Fourier变换。即Hilbert变换的一个重要性质是对
信号产生$\frac{\pi}{2}$的相移。\footnote{本段内容不够严谨,正负号可能有误,
其中的细节也被省略,因而仅供参考。}

该命令通过将原始信号与一个201点FIR滤波器进行时间域卷积以实现Hilbert变换。
此FIR滤波器是通过对理想Hilbert变换的脉冲响应加Hanning窗获得的。在频率域,
每个频率处的振幅响应为1,相位为90度。Hilbert变换后的结果将替代内存中的
原始信号。

需要注意的是,此操作在零频和Nyquist频率附近的小区域内是不精确的。若要
对很低频率的数据进行Hilbert变换(比如长周期面波),则需要先对信号进行
减采样。由于该变换是在时间域完成的,所以计算时在原地使用重叠储存算法,
其对于文件长度没有限制。

理论上,Hilbert变换可以从振幅谱的对数中计算最小延迟相位,本质上Hilbert
变换是一个非带限的低通滤波器,因而本命令中的Hilbert变换无法用于计算最小
延迟相位。

SAC提供了Hilbert变换的函数库,可以直接在C或Fortran程序中调用,详情参考
``\nameref{sec:libsac}''一节。

\SACTitle{头段变量}
depmin、depmax、depmen
