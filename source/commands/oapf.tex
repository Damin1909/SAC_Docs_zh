\SACCMD{oapf}
\label{cmd:oapf}

\SACTitle{概要}
打开一个字母数字型震相拾取文件

\SACTitle{语法}
\begin{SACSTX}
OAPF [STANDARD|NAME] [file]
\end{SACSTX}

\SACTitle{输入}
\begin{description}
\item [STANDARD] 在写震相拾取文件时使用标准文件id。标准文件id包含了头段
    中事件名、台站名、分量方位角以及入射角。
\item [NAME] 使用SAC文件名代替标准文件id
\item [file] 要打开的字符数字型震相拾取文件,如果该文件已经存在,则其将
    会被打开,新的震相拾取会加到文件的底部
\end{description}

\SACTitle{缺省值}
\begin{SACDFT}
oapf standard apf
\end{SACDFT}

\SACTitle{说明}
震相拾取文件可以作为自动拾取(\nameref{cmd:apk})以及人工拾取(\nameref{cmd:plotpk})
命令产生的简单数据库。每个震相拾取被写在文件的一行上。这个文件的每个常规
行包含一个文件id、一个震相拾取id、震相拾取的时间、拾取的幅度以及一些格式
信息。这些行为80个字符长。文件id是一些标准的头段信息或者文件名。拾取到的
时间是GMT时间或者时间偏移。这依赖于产生这些拾取的命令如 !apk! 或
!ppk! 中指定的选项。这将导致4个不同的格式,在第79列上以不同的字符
标识这些行,如那些波形和峰峰值的读取数据,在80列以后不附加字段。一行的
最大长度为200。下面会展示不同格式的行。
