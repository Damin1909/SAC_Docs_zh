\SACCMD{window}
\label{cmd:window}

\SACTitle{概要}
设置图形窗口位置和宽高比

\SACTitle{语法}
\begin{SACSTX}
WIN!DOW! n [X!SIZE! xwmin xwmax] [Y!SIZE! ywmin ywmax] [ASPECT [value|ON|OFF]]
\end{SACSTX}

\SACTitle{输入}
\begin{description}
\item [n] 要设置属性的图形窗口号,n取值1到10
\item [XSIZE xwmin xwmax] 设置图形窗口的水平位置。
    其中xwmin和xwmax分别是窗口左/右边界位置,其可以取值为0.0到1.0。
\item [YSIEZ ywmin ywmax] 设置图形窗口的垂直位置。
    其中ywmin和ywmax分别是窗口左/右边界位置,其可以取值为0.0到1.0。
\item [ASPECT value|ON|OFF] 设置宽纵比为value。若打开ASPECT选项,则自动计算xwmax,
    使得xsize与ysize的比值为value,若value则未设定,则使用系统默认值。若打开了ASPECT
    选项,但是却没有指定xsize选项,则APSECT选项被关闭,并且使用默认的xwmin和xwmax值。
\end{description}

\SACTitle{缺省值}
下面列出前5个绘图窗口位置的缺省值:
\begin{table}[!ht]
\centering
\caption{SAC标准窗口}
\begin{tabular}{ccccc}
\toprule
 n & xwmin & xwmax & ywmin & ywmax  \\
\midrule
 1 & 0.05  & 0.65  & 0.45  & 0.95 \\
 2 & 0.07  & 0.67  & 0.43  & 0.93 \\
 3 & 0.09  & 0.69  & 0.41  & 0.91 \\
 4 & 0.11  & 0.71  & 0.39  & 0.89 \\
 5 & 0.13  & 0.73  & 0.37  & 0.87 \\
\bottomrule
\end{tabular}
\end{table}

缺省情况下ASPECT选项是打开的,其值为11.0/8.5=1.294,因而xwmax默认不使用。

\SACTitle{说明}
SAC使用的X11图形系统支持多窗口绘图。beginwindow命令使得你可以选择接下来的绘图
命令要绘制在哪个图形窗口中。如果想要修改窗口的属性,则必须在使用beginwindow命令
前使用window命令。

该命令可以控制每个X图形窗口出现时相对于屏幕左下角的位置以及窗口的宽高比。
屏幕左下角的坐标为(0,0),右上角的坐标为(1,1)。

默认情况下,使用编号为1的图形窗口。其水平方向的位置为0.05到0.65,垂直方向的位置
为0.45到0.95,即窗口位于屏幕的左上角。图形窗口随着编号的增加不断右下角移动。

若关闭ASPECT选项,则图形窗口的宽高比由屏幕的宽高比决定。对于4:3的屏幕,默认宽高比为
1.6:1;对于16:10的屏幕,默认宽高比为1.9:1。SGF文件的宽高比为4:3。

\SACTitle{示例}
设定图形窗口1的水平位置,垂直位置不变:
\begin{SACCode}
SAC> window 1 x 0.25 0.85
SAC> beginwindow 1
\end{SACCode}
在这种情况下,显式指定了XSIZE,因而ASPECT被自动设置为OFF。

\begin{SACCode}
SAC> window 1 aspect 1.33 x 0.25 0.85
SAC> beginwindow 1
\end{SACCode}
该命令与上面的命令相同,虽然设置了aspect的值,但由于指定了XSIZE,因而XSIZE具有
更高的优先级。

\begin{SACCode}
SAC> window 1 x 0.25 0.85 aspect 1.33
SAC> beginwindow 1
\end{SACCode}
由于APSECT位于XSIZE后面,因而ASPECT的优先级高于XSIZE的优先级,该命令会忽略xwmax,
并固定宽高比为1.33。
