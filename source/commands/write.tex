\SACCMD{write}
\label{cmd:write}

\SACTitle{概要}
将内存中的数据写入磁盘

\SACTitle{语法}
\begin{SACSTX}
W!RITE! [SAC|ALPHA|XDR] [DIR OFF|CURRENT|name] [KSTCMP]
    [OVER|APPEND text|PREPEND text|DELETE text|CHANGE text1 text2] filelist
\end{SACSTX}

\SACTitle{输入}
\begin{description}
\item [无参数] 使用以前的数据格式和文件列表
\item [SAC] 以SAC二进制文件格式写入磁盘
\item [ALPHA] 写SAC字符数字型写入磁盘
\item [XDR] 用SAC二进制 !XDR! 格式写文件。这个格式用于实现不同构架
    的二进制数据的转换
\item [DIR OFF] 关闭目录选项,即写入当前目录
\item [DIR CURRENT] 打开目录选项并设置写目录为当前目录
\item [DIR name] 打开目录选项并设置写目录为 !name!。将所有的文件
    写入目录 !name! 中,其可以是相对路径或绝对路径
\item [KSTCMP] 使用 !KSTNM! 和 !KCMPNM! 头段变量为内存中
    每个数据文件定义一个文件名。生成的文件名将检查是否唯一,如果不唯一,
    则在文件名后加序号以避免文件名冲突
\item [OVER] 使用当前读文件列表作为写文件列表,用内存中的文件覆盖磁盘中的文件
\item [APPEND text] 在当前读文件列表的文件名后附加字符串 !text!
    以创建写文件列表
\item [PREPEND text] 在当前读文件列表的文件名前附加字符串 !text!
    以创建写文件列表
\item [DELETE text] 在当前读文件列表的文件名中删除第一次出现的 !text!
    以创建写文件表
\item [CHANGE text1 text2] 将当前读文件表中每个文件名中第一次出现的 !text1!
    修改为 !text2! 来创建写文件表
\item [filelist] 将写文件列表设置为filelist,这个列表可以包含文件名、相对/
    绝对路径,不可以包含通配符
\end{description}

\SACTitle{缺省值}
\begin{SACDFT}
write sac
\end{SACDFT}

\SACTitle{说明}
该命令允许你在数据处理的过程中将结果写入磁盘。写磁盘文件时可以选择几种
数据格式,内存中的每个文件都将完整地写入到磁盘中。

多数情况下,你会选择使用SAC数据文件格式,这是SAC软件的标准输入输出格式,
用于快速读写,其包含了以一个头段区和一个数据区。具体可以参考\nameref{chap:sac-file-format}一章。

你可以直接指定写文件名,也可以通过修改内存中的当前文件名间接地指定它们。
!OVER! 选项把写文件表设置为读文件表。它用于覆盖包含当前内存的数据
的读入的最后一组磁盘文件。!APPEND!、!PREPEND!、!DELETE!、
!CHANGE! 选项通过以所需要的方式修改每个读文件名的方式建立一个
写文件表,这在宏命令中非常有用,在宏命令中你通常需要自动处理大量数据文件,
并保持输出文件风格的一致。当使用这四个选项中的任意一个时,命令执行时会在
终端输出文件列表,使得你可以看到实际写入磁盘的文件名。

\SACTitle{示例}
该命令的简单示例可以参考\nameref{sec:read-and-write}一节。

对一组数据文件进行滤波,然后将结果存入一组新数据文件:
\begin{SACCode}
SAC> read d1 d2 d3
SAC> lowpass butter npoles 4
SAC> write f1 f2 f3
\end{SACCode}

也可以使用 !CHANGE! 选项完成这一操作:
\begin{SACCode}
SAC> read d1 d2 d3
SAC> lowpass butter npoles 4
SAC> write change d f
\end{SACCode}

若想要用滤波后的数据替换磁盘中的原始数据,则上例的第三行要变成:
\begin{SACCode}
SAC> write over
\end{SACCode}

\SACTitle{BUGS}
\begin{itemize}
\item 使用 !dir off! 和 !dir current! 选项会直接报错,因而
    关键字 !off! 和 !current! 会被当作普通目录名,而由于
    目录不存在因而无法写入(v101.6a)
\end{itemize}
