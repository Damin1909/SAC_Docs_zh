\SACCMD{plotc}
\label{cmd:plotc}

\SACTitle{概要}
使用光标标注SAC图形和创建图件

\SACTitle{语法}
\begin{SACSTX}
P!LOT!C [R!EPLAY!|C!REATE!] [F!ILE!|M!ACRO! filename] [B!ORDER! ON|OFF]
\end{SACSTX}

\SACTitle{输入}
\begin{description}
\item [REPLAY] 重新显示或绘制一个已经存在的文件或宏,关于文件和宏的区别见下
\item [CREATE] 创建一个新的文件或宏
\item [FILE filename] 重新显示或创建一个文件。如果省略文件名则使用上一个文件
\item [MACRO filename] 重新显示或创建一个宏
\item [BORDER ON|OFF] 打开/关闭图形的边界
\end{description}

\SACTitle{缺省值}
\begin{SACDFT}
plotc create file out border on
\end{SACDFT}

\SACTitle{说明}
这个命令让你可以注释一个SAC绘图或者创建一个用于会议或报告的图件。简单的
说就是一个简单的``画图''软件。你需要一个具有光标的图形设备。你可以通过在
终端屏幕上放置一个目标(比如圆、方)或者文本创建一个图件。光标的位置决定
了目标绘制的位置,敲入的字符决定了要绘制哪个目标。目标包括圆、矩形、多边形、
线、箭头、弧,还有多种放置文本的方法。

这个命令绘制出的图可以直接截图利用,绘制过程中用到的所有命令将作为输出
文件输出。这个命令有两种输出文件格式:简单文件或宏文件。两者都是字符
数字型文件,可以直接用文本编辑器修改。它们包含了 !plotc! 命令中
光标响应的历史以及位置。宏文件一旦被创建,可以用于更多的绘图。其比例和
旋转角度均可以修改。简单的 !plotc! 文件名以 !.PCF!为后缀,
宏文件名则以 !.PCM! 为后缀。这使得你可以区分你的目录下的这些文件。

当你创建一个新的文件或宏时,SAC在屏幕上绘制一个矩形,代表你可以绘图的区域,
你可以将光标移动到该区域内任何你想放置的位置,并输入代表你想绘制的目标或
你想要光标执行的动作所对应的字符。

有两种光标选项类型:动作类和参数设置类。动作类选项将做一些事情(绘制一个
矩形,放一些文本),他们如何执行这个操作部分基于当前参数设置选项的值(例
如多边形的边数,文本大小等)。这个区别与SAC自己的操作命令和参数设定命令
相同。下面会列出动作和参数设定选项。

当你重新显示一个文件或者宏时,图形在终端屏幕上将会重新绘制,光标也将打开。
你可以像你当初创建这个文件一样向其中加入目标。当你完成创建图件之后你可以
将其发送到不同的图形设备,使用 \nameref{cmd:begindevices} 命令临时关闭
终端屏幕打开其他图形设备(比如 \nameref{cmd:sgf}),然后重新显示这个文件。

为了注释一个SAC绘图,要执行 \nameref{cmd:vspace} 命令设置正确的横纵比,
然后执行 \nameref{cmd:beginframe} 命令关闭自动刷新,执行需要的SAC绘图命令,
执行 \nameref{cmd:plotc} 命令(创建或者重新显示),然后执行
\nameref{cmd:endframe} 命令恢复自动刷新。

\SACTitle{示例}
下面的例子展示了如何使用 !plotc! 命令给一个SAC标准绘图添加注释:
\begin{SACCode}
SAC> fg impulse npts 1024                      //生成文件
SAC> lp c2 n 7 c 0.2 t 0.25 a 10               //低通滤波
SAC> fft
 DC level after DFT is 1
SAC> axes only l b                             //左和下坐标轴设置
SAC> ticks only l b
SAC> border off
SAC> fileid off
SAC> qdp off
SAC> vspace 0.75                              //修改图形尺寸
SAC> beginframe                               //开始绘图
SAC> psp am linlin                            //绘图
SAC> plotc create file bandpass               //开始在图上做注释
...用光标和键盘进行各种操作...
SAC> endframe
\end{SACCode}

\nameref{cmd:plotsp} 用于绘制滤波响应曲线以及两个轴,\nameref{cmd:plotc}
用于交互式地添加注释。\nameref{cmd:vspace} 命令限制了图形中纵横比为3:4的
区域为绘图区域。这个对于之后将输出发送到具有纵横比3:4的SGF设备来说很有必要。
在这之后你将有一个叫做 !BANDPASS.PCF! 的文件,其中包很了这个图形的
注释信息。

为了将注释写入SGF文件:
\begin{SACCode}
SAC> begindevices sgf                  // 打开sgf设备
SAC> beginframe
SAC> plotsp
SAC> plotc replay                      // 重新绘制上一注释图
SAC> endframe
\end{SACCode}
这样一个包含注释绘图的SGF文件就建立了。

\SACTitle{注意}
\begin{enumerate}
    \item 只有当设置正方形视窗(!vspace 1.0!)时绘制的圆形和扇形
        才是正确的,否则只能产生一个椭圆,其纵横比等于视窗的纵横比。
\item 除文本之外的所有操作码都按比例适应图形窗口。
\end{enumerate}
文本尺寸并不是当前标度的。当你生成一个图像并想要将文本放在一个矩形或圆中
时会产生一个问题。在这种情况下,图形窗口必须与输出页具有相同的尺寸,以
避免图形的偏差。这可以通过使用 \nameref{cmd:window} 命令设置窗的水平X
尺寸为0.75,垂直Y尺寸为0.69。例如:!WINDOW 1 X 0.05 0.80 Y 0.05 0.74!。
这个命令必须在窗口被创建之前执行。(即在 \nameref{cmd:beginwindow} 或
\nameref{cmd:begindevices} 之前)

\begin{table}[!ht]
\centering
\ttfamily
\small
\caption{plotc命令表}
\begin{tabular}{p{1cm}p{10cm}}
	\toprule
	字符	& 	含义	\\
	\midrule
	A		&	绘制一条到ORIGIN到CURSOR的箭头	\\
	B		&	在绘图区周围绘制边界的tick标记  \\
	C		&	绘制一个圆心在ORIGIN,且经过CURSOR的圆	\\
	D		&	从replay文件中删除最后一个动作选项	\\
	G		&	设置ORIGIN,并将其全局化	\\
	L		& 	绘制一条从ORIGIN到CURSOR的线	\\
	M		&	在CURSOR处插入一个宏文件(输入宏文件名,比例因子和旋转角。
                若没有指定,则使用上一次的值,默认是OUT,1.0,0)\\
	O		&	设置ORIGIN为CURSOR		\\
	N		&	绘制一个中心在ORIGIN,一个顶点位于CURSOR的n边形 \\
	Q		&	退出PLOTC	\\
	R		&	绘制对脚位于ORIGIN和CURSOR的长方形	\\
	S		&	绘制一个圆心位于ORIGIN的扇形(用光标的移动来
                指定扇形的角度,键入S来绘制一个小于180度的扇形,或者键入C绘制它的补集)\\
	T		&	在CURSOR处放置一行文本,文本以回车键结束	\\
	U		&	在CURSOR处放置多行文本,文本以空白行结束	\\
	\bottomrule
\end{tabular}
\end{table}
\SACTitle{关于PLOTC命令表的说明}
\begin{itemize}
\item !CURSOR! 表示当前光标位置
\item !ORIGIN! 一般为上次光标的位置
\item !G! 选项强制ORIGIN固定
\item !O! 选项再次允许ORIGIN移动
\item !Q! 选项不自动拷贝至文件,但是可以通过文本编辑器直接加入
\end{itemize}
如果SAC在replay模式没有在文件中看到Q选项,则其在显示文件内容之后回到光标
模式,这使得你可以在文件结束之后继续增加更多的选项。如果SAC在文件中看到
Q选项,则显示其内容并退出。文件中以星号开头的行为注释行。
!plotc! 还有一些更复杂的选项,但是运行起来好像有点问题,有兴趣的
可以试试 !help plotctable!。
