\section*{功能命令列表}

\subsection*{信息模块}
\begin{itemize}
\item \nameref{cmd:comcor}:控制SAC的命令校正选项
\item \nameref{cmd:production}:控制作业模式选项
\item \nameref{cmd:report}:报告SAC选项的当前状态
\item \nameref{cmd:trace}:追踪黑板变量和头段变量
\item \nameref{cmd:echo}:控制输入输出回显到终端
\item \nameref{cmd:history}:打印最近执行的SAC命令列表
\item \nameref{cmd:message}:发送信息到用户终端
\item \nameref{cmd:quitsub}:退出子程序
\item \nameref{cmd:about}:显示版本和版权信息
\item \nameref{cmd:news}:终端显示关于SAC的一些信息
\item \nameref{cmd:quit}:退出SAC
\item \nameref{cmd:help}:在终端显示SAC命令的语法和功能信息
\item \nameref{cmd:printhelp}:调用打印机打印帮助文档
\item \nameref{cmd:inicm}:重新初始化SAC
\item \nameref{cmd:transcript}:控制输出到副本文件
\end{itemize}

\subsection*{执行模块}
\begin{itemize}
\item \nameref{cmd:evaluate}:对简单算术表达式求值
\item \nameref{cmd:setbb}:设置黑板变量的值
\item \nameref{cmd:unsetbb}:删除黑板变量
\item \nameref{cmd:getbb}:获取或打印黑板变量的值
\item \nameref{cmd:mathop}:控制数学操作符的优先级
\item \nameref{cmd:macro}:执行SAC宏文件
\item \nameref{cmd:installmacro}:将宏文件安装到SAC全局宏目录中
\item \nameref{cmd:setmacro}:定义执行SAC宏文件时搜索的一系列目录
\item \nameref{cmd:systemcommand}:从SAC中执行系统命令
\end{itemize}

\subsection*{一元操作模块}
\begin{itemize}
\item \nameref{cmd:add}:为每个数据点加上同一个常数
\item \nameref{cmd:sub}:给每个数据点减去同一个常数
\item \nameref{cmd:mul}:给每个数据点乘以同一个常数
\item \nameref{cmd:div}:对每个数据点除以同一个常数
\item \nameref{cmd:sqr}:对每个数据点做平方
\item \nameref{cmd:sqrt}:对每个数据点取其平方根
\item \nameref{cmd:abs}:对每一个数据点取其绝对值
\item \nameref{cmd:log}:对每个数据点取其自然对数($\ln y$)
\item \nameref{cmd:log10}:对每个数据点取以10为底的对数($\log_{10} y$)
\item \nameref{cmd:exp}:对每个数据点取其指数($e^y$)
\item \nameref{cmd:exp10}:对每个数据点取以10为底的指数($10^y$)
\item \nameref{cmd:int}:利用梯形法或矩形法对数据进行积分
\item \nameref{cmd:dif}:对数据进行微分操作
\end{itemize}

\subsection*{二元操作模块}
\begin{itemize}
\item \nameref{cmd:addf}:使内存中的一组数据加上另一组数据
\item \nameref{cmd:subf}:使内存中的一组数据减去另一组数据
\item \nameref{cmd:mulf}:使内存中的一组数据乘以另一组数据
\item \nameref{cmd:divf}:使内存中的一组数据除以另一组数据
\item \nameref{cmd:binoperr}:控制二元操作addf、subf、mulf、divf中的错误
\item \nameref{cmd:merge}:将多个数据文件合并成一个文件
\end{itemize}

\subsection*{信号校正模块}
\begin{itemize}
\item \nameref{cmd:rq}:从谱文件中去除Q因子
\item \nameref{cmd:rglitches}:去掉信号中的坏点
\item \nameref{cmd:rmean}:去除均值
\item \nameref{cmd:rtrend}:去除线性趋势
\item \nameref{cmd:taper}:对数据两端应用对称的taper函数,使得数据两端平滑地衰减到零
\item \nameref{cmd:rotate}:将成对的正交分量旋转一个角度
\item \nameref{cmd:quantize}:将连续数据数字化
\item \nameref{cmd:interpolate}:对等间隔或不等间隔数据进行插值以得到新采样率
\item \nameref{cmd:stretch}:拉伸(增采样)数据,包含了一个可选的FIR滤波器
\item \nameref{cmd:decimate}:对数据减采样,包含了一个可选的抗混叠FIR滤波器
\item \nameref{cmd:smooth}:对数据应用算术平滑算法
\item \nameref{cmd:reverse}:将所有数据点逆序
\end{itemize}

\subsection*{数据文件模块}
\begin{itemize}
\item \nameref{cmd:funcgen}:生成一个函数并将其存在内存中
\item \nameref{cmd:datagen}:产生样本波形数据并储存在内存中
\item \nameref{cmd:read}:从磁盘读取SAC文件到内存
\item \nameref{cmd:readbbf}:将黑板变量文件读入内存
\item \nameref{cmd:readcss}:从磁盘读取CSS数据到内存
\item \nameref{cmd:readerr}:控制在执行read命令过程中的错误的处理方式
\item \nameref{cmd:readhdr}:从SAC数据文件中读取头段到内存
\item \nameref{cmd:write}:将内存中的数据写入磁盘
\item \nameref{cmd:writebbf}:将黑板变量文件写入到磁盘
\item \nameref{cmd:writecss}:将内存中的数据以 !CSS 3.0! 格式写入磁盘
\item \nameref{cmd:writehdr}:用内存中文件的头段区覆盖磁盘文字中的头段区
\item \nameref{cmd:listhdr}:列出指定的头段变量的值
\item \nameref{cmd:chnhdr}:修改指定的头段变量的值
\item \nameref{cmd:readtable}:从磁盘读取列数据文件到内存
\item \nameref{cmd:copyhdr}:从内存中的一个文件复制头段变量给其他所有文件
\item \nameref{cmd:convert}:实现数据文件格式的转换
\item \nameref{cmd:cut}:定义要读入的数据窗
\item \nameref{cmd:cuterr}:控制坏的截窗参数引起的错误
\item \nameref{cmd:cutim}:截取内存中的文件
\item \nameref{cmd:deletechannel}:从内存文件列表中删去一个或多个文件
\item \nameref{cmd:synchronize}:同步内存中所有文件的参考时刻
\item \nameref{cmd:sort}:根据头段变量的值对内存中的文件进行排序
\item \nameref{cmd:wild}:设置读命令中用于扩展文件列表的通配符
\end{itemize}

\subsection*{图形环境模块}
\begin{itemize}
\item \nameref{cmd:saveimg}:将绘图窗口中的图像保存到多种格式的图像文件中
\item \nameref{cmd:xlim}:设定图形中X轴的范围
\item \nameref{cmd:ylim}:设定图形中Y轴的范围
\item \nameref{cmd:linlin}:设置X、Y轴均为线性坐标
\item \nameref{cmd:loglog}:设置X、Y轴均为对数坐标
\item \nameref{cmd:linlog}:设置X轴为线性坐标,Y轴为对数坐标
\item \nameref{cmd:loglin}:设置X轴为对数坐标,Y轴为线性坐标
\item \nameref{cmd:xlin}:设置X轴为线性坐标
\item \nameref{cmd:ylin}:设置Y轴为线性坐标
\item \nameref{cmd:xlog}:设置X轴为对数坐标
\item \nameref{cmd:ylog}:设置Y轴为对数坐标
\item \nameref{cmd:xdiv}:控制X轴的刻度间隔
\item \nameref{cmd:ydiv}:控制Y轴的刻度间隔
\item \nameref{cmd:xfull}:控制X轴的绘图为整对数方式
\item \nameref{cmd:yfull}:控制Y轴的绘图为整对数方式
\item \nameref{cmd:xfudge}:设置X轴范围的附加因子
\item \nameref{cmd:yfudge}:设置Y轴范围的附加因子
\item \nameref{cmd:axes}:控制注释轴的位置
\item \nameref{cmd:ticks}:控制绘图上刻度轴的位置
\item \nameref{cmd:border}:控制图形四周边框的绘制
\item \nameref{cmd:grid}:控制绘图时的网格线
\item \nameref{cmd:xgrid}:控制绘图时的X方向的网格线
\item \nameref{cmd:ygrid}:控制绘图时的Y方向的网格线
\item \nameref{cmd:title}:定义绘图的标题和属性
\item \nameref{cmd:gtext}:控制绘图中文本质量以及字体
\item \nameref{cmd:tsize}:控制文本尺寸属性
\item \nameref{cmd:xlabel}:定义X轴标签及属性
\item \nameref{cmd:ylabel}:定义Y轴标签及属性
\item \nameref{cmd:plabel}:定义通用标签及其属性
\item \nameref{cmd:filenumber}:控制绘图时文件号的显示
\item \nameref{cmd:fileid}:控制绘图时文件ID的显示
\item \nameref{cmd:picks}:控制时间标记的显示
\item \nameref{cmd:qdp}:控制低分辨率快速绘图选项
\item \nameref{cmd:loglab}:控制对数轴的标签
\item \nameref{cmd:beginframe}:打开frame,用于绘制组合图
\item \nameref{cmd:endframe}:关闭frame
\item \nameref{cmd:beginwindow}:启动/切换至指定编号的X图形窗口
\item \nameref{cmd:window}:设置图形窗口位置和宽高比
\item \nameref{cmd:xvport}:定义X轴的视口
\item \nameref{cmd:yvport}:定义Y轴的视口
\item \nameref{cmd:null}:控制空值的绘制
\item \nameref{cmd:floor}:对数数据的最小值
\item \nameref{cmd:width}:控制图形设备的线宽
\item \nameref{cmd:color}:控制彩色图形设备的颜色选项
\item \nameref{cmd:line}:控制绘图中的线型
\item \nameref{cmd:symbol}:控制符号绘图属性
\end{itemize}

\subsection*{图像控制模块}
\begin{itemize}
\item \nameref{cmd:setdevice}:定义后续绘图时使用的默认图形设备
\item \nameref{cmd:begindevices}:启动某个图像设备
\item \nameref{cmd:enddevices}:结束某个图像设备
\item \nameref{cmd:vspace}:设置图形的最大尺寸和长宽比
\item \nameref{cmd:sgf}:控制SGF设备选项
\item \nameref{cmd:pause}:发送信息到终端并暂停
\item \nameref{cmd:wait}:控制SAC在绘制多个图形时是否暂停
\item \nameref{cmd:print}:打印最近的SGF文件
\end{itemize}

\subsection*{图像绘制模块}
\begin{itemize}
\item \nameref{cmd:plot}:绘制单波形单窗口图形
\item \nameref{cmd:plot1}:绘制多波形多窗口图形
\item \nameref{cmd:plot2}:产生一个多波形单窗口绘图
\item \nameref{cmd:plotpk}:绘图并拾取震相到时
\item \nameref{cmd:plotdy}:绘制一个带有误差棒的图
\item \nameref{cmd:plotxy}:以一个文件为自变量,一个或多个文件为因变量绘图
\item \nameref{cmd:plotalpha}:从磁盘读入字符数据型文件到内存并将数据绘制出来
\item \nameref{cmd:plotc}:使用光标标注SAC图形和创建图件
\item \nameref{cmd:plotsp}:用多种格式绘制谱数据
\item \nameref{cmd:plotpm}:针对一对数据文件产生一个``质点运动''图
\item \nameref{cmd:erase}:清除图形显示区域
\end{itemize}

\subsection*{谱分析模块}
\begin{itemize}
\item \nameref{cmd:hanning}:对每个数据文件应用一个``hanning''窗
\item \nameref{cmd:mulomega}:在频率域进行微分操作
\item \nameref{cmd:divomega}:在频率域进行积分操作
\item \nameref{cmd:fft}:对数据做快速离散傅立叶变换
\item \nameref{cmd:ifft}:对数据进行离散反傅立叶变换
\item \nameref{cmd:keepam}:保留内存中谱文件的振幅部分
\item \nameref{cmd:khronhite}:对数据应用Khronhite滤波器
\item \nameref{cmd:correlate}:计算自相关和互相关函数
\item \nameref{cmd:convolve}:计算主信号与内存中所有信号的卷积
\item \nameref{cmd:hilbert}:应用Hilbert变换
\item \nameref{cmd:envelope}:利用Hilbert变换计算包络函数
\item \nameref{cmd:benioff}:对数据使用Benioff滤波器
\item \nameref{cmd:unwrap}:计算振幅和展开相位
\item \nameref{cmd:wiener}设计并应用一个自适应Wiener滤波器
\item \nameref{cmd:plotsp}:用多种格式绘制谱数据
\item \nameref{cmd:readsp}:读取writesp和writespe写的谱文件
\item \nameref{cmd:writesp}:将谱文件作为一般文件写入磁盘
\item \nameref{cmd:bandpass}:对数据文件使用无限脉冲带通滤波器
\item \nameref{cmd:highpass}:对数据文件应用一个无限脉冲高通滤波器
\item \nameref{cmd:lowpass}:对数据文件应用一个无限脉冲高通滤波器
\item \nameref{cmd:bandrej}:应用一个无限脉冲带阻滤波器
\item \nameref{cmd:fir}:应用一个有限脉冲响应滤波器
\end{itemize}

\subsection*{分析工具}
\begin{itemize}
\item \nameref{cmd:linefit}:对内存中数据的进行最小二乘线性拟合
\item \nameref{cmd:correlate}:计算自相关和互相关函数
\item \nameref{cmd:convolve}:计算主信号与内存中所有信号的卷积
\item \nameref{cmd:envelope}:利用Hilbert变换计算包络函数
\item \nameref{cmd:filterdesign}:产生一个滤波器的数字和模拟特性的图形显示,包括:振幅,相位,脉冲响应和群延迟
\item \nameref{cmd:map}:利用SAC内存中的所有数据文件生成GMT地图
\item \nameref{cmd:whiten}:平滑输入的时间序列的频谱
\item \nameref{cmd:arraymap}:利用SAC内存中的所有文件产生一个台阵或联合台阵的分布图
\end{itemize}

\subsection*{事件分析模块}
\begin{itemize}
\item \nameref{cmd:ohpf}:打开一个Hypo格式的震相文件
\item \nameref{cmd:chpf}:关闭当前打开的Hypo震相拾取文件
\item \nameref{cmd:whpf}:将辅助内容写入Hypo格式的震相拾取文件中
\item \nameref{cmd:oapf}:打开一个字母数字型震相拾取文件
\item \nameref{cmd:capf}:关闭目前打开的字符数字型震相拾取文件
\item \nameref{cmd:apk}:对波形使用自动事件拾取算法(由连续信号判断是否其中是否包含地震事件)
\item \nameref{cmd:plotpk}:产生一个用于拾取到时的图
\item \nameref{cmd:mtw}:决定接下来命令中所使用的测量时间窗
\item \nameref{cmd:markptp}:在测量时间窗内测量并标记最大峰峰值
\item \nameref{cmd:marktimes}:根据一个速度集得到走时并对数据文件进行标记
\item \nameref{cmd:markvalue}:在数据文件中搜索并标记某个值
\item \nameref{cmd:rms}:计算测量时间窗内的信号的均方根
\item \nameref{cmd:traveltime}:根据预定义的速度模型计算指定震相的走时
\end{itemize}

\subsection*{XYZ数据模块}
\begin{itemize}
\item \nameref{cmd:spectrogram}:使用内存中的所有数据计算频谱图
\item \nameref{cmd:sonogram}:计算一个频谱图,其等价于同一个谱图的两个不同的平滑版本的差
\item \nameref{cmd:image}:利用内存中的数据文件绘制彩色图
\item \nameref{cmd:loadctable}:允许用户在彩色绘图中选择一个新的颜色表
\item \nameref{cmd:grayscale}:产生内存中数据的灰度图像
\item \nameref{cmd:contour}:利用内存中的数据绘制等值线图
\item \nameref{cmd:zlevels}:控制后续等值线图上的等值线间隔
\item \nameref{cmd:zcolors}:控制等值线的颜色显示
\item \nameref{cmd:zlines}:控制后续等值线绘图上的等值线线型
\item \nameref{cmd:zticks}:用方向标记标识等值线
\item \nameref{cmd:zlabels}:根据等值线的值控制等值线的标记
\end{itemize}

\subsection*{仪器校正模块}
\begin{itemize}
\item \nameref{cmd:transfer}:反卷积以去除仪器响应并卷积以加入其它仪器响应
\end{itemize}

\subsection*{FK谱}
\begin{itemize}
\item \nameref{cmd:bbfk}:利用SAC内存中的所有文件计算宽频频率-波数谱估计
\item \nameref{cmd:beam}:利用内存中的全部数据文件计算射线束
\end{itemize}

\newpage
