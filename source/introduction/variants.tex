\section{SAC变体}

SAC的发展史还是很曲折的,这也导致SAC存在多个不同的变体。

\begin{description}
\item[Fortran SAC]  即SAC的Fortran语言实现。最后一个分发版本发布于2003年,
                    版本号10.6f。曾经以限制性的形式在iaspei软件库中分发。
\item[SAC2000]      从Fortran源码转换为C源码,并以C源码为基础继续维护。
                    该版本加入了数据库特性以及一些新的命令。目前该版本
                    已不再分发。
\item[SAC/IRIS]     由SAC2000衍生的版本,不包含数据库特性\footnote{目前
                    的SAC/IRIS中还可以看到一些与数据库特性相关的命令和
                    选项,比如很多命令中的commit、rollback、recalltrace
                    选项,这些选项的存在属于历史遗留问题,且已经基本不再
                    维护,因而本文档中完全没有提及。},也就是本文档所介
                    绍的版本,在本文档中称为SAC。现在由IRIS下的SAC开发
                    小组负责维护,并由IRIS分发。
\item[SAC/BRIS]     也称为 MacSAC,主要在 MacOS X 下使用。该变种由
                    10.6d Fortran源码衍生而来,后期与10.6f集成。该变种
                    除了提供 MacOS X 下预编译的安装包外,还提供了编译脚本方便
                    用户自行在其他 Linux 系统下编译使用。该变种的功能是 SAC/IRIS 的
                    超集,相对于 SAC/IRIS 的最主要扩展在于增强了宏语言的功能以及处理
                    台阵数据的能力,以及支持直接读取 miniSEED 格式的数据文件。
                    其作者为
                    \href{http://www1.gly.bris.ac.uk/~george/gh.html}{George Helffrich}。
\end{description}
