\section{数据重采样}
相关命令:\nameref{cmd:decimate}、\nameref{cmd:interpolate}

如下情形,需要对数据进行重采样:
\begin{itemize}
\item 不同仪器的采样周期可能不同,需要将所有的数据重采样到相同的采样周期;
\item 数据的采样周期很小,导致数据量很大,而实际研究中不需要如此小的采样
    周期,因而可以对数据做减采样以减小数据量;
\item 数据的采样周期过大,实际研究中需要更小的采样周期,此时需对数据做
    插值重采样;
\item 数据为不等间隔数据,需要插值成为等间隔数据;
\end{itemize}

\subsection{decimate}
\nameref{cmd:decimate} 专门用于解决上面所说的第二种情形,即等间隔数据的
减采样。在减采样过程中,根据Nyquist采样定理,可能会出现混叠现象,而 
!decimate! 对数据自动做了低通滤波,以避免混叠现象的产生。

下面的示例中,将一个等间隔数据减采样10倍:
\begin{SACCode}
SAC> fg seis
SAC> lh delta npts

     delta = 1.000000e-02       //采样间隔delta=0.01
      npts = 1000
SAC> decimate 5; decimate 2     // 减采样10倍
SAC> lh delta npts

     delta = 9.999999e-02       //采样间隔delta=0.1,忽略浮点数误差
      npts = 100
\end{SACCode}

\subsection{interpolate}
与 \nameref{cmd:decimate} 相比,\nameref{cmd:interpolate} 命令功能
更加强大,其可以对等间隔或不等间隔数据进行增采样或减采样。

比如增采样,即插值:
\begin{SACCode}
SAC> fg seis
SAC> lh delta npts

     delta = 1.000000e-02
     npts = 1000
SAC> interp delta 0.005         // 增采样2倍
SAC> lh

     delta = 5.000000e-03
     npts = 1999
\end{SACCode}

对于减采样,!interpolate! 与 !decimate! 的功能略有重复,
但 !interpolate! 在减采样时不会对数据进行低通滤波,因而使用 
!interpolate! 进行减采样时可能会出现混叠现象,故而需要手动进行
低通滤波。

下面的示例将数据减采样到 \SI{20}{\Hz}。根据Nyquist采样定理,为了保证
不产生混叠现象,应首先对数据做 \SI{10}{\Hz} 的低通滤波。
\begin{SACCode}
SAC> fg seis
SAC> lh npts delta

     npts = 1000
     delta = 1.000000e-02
SAC> lp c 10
SAC> interpolate delta 0.05
WARNING potential for aliasing. new delta: 0.0500 data delta: 0.0100
    //  在做了lowpass之后,此处的警告可忽略
\end{SACCode}
