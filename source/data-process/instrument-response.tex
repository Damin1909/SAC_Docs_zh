\section{去仪器响应}
\label{sec:instrument-response}
相关命令:\nameref{cmd:transfer}

相关脚本:\hyperref[subsec:transfer-perl]{Perl脚本}、
          \hyperref[subsec:transfer-python]{Python脚本}

SAC中可以使用命令 \nameref{cmd:transfer} 实现去仪器响应,本节只列出
日常数据处理过程中最常用的命令。关于仪器响应的详细介绍,请参考附录
``\nameref{chap:resp}''以及教科书中的相关内容。关于 \nameref{cmd:transfer}
命令的具体用法,参考该命令的语法说明及示例。

SAC中常用的仪器响应文件有两种格式,即RESP文件和SAC PZ文件。本节会介绍
RESP文件和PZ文件的几种用法,并对每种方法的优缺点进行比较。

\subsection{RESP文件}

\subsubsection{RESP方法1}
使用 !evalresp! 选项但不指定RESP文件时,!transfer! 会对
内存中的所有SAC数据进行循环。对于内存中的每个SAC数据,从头段中提取台站
分量信息,然后在当前目录下寻找并使用对应的仪器响应文件
!RESP.NET.STA.LOC.CHN!。
\begin{SACCode}
SAC> r *.SAC
SAC> trans from evalresp to none freq 0.004 0.007 0.2 0.4
\end{SACCode}
该方法的好处是,可以一次处理多个SAC数据,且无需指定仪器响应文件的文件名。

\subsubsection{RESP方法2}
可以使用 !evalresp fname! 选项为每个波形分别指定RESP文件:
\begin{SACCode}
SAC> r 2006.253.14.30.24.0000.TA.N11A..LHZ.Q.SAC
SAC> rmean; rtr; taper
SAC> trans from evalresp fname RESP.TA.N11A..LHZ to none \
                                freq 0.004 0.007 0.2 0.4
\end{SACCode}
该方法的优点在于,RESP文件的文件名可以任意,使用起来更灵活。缺点在于,
一次只能处理一个SAC数据,数据的批量处理需要写脚本实现。

\subsubsection{RESP方法3}
可以将所有台站的RESP文件都合并到同一个文件中(!cat RESP.*.*.*.* >> RESP.ALL!),
并指定该总RESP文件为仪器响应文件,此时命令会从总RESP文件中自动寻找匹配
的仪器响应。
\begin{SACCode}
SAC> r *.SAC
SAC> trans from evalresp fname RESP.ALL to none \
                            freq 0.004 0.007 0.2 0.4
\end{SACCode}

\subsection{PZ文件}

\subsubsection{PZ方法1}
手动为每个波形指定PZ响应文件:
\begin{SACCode}
SAC> r OR075_LHZ.SAC
SAC> rmea; rtr; taper
SAC> trans from polezero subtype SAC_PZs_XC_OR075_LHZ to none \
                        freq 0.008 0.016 0.2 0.4
SAC> mul 1.0e9      // 用PZ文件transfer to none得到的位移数据的单位为m
                    // 而SAC默认的单位为nm,因而必须乘以1.0e9
SAC> w OR075.z      // 此时位移数据的单位为nm
\end{SACCode}
该方法的缺点是,一次只能处理一个波形数据,且需要用户编程指定PZ文件名。

\subsubsection{PZ方法2}
可以将所有台站的PZ文件合并到同一个文件中(!cat SAC_PZs_* >> SAC.PZs!),
并指定该总PZ文件为仪器响应文件,此时命令会从总PZ文件中自动寻找匹配的仪器
响应。
\begin{SACCode}
SAC> r *.SAC
SAC> trans from pol s SAC.PZs to none freq 0.008 0.016 0.2 0.4
SAC> mul 1.0e9
SAC> w over
\end{SACCode}
该方法的优点是一次可以处理多个波形数据。

\subsection{对比}
从易用性来看,RESP方法1、RESP方法3和PZ方法2都是比较易于使用的,
只需要一个简单的命令,即可同时对所有波形数据进行处理。而RESP方法2和PZ
方法1,需要用户自己从数据文件的文件名或头段中提取信息,并指定对应的
响应文件,这需要通过写少量的脚本来实现。

从执行效率来看,做了一个简单的测试,共670个波形数据,用不同的方法去
仪器响应,所需时间如下:
\begin{description}
\item [RESP方法1] 58秒;
\item [RESP方法2] 43秒;
\item [RESP方法3] 227秒;
\item [PZ方法1] 8秒;
\item [PZ方法2] 90秒;
\end{description}
从中可以总结出执行效率的如下规律:
\begin{enumerate}
\item RESP2和PZ1相比,RESP3与PZ2相比,可知,PZ文件的效率要高于RESP
    文件。这很容易理解,毕竟RESP文件中包含了更为完整的信息,计算量要
    更大一些;PZ文件中仅包含了零极点信息和总增益信息,对于日常的
    使用来说,已经足够;
\item RESP1和RESP2相比,区别在于:后者使用指定的文件,前者则需要从数据
    中提取信息、构建文件名并在当前目录下搜索,因而RESP1要比RESP2慢一些。
\item RESP3和PZ2方法,都是把多个响应函数放在同一个响应文件中,
    对于每个波形都需要对响应文件做遍历以找到匹配的响应函数,因而是所有
    方法中速度最慢的。
\end{enumerate}

总结下来:
\begin{itemize}
\item 想要写起来简单,用RESP方法1;
\item 想要执行快,可以用PZ方法1;
\end{itemize}
