\SACCMD{plotrecordsection}
\label{sss:plotrecordsection}

\SACTitle{概要}
用叠加文件列表中的文件绘制剖面图

\SACTitle{语法}
\begin{SACSTX}
P!LOT!R!ECORD!S!ECTION! [L!ABLES! ON|OFF|headerfield] [O!RIGIN! D!EFAULT!|R!EVERSED!]
    [R!EFERENCELINE! ON|OFF] [S!IZE! v] [W!EIGHT! ON|OFF] [P!OLARITY! ON|OFF]
    [C!URSOR! ON|OFF] [RED!UCED! ON|OFF|P!HASE! phasename|V!ELOCITY! velocity]
    [A!SPECT! ON|OFF] [ORIE!NT! P!ORTRAIT!|L!ANDSCAPE!] [T!TIME! ON|OFF|D!EFAULT!|TEXT]
    [X!LABEL! ON|OFF|D!EFAULT!|TEXT] [Y!LABEL! ON|OFF|D!EFAULT!|TEXT]
\end{SACSTX}

\SACTitle{输入}
\begin{description}
\item [LABELS ON|OFF] 打开/关闭标签选项。若打开,则每个文件都用头段变量进行标签
\item [LABELS headerfield] 打开标签选项,病设置头段变量名
\item [ORIGIN DEFAULT|REVERSED] 在Portrait模式中,距离沿着Y轴,默认情况下距离原点位于左上角。在landscape模式下,距离沿着X轴,默认情况下原点位于左下角。
\item [REFERENCELINE ON|OFF] 开启/关闭参考线选项。若打开,则每个文件在距离属性值对应的地方绘制一条垂直虚线
\item [SIZE v] ?
\item [WEIGHT ON|OFF] 打开/关闭权重选项
\item [POLARITY ON|OFF] 打开/关闭极性选项
\item [CURSOR ON|OFF]
\item [REDUCED ON|OFF|VELOCITY vel|PHASE phase] reduced走时曲线。可以指定reduce速度或者一个参考震相
\item [ORIENT PORTRAIT|LANDSCAPE] portrait模式中,水平轴为时间,纵轴为震中距;landscape模式下,水平轴为震中距,垂直轴为时间
\item [TTIME ON|OFF|DEFAULT|TEXT] 绘制走时曲线。需要首先用 \nameref{sss:traveltime} 命令计算走时曲线
\item [XLABEL ON|OFF|DEFAULT|TEXT] 打开/关闭/设置X轴标签
\item [YLABEL ON|OFF|DEFAULT|TEXT] 打开/关闭/设置Y轴标签
\end{description}

\SACTitle{缺省值}
\begin{SACDFT}
plotrecordsection labels filename origin default referenceline on size 0.1
    weight on polarity on orient portrait reduced off cursor off ttime off
\end{SACDFT}

\SACTitle{说明}
该命令将利用叠加文件列表中绘制剖面图。在portrait模式下,X轴为时间,Y轴为震中距,
在landscape模式下则交换XY轴。每个文件的零振幅将会画在距离轴上对应的震中距处。

为了能够正确绘图,叠加列表中的所有文件必须定义震中距属性,该属性可以来自于文件
头段,也可以在 \nameref{sss:globalstack}、\nameref{sss:addstack}、\nameref{sss:changestack}
等命令的DISTANCE选项中定义。

\nameref{sss:distancewindow} 和 \nameref{sss:timewindow} 命令可以控制要显示的数据窗。
\nameref{sss:distanceaxis} 和 \nameref{sss:timeaxis} 命令控制横纵轴的尺寸。
\nameref{sss:velocitymodel} 定义了速度模型,用于计算动态延迟。
\nameref{sss:velocityroset} 命令用于控制速度rosette的显示效果。

\SACTitle{光标模式}
在光标模式下,有两个额外的功能:缩放和决定视速度。

缩放功能需要用户指定要显示的区域。用户首先将光标放在当前图形区域的一个角落,键入!c1!,
再将光标移动到对角的另一个角落,键入 !c2!。两次键入
确定了唯一的矩形区域,也确定了要绘制的区域的时间范围和距离范围,此时,会自动重新
绘制缩放后的剖面图,用户可以键入 !o! 命令重新绘制原始图形。缩放功能最多
可以递归5次。

视速度确定功能需要用于移动光标,并分别键入 !v1! 和 !v2! 以标记点,
SAC会自动计算视速度,显示在输出设备上并保持到黑板变量vapp中。可以多次设置v2,但
只有最后一次的值会保存到黑板变量中。

除了c1、c2、v1、v2之外,光标模式下还有一个命令,即 !q!,用于退出光标模式。
