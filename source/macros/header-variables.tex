\section{引用头段变量值}
前面已经介绍了SAC中的很多头段变量,也知道如何使用 \nameref{cmd:listhdr}
查看头段变量的值,!lh! 命令的输出对于人来说很直观,但是对于机器
来说却很不友好。有些时候需要直接使用头段变量的值,这就需要一些特殊的技巧。

最常见的情况是第 \pageref{code:origin-time} 页给出的例子。在使用
``!ch o gmt!''指定发震时刻后,需要获取头段变量 !o! 的值,
对该值取负值,并用于``!ch allt!''中。

本例中,需要先获取头段变量 !o! 的值,再将其值用于其它命令中,
准确的说这叫变量值的引用。在SAC命令中引用头段变量的值有两种方式,
分别是``!&fname,header&!''和``!&fno,header&!''\footnote{实际上,
SAC官方文档给出的引用方式中没有末尾的 !&! 符号,仅当一些特殊的情况
下才使用,这样容易使得整个语法混乱不堪,所以这里采用了另外一种引用
方式。所有示例均已通过测试。}。

!fname! 和 !fno! 都唯一指向了内存中的某个波形数据,其中
!fname! 表示文件名,!fno! 表示文件号(即内存中的第几个文件,
索引值从1开始),!header! 则为头段变量名。

下例展示了如何通过两种方式引用头段变量的值:
\begin{SACCode}
SAC> fg seis
SAC> w seis.SAC
SAC> r ./seis.SAC               // 注意"./"
SAC> lh kevnm o stla            // 查看三个头段变量的值

     kevnm = K8108838
         o = -4.143000e+01
      stla = 4.800000e+01
SAC> echo on processed          // 打开回显,显示处理信息
SAC> ch kuser0 &1,kevnm&        // 通过文件号引用头段变量kevnm
 ==>  ch kuser0 K8108838        // 实际执行的效果
SAC> ch user0 &./seis.SAC,o&    // 利用文件名,引用头段变量o
 ==>  ch user0 -41.43
SAC> ch user1 &seis.SAC,stla&   // 文件名少了"./"
 ERROR 1363: Illegal data file list name: seis.SAC
SAC> lh kuser0 user0 user1

     kuser0 = K8108838
     user0 = -4.143000e+01
\end{SACCode}

在通过文件名指定波形数据时要注意:SAC记录的是文件的全路径。一般情况下,
使用文件号会更方便些。
