\section{SAC与脚本的运行速度}
以一个Perl脚本来解释这个陷阱。该脚本中读取一个文件,做简单操作后保存到
新文件中,并删除原文件。实际情况下,当然可以直接覆盖旧文件,此处这样
写完全是出于演示目的。
\begin{minted}{perl}
#/usr/bin/env perl
open(SAC, "|sac");
print SAC "r seis.SAC \n";
print SAC "bp c 0.1 5 \n";
print SAC "w seis.SAC.bp \n";
unlink "seis.SAC";
print SAC "q\n";
close(SAC);
\end{minted}
该脚本看似正确,实际运行时却会报错。出错的根本原因在于Perl的运行速度比
SAC的运行速度快很多。无论是Perl还是Python,或其他脚本语言,调用SAC时只是
简单地将一堆命令传递给SAC,而不去管具体命令的执行过程。上面的示例脚本中,
Perl语句是按顺序执行的,在将三个命令传递给SAC后,Perl用 !unlink!
函数删除了原文件,由于SAC的运行速度比Perl要慢很多,因而``快Perl''已经删除了
文件,``慢SAC''还没来得及执行命令。故而当``慢SAC''开始执行命令时,由于
文件已被``快Perl''删除而报错。

解决办法是,在退出SAC后再使用 !unlink! 函数,即将 !unlink!
函数放在 !close(SAC)! 语句后。
