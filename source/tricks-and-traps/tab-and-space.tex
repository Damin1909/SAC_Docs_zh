\section{Tab与空格}
编程界有诸多圣战,编辑器 vs IDE,vim vs emacs,以及谁是最好的语言等等。
这一节提到的``Tab与空格''也算是编程界的圣战之一了。这一节并不打算争论
究竟该使用Tab还是空格,而是讨论一下在SAC中使用Tab会遇到的陷阱。

从终端启动SAC,试着在SAC中键入Tab键,你会发现这些Tab键仅仅起到了命令
补全的作用,因而你是无法直接在交互式SAC中输入一个Tab键的。因此,在
交互式SAC下,Tab键不会造成任何问题。但,当在脚本中调用SAC,或者写SAC
宏时,Tab键就会产生影响。

以在Bash中调用SAC为例:
\begin{minted}{bash}
sac << EOF
    fg seis     // 此行行首有Tab,只是你看不见
    w seis.sac  // 此行行首有Tab,只是你看不见
    q           // 此行行首有Tab,只是你看不见
EOF
\end{minted}
由于SAC命令前有Tab键,导致执行该脚本的结果如下:
\begin{minted}{console}
sh: line 0: fg: no job control
22:24:49 up 12 days, 12:24,
USER
TTY
FROM
8 users,
load average: 1.45, 1.45, 1.42
LOGIN@
IDLE
JCPU
sh: q: command not found
SAC Error: EOF/Quit
SAC executed from a script: quit command missing
Please add a quit to the script to avoid this message
If you think you got this message in error,
please report it to: sac-help@iris.washington.edu
\end{minted}

引起该问题的原因尚不清楚,但比较明显的是,由于命令的行首有Tab,
导致SAC对命令的解析出现了问题,所有的命令都被当做外部命令来解释了。
在任意脚本语言中调用SAC,都有可能会出现类似的问题:
\begin{minted}{perl}
#!/usr/bin/env perl
use strict;
use warnings;
open(SAC, "|sac");
print SAC "     fg seis\n";     # fg 前面为Tab键

print SAC "\tfg seis\n";        # \t 为显式Tab键

print SAC qq [
    fg seis                     # fg 前面有Tab
    w seis.sac
    q
]
close(SAC);
\end{minted}
这个Perl脚本的例子中,展示了三种会出现问题的情况,其中第一种和第二种
一般都不会这写,所以很少碰到,而第三种写法是常见的写法,经常会需要用
缩进来将代码对齐,使得代码更整齐,就可能导致Tab键的问题。

因而,在脚本中调用SAC时要尽量避免使用Tab键。目前大多数主流编辑器都具备
将Tab自动转换成空格的功能,建议将这一功能打开,使得在写代码时无论输入
Tab还是输入空格都会自动转换成空格。
