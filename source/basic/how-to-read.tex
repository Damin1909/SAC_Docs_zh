\section{如何阅读本文档?}
本文档的内容分为两个部分:教程部分、命令部分和附录。

\begin{itemize}
\item 教程部分介绍了SAC的基础及进阶知识,并通过尽可能多的示例来演示如何
    操作和使用SAC。初学者应该坐在计算机前,打开终端,键入\footnote{严禁
    复制!不许偷懒!}书中的示例,试着理解每一个步骤的原理以及结果,并不
    断熟悉常用的SAC命令。
\item 命令部分详细地列出了SAC中的每一个命令的语法、参数以及一些技术细节,
    并包含了大量示例,适合作为参考,在需要的时候查阅。
\item 附录部分包含了一些与SAC有关但又稍微有些偏离本文档的主线的内容。
\end{itemize}

在阅读教程的同时,应随时翻看相应命令的说明,在实践的过程中掌握基础命令的
语法和用法。这样基本就完成了SAC初阶的要求。

在读完教程部分之后,应浏览SAC的几乎所有命令,并挑选其中感兴趣的一些
进行尝试。此后,在平常的科研工作中经常使用SAC,有了实践经验和对SAC的
进一步认识之后,可以阅读文档中的进阶内容,达到SAC进阶的要求。

最后,如果对SAC的内部机理感兴趣,可以阅读SAC的源码,重新实现一些SAC
底层的功能。
