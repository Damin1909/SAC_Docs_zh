\section{文档更新历史}
\subsection*{2012-01-08 1.0版}
\begin{itemize}
\item 第一版发布,由DOC转换为PDF
\item 参考《数字地震波形分析》一书,翻译了大部分官方文档中的内容
\item 结合SAC 101.4版本,增加、删除和修改了一些命令
\item 增加了书签,方便定位,支持全文搜索
\end{itemize}

\subsection*{2012-09-03 1.1版}
\begin{itemize}
\item 重新格式化整个文档,使得其看上去更规范,也易于以后的修改
\item 代码从NotePad++中直接导出,支持语法高亮
\item 代码及正文英文字体采用Consolas字体
\item 增加了``在脚本中调用SAC''一节
\item 新增命令 \texttt{transfer}、\texttt{traveltime}、\texttt{saveimg}、\texttt{datagen}
\item 更新至SAC v101.5c
\item 公式用公式编辑器编辑
\end{itemize}

\subsection*{2012-09-18 1.2版}
\begin{itemize}
\item 增加了封面配图
\end{itemize}

\subsection*{2013-03-29 2.0版}
用LaTeX重新排版文档
\begin{itemize}
\item 操作系统:CentOS 6.4
\item 编译环境:TeX Live 2012
\item 编译命令:xeLaTeX
\item 中文实现:ctex宏包
\item 中文字体:宋体+黑体
\item 英文主字体:Liberation Sans
\item 代码字体:Courier 10 Pitch
\end{itemize}

\subsection*{2013-04-06 2.1版}
\begin{itemize}
\item 重新整理了第一章
\item 修复bugs
\end{itemize}

\subsection*{2013-04-12 2.2版}
\begin{itemize}
\item 重新排版了全部命令
\item 重新设计了封面
\end{itemize}

\subsection*{2014-02-22 2.3版}
\begin{itemize}
\item 使用git管理源码
\item 整理结构和布局的修改
\item 新的小节:``SAC IO升级版''、``黑板变量的读写''、``SAC保存图像''
\item 修复bugs;
\end{itemize}

\subsection*{2014-04-18 3.0版}
\begin{itemize}
\item 源码托管在GitHub上,正式开源
\item 丢弃了之前的提交历史,重新开始
\item 重写了LaTeX导言区
\item 重新设计了整个文档的结构
\item 重写了教程部分的大多数内容
\item 教程部分根据SAC v101.6a进行修正
\item 修复bugs
\end{itemize}

\subsection*{2014-09-25 3.1版}
\begin{itemize}
\item 重新整理了大部分命令的语法说明
\item 对``SAC图像''一章进行了修订
\item 新增``信号迭加子程序''一章
\item 新增``谱估计子程序''一章
\item 新增``在Python中调用SAC''一节
\item 修复bugs
\end{itemize}

\subsection*{2015-05-02 3.2版}
对于用户:
\begin{itemize}
\item 修复bugs和typos
\item 命令整理:\texttt{systemcommand}、\texttt{transfer}
\item 新增章节
    \begin{itemize}
    \item 波形排序
    \item 标记震相理论走时的三种方法
    \item 图像格式转换
    \item SAC初始化宏文件
    \item SAC命令的长度上限
    \item 字节序
    \item 新增附录``仪器响应'',整理了``去仪器响应''一节
    \end{itemize}
\item 新增示例:调用SAC的Hilbert函数
\end{itemize}

对于维护者:
\begin{itemize}
\item 新增ChangeLog
\item 更新README,可根据说明自行编译源码生成PDF
\item 修改Makefile,对依赖的检测更加智能
\item 简化用于调用绘图脚本的Makefile
\item 英文使用TeX默认字体,中文使用开源中文字体Fandol
\item 使用 \texttt{minted} 实现代码的语法高亮,替代 \texttt{listings}
\item \texttt{datetime} 宏包升级至 \texttt{datetime2}
\end{itemize}

\subsection*{2015-06-06 3.3版}
对于用户:
\begin{itemize}
\item 修改bugs和typos
\item 命令整理:\texttt{hilbert}、\texttt{transfer}
\item 新增内容:
    \begin{itemize}
    \item 四个文件重命名脚本
    \item 读取某个目录下全部文件遇到的问题
    \item 使用Tab遇到的问题
    \item 数据命名规则
    \item 时区校正
    \item 错误与警告消息
    \item 未定义变量
    \item SAC debug
    \item \texttt{wh} 与 \texttt{w over} 的区别
    \end{itemize}
\end{itemize}

对于维护者:
\begin{itemize}
\item 删除之前的提交历史,精简repo尺寸,维护者需要重新Fork
\item 删除了scons脚本
\end{itemize}

\subsection*{xxxxx-xx-xx 3.4版}
对于用户:
\begin{itemize}
\item 命令的``错误消息''和``警告消息''集中在附录中
\item 在C程序中调用 \texttt{distaz}
\item 命令整理:\texttt{mtw}、\texttt{markptp}、\texttt{markvalue}、\texttt{readcss}
\item 在C程序中读写SAC文件
\end{itemize}

对于维护者:
\begin{itemize}
\item 更新至TeXLive 2015
\item ctex宏包更新至2.2
\end{itemize}