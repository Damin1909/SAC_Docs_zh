\section{数据来源}
地震波形数据的来源有很多,下面列举并简单介绍常见的数据来源。

\subsection{国家测震台网数据备份中心}
\href{http://www.seisdmc.ac.cn/}{国家测震台网数据备份中心},隶属于
中国地震局地球物理研究所。到2014年底,国家测震台网已建成由170个台站
和3个小孔径台阵(共30个子台)组成的国家地震台网;859个台站组成的31个
区域地震台网,33个子台组成的6个火山监测台网;291套地震仪器组成的32个
应急流动观测台网。国家测震台网数据备份中心可以提供从2007年8月起的
全球M5.5级以上地震事件以及国内及周边地区M3.5级以上地震事件的波形数据。

要申请数据备份中心的数据,需要注册账户并升级账户属性。
要升级账户属性,需要按程序进行申请,并且签署和遵守
\href{http://www.seisdmc.ac.cn/class/view?id=8}{相关协议}。

\subsection{IRIS DMC}
\label{subsec:IRIS}
\href{http://ds.iris.edu/ds/nodes/dmc/}{IRIS DMC} 是世界上最大的地震
波形数据中心。

IRIS DMC的大部分数据是完全公开的,无需注册即可直接申请下载波形数据。
从IRIS DMC申请数据的工具有很多:
\href{http://docs.obspy.org/}{ObsPy}、
\href{https://ds.iris.edu/ds/nodes/dmc/manuals/breq_fast/}{BREQ\_FAST}、
\href{http://ds.iris.edu/wilber3/find_event}{Wilber III}、
\href{http://service.iris.edu/}{Web Service}、
\href{http://ds.iris.edu/ds/nodes/dmc/software/downloads/irisfetch.m/}{irisfetch.m}、
\href{https://seiscode.iris.washington.edu/projects/ws-fetch-scripts}{Web Service Fetch scripts}、
\href{http://www.seis.sc.edu/sod/}{SOD} 和
\href{https://ds.iris.edu/ds/nodes/dmc/software/downloads/jweed/}{JWEED}。

\subsection{NIED}
\label{subsec:NIED}
\href{http://www.bosai.go.jp/}{NIED}是日本的国家防灾科学技术研究所。其
下包含若干台网:高感度地震观测网 \href{http://www.hinet.bosai.go.jp/}{Hi-net}、
宽频带地震台网 \href{http://www.fnet.bosai.go.jp/}{F-net}、强地面运动地震
台网 \href{http://www.kyoshin.bosai.go.jp/}{K-net和KiK-net} 和
火山观测网 \href{http://www.vnet.bosai.go.jp/}{V-net}。

\subsection{Natural Resources Canada}
\label{subsec:nrcan}
加拿大政府的网站 \href{http://www.nrcan.gc.ca/home}{Natural Resources Canada}
提供了
\href{http://www.earthquakescanada.nrcan.gc.ca/stndon/CNSN-RNSC/index-eng.php}{Canadian National Seismic Network}、
\href{http://can-ndc.nrcan.gc.ca/yka/index-en.php}{Yellowknife Seismic Array}、
POLARIS Network等台网/台阵的连续波形数据以及这些台网/台阵1975年至今的事件
波形数据。

事件波形数据可以直接 \href{http://www.earthquakescanada.nrcan.gc.ca/stndon/NWFA-ANFO/eve/index-eng.php}{点击下载};
连续波形数据则可以通过工具
\href{http://www.earthquakescanada.nrcan.gc.ca/stndon/AutoDRM/index-eng.php}{AutoDRM} 下载。

\subsection{其他}
\begin{itemize}
\item \href{http://www.ncedc.org/}{Northern California Earthquake Data Center}
\item \href{http://pnsn.org/}{Pacific Northwest Seismic Network}
\item \href{http://www.scsn.org/}{Southern California Seismic Network}
\item \href{http://scedc.caltech.edu/}{Southern California Seismic Network at Caltech}
\item \href{http://www.chinarraydmc.org/}{中国地震科学探测台阵数据中心}
\end{itemize}
