\section{仪器响应文件}
在SAC诞生的80年代,模拟地震仪的类型很少,SAC将这些地震仪器的响应函数
都内置到程序中,可以直接使用。随着数字地震仪的出现和不断发展,地震仪器
的类型越来越丰富,不可能把这些地震仪器的响应函数都内置到SAC中,这就需要
有更通用的方式来描述仪器响应,即仪器响应文件:RESP、PZ和FAP。

\subsection{内置仪器响应}
SAC内置了很多标准地震仪器的仪器响应,如表 \ref{table:instrument-type}
所示。部分仪器类型还拥有子类型,如表 \ref{table:instrument-subtype} 所示。
在SAC命令中,可以直接使用这些仪器类型。

\begin{table}[tp]
\centering
\ttfamily
\small
\caption{SAC内置仪器类型列表}
\label{table:instrument-type}
\begin{tabular}{ll}
\toprule
type     &  说明  \\
\midrule
BBDISP   &  Blacknest specification of Broadband Displacement \\
BBVEL    &  Blacknest specification of Broadband Velocity   \\
BENBOG   &  Blacknest specification of Benioff by Bogert    \\
DSS      &  LLNL Digital Seismic System \\
DWWSSN   &  Digital World Wide Standard Seismograph Station \\
EKALP6   &  Blacknest specification of EKA LP6  \\
EKASP2   &  Blacknest specification of EKA SP2  \\
ELMAG    &  Electromagnetic \\
GBALP    &  Blacknest specification of GBA LP   \\
GBASP    &  Blacknest specification of GBA SP   \\
GENERAL  &  General seismometer \\
GSREF    &  USGS Refraction \\
HFSLPWB  &  Blacknest specification of HFS LPWB \\
IW       &  EYEOMG-spectral differentiation \\
LLL      &  LLL broadband analog seismometer    \\
LLSN     &  LLSN L-4 seismometer    \\
LNN      &  Livermore NTS Network instrument    \\
LRSMLP   &  Blacknest specification of LRSM LP  \\
LRSMSP   &  Blacknest specification of LRSM SP  \\
NORESS   &  NORESS (NRSA)   \\
NORESSHF &  NORESS high frequency element   \\
OLDBB    &  Old Blacknest specification of BB   \\
OLDKIR   &  Old Blacknest specification of Kirnos   \\
PORTABLE &  Portable seismometer with PDR2  \\
PTBLLP   &  Blacknest specification of PTBL LP  \\
REDKIR   &  Blacknest specification of RED Kirnos   \\
REFTEK   &  Reftek 97-01 portable instrument    \\
RSTN     &  Regional Seismic Test Network   \\
S750     &  S750 Seismometer    \\
SANDIA   &  Sandia system 23 instrument \\
SANDIA3  &  Sandia new system with SL-210   \\
SRO      &  Seismic Research Observatory    \\
WA       &  Wood-Anderson   \\
WABN     &  Blacknest specification of Wood-Anderson    \\
WIECH    &  Wiechert seismometer    \\
WWLPBN   &  Blacknest specification of WWSSN long period    \\
WWSP     &  WWSSN short period  \\
WWSPBN   &  Blacknest specification of WWSSN short period   \\
YKALP    &  Blacknest specification of YKA long period  \\
YKASP    &  Blacknest specification of YKA short period \\
\bottomrule
\end{tabular}
\end{table}

\begin{table}[htb]
\centering
\ttfamily
\small
\caption{部分仪器子类型}
\label{table:instrument-subtype}
\begin{tabular}{ll}
\toprule
主类型 &   子类型 \\
\midrule
LLL       &       LV, LR, LT, MV, MR, MT, EV, ER, ET, KV, KR, KT    \\
LNN       &       BB|HF                                 \\
NORESS    &       LP|IP|SP                              \\
RSTN      &       [CP|ON|NTR|NY|SD][KL|KM|KS|7S][Z|N|E] \\
SANDIA    &       [N|O][T|L|B|D|N|E][V|R|T]             \\
SRO       &       BB|SP|LPDE                            \\
FREEPERIOD v &    ELMAG, GENERAL, IW, LLL SUBTYPE BB, REFTEK    \\
MAGNIFICATION n & ELMAG, GENERAL  \\
NZEROS n &        GENERAL, IW   \\
DAMPING v &       GENERAL, LLL SUBTYPE BB, REFTEK   \\
CORNER v &        LLL SUBTYPE BB, REFTEK    \\
GAIN v &            \\
HIGHPASS v &      REFTEK    \\
\bottomrule
\end{tabular}
\end{table}

除了表 \ref{table:instrument-type} 中列出的众多仪器类型之外,还有几个
特别的仪器类型:
\begin{itemize}
\item !none!:即位移,也是SAC的默认值
\item !vel!:速度
\item !acc!:加速度
\end{itemize}

\subsection{RESP文件}
RESP文件是用于描述仪器响应的文件,其包含了描述仪器响应所需要的全部信息。

RESP仪器响应文件可以通过如下几种方式获得:
\begin{itemize}
\item 用rdseed程序从SEED数据中提取;
\item 用evalresp程序从SEED数据中提取;
\item 从IRIS DMC resp Web Service\footnote{\url{http://service.iris.edu/irisws/resp/1/}}下载;
\item 手写RESP文件;
\end{itemize}

一个RESP文件中可以只包含一个仪器响应函数,也可以包含多个台站、多通道、
多时间段的多个仪器响应函数。每个仪器响应函数中包含了台站名、台网名、
通道名、开始时间和结束时间等台站的基本信息。具体的仪器响应函数部分又
分成多个Stage,每个Stage中又分为多个block,包含了仪器响应的不同信息。

\begin{itemize}
\item Stage1一般对应模拟信号阶段,从中可以提取中这一阶段的输入单位、
    零极点、归一化因子$A_0$以及第一阶段的增益。
\item Stage2一般对应ADC阶段,从中可以提取出这一阶段的放大系数。
\item Stage3一般对应于数字滤波和减采样阶段。通常需要对数字信号多次
    滤波或减采样,因而Stage3后面可能会接多个类似的Stage。从这几个
    Stage中提取的信息是增益,一般值为1。
\item Stage0是会给出前面所有Stage的增益的乘积,主要是起到了辅助验证的作用。
\end{itemize}

\subsection{SAC PZ文件}
RESP文件中包含了仪器响应的完整信息,同时也包含了不少冗余信息。SAC从RESP
文件中提取处仪器响应中的重要信息,定义了新的零极点响应文件(即SAC PZ)。
相对于RESP文件而言,PZ文件中仅包含仪器响应中的零极点和增益信息,在去
仪器响应时更方便。

SAC PZ文件可以用rdseed程序从SEED文件中提取,也可以从IRIS DMC SAC PZ Web Service
\footnote{\url{http://service.iris.edu/irisws/sacpz/1/}}获取,当然也可以
手写SAC PZ文件。

下面是某个台站的SAC PZ文件:
\begin{verbatim}
* **********************************
* NETWORK   (KNETWK): IU
* STATION    (KSTNM): COLA
* LOCATION   (KHOLE): 00
* CHANNEL   (KCMPNM): BHZ
* CREATED           : 2013-06-22T14:12:09
* START             : 2012-09-14T04:00:00
* END               : 2599-12-31T23:59:59
* DESCRIPTION       : College Outpost, Alaska, USA
* LATITUDE          : 64.873599
* LONGITUDE         : -147.861600
* ELEVATION         : 84.0
* DEPTH             : 116.0
* DIP               : 0.0
* AZIMUTH           : 0.0
* SAMPLE RATE       : 20.0
* INPUT UNIT        : M
* OUTPUT UNIT       : COUNTS
* INSTTYPE          : Geotech KS-54000 Borehole Seismometer
* INSTGAIN          : 2.013040e+03 (M/S)
* COMMENT           : N/A
* SENSITIVITY       : 3.377320e+09 (M/S)
* A0                : 8.627050e+04
* **********************************
ZEROS   3
        +0.000000e+00   +0.000000e+00
        +0.000000e+00   +0.000000e+00
        +0.000000e+00   +0.000000e+00
POLES   5
        -5.943130e+01   +0.000000e+00
        -2.271210e+01   +2.710650e+01
        -2.271210e+01   -2.710650e+01
        -4.800400e-03   +0.000000e+00
        -7.384400e-02   +0.000000e+00
CONSTANT        +2.913631e+14
\end{verbatim}

SAC PZ文件中,以星号开始的行为注释行,给出了该PZ文件所对应的台站信息,
其中 !INPUT UNIT! 表明了该PZ文件的输入是位移、速度还是加速度。
用 !rdseed! 从SEED数据中提取出来的PZ文件,输入都是位移,
且单位为 \si{\m}。

以关键字 !ZEROS! 起始的行给出了零点数目,接下来几行列出了每个
零点的实部和虚部。以关键字 !POLES! 起始的行给出了极点数目,
接下来几行列出了每个极点的实部和虚部。最后一行给出了仪器响应中的常数
!CONSTANT!。

根据零极点以及 !CONSTANT!,即可计算得到仪器响应函数:
\[
    H(s) = C_0 * \frac{(s-z_1)(s-z_2)...(s-z_{nz})}{(s-p_1)(s-p_2)...(s-p_{nz})}
\]
其中$s=2\pi i f$。

一些说明:
\begin{itemize}
\item 若有零点 !(0.0,0.0)!,则这样的``零''零点可以省略。因而
    列出的零点数可能会少于``ZEROS''行给出的零点数;上例中的三个零点
    可以不列出;
\item !CONSTANT! 对应于RESP文件中所有阶段的增益$Sd_0$以及归一化
    因子$A_0$的乘积;
\item 若未指定 !CONSTANT!,则默认值为1.0;
\end{itemize}

\subsection{FAP文件}
FAP文件是响应函数的另一种表现形式,其包含了很多记录行,每行三个字段,
分别是频率(\si{Hz})、振幅及相位。

频率不需要等间隔分段。在执行 !transfer! 时,低于第一行频率的
频段将使用第一行的振幅和相位;同理大于最后一行频率的频段将使用最后
一行的振幅和相位。

FAP文件可以从程序evalresp v3.3.2中获得,FAP相对于PZ文件的优势在于,
其给出了每个频率的振幅和相位响应,因而包含更丰富的信息,且方便人工
修改以控制需要校正的频率段。

\subsection{RESP vs PZ vs FAP}
RESP、PZ和FAP都可以用于表征仪器的响应函数,常见的是RESP和PZ,而这两种
还是有很大区别的:
\begin{itemize}
\item RESP文件包含了仪器响应的完整信息,而PZ文件中仅包含了零极点
    和增益信息,二者的主要差异在于PZ文件中未包含FIR滤波器的信息;
\item RESP文件中可以知道输入数据是位移、速度还是加速度,而PZ文件默认输入
    为位移。因而若RESP文件中输入是速度,则PZ文件中会多一个``零''零点;
    若RESP文件中输入是加速度,则PZ文件中会多两个``零''零点;
\item SAC中的默认位移单位是 \si{\nm},RESP文件中有指定输入单位为 \si{\m},
    因而在用RESP去仪器响应时,transfer会在去除仪器响应之后在对数据做单位
    上的变换以使得得到的位移数据的单位是 \si{\nm},即与SAC的标准相一致。
    而PZ文件中并未提供输入单位信息,或者说即便提供了也没有被利用到,故而
    用PZ文件去除仪器响应得到的位移物理量单位是 \si{\m},为了与SAC标准相
    一致,需要对数据乘以10的9次方将数据单位由 \si{\m} 转换成 \si{\nm};
\end{itemize}

对于大多数情况,建议使用PZ文件,数据处理速度要快很多。
