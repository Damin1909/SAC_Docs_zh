\section{物理与数学}
地震仪一般固定在地表或地下浅层,也有放置在钻孔深处或海底的,因而地震仪
可以直接感知到地面的运动。在地面运动物理量被地震仪接收后,首先要将其转
换为电信号,然后对电信号振幅进行放大以及滤波,再将连续的时间序列离散化,
最终以常见的波形数据的形式表现出来。

概括地说,地面的运动在被地震仪感知之后,需要经历如下三个阶段,最终成为
用户拿到的波形数据:
\begin{enumerate}
\item 模拟信号阶段
\item 模数转换
\item 数字信号阶段
\end{enumerate}

地震仪的仪器响应可以表示为频率的复函数,即常说的振幅响应和频率响应。
上面提到的三个阶段,每个阶段的响应函数都是频率的复函数,表示为$G_i(f)$,
整个仪器的响应函数是各个阶段响应函数的乘积:
\[
    G(f)=\prod_i G_i(f)
\]

\subsection{预备知识}
在介绍地震仪的三个阶段之前,需要先了解一些基础的数学知识。
\subsubsection{响应函数}
模拟信号的响应函数用Laplace变换表示:
\[
    H(s)=\int_0^{\infty}h(t)e^{-st}dt
\]

数字信号的响应函数用Z变换表示:
\[
    H(z)=\sum_{-\infty}^{+\infty}h_m z^{-m}
\]

\subsubsection{归一化}
频率响应可以表示为
\[
    G(f)=S_d R(f)
\]
其中$R(f)$是频率的函数。在某个特定的频率$f_s$,有$|R(f_s)|=1.0$,
即$R(f)$在频率$f_s$处进行归一化。$S_d$是放大系数,也称为Sensitivity或者Gain。

R(f)可以表示为
\[
    R(f)= A_0 H_p(s)
\]
其中$H_p(s)$是用零极点表示的transfer函数,一般来说在频率$f_s$处这个transfer
函数的振幅响应不为1,所以需要归一化因子$A_0$。

\subsubsection{FIR滤波器}
FIR滤波器从数学上看就是对数据采样点做加权平均,设计不同的加权系数就得到
不同的FIR滤波器。FIR滤波器一般设计为振幅响应为方阶跃函数(boxcar),因
而其在通带内有较平的振幅响应,在拐角频率处有很尖锐很陡峭的振幅响应变化。
与此同时,滤波器具有线性相位,在时间域造成信号的时间延迟,一般数据采集
系统会对这个时间延迟做校正。因此,用户基本不需要考虑FIR滤波器对仪器响应的影响。

\subsection{模拟信号阶段}
模拟信号阶段会将连续的地面运动物理量(比如速度)转换为连续的电压信号,
并进行电压放大。因而此阶段的输入单位是运动物理量的单位(比如 \si{\m\per\s}),
输出单位是伏特(\si{\V})。

这个阶段的响应函数可以表示为
\[
    G(f)=S_d A_0 \frac{\prod_{i=1}^{n} (s-r_i)}{\prod_{j=1}^{n} (s-p_j)}=S_d A_0 H_p(s)
\]

其中,$S_d$是放大系数,$H_p(s)$是用零极点表示的transfer函数,这里有$n$个零点
和$m$个极点。$A_0$是归一化因子,且$s=i 2\pi f$。

假定某仪器的自然频率为$f_0=\SI{1}{\Hz}$, 阻尼常数为$\lambda=0.7$。如果该仪器是一个
相对简单的仪器,则该仪器的加速度传递函数为:

\[
    H(s) = \frac{s}{s^2+2\lambda \omega_0 s + \omega_0^2}
\]

几点说明:
\begin{itemize}
\item 该传递函数仅表示某一类地震仪的传递函数,现代地震仪的传递函数可能
    比这个要复杂;
\item 根据该传递函数,很容易计算出它的一个零点和两个极点;
\end{itemize}

已知该仪器在$f_s=\SI{1}{\Hz}$时的增益为$Sd=\SI{150}{\V\per(\m\per\square\s)}$,
即若仪器接收到 \SI{1}{\Hz} 频率的加速度 \SI{1}{\m\per\square\s},
则仪器的输出电压为 \SI{150}{\V}。因而仪器真实的传递函数还需要把增益加进去:

\[
    G(f) = R(f)*S_d = A_0*H_p(s)*S_d
\]

其中$A_0$是归一化因子,保证在频率$f=f_s$处有$|R(f_s)|=1.0$。

\subsection{模数转换}
模数转换器,将上一阶段产生的连续电压信号转换为离散的电压信号。
输入的单位是伏特 \si{\V},输出单位是 \si{counts}。这个阶段,所有频段
有相同的振幅响应,即只存在一个放大系数,同时可能存在一个时间延迟。

假设模数转换器是 $m$ 位,电压范围是 $\pm n$ \si{\V},那么
$\SI{1}{counts}=\frac{2n}{2^m} \si{\V}$,电压也能精确到$\frac{2n}{2^m} \si{\V}$。
例如,\num{24} 位模数转换器的输入电压范围是 \SI{+-20}{\V},则输出的最大
范围是$\pm 2^{23}$ \si{counts}。因而ADC的放大系数为
\[
    S_d = \SI[exponent-base = 2]{e24/40}{counts\per\V} = \SI{4.1943e5}{counts\per\V}
\]
即若ADC接收到的输入电压为 \SI{1}{\V},则其输出为 \SI{4.1983e5}{counts}。
结合地震仪的放大系数可知,频率为 \SI{1}{\Hz} 的 \SI{1}{\m\per\square\s}
的地面运动加速度在ADC的输出为:

\[
    \SI{1}{\m\per\square\s} *
    \SI{150}{\V\per{(\m\per\square\s)}} *
    \SI{4.1943e5}{counts\per\V} =
    \SI{6.29145e7}{counts}
\]

另一方面,由于ADC的输入电压的上限为 \SI{20}{\V},因而仪器所能记录的
最大加速度为$20/150=\SI{0.13}{\m\per\square\s}$。

\subsection{数字信号阶段}
这个阶段会对数据信号进行进一步的处理,主要包含三个部分,即离散信号
滤波、数据重采样、时间延迟校正。

离散信号滤波可以采用FIR滤波器,也可以采用IIR滤波器。多数情况下采用
FIR滤波器,而FIR滤波器的振幅响应函数可以认为在全频段内为1\footnote{
FIR滤波器在Nyquist频率附近会有5\%左右的震荡,因而若感兴趣的频率与
Nyquist频率相差较大,则可以忽略这一阶段的响应函数},因而这个阶段只
需要考虑放大系数,而不需要再考虑由于滤波引入的响应函数。同样,对于
数据重采样以及时间校正也不会引入新的响应函数。

\subsection{小结}
综上所述,三个阶段中,第一个阶段最为复杂,需要给出放大系数$Sd_{1}$、归一化
因子$A_0$以及零极点信息;第二个阶段以及第三个阶段都只需要给出放大系数$Sd_{2}$和$Sd_3$。
最终得到仪器的响应函数为
\[
    G(f)=Sd_1 A_0 H_p(s) Sd_2 Sd_3=Sd_0 A_0 H_p(s)
\]
即需要仪器在第一个阶段的零极点信息、归一化因子$A_0$以及三个阶段的放大系数的
乘积$Sd_0$即可以近似表示地震仪的仪器响应。
