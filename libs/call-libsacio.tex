\section{调用libsacio库}
\label{sec:libsacio}
\subsection{rsac1}
子函数 \texttt{rsac1} 用于读取等采样间隔的SAC数据。

函数定义如下:
\begin{minted}{c}
void
rsac1(  char    *kname,     // 要读入的文件名
        float   *yarray,    // 数据被保存到yarray数组中
        int     *nlen,      // 数据长度
        float   *beg,       // 数据开始时间,即头段变量b
        float   *del,       // 数据采样周期,即头段变量delta
        int     *max_,      // 数组yarray的最大长度,若nlen>max_则截断
        int     *nerr,      // 错误标记,0代表成功,非零代表失败
        int      kname_s    // 数组kname的长度
)
\end{minted}

相关示例代码为 \texttt{rsac1c.c} \footnote{代码位于 \texttt{sac/doc/examples},
下同。}和 \texttt{rsac1f.f}。

\subsection{rsac2}
子函数 \texttt{rsac2} 用于读取非等间隔采样的。
\begin{minted}{c}
void
rsac2(  char    *kname,     // 要读入的文件名
        float   *yarray,    // 因变量数组
        int     *nlen,      // 数据长度
        float   *xarray,    // 自变量数组
        int     *max_,      // 数组最大长度
        int     *nerr,      // 错误标记
        int      kname_s    // 数组kname的长度
)
\end{minted}
相关示例代码为 \texttt{rsac2c.c} 和 \texttt{rsac2f.f}。

\subsection{wsac1}
子函数 \texttt{wsac1} 用于写等间隔SAC文件。
\begin{minted}{c}
void
wsac1(  char  *kname,       // 要写入的文件名
        float *yarray,      // 要写入文件的数组
        int   *nlen,        // 数组长度
        float *beg,         // 数据起始时刻
        float *del,         // 数据采样周期
        int   *nerr,        // 错误标记
        int    kname_s      // 文件名长度
)
\end{minted}

相关示例代码为 \texttt{wsac1c.c} 和 \texttt{wsac1f.f}。

\subsection{wsac2}
写非等间隔SAC文件。

\begin{minted}{c}
void
wsac2(  char  *kname,       // 文件名
        float *yarray,      // 因变量数组
        int   *nlen,        // 数组长度
        float *xarray,      // 自变量数组
        int   *nerr,        // 错误码
        int    kname_s      // 文件名长度
)
\end{minted}

相关示例代码为 \texttt{wsac2c.c} 和 \texttt{wsac2f.f}。

\subsection{wsac0}
子函数 \texttt{wsac0} 相对来说更加通用也更复杂,利用该函数可以创建
包含更多头段的SAC文件。

\begin{minted}{c}
void
wsac0(  char  *kname,       // 文件名
        float *xarray,      // 自变量数组
        float *yarray,      // 因变量数组
        int   *nerr,        // 错误码
        int    kname_s      // 文件名长度
)
\end{minted}

要使用子函数 \texttt{wsac0},首先要调用子函数 \texttt{newhdr} 创建一个
完全未定义的头段区,并利用其它头段变量相关子函数设置头段变量的值,
并由 \texttt{wsac0} 写入到文件中。必须要赋值的头段变量为 \texttt{delta}、
\texttt{b}、\texttt{e}、\texttt{npts}、\texttt{iftype}。

相关示例代码为 \texttt{wsacnc.c} 、\texttt{wsacnf.f}。n=3、4、5。

\subsection{getfhv}
获取浮点型头段变量的值。
\begin{minted}{c}
void
getfhv( char  *kname,       // 头段变量名
        float *fvalue,      // 浮点型头段变量的值
        int   *nerr,        // 错误码
        int    kname_s      // 变量名长度
)
\end{minted}

相关示例代码为 \texttt{gethvc.c} 和 \texttt{gethvf.f}。

至于如何获取和设置其它类型的头段变量,方法类似,不再多说。

\subsection{readbbf}
读取一个黑板变量文件。
\begin{minted}{c}
void
readbbf( char   *kname,     // 要读取的文件
         int    *nerr,      // 错误码
         int    kname_s     // 文件名长度
)
\end{minted}

\texttt{writebbf} 与之类似,不再列出。

\subsection{getbbv}
获取一个黑板变量的值。
\begin{minted}{c}
void
getbbv( char    *kname,     // 黑板变量名
        char    *kvalue,    // 黑板变量的值
        int     *nerr,      // 错误码
        int     kname_s,    // 变量名长度
        int     kvalue_s    // 变量值的长度
)
\end{minted}
\texttt{setbbv} 的函数定义与其类似,不再列出。

\subsection{distaz}
计算地球上任意两点之间的震中距、方位角和反方位角。
\begin{minted}{c}
void
distaz( double    evla,     // 事件纬度
        double    evlo,     // 事件经度
        float    *stla,     // 台站纬度
        float    *stlo,     // 台站经度
        int       nsta,     // 台站数目
        float    *dist,     // 震中距(km)
        float    *az,       // 方位角
        float    *baz,      // 反方位角
        float    *deg,      // 震中距(度)
        int      *nerr      // 错误码
)
\end{minted}
示例代码如下:
\inputminted{c}{./libs/distaz.c}
本例中,台站数目 \texttt{nsta=1} 。实际上该函数可以计算任意数目的台站
到事件的震中距、方位角信息,若台站数目 \texttt{nsta} 不为1,则
\texttt{stla}、\texttt{stlo}、\texttt{dist}、\texttt{az}、\texttt{baz}、
\texttt{deg} 均可用数组或指针表示。
