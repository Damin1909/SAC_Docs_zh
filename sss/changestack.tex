\SACCMD{changestack}
\label{sss:changestack}

\SACTitle{概要}
修改当前迭加文件列表中的文件属性

\SACTitle{语法}
\begin{SACSTX}
C!HANGE!S!TACK! filename|filenumber [W!EIGHT! v] [DI!STANCE! v]
    [BE!GINTIME! v] [END!TIME! v] [DE!LAY! v S!ECONDS!|P!OINTS!]
    [I!NCREMENT! v S!ECONDS!|P!OINTS!] [N!ORMAL!|R!EVERSED!]
\end{SACSTX}

\SACTitle{输入}
\begin{description}
\item [filename] 迭加文件列表中的文件名
\item [filenumber] 迭加文件列表中的文件号
\item [WEIGHT v] 当前文件的权重因子。v的取值范围为0到1,在迭加之前会首先对文件的每个值乘以该权重因子再做迭加。
\item [DISTANCE v] 该文件所对应的震中距,单位为 \si{\km}。用于计算动态时间延迟
\item [BEGINTIME v] 事件开始的时间
\item [ENDTIME] 事件结束时间
\item [DELAY v SECONDS|POINTS] 该文件的静态时间延迟,单位为秒或数据点数
\item [INCREMENT v SECONDS|POINTS] 该文件的静态时间延迟增量,单位为秒或数据点数。在每次执行incrementstack命令时,静态时间延迟会增加一个常数。
\item [NORMAL|REVERSED] 文件拥有正/负极性
\end{description}

\SACTitle{说明}
该命令允许你修改修改迭加文件列表中任意文件的任意属性。详情参考 \nameref{sss:addstack} 命令。

\SACTitle{错误消息}
\begin{itemize}
\item 5106 文件不在迭加文件列表中
\end{itemize}

\SACTitle{相关命令}
\nameref{sss:addstack}
