\SACCMD{incrementstack}
\label{sss:incrementstack}

\SACTitle{概要}
迭加文件列表中的增量属性

\SACTitle{语法}
\begin{SACSTX}
I!NCREMENT!S!TACK!
\end{SACSTX}


\SACTitle{缺省值}
缺省值为0

\SACTitle{说明}
可以设定增量的属性包括静态时间延迟、视速度和速度模型中的截距时间。若属性
增量为0.0,则属性值不改变。

可以为视速度或速度模型截距时间设置增量,其他属性值则自动计算以保持在特定点的零时间延迟。

\SACTitle{示例}
\begin{SACCode}
SAC/SSS> addstack filea
SAC/SSS> addstack fileb
SAC/SSS> addstack filec
SAC/SSS> addstack filed
SAC/SSS> velocitymodel 1 refr vapp 7.9 vappi 0.1 tovm calc dist 320. tvm 45.
SAC/SSS> sumstack
SAC/SSS> writestack stack1
SAC/SSS> incrementstack
SAC/SSS> sumstack
SAC/SSS> writestack stack2
SAC/SSS> incrementstack
SAC/SSS> sumstack
SAC/SSS> writestack stack3
\end{SACCode}

上面的命令会产生三个迭加文件,即stack1、stack2、stack3。迭加时使用折射波速度
模型,视速度VAPP分别为7.9、8.0、8.1。速度模型截距时间TOVM自动计算以保证在320km、45秒处具有零时间延迟。

\SACTitle{相关命令}
\nameref{sss:velocitymodel}
