\SACCMD{velocitymodel}
\label{sss:velocitymodel}

\SACTitle{概要}
设置计算动延迟时所使用的迭加速度模型参数

\SACTitle{语法}
\begin{SACSTX}
V!ELOCITY!M!ODEL! n [ON|OFF] [REFRACTEDWAVE|NORMALMOVEOUT]
    [FLIP] [VAPP v|CALCULATE] [T0VM v|CALCULATE]
    [DVM v1 [v2]] [TVM v1 [v2]] [VAPPI v] [T0VMI v]
\end{SACSTX}

\SACTitle{输入}
\begin{description}
\item [n] 设置速度模型号,取值为1或2
\item [ON|OFF] 打开/关闭速度模型选项。若打开则使用速度模型,否则忽略
\item [REFRACTEDWAVE] 打开速度模型选项,并修改为折射波模型
\item [NORMALMOVEOUT] 打开速度模型选项,并修改为Normal moveout模型
\item [FLIP] 交换两个速度模型的属性
\item [VAPP v] 设置视速度为v
\item [VAPP CALCULATE] SAC自动计算视速度
\item [TOVM v] 设置时间轴截距为v
\item [TOVM CALCULATE] SAC自动计算截距
\item [DVM v1 v2] 定义一/二个参考距离
\item [TVM v1 v2] 定义一/二个参考时间
\item [VAPPI v] 设置视速度增量为v。每次~\nameref{sss:incrementstack}~命令执行时视速度增加v
\item [TOVMI v] 设置时间轴截距的增量为v。每次~\nameref{sss:incrementstack}~命令执行时视速度增加v
\end{description}

\SACTitle{缺省值}
\begin{SACDFT}
velocitymodel 1 off
velocitymodel 2 off
\end{SACDFT}

\SACTitle{说明}
第一个速度模型用于计算某个特定震相的动态台站延迟。在信号迭加(\nameref{sss:sumstack})、绘图叠加图(\nameref{sss:plotstack})、绘制剖面图(\nameref{sss:plotrecordsection})时会使用该模型。第二个速度模型用于在绘图剖面图时显示相对于第二震相的延迟。这两个模型的参数可以很容易的进行交换。

可以使用两种速度模型,即折射波速度模型:
\[ T_{delay} = TVM(1) - \frac{TOVM+DIST}{VAPP} \]
以及normal moveout速度模型:
\[ T_{delay} = TVM(1) - \sqrt{TOVM^2 + (\frac{DIST}{VAPP})^2} \]

这些速度模型延迟可以通过多种方式得到:
\begin{itemize}
\item 直接输入VAPP、TOVM、TVM(1)
\item 输入DVM(1)、TVM(1)以及VAPP或TOVM,SAC自动计算所需的变量以保证在距离DVM(1)处时间延迟为零
\item 输入DVM(1)、TVM(1)、DVM(2)和TVM(2)。SAC将计算VAPP和TOVM,以保证在距离DVM(1)处的时间延迟为零
\end{itemize}

\SACTitle{示例}
设置第一个迭加速度模型为折射波模型,视速度为6.5 km/s,让SAC自动计算TOVM以使得200 km处的时间延迟为零:

\begin{SACCode}
velocitymodel 1 refractedwave vapp 6.5 tovm calculate dvm 200 tvm 35
\end{SACCode}

\SACTitle{相关命令}
\nameref{sss:sumstack}、\nameref{sss:plotstack}、\nameref{sss:plotrecordsection}
