\SACCMD{sumstack}
\label{sss:sumstack}

\SACTitle{概要}
对迭加文件列表中的文件进行迭加

\SACTitle{语法}
\begin{SACSTX}
S!UM!S!TACK! [N!ORMALIZATION! ON|OFF]
\end{SACSTX}

\SACTitle{输入}
\begin{description}
\item [NORMALIZATION ON|OFF] 打开/关闭归一化选项。若该选项打开,则对于迭加结果中
的每个数据点除以所有文件的权重因子的和。
\end{description}

\SACTitle{缺省值}
该命令用于将迭加文件列表中的文件进行迭加。在该命令执行之前必须通过~\nameref{sss:timewindow}~命令设置迭加时间窗。每个数据会根据其静/动时间延迟做相应的时移。对于不迭加时间窗内的数据直接按零值处理。每个文件可以给定权重以及极性。

在迭加之后,会自动生成迭加结果的绘图。迭加结果可以通过~\nameref{sss:writestack}~命令保存到磁盘中。

\SACTitle{错误消息}
\begin{itemize}
\item 5130:未定义时间窗
\end{itemize}

\SACTitle{相关命令}
\nameref{sss:timewindow}、\nameref{sss:writestack}
