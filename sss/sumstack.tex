\SACCMD{sumstack}
\label{sss:sumstack}

\SACTitle{概要}
对叠加文件列表中的文件进行叠加

\SACTitle{语法}
\begin{SACSTX}
S!UM!S!TACK! [N!ORMALIZATION! ON|OFF]
\end{SACSTX}

\SACTitle{输入}
\begin{description}
\item [NORMALIZATION ON|OFF] 打开/关闭归一化选项。若该选项打开,则对于叠加结果中
的每个数据点除以所有文件的权重因子的和。
\end{description}

\SACTitle{缺省值}
该命令用于将叠加文件列表中的文件进行叠加。在该命令执行之前必须通过 \nameref{sss:timewindow} 命令设置叠加时间窗。每个数据会根据其静/动时间延迟做相应的时移。对于不叠加时间窗内的数据直接按零值处理。每个文件可以给定权重以及极性。

在叠加之后,会自动生成叠加结果的绘图。叠加结果可以通过 \nameref{sss:writestack} 命令保存到磁盘中。
