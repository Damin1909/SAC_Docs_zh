\section{在Python中调用SAC}
\label{sec:sac-python}

\subsection{简介}
下面的脚本中给出了一个简单的例子,展示了如何在Python中调用SAC。
\inputminted{python}{./call-in-script/simple-script.py}

\subsection{头段变量}
Python无法直接引用SAC文件的头段变量值,依然需要利用SAC宏的的语法。
但Python不需要使用SAC的黑板变量功能。
\inputminted{python}{./call-in-script/variables.py}
本例中第13行,在SAC运行的同时Python临时定义了一个变量,并成功将其
传递给了SAC,这在Bash中是不容易做到的。

\subsection{内联函数}
Python可以完成各种复杂的数学运算:
\inputminted{python}{./call-in-script/arithmetic-functions.py}

Python对字符串的处理也很简单:
\inputminted{python}{./call-in-script/string-functions.py}

\subsection{条件判断和循环控制}
Python也可以很容易地实现条件判断和循环控制:
\inputminted{python}{./call-in-script/do-loops.py}
这个例子中只启动了一次SAC,然后开始Python的循环控制,读取当前目录下的
每一个SAC文件,做一些数据处理,然后写到新文件中。

该Python脚本与上一个Bash脚本相比,实现了几乎相同的功能,但Python脚本中
仅启动和退出SAC一次,与Bash脚本的多次启动相比,其效率要高很多。

\subsection{修改发震时刻}
要将发震时刻写入到SAC文件中,首先要将计算发震日期是一年的第几天。Python
内置的 \texttt{datetime} 模块可以很容易实现。
\inputminted{python}{./call-in-script/ch-origin.py}

\subsection{文件重命名}
\label{subsec:rename-in-python}
与前面的Perl脚本类似,先用 \texttt{split} 方法将文件名分割成列表,然后
直接从列表中取需要的元素。
\inputminted{python}{./call-in-script/rename.py}
