\section{Perl中调用SAC}
\label{sec:sac-perl}

\subsection{简单示例}
下面的脚本展示了如何在Perl中调用SAC。
\inputminted{perl}{./call-in-script/0.simple-script.pl}
Perl中调用SAC本质上是使用 \texttt{open(SAC, "| sac")} 语句定义了一个名为
\texttt{SAC} 指向 \texttt{sac} 的句柄,然后通过 \texttt{print SAC} 语句
将要执行的SAC命令传递给SAC。

\subsection{数据转换}
首先要将SEED格式的数据转换成SAC格式。
\begin{itemize}
\item 假定每个目录下有且只有一个SEED数据
\item \texttt{rdseed} 一次只能处理一个SEED数据
\item \texttt{rdseed} 的 \texttt{-pdf} 选项会提取出SAC波形数据和PZ格式的仪器响应文件
\end{itemize}
\inputminted{perl}{./call-in-script/1.rdseed.pl}

\subsection{文件合并}
\label{subsec:merge-in-perl}
SEED文件的波形数据可能会因为多种原因而出现间断,导致同一个通道会解压出来
多个SAC文件,因而需要将属于同一个通道的SAC数据合并起来。
\begin{itemize}
\item 此处使用了新版 \nameref{cmd:merge} 命令的语法,要求SAC版本大于v101.6
\item \texttt{merge} 命令还有更多选项可以控制数据合并的细节,见命令的语法介绍
\item 合并后的数据,以最早的数据段的文件名命名
\item 多余的数据段均被删除,以保证每个通道只有一个SAC文件
\item 由于脚本运行速度比SAC运行速度快,因而应先退出SAC再删除多余的数据段
\end{itemize}
\inputminted{perl}{./call-in-script/2.merge.pl}

\subsection{文件重命名}
\label{subsec:rename-in-perl}
从SEED解压出的SAC文件名较长,因而对其重命名以简化。
\begin{itemize}
\item SEED解压出的默认文件名格式为 \texttt{yyyy.ddd.hh.mm.ss.ffff.NN.SSSSS.LL.CCC.Q.SAC}
\item 重命名后的文件名为 \texttt{NET.STA.LOC.CHN.SAC}
\end{itemize}
\inputminted{perl}{./call-in-script/3.rename.pl}

\subsection{添加事件信息}
\label{subsec:event-info-perl}
若SEED中不包含事件信息,则解压得到的SAC文件中也不会包含事件信息。因而
需要用户手动添加事件的发震时刻、经纬度、深度和震级信息。
\begin{itemize}
\item 输入参数包括:目录名、发震时刻、经度、纬度、深度、震级
\item 发震时刻的格式为 \texttt{yyyy-mm-ddThh:mm:ss.xxx},其中 \texttt{T}
    用于分隔日期和时间
\end{itemize}
\inputminted{perl}{./call-in-script/4.eventinfo.pl}

\subsection{去仪器响应}
\label{subsec:transfer-perl}
使用PZ文件去仪器响应。若数据的时间跨度太长,在该时间跨度内可能仪器响应会
发生变化,因而会存在一个通道有多个PZ文件的情况。目前该脚本在遇到这种情况
时会自动退出。
\inputminted{perl}{./call-in-script/5.transfer.pl}

\subsection{分量旋转}
\label{subsec:rotate-perl}
将成对的水平分量旋转到大圆路径。
\begin{itemize}
\item 检查三分量是否缺失
\item 检查 \texttt{delta} 是否相等
\item 取三分量中的最大 \texttt{b} 和最小 \texttt{e} 值作为数据窗口,此
    操作要求三分量的 \texttt{kzdate} 和 \texttt{kztime} 必须相同,这一点
    在添加事件信息时使用 \nameref{cmd:synchronize} 已经实现
\item 未检查分量的 \texttt{cmpinc} 和 \texttt{cmpaz} 是否符合要求。若不
    符合要求,SAC会报错且不会旋转,写到磁盘中的R和T分量实际上是原数据,
    暂不知如何处理。
\end{itemize}
\inputminted{perl}{./call-in-script/6.rotate.pl}

\subsection{数据重采样}
\label{subsec:resample-perl}
通常有两种情况下需要对数据进行重采样:
\begin{itemize}
\item 原始数据的采样率过高,而实际研究中不需要如此高的采样率,此时,对数据
    做减采样可以大大减少数据量;
\item 原始数据中,不同台站的采样率不同,可能会影响到后期的数据处理,因而
    需要让所有数据使用统一的采样率;
\end{itemize}
下面的Perl脚本中使用 \nameref{cmd:interpolate} 命令将所有数据重采样到相同
的采样周期。用户可以在命令行中直接指定要使用的重采样后的采样周期,若命令行
中的采样周期指定为0,则以大多数数据所使用的采样周期作为重采样后的采样周期。
\inputminted{perl}{./call-in-script/7.resample.pl}
