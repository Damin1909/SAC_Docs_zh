\section{Bash中调用SAC}
\label{sec:sac-bash}

\subsection{简介}
SAC宏的功能相对比较单一,难以满足日常数据处理的需求,可以在Bash脚本中
直接调用SAC,这样可以利用Bash脚本的更多特性。

下面的例子展示了如何在Bash脚本中调用SAC:
\inputminted{bash}{./call-in-script/simple-script.sh}

SAC在启动是默认会显示版本信息,当用脚本多次调用SAC时,版本信息也会
显示多次,可以通过设置变量 \verb|SAC_DISPLAY_COPYRIGHT=0| 的方式隐藏
版本信息。

脚本中从``\texttt{sac << EOF}''开始到``\texttt{EOF}''的全部内容,都会被
Bash传递给SAC,SAC会逐一解释并执行每行命令。

\subsection{头段变量和黑板变量}
想要在Bash脚本中引用头段变量,需要借助于SAC宏的语法。
\inputminted{bash}{./call-in-script/variables.sh}

\subsection{内联函数}
bash可以完成基本的数学运算,但是所有的运算只支持整型数据,浮点型运算
或者其它更高级的数学运算需要借助 \texttt{bc} 或者 \texttt{awk} 来完成。
Bash中的变量以``\verb|$|''作为标识符,Bash会首先做变量替换再将替换后的
命令传递给SAC。
\inputminted{bash}{./call-in-script/arithmetic-functions.sh}

本例中的变量``\verb|$var1|'' 和 ``\verb|$var2|''会首先被SAC解释成
为1和2,因而SAC实际接收到的命令是``\texttt{bp c 1 2} ''。

借助于 \texttt{awk}、\texttt{sed} 等工具,也可以实现部分字符串处理函数:
\inputminted{bash}{./call-in-script/string-functions.sh}

\subsection{条件判断和循环控制}
Bash具有更灵活的条件判断和循环控制功能,但由于Bash自身的限制,这些特性
仅能在SAC外部使用,因而下例中需要多次调用SAC,在某些情况下会相当耗时。
\inputminted{bash}{./call-in-script/do-loops.sh}

\subsection{文件重命名}
\label{subsec:rename-in-bash}
Bash下可以借助于 \texttt{awk} 来实现文件重命名。下面的例子中,首先用
点号对文件名做分割,\verb|$0| 表示原始文件名,\verb|$7| 表示用逗号分
割后的第7段字符,即台网名,其他同理。最后将 \texttt{awk} 的输出传给
\texttt{sh} 去执行。
\begin{minted}{console}
$ ls *.SAC | awk -F "." '{printf "mv %s %s.%s.%s", $0, $7, $8, $10}' | sh
\end{minted}
