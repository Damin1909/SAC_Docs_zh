\section{字节序}
\subsection{定义}
字节序,即一个多字节的数据在内存中存储方式。计算机领域中,主要有两种字节序,
即小端序(little endian)和大端序(big endian)。

举例说明,假设一个整型变量占四个字节,其十六进制表示为~\verb+0x01234567+,
首地址为~\verb+0x101+,则该整型变量的四个字节将被存储在内存~\verb+0x101~0x104+中。
若按照大端序存储,则~\verb+01+位于地址~\verb+0x100+,\verb+23+位于地~址\verb+0x101+,
以此类推。而小端序则是~\verb+67+位于地址~\verb+0x100+,~\verb+45+位于~\verb+0x101+。
直观的来看,大端序即数据的最高字节在地址上更靠前,小端序则是数据的最低字节
在地址上更靠前。关于字节序的详细说明,请参考维基百科相关条目。

\subsection{麻烦}
不同的处理器可能使用不同的字节序,这在数据交换的时候会带来很多麻烦。

比如用一个大端序的机器将数据~\verb+0x01234567+写入文件中,然后将文件复制
到一个小端序的机器中,小端序机器在读取数据时会顺序读取,即将~\verb+01+放
入低地址位,将~\verb+67+放在高地址位,而在解释时会认为低地址位的是最低字节。
所以,大端序的~\verb+0x1234567+会被小字节序当做~\verb+0x67452301+。通常在
不同字节序的机器间传输数据时都需要先进行字节序的转换。

\subsection{字节序的判断}
大多数个人计算机的字节序都是小端序,可以通过下面的命令查看当前机器的字节序:
\begin{minted}{console}
$ lscpu | grep -i byte
\end{minted}

对于SAC文件来说,由于SAC头段中并没有包含当前字节序的相关信息,
所以SAC只能通过当前文件的SAC版本号,即~\verb+nvhdr+来判断。若~\verb+nvhdr+的
值在0到6之间,则SAC文件的字节序与当前机器字节序相同,否则不同则需要做字节序转换。

如果想自己确认数据与机器的字节序是否相同,可以用如下命令查看:
\begin{minted}{console}
$ od -j304 -N4 -An -td file.SAC
\end{minted}
该命令会输出SAC文件中的第305到308字节,若输出为6则表示文件字节序与
当前机器字节序相同,否则不同。

\subsection{字节序的转换}
SAC程序在读入数据时,会自动检测当前机器的字节序以及当前数据的字节序。
若二者不相同,则会对已读入内存中的数据进行字节序的转换。

但SAC相关的工具却不一定可以。比如,旧版本的~\verb+saclst+以及网上的某些
matlab处理脚本大多是没有对SAC文件的字节序做判断的。这会导致一个让人困惑的问题:
数据用SAC读取和查看一切正常,但是用其他工具读却乱七八糟,数据点数、采样间隔以及
具体的数据值没有一个对的。这种问题很多情况下都是因为字节序导致的。

如果SAC文件的字节序跟当前机器的字节序不同,最好将SAC文件转换到当前机器的
字节序,这样不论是SAC还是其他工具都可以正常对数据进行处理。字节序转换的
办法很简单:
\begin{SACCode}
SAC> r *.SAC
SAC> w over
\end{SACCode}
直接将所有SAC数据读入SAC,然后再~\verb+w over+即可,SAC会自动以当前机器的
字节序覆盖磁盘中的文件。
