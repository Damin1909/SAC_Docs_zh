\section{字节序}
字节序,即一个多字节的数据在内存中存储方式。计算机领域中,有两种字节序,
即小端序(little endian)和大端序(big endian)。

举例说明,假设一个整型变量占四个字节,其地址为\verb+0x100+。则
\verb+x+的四个字节将被存储在存储器的\verb+0x100~0x103+字节中。
若这个整型变量的十六进制为\verb+0x01234567+,若按照大端序存储,则
\verb+01+位于地址\verb+0x100+,\verb+23+位于地址\verb+0x101+,以此
类推。而小端序则是\verb+67+位于地址\verb+0x100+,\verb+45+位于
\verb+0x101+。直观的来看,大端序即数据的最高字节在地址上更靠前,
小端序则是数据的最低字节在地址上更靠前。

关于字节序的详细说明,请参考维基百科相关条目。

不同的处理器使用可能使用不同的字节序,这在数据交换的时候会带来很多麻烦。

比如用一个大端序的机器将数据\verb+0x01234567+写入文件中,然后将文件复制
到一个小端序的机器中,小端序机器在读取数据时会顺序读取,即将\verb+01+放
入低地址位,将\verb+67+放在高地址位,而在解释时会认为低地址位的是最低字节。
所以,大端序的\verb+0x1234567+会被小字节序当做\verb+0x67452301+。通常在
不同字节序的机器间传输数据时都需要先进行字节序的转换。

在SAC中也会遇到类似的问题,幸运的是sac程序在读取数据时会自动检测当前
机器的字节序以及当前数据的字节序,若二者不相同,则会对已读入内存中的
数据进行字节序的转换。

由于SAC头段中并没有包含当前字节序的相关信息,所以SAC只能通过判断当前
文件的SAC版本号,即\verb+nvhdr+来判断,若\verb+nvhdr+的值在0到6之间,
则SAC文件的字节序与当前机器字节序相同,否则不同则需要做字节序转换。

SAC能够自动判断文件的字节序并在必要的时候做字节序转换,但其他与SAC相关
的工具却不一定可以。比如,旧版本的\verb+saclst+以及网上的某些matlab处理
本大多是没有对SAC文件的字节序做判断的。这会导致一个让人困惑的问题:某些
数据用SAC读一切正常,但是用其他工具读却乱七八糟,数据点数、采样间隔以及
具体的数据值没有一个对的。这种问题很多情况下都是因为字节序导致的。

如果SAC文件的字节序跟当前机器的字节序不同,最好将SAC文件转换到当前机器的
字节序,这样不论是SAC还是其他工具都可以正常对数据进行处理。字节序转换的
办法很简单:
\begin{SACCode}
SAC> r *.SAC
SAC> w over
\end{SACCode}
直接将所有SAC数据读入SAC,然后再\verb+w over+即可,SAC会自动以当前机器的
字节序覆盖磁盘中的文件。
