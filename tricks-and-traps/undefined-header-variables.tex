\section{未定义的头段变量}
处理数据的时候会遇到一种情况,用SAC读入数据后,可以用lh查看到某个头段
变量的值,但是直接使用 \texttt{saclst} 却发现头段变量的值未定义。

直接读入SAC查看头段变量 \texttt{dist} 的值:
\begin{SACCode}
SAC> r XXXX.SAC
SAC> lh dist
     dist = 3.730627e+02
\end{SACCode}

用saclst查看同一个文件的头段变量dist的值:
\begin{minted}{console}
$ saclst dist f XXXX.SAC
XXXX.SAC        -12345.0
\end{minted}

用两种方法查看头段变量的值得到的结果不同,出现这种情况的原因,这个SAC
数据中 \texttt{dist} 本身是没有定义的,当SAC读入该数据时,会自动计算并
更新 \texttt{dist} 的值,所以使用 \texttt{lh} 会得到正确的 \texttt{dist}
值,而 \texttt{saclst} 是直接读取数据文件的头段,并不会对重新计算,因而
\texttt{saclst} 得到的是未定义值。也就是说,\texttt{saclst} 得到的是
文件中保存的值,\texttt{lh} 得到的是数据中应该保存的值。

如果想要 \texttt{saclst} 也获取正确的值,可以先用SAC把数据读进去,待
SAC把头段更新后,再写回到磁盘中:
\begin{SACCode}
SAC> r *.SAC
SAC> wh
SAC> q
\end{SACCode}

经常会出现这些问题的头段变量,换句话说,SAC在读入数据时会自动更新的
头段变量包括:\texttt{depmax}、\texttt{depmin}、\texttt{depmen}、
\texttt{e}、\texttt{gcarc}、\texttt{dist} 等。
