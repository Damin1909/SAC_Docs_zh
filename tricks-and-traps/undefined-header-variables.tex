\section{未定义的头段变量}
处理数据的时候会遇到一种情况,用SAC读入数据后,可以用lh查看到某个头段变量的值,
但是直接使用 \texttt{saclst} 却发现头段变量的值未定义。

直接读入SAC查看头段变量dist的值:
\begin{SACCode}
SAC> r XXXX.SAC
SAC> lh dist
     dist = 3.730627e+02
\end{SACCode}

用saclst查看同一个文件的头段变量dist的值:
\begin{minted}{console}
$ saclst dist f XXXX.SAC
XXXX.SAC        -12345.0
\end{minted}

用两种方法查看头段变量的值得到的结果不同,出现这种情况的原因,这个SAC数据中
dist本身是没有定义的,当SAC读入该数据时,会自动计算并更新dist的值,所以使用
lh会得到正确的dist值,而saclst是直接读取数据文件的头段,并不会对重新计算,
因而saclst得到的是未定义值。也就是说,saclst得到的是文件中保存的值,lh得到的
是数据中应该保存的值。

如果想要saclst也获取正确的值,可以先用SAC把数据读进去,待SAC把头段更新后,
再写回到磁盘中:
\begin{SACCode}
SAC> r *.SAC
SAC> wh
SAC> q
\end{SACCode}

经常会出现这些问题的头段变量,换句话说,SAC在读入数据时会自动更新的头段变量
包括:depmax、depmin、depmen、e、gcarc、dist等。
