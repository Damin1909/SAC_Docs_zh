\section{读取目录下的SAC文件}
\label{sec:read-dir}
假设目录 !data! 下有一堆SAC文件,现在想要将这些SAC文件读入内存中,
有如下几种方式。

第一种,先cd进入该目录,再读取SAC文件:
\begin{SACCode}
$ cd data/
$ sac
SAC> r *.SAC
SAC> ...
SAC> q
$ cd ..
\end{SACCode}

第二种,直接用相对路径读取SAC文件:
\begin{SACCode}
SAC> r data/*.SAC
SAC> ...
\end{SACCode}

第三种,使用 \nameref{cmd:read}命令的 !dir! 选项:
\begin{SACCode}
SAC> r dir data *.SAC
SAC> ...
\end{SACCode}

以上算是技巧,下面来说说这其中的陷阱。假设有一堆SAC文件,保存在一个名为
!dirraw! 的目录中,现在想要用第二种方式读取SAC文件:
\begin{SACCode}
SAC> r dirraw/*.SAC
 ERROR 1301: No data files read in.
\end{SACCode}
会发现出现了无法读入数据的错误。这是为什么呢?

严格来说,这算是SAC的一个bug。如上面的第三种方法所示,SAC的
!read! 命令有 !dir! 选项,用于指定要从哪个目录中读取SAC
数据。在上面例子中,因为目录名为 !dirraw!,SAC在解释
!r dirraw/*.SAC! 时出现了一些问题,SAC会从 !dirraw/*.SAC!
中识别处关键字 !dir!,然后忽略了后面的其他字符,因而在SAC看来,
这个命令实际上等效于 !r dir!,只给定了关键字 !dir!,却
没有给定目录名以及要读取的SAC文件名,因而出现了如上所示错误。

因而,要避免这个错误,就要求目录名不要以 !dir! 开头。

那么,如果目录名真的是以 !dir! 开头的,怎么办呢?第一种方法可以,
第三种方法也可以,第二种方法的修改版也可以:
\begin{SACCode}
SAC> r ./dirraw/*.SAC
\end{SACCode}
