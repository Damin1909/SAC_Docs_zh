\section{读取目录下的SAC文件}
\label{sec:read-dir}
假设目录 \texttt{data} 下有一堆SAC文件,现在想要将这些SAC文件读入内存中,
有如下几种方式。

第一种,先cd进入该目录,再读取SAC文件:
\begin{SACCode}
$ cd data/
$ sac
SAC> r *.SAC
SAC> ...
SAC> q
$ cd ..
\end{SACCode}

第二种,直接用相对路径读取SAC文件:
\begin{SACCode}
SAC> r data/*.SAC
SAC> ...
\end{SACCode}

第三种,使用 \nameref{cmd:read}命令的 \texttt{dir} 选项:
\begin{SACCode}
SAC> r dir data *.SAC
SAC> ...
\end{SACCode}

以上算是技巧,下面来说说这其中的陷阱。假设有一堆SAC文件,保存在一个名为
\texttt{dirraw} 的目录中,现在想要用第二种方式读取SAC文件:
\begin{SACCode}
SAC> r dirraw/*.SAC
 ERROR 1301: No data files read in.
\end{SACCode}
会发现出现了无法读入数据的错误。这是为什么呢?

严格来说,这算是SAC的一个bug。如上面的第三种方法所示,SAC的
\texttt{read} 命令有 \texttt{dir} 选项,用于指定要从哪个目录中读取SAC
数据。在上面例子中,因为目录名为 \texttt{dirraw},SAC在解释
\texttt{r dirraw/*.SAC} 时出现了一些问题,SAC会从 \texttt{dirraw/*.SAC}
中识别处关键字 \texttt{dir},然后忽略了后面的其他字符,因而在SAC看来,
这个命令实际上等效于 \texttt{r dir},只给定了关键字 \texttt{dir},却
没有给定目录名以及要读取的SAC文件名,因而出现了如上所示错误。

因而,要避免这个错误,就要求目录名不要以 \texttt{dir} 开头。

那么,如果目录名真的是以 \texttt{dir} 开头的,怎么办呢?第一种方法可以,
第三种方法也可以,第二种方法的修改版也可以:
\begin{SACCode}
SAC> r ./dirraw/*.SAC
\end{SACCode}
