\section{wh与w over的用法与区别}
将SAC数据读入内存,做些许修改,然后再写回到磁盘,这是最常见的操作。有两个命令
可以完成将数据写回磁盘的操作,即~\verb+wh+和~\verb+w over+。

先说说这两者的区别,~\verb+w over+会用内存中SAC文件的头段区和数据区覆盖磁盘文件的
头段区和数据区,而~\verb+wh+则只会用内存中SAC文件的头段区覆盖磁盘文件中的头段区。

那么这两者该如何用呢?将SAC数据读入内存后,如果修改了数据,则必须使用~\verb+w over+;
如果仅修改了头段,则使用~\verb+wh+或者~\verb+w over+都可以,但是推荐使用~\verb+wh+,
原因很有两点:一方面,使用~\verb+wh+就已经足够完成操作;另一方面,~\verb+wh+与
~\verb+w over+相比,前者只需要写入很少字节的数据,而后者则需要写入更多字节,因而前者
在效率上比后者要高很多,尤其是当文件较大时。
