\section{wh与w over}
\label{sec:wh-and-wover}
将SAC数据读入内存,做些许修改,然后再写回到磁盘,这是最常见的操作。
有两个命令可以完成将数据写回磁盘的操作,即 \texttt{wh} 和 \texttt{w over}。

先说说这两者的区别,\texttt{w over} 会用内存中SAC文件的头段区和数据区
覆盖磁盘文件的头段区和数据区,而 \texttt{wh} 则只会用内存中SAC文件的
头段区覆盖磁盘文件中的头段区。

那么这两者该如何用呢?将SAC数据读入内存后,如果修改了数据,则必须使用
\texttt{w over};如果仅修改了头段,则使用 \texttt{wh} 或者 \texttt{w over}
都可以,但是推荐使用 \texttt{wh},原因很有两点:一方面,使用 \texttt{wh}
就已经足够完成操作;另一方面,\texttt{wh} 与 \texttt{w over} 相比,
前者只需要写入很少字节的数据,而后者则需要写入更多字节,因而前者在效率
上比后者要高很多,尤其是当文件较大时。
